Topics
Topic
Formula_Id q_4
Latex f(x)= \frac{x^2 + x + c}{x^2 + 2x + c}
Title Finding value of <span class="math-container" id="q_1">$c$</span> such that the range of the rational function <span class="math-container" id="q_2">$f(x) = \frac{x^2 + x + c}{x^2 + 2x + c}$</span> does not contain <span class="math-container" id="q_3">$[-1, -\frac{1}{3}]$</span>
Question <p>I am comfortable when I am asked to calculate the range of a rational function, but how do we do the reverse? I came across this problem.</p>  <p>If <span class="math-container" id="q_4">$$f(x)= \frac{x^2 + x + c}{x^2 + 2x + c}$$</span> then find the value of <span class="math-container" id="q_5">$c$</span> for which the range of <span class="math-container" id="q_6">$f(x)$</span> does not contain <span class="math-container" id="q_7">$[-1, -\frac{1}{3}]$</span>.</p>
Tags functions
_number B.1
Formula_Id q_9
Latex \frac{df}{dx} = f(x+1)
Title Solving differential equations of the form <span class="math-container" id="q_8">$f'(x)=f(x+1)$</span>
Question <p>How to solve differential equations of the following form: </p>  <p><span class="math-container" id="q_9">$\frac{df}{dx} = f(x+1)$</span></p>
Tags ordinary-differential-equations
_number B.2
Formula_Id q_13
Latex 10^{-10}
Title Approximation to <span class="math-container" id="q_10">$\sqrt{5}$</span> correct to an exactitude of <span class="math-container" id="q_11">$10^{-10}$</span>
Question <p>I am attempting to resolve the following problem:</p>  <blockquote>   <p>Find an approximation to <span class="math-container" id="q_12">$\sqrt{5}$</span> correct to an exactitude of <span class="math-container" id="q_13">$10^{-10}$</span> using the bisection algorithm.</p> </blockquote>  <p>From what I understand, <span class="math-container" id="q_14">$\sqrt{5}$</span> has to be placed in function of <span class="math-container" id="q_15">$x$</span> but I am not sure where to go from there.</p>  <p>Also, a function in Mathematica are given to do the calculations in which the function <span class="math-container" id="q_16">$f(x)$</span>, <span class="math-container" id="q_17">$a$</span> and <span class="math-container" id="q_18">$b$</span> (from the interval <span class="math-container" id="q_19">$[a, b]$</span> where <span class="math-container" id="q_20">$f(a)$</span> and <span class="math-container" id="q_21">$f(b)$</span> have opposite signs), the tolerance and the number of iterations.</p>
Tags numerical-methods,algorithms,bisection
_number B.3
Formula_Id q_22
Latex \sum_{k=0}^{n} \binom{n}{k} k
Title How to compute this combinatoric sum?
Question <p>I have the sum</p>  <p><span class="math-container" id="q_22">$$\sum_{k=0}^{n} \binom{n}{k} k$$</span></p>  <p>I know the result is <span class="math-container" id="q_23">$n 2^{n-1}$</span> but I don't know how you get there. How does one even begin to simplify a sum like this that has binomial coefficients.</p>
Tags combinatorics,number-theory,summation,proof-explanation
_number B.4
Formula_Id q_25
Latex 5^{133} \mod 8.
Title How to calculate mod of number with big exponent
Question <p>I want to find</p>  <p><span class="math-container" id="q_25">$$ 5^{133} \mod 8. $$</span> I have noticed that <span class="math-container" id="q_26">$5^n \mod 8 = 5$</span> when <span class="math-container" id="q_27">$n$</span> is uneven and 1 otherwise, which would lead me to say that <span class="math-container" id="q_28">$5^{133} \mod 8 = 5$</span> But I don't know how to prove this. How can I prove that this is the case (or find another solution if it is not)?</p>
Tags algebra-precalculus,arithmetic
_number B.6
Formula_Id q_49
Latex \lim_{n\rightarrow \infty}\sqrt[n]{\frac{(27)^n(n!)^3}{(3n)!}}
Title finding value of <span class="math-container" id="q_48">$\lim_{n\rightarrow \infty}\sqrt[n]{\frac{(27)^n(n!)^3}{(3n)!}}$</span>
Question <p>Finding value of <span class="math-container" id="q_49">$\lim_{n\rightarrow \infty}\sqrt[n]{\frac{(27)^n(n!)^3}{(3n)!}}$</span></p>  <p>what i try</p>  <p><span class="math-container" id="q_50">$\displaystyle l=\lim_{n\rightarrow \infty}\bigg(\frac{(27)^n(n!)^3}{(3n)!}\bigg)^{\frac{1}{n}}$</span></p>  <p><span class="math-container" id="q_51">$\displaystyle \ln(l)=\lim_{n\rightarrow \infty}\frac{1}{n}\bigg[n\ln(27)+3\ln(n!)-\ln((3n)!)\bigg]$</span></p>  <p>How do i solve it help me please </p>
Tags limits
_number B.8
Formula_Id q_52
Latex \sum_{n=0}^N nx^n
Title Simplifying this series
Question <p>I need to write the series </p>  <p><span class="math-container" id="q_52">$$\sum_{n=0}^N nx^n$$</span> </p>  <p>in a form that does not involve the summation notation, for example <span class="math-container" id="q_53">$\sum_{i=0}^n i^2 = \frac{(n^2+n)(2n+1)}{6}$</span>. Does anyone have any idea how to do this? I've attempted multiple ways including using generating functions however no luck</p>
Tags sequences-and-series
_number B.9
Formula_Id q_55
Latex \int_{0}^{\infty}\frac{\sin x}{x^{a}}
Title Find the values of a>0 for which the improper integral <span class="math-container" id="q_54">$\int_{0}^{\infty}\frac{\sin x}{x^{a}} $</span> converges .
Question <p>Find the values of a>0 for which the improper integral <span class="math-container" id="q_55">$\int_{0}^{\infty}\frac{\sin x}{x^{a}} $</span> converges .  Do  I have to expand integrand using series expansion??</p>
Tags improper-integrals
_number B.10
Formula_Id q_71
Latex \iint_{V} f(x,y) dx\ dy = \iint_{Q} f(\Phi(u,v) \Bigg| \frac{\partial{\Phi}}{\partial{u}} \times \frac{\partial{\Phi}}{\partial{v}} \Bigg|
Title What's the cross product in 2 dimensions?
Question <p>The math book i'm using states that the cross product for two vectors is defined over <span class="math-container" id="q_56">$R^3$</span>:</p>  <p><span class="math-container" id="q_57">$$u = (a,b,c)$$</span><br> <span class="math-container" id="q_58">$$v = (d,e,f)$$</span></p>  <p>is:</p>  <p><span class="math-container" id="q_59">$$u \times v = \begin{vmatrix} \hat{i} &amp; \hat{j} &amp; \hat{k} \\ a &amp; b &amp; c \\ d &amp; e &amp; f \\ \end{vmatrix} $$</span></p>  <p>and the direction of the resultant is determined by curling fingers from vector v to u with thumb pointing in direction of the cross product of u x v.</p>  <hr>  <p>Out of curiosity, what's the cross product if u and v are defined over <span class="math-container" id="q_60">$R^2$</span> instead of <span class="math-container" id="q_61">$R^3$</span> instead:</p>  <p><span class="math-container" id="q_62">$$u = (a,b)$$</span><br> <span class="math-container" id="q_63">$$v = (d,e)$$</span></p>  <p>Is there a "degenerate" case for the cross product of <span class="math-container" id="q_64">$R^2$</span> instead <span class="math-container" id="q_65">$R^3$</span>?  like this is some type of 2x2 determinant instead?</p>  <p>for instance if had a parameterization:</p>  <p><span class="math-container" id="q_66">$$\Phi(u,\ v) = (\ f(u),\ \ g(v)\ )$$</span></p>  <p>and needed to calculate in <span class="math-container" id="q_67">$R^2$</span>:</p>  <p><span class="math-container" id="q_68">$$ D = \Bigg| \frac{\partial{\Phi}}{\partial{u}} \times \frac{\partial{\Phi}}{\partial{v}} \Bigg| $$</span></p>  <p>There are plenty of examples in the book for calculating the determinate D in <span class="math-container" id="q_69">$R^3$</span> but none at all for <span class="math-container" id="q_70">$R^2$</span> case.</p>  <p>As in:</p>  <p><span class="math-container" id="q_71">$$ \iint_{V} f(x,y) dx\ dy = \iint_{Q} f(\Phi(u,v) \Bigg| \frac{\partial{\Phi}}{\partial{u}} \times \frac{\partial{\Phi}}{\partial{v}} \Bigg| $$</span></p>  <p><span class="math-container" id="q_72">$$ \Phi(u,v)=(2u \cos v,\ \ u \sin v) $$</span></p>
Tags multivariable-calculus,vectors
_number B.11
Formula_Id q_73
Latex (1+i\sqrt{3})^{1/2}
Title Finding the roots of a complex number
Question <p>I was solving practice problems for my upcoming midterm and however I got stuck with this question type.</p>  <p>It is asking me to find all roots and then sketch it.</p>  <p><span class="math-container" id="q_73">$(1+i\sqrt{3})^{1/2}$</span></p>  <p>How do we proceed? </p>
Tags linear-algebra,complex-numbers,polar-coordinates
_number B.12
Formula_Id q_76
Latex bf(b)-af(a)
Title How to simplify expression <span class="math-container" id="q_74">$\int_a^b f(x)dx+\int_{f(a)}^{f(b)} f^{-1}(x)dx \ ?$</span>
Question <p>How to simplify expression</p>  <p><span class="math-container" id="q_75">$$\int_a^b f(x)dx+\int_{f(a)}^{f(b)} f^{-1}(x)dx \ ?$$</span></p>  <p>The answer is <span class="math-container" id="q_76">$bf(b)-af(a)$</span>  but I am wondering how to get the answer.</p>
Tags calculus
_number B.13
Formula_Id q_77
Latex y=xy'+ \frac{1}{2}(y')^{2}
Title Help solving first-order differential equation
Question <p>I have first-order differential equation <span class="math-container" id="q_77">$$y=xy'+ \frac{1}{2}(y')^{2}$$</span> Maybe, with this someone will find way to solve it <span class="math-container" id="q_78">$$\frac{1}{2}y'(2x+y')=y$$</span> I thought I can use <span class="math-container" id="q_79">$x^2+y=t$</span> for subtitution and when I derivate, I have <span class="math-container" id="q_80">$t'=2x+y'\\(t'-2x)t'=2t-2x^2$</span> which is acctualy the same as previous. I don't have idea how to start.. </p>
Tags ordinary-differential-equations
_number B.14
Formula_Id q_82
Latex 1 + 2x + 3x^2 + 4x^3 + 5x^4 + ... + nx^{n-1}+...
Title Derive the sum of <span class="math-container" id="q_81">$\sum_{i=1}^n ix^{i-1}$</span>
Question <p>For the series </p>  <blockquote>   <p><span class="math-container" id="q_82">$$1 + 2x + 3x^2 + 4x^3 + 5x^4 + ... + nx^{n-1}+... $$</span> </p> </blockquote>  <p>and <span class="math-container" id="q_83">$x \ne 1, |x| &lt; 1$</span>.</p>  <p>I need to find partial sums and finally, the sum <span class="math-container" id="q_84">$S_n$</span> of series. Here is what I've tried: </p>  <ol> <li>We can take a series <span class="math-container" id="q_85">$S_2 = 1 + x + x^2 + x^3 + x^4 + ...$</span> so that <span class="math-container" id="q_86">$\frac{d(S_2)}{dx} = S_1$</span> (source series).</li> <li>For the <span class="math-container" id="q_87">$|x| &lt; 1$</span> the sum of <span class="math-container" id="q_88">$S_2$</span> (here is geometric progression): <span class="math-container" id="q_89">$\frac{1-x^n}{1-x} = \frac{1}{1-x}$</span></li> <li><span class="math-container" id="q_90">$S_1 = \frac{d(S_2)}{dx} = \frac{d(\frac{1}{1-x})}{dx} = \frac{1}{(1-x)^2}$</span></li> </ol>  <p>But this answer is incorrect. Where is my mistake? Thank you.</p>
Tags sequences-and-series,convergence,summation,power-series
_number B.15
Formula_Id q_92
Latex \int_0^1\frac{\ln(1+x)\ln(1-x)}{1+x}\,dx
Title Finding <span class="math-container" id="q_91">$ \int_0^1\frac{\ln(1+x)\ln(1-x)}{1+x}dx$</span>
Question <blockquote>   <p>Calculate   <span class="math-container" id="q_92">$$\int_0^1\frac{\ln(1+x)\ln(1-x)}{1+x}\,dx$$</span></p> </blockquote>  <p>My try : </p>  <p>Let : <span class="math-container" id="q_93">$$I(a,b)=\int_0^1\frac{\ln(1-ax)\ln(1+bx)}{1+x}\,dx$$</span></p>  <p>Then compute <span class="math-container" id="q_94">$\frac{d^2 I(a,b)}{dadb}$</span>.</p>  <p>I'm happy to see ideas in order to kill this integral.</p>
Tags integration,sequences-and-series,definite-integrals,closed-form
_number B.16
Formula_Id q_97
Latex \int _{x=0}^{\infty} \frac{\sin(x)}{x}
Title Calculate <span class="math-container" id="q_95">$\int _{x=0}^{\infty} \frac{\sin(x)}{x}$</span> with the function <span class="math-container" id="q_96">$\frac{e^{iz}}{z}$</span>
Question <p>I want to calculate <span class="math-container" id="q_97">$\int _{x=0}^{\infty} \frac{\sin(x)}{x}$</span> with the function <span class="math-container" id="q_98">$f(z) = \frac{e^{iz}}{z}$</span>.</p>  <p>I thought about using the closed path <span class="math-container" id="q_99">$\Gamma = \gamma _1 + \gamma _R + \gamma _2 + \gamma _{\epsilon}$</span>, when:</p>  <p><span class="math-container" id="q_100">$\gamma_1 (t) = t, t \in [i\epsilon, iR]$</span></p>  <p><span class="math-container" id="q_101">$\gamma_R (t) = Re^{it}, t \in [-\frac{\pi}{2}, \frac{\pi}{2}]$</span></p>  <p><span class="math-container" id="q_102">$\gamma_2 (t) = t, t \in [-iR, -i\epsilon]$</span></p>  <p><span class="math-container" id="q_103">$\gamma_{\epsilon} (t) = \epsilon e^{it}, t \in [-\frac{\pi}{2}, \frac{\pi}{2}]$</span></p>  <p>I use the fact that <span class="math-container" id="q_104">$\frac{\sin(x)}{x}$</span> is an even function and has an anti derivative, so the integral on a closed path is zero.</p>  <p>I managed to show that <span class="math-container" id="q_105">$\int_{\gamma _{\epsilon}} f = -i\pi$</span> when <span class="math-container" id="q_106">$\epsilon \to 0$</span>.</p>  <p>However I am struggling to show that <span class="math-container" id="q_107">$\int_{\gamma _R} f = 0$</span> when <span class="math-container" id="q_108">$R \to \infty$</span></p>  <p>Help would be appreciated</p>
Tags complex-analysis,improper-integrals
_number B.17
Formula_Id q_111
Latex \log{2}+n\log\cfrac{n}{n+1}
Title Evaluate <span class="math-container" id="q_109">$\lim_{n \rightarrow \infty } \frac {[(n+1)(n+2)\cdots(n+n)]^{1/n}}{n}$</span>
Question <p>Evaluate <span class="math-container" id="q_110">$$\lim_{n \rightarrow \infty~} \dfrac {[(n+1)(n+2)\cdots(n+n)]^{\dfrac {1}{n}}}{n}$$</span> using Cesáro-Stolz theorem.</p>  <p>I know there are many question like this, but i want to solve it using Cesáro-Stolz method and no others.</p>  <p>I took log and applied Cesáro-Stolz, I get <span class="math-container" id="q_111">$$\log{2}+n\log\cfrac{n}{n+1}$$</span></p>  <p>Which gives me answer as <span class="math-container" id="q_112">$\frac{2}{e}$</span> . But answer is <span class="math-container" id="q_113">$\frac{4}{e}$</span>. Could someone help?.</p>  <p>Edit:  On taking log,  <span class="math-container" id="q_114">$$\lim_{n \to \infty} \frac{-n\log n + \sum\limits_{k=1}^{n} \log \left(k+n\right)}{n} \\= \lim_{n \to \infty} \left(-(n+1)\log (n+1) + \sum\limits_{k=1}^{n+1} \log \left(k+n\right)\right) - \left(-n\log n + \sum\limits_{k=1}^{n} \log \left(k+n\right)\right) \\ = \lim_{n \to \infty} \log \frac{2n+1}{n+1} - n\log \left(1+\frac{1}{n}\right) = \log 2 - 1$$</span> Which gives <span class="math-container" id="q_115">$2/e$</span></p>
Tags sequences-and-series
_number B.18
Formula_Id q_130
Latex \phi(n) = 40
Title Calculate all <span class="math-container" id="q_127">$n \in \Bbb N \setminus \{41\}$</span> such that <span class="math-container" id="q_128">$\phi(n)=40$</span>?
Question <blockquote>   <p>I'm looking for an <span class="math-container" id="q_129">$n \in \Bbb N$</span> for which <span class="math-container" id="q_130">$\phi(n) = 40$</span> where <span class="math-container" id="q_131">$\phi$</span> is a Euler-Totient Function </p> </blockquote>  <p>I already found one, namely, <span class="math-container" id="q_132">$n=41$</span></p>  <p>How the calculate the <span class="math-container" id="q_133">$n's$</span>?</p>
Tags totient-function
_number B.20
Formula_Id q_136
Latex 9^{9^{9^{…{^9}}}} ? x (\text{mod } 100)
Title Finding the last two digits of <span class="math-container" id="q_134">$9^{9^{9^{…{^9}}}}$</span> (nine 9s)
Question <p>I'm continuing on my journey learning about modular arithmetic and got confused with this question:</p>  <p>Find the last two digits of <span class="math-container" id="q_135">$9^{9^{9^{…{^9}}}}$</span> (nine 9s). The phi function is supposed to be used in this problem and so far this is what I've got:</p>  <p><span class="math-container" id="q_136">$9^{9^{9^{…{^9}}}} ? x (\text{mod } 100)$</span> Where <span class="math-container" id="q_137">$0 ? x ? 100$</span></p>  <p><span class="math-container" id="q_138">$9^{9^{9^{…{^9}}}} \text{ (nine 9s) }= 9^a$</span> In order to know <span class="math-container" id="q_139">$9^a (\text{mod } 100)$</span>, we need to know <span class="math-container" id="q_140">$a (\text{mod } \phi(100))$</span> As <span class="math-container" id="q_141">$\phi(100)= 40$</span>, we get <span class="math-container" id="q_142">$a = b (\text{mod } 40)$</span></p>  <p><span class="math-container" id="q_143">$9^{9^{9^{…{^9}}}} \text{ (eight 9s) }= 9^b$</span> In order to know <span class="math-container" id="q_144">$9^b (\text{mod } 40)$</span>, we need to know <span class="math-container" id="q_145">$b (\text{mod } \phi(40))$</span> As <span class="math-container" id="q_146">$\phi(40)= 16$</span>, we get <span class="math-container" id="q_147">$b = c ( \text{mod }16)$</span></p>  <p><span class="math-container" id="q_148">$9^{9^{9^{…{^9}}}}\text{ (seven 9s) }= 9^c $</span> In order to know <span class="math-container" id="q_149">$9^c (mod 16)$</span>, we need to know <span class="math-container" id="q_150">$c (\text{mod } phi(16))$</span> as <span class="math-container" id="q_151">$\phi(16)= 8 $</span> we need to find <span class="math-container" id="q_152">$c (\text{mod } 8)$</span></p>  <p>As <span class="math-container" id="q_153">$9 = 1 (\text{mod } 8)$</span> <span class="math-container" id="q_154">$c = 1 (\text{mod } 8)$</span></p>  <p>I feel like I might have made a mistake somewhere along the way because I'm having a lot of trouble stitching it all back together in order to get a value for the last two digits. Could anyone please help me with this? Thank you!</p>
Tags number-theory,modular-arithmetic
_number B.21
Formula_Id q_172
Latex \sqrt{2i-1}?
Title Is this the only way to evaluate <span class="math-container" id="q_171">$\sqrt{2i-1}?$</span>
Question <p>work out the <span class="math-container" id="q_172">$\sqrt{2i-1}?$</span></p>  <p><span class="math-container" id="q_173">$2i-1=(a+bi)^2$</span></p>  <p><span class="math-container" id="q_174">$a^2+2abi-b^2$</span></p>  <p><span class="math-container" id="q_175">$a^2-b^2=-1$</span></p>  <p><span class="math-container" id="q_176">$2ab=2$</span></p>  <hr>  <p><span class="math-container" id="q_177">$a^2=b^{-2}$</span></p>  <p><span class="math-container" id="q_178">$b^{-2}-b^2=-1$</span></p>  <p><span class="math-container" id="q_179">$-b^{4}+1=-1$</span></p>  <p><span class="math-container" id="q_180">$b^4=2$</span></p>  <p><span class="math-container" id="q_181">$b=\sqrt[4]{2}$</span></p>  <p>Can we solve <span class="math-container" id="q_182">$\sqrt{2i-1}$</span> in another way?</p>
Tags algebra-precalculus
_number B.24
Formula_Id q_199
Latex \lim_{x??}\log_xP(x)
Title What can be P(0), when <span class="math-container" id="q_183">$P(x^2+1)=(P(x))^2+1$</span> and P(x) is polynomial?
Question <p>What can be <span class="math-container" id="q_184">$P(0)$</span>, when <span class="math-container" id="q_185">$P(x^2+1)=(P(x))^2+1$</span> and <span class="math-container" id="q_186">$P(x)$</span> is polynomial?</p>  <p>Let <span class="math-container" id="q_187">$P(0)=0$</span>, then <span class="math-container" id="q_188">$P(1)=1$</span>, <span class="math-container" id="q_189">$P(2)=2$</span>, <span class="math-container" id="q_190">$P(5)=5$</span>, <span class="math-container" id="q_191">$P(26)=26$</span>, <span class="math-container" id="q_192">$P(677)=677$</span> ... and so on. Then <span class="math-container" id="q_193">$P(x)=x$</span>, because all the points on <span class="math-container" id="q_194">$y=P(x)$</span> are <span class="math-container" id="q_195">$y=x$</span>.</p>  <p>If <span class="math-container" id="q_196">$P(0)=2$</span>, then <span class="math-container" id="q_197">$P(x)=(x^2+1)^2+1$</span> for the same reason.</p>  <p>But when <span class="math-container" id="q_198">$P(0)=3$</span>, we have that <span class="math-container" id="q_199">$\lim_{x??}\log_xP(x)$</span> does not converge into an integer. So I think <span class="math-container" id="q_200">$P(x)$</span> cannot be a polynomial.</p>  <p>Then what are the values of <span class="math-container" id="q_201">$P(0)$</span> that makes <span class="math-container" id="q_202">$P(x)$</span> polynomial?</p>
Tags polynomials
_number B.25
Formula_Id q_205
Latex x-\frac{x^3}{3 \times 3!}+\frac{x^5}{5\times5!}-\frac{x^7}{7 \times 7!}+\cdots = \sum_{n=0}^\infty (-1)^n\frac{x^{(2n+1)}}{(2n+1) \times (2n+1)!}
Title How to solve an indefinite integral using the Taylor series?
Question <p>I am trying to show that the following integral is convergent but not absolutely.</p>  <blockquote>   <p><span class="math-container" id="q_203">$$\int_0^\infty\frac{\sin x}{x}dx.$$</span></p> </blockquote>  <p>My attempt:</p>  <blockquote>   <p>I first obtained the taylor series of <span class="math-container" id="q_204">$\int_0^x\frac{sin x}{x}dx$</span> which is as follows:   <span class="math-container" id="q_205">$$x-\frac{x^3}{3 \times 3!}+\frac{x^5}{5\times5!}-\frac{x^7}{7 \times 7!}+\cdots = \sum_{n=0}^\infty (-1)^n\frac{x^{(2n+1)}}{(2n+1) \times (2n+1)!} $$</span></p> </blockquote>  <p>Now <span class="math-container" id="q_206">$\int_0^\infty\frac{\sin x}{x}dx=\lim_{x\to \infty} \sum_{n=0}^\infty (-1)^n\frac{x^{(2n+1)}}{(2n+1) \times (2n+1)!}$</span></p>  <p>and I got stuck here! What is the next step?</p>
Tags real-analysis,calculus,integration,taylor-expansion,riemann-integration
_number B.26
Formula_Id q_207
Latex e^{3\pi i/2}
Title What is the value of <span class="math-container" id="q_207">$e^{3i \pi /2}$</span>?
Question <p>When solving for the value, we know that <span class="math-container" id="q_208">$e^{\pi i}=-1$</span> . I am confused as to what is the right answer when you evaluate this.I am getting two possible answers: <span class="math-container" id="q_209">$e^{3\pi i/2}$</span> = <span class="math-container" id="q_210">$(e^{\pi i})^{3/2}$</span> so this could be <span class="math-container" id="q_211">$(\sqrt{-1})^3=i^3=-i$</span> or it could be <span class="math-container" id="q_212">$\sqrt{(-1)^3}=\sqrt{-1}=i$</span>. Which one is the correct answer, and where am I going wrong? Thanks.</p>
Tags complex-numbers,exponentiation
_number B.27
Formula_Id q_213
Latex \sin(18^\circ)=\frac{a + \sqrt{b}}{c}
Title If <span class="math-container" id="q_213">$\sin(18^\circ)=\frac{a + \sqrt{b}}{c}$</span>, then what is <span class="math-container" id="q_214">$a+b+c$</span>?
Question <p>If <span class="math-container" id="q_215">$\sin(18)=\frac{a + \sqrt{b}}{c}$</span> in the simplest form, then what is <span class="math-container" id="q_216">$a+b+c$</span>? <span class="math-container" id="q_217">$$ $$</span> <strong>Attempt:</strong> <span class="math-container" id="q_218">$\sin(18)$</span> in a right triangle with sides <span class="math-container" id="q_219">$x$</span> (in front of corner with angle <span class="math-container" id="q_220">$18$</span> degrees), <span class="math-container" id="q_221">$y$</span>, and hypotenuse <span class="math-container" id="q_222">$z$</span>, is actually just <span class="math-container" id="q_223">$\frac{x}{z}$</span>, then <span class="math-container" id="q_224">$x = a + \sqrt{b}, z = c$</span>. We can find <span class="math-container" id="q_225">$y$</span> as <span class="math-container" id="q_226">$$ y = \sqrt{c^{2}- (a + \sqrt{b})^{2}} $$</span> so we have  <span class="math-container" id="q_227">$$ \cos(18) = \frac{y}{z} = \frac{\sqrt{c^{2}- (a + \sqrt{b})^{2}}}{c}$$</span></p>  <p>I also found out that <span class="math-container" id="q_228">$$b = (c \sin(18) - a)^{2} = c^{2} \sin^{2}(18) - 2ac \sin(18) + a^{2}$$</span></p>  <p>I got no clue after this.</p>  <hr>  <p>The solution says that <span class="math-container" id="q_229">$$ \sin(18) = \frac{-1 + \sqrt{5}}{4} $$</span></p>  <p>I gotta intuition that we must find <span class="math-container" id="q_230">$A,B,C$</span> such that <span class="math-container" id="q_231">$$ A \sin(18)^{2} + B \sin(18) + C = 0 $$</span>  then <span class="math-container" id="q_232">$\sin(18)$</span> is a root iof <span class="math-container" id="q_233">$Ax^{2} + Bx + C$</span>, and <span class="math-container" id="q_234">$a = -B, b = B^{2} - 4AC, c = 2A$</span>.</p>  <hr>  <p><strong>Totally different. This question is not asking to prove that <span class="math-container" id="q_235">$sin(18)=(-1+\sqrt{5})/4$</span>, that is just part of the solution.</strong></p>
Tags algebra-precalculus,trigonometry,euclidean-geometry,contest-math
_number B.28
Formula_Id q_238
Latex i=\sqrt{-1}
Title Dividing Complex Numbers by Infinity
Question <p>My PreCalculus teacher recently reviewed the properties of limits with us before our test and stated that <em>any real number divided by infinity equals zero</em>. This got me thinking and I asked them whether a <strong>complex number</strong> (i.e. <span class="math-container" id="q_236">$3+2i$</span> or <span class="math-container" id="q_237">$-4i$</span>) <strong>divided by infinity would equal zero</strong>. </p>  <p>This completely stumped them and I was unable to get an answer. After doing some theoretical calculation, knowing that <span class="math-container" id="q_238">$i=\sqrt{-1}$</span>, I calculated that a complex number such as <span class="math-container" id="q_239">$\frac{5i}{\infty}=0$</span> since  <span class="math-container" id="q_240">$$\frac{5}{\infty}\cdot \frac{\sqrt{-1}}{\infty} = 0\cdot 0 = 0,$$</span>  using properties utilized with real numbers that would state that <span class="math-container" id="q_241">$\frac{5x}{\infty} = 0$</span> since <span class="math-container" id="q_242">$$\frac{5}{\infty}\cdot \frac{x}{\infty} = 0\cdot 0 = 0.$$</span> Is this theoretical calculation correct or is there more to the concept than this?</p>
Tags algebra-precalculus,limits,complex-numbers,infinity
_number B.29
Formula_Id q_245
Latex a^3+b^3+c^3-3abc
Title Find <span class="math-container" id="q_243">$a^3+b^3+c^3-3abc$</span> (binomial theorem)
Question <blockquote>   <p><span class="math-container" id="q_244">$$a=\sum_{n=0}^\infty\frac{x^{3n}}{(3n)!}\\b=\sum_{n=1}^\infty\frac{x^{3n-2}}{(3n-2)!}\\c=\sum_{n=1}^\infty\frac{x^{3n-1}}{(3n-1)!}$$</span>Find <span class="math-container" id="q_245">$a^3+b^3+c^3-3abc$</span>:</p>      <p><span class="math-container" id="q_246">$(a)\ 1$</span></p>      <p><span class="math-container" id="q_247">$(b)\ 0$</span></p>      <p><span class="math-container" id="q_248">$(c)-1$</span></p>      <p><span class="math-container" id="q_249">$(d)-2$</span></p> </blockquote>  <p>Please help me solve this question.</p>  <p>I added <span class="math-container" id="q_250">$a,b$</span> and <span class="math-container" id="q_251">$c$</span>. It gives me the expansion of <span class="math-container" id="q_252">$e^x$</span>.</p>  <p>But i dont know how to use it.</p>
Tags binomial-theorem
_number B.30
Formula_Id q_262
Latex Empty(x) \iff \not \exists y (y \in x)
Title Are definitions axioms?
Question <p>I just want to ask a very elementary question.</p>  <p>When we introduce a "definition" in a first order logical system. For example when we say </p>  <p>Define: <span class="math-container" id="q_262">$Empty(x) \iff \not \exists y (y \in x) $</span> </p>  <p>Isn't that definition itself an "axiom", call it a definitional axiom.</p>  <p>I'm asking this because the one place predicate symbol Empty() is actually new, it is not among the listed primitives of say Zermelo, which has only identity and membership as primitive symbols. </p>  <p>So when we are stating definitions are we in effect stating axioms? but instead of being about characterizing a primitive, they are definitional axioms giving a complete reference to a specified set of symbols in the system.</p>  <p>Is that correct?</p>  <p>Now if that is the case, then why we don't call it axiom when we state it, I mean why we don't say for example:</p>  <p>Definitional axiom 1) <span class="math-container" id="q_263">$Empty(x) \iff \not \exists y (y \in x)$</span></p>  <p>Zuhair</p>
Tags terminology,definition,first-order-logic,axioms
_number B.32
Formula_Id q_267
Latex \frac{\partial^3 f}{\partial x^3}
Title Physical meaning and significance of third derivative of a function
Question <p>Given a physical quantity represented by a function <span class="math-container" id="q_264">$f(t,x)$</span> what is (if there is any) the actual meaning of the third derivative of <span class="math-container" id="q_265">$f$</span>, <span class="math-container" id="q_266">$\frac{\partial^3 f}{\partial t^3}$</span> or <span class="math-container" id="q_267">$\frac{\partial^3 f}{\partial x^3}$</span></p>
Tags physics
_number B.33
Formula_Id q_278
Latex a \uparrow^n b
Title Extending Knuth up-arrow/hyperoperations to non-positive values
Question <p>So... I had the silly idea to extend Knuth's up-arrow notation so that it included zero and negative arrows. It is normally defined as <span class="math-container" id="q_268">$$\begin{align*} a \uparrow b &amp; = a^b \\ a \uparrow^n b &amp; = \underbrace{a \uparrow^{n - 1} (a \uparrow^{n - 1} (\dots(a \uparrow^{n - 1} a) \dots ))}_{b\text{ copies of } a} \end{align*}$$</span> so, basically the hyperoperation sequence starting from exponentiation. For now, I will only consider <span class="math-container" id="q_269">$a,b &gt; 0$</span>.</p>  <p>If we try to go backwards from <span class="math-container" id="q_270">$a \uparrow b$</span>, the "trivial" extension (letting down arrows represent negative up arrows, because why the heck not) is: <span class="math-container" id="q_271">$$\begin{align*} a \;b &amp; = a \cdot b \\ a \downarrow b &amp; = a + b \\ a \downarrow \downarrow b &amp; = \text{see below} \end{align*} \\ \vdots$$</span> But I had trouble coming up with an expression for <span class="math-container" id="q_272">$a \downarrow \downarrow \downarrow b$</span>.</p>  <p>Maybe it doesn't exist. Alternatively, maybe there is a way of defining <span class="math-container" id="q_273">$a \; b$</span> (zero arrows) such that it does exist. So my question is: <strong>Is there an extension of Knuth's up-arrow notation such that <span class="math-container" id="q_274">$a \downarrow^n b$</span> exists for all <span class="math-container" id="q_275">$n \geq 3$</span>?</strong> <hr> <strong>Edit</strong>: Welp, I messed this question up. I initially thought <span class="math-container" id="q_276">$a \downarrow \downarrow b = a + 1$</span> was correct, but it is actually <span class="math-container" id="q_277">$b + 1$</span>. So I thought I had an example of an extension when I did not. I have modified the question accordingly.</p>  <p>An extension would define <span class="math-container" id="q_278">$a \uparrow^n b$</span> for each <span class="math-container" id="q_279">$n \leq 0$</span> which satisfies the recursive definition of the notation.</p>  <hr>  <p><strong>Edit 2</strong>: Okay, turns out <span class="math-container" id="q_280">$a \downarrow \downarrow b = b + 1$</span> isn't correct either, as this would imply <span class="math-container" id="q_281">$a \downarrow b = a + b - 1$</span>. For example, <span class="math-container" id="q_282">$4 \downarrow 3 = 4 \downarrow \downarrow (4 \downarrow \downarrow 4) = 4 \downarrow \downarrow (4 + 1) = (4 + 1) + 1 = 6 = 4 + 3 - 1$</span>. But it is really close; perhaps we need an exception, such as <span class="math-container" id="q_283">$$\begin{align*} a \downarrow \downarrow b = \begin{cases}b + 1 &amp; \text{if } a &lt; b \\ b + 2 &amp; \text{if } a = b \end{cases}\end{align*}.$$</span> The case <span class="math-container" id="q_284">$a &gt; b$</span> does not show up when evaluating <span class="math-container" id="q_285">$a \downarrow b$</span>, but it will be need to be defined if we try to extend further to <span class="math-container" id="q_286">$a \downarrow \downarrow \downarrow b$</span>. For instance, we could abuse the fact that the case <span class="math-container" id="q_287">$a &gt; b$</span> is allowed to be anything, and let <span class="math-container" id="q_288">$$a \downarrow \downarrow b = b + 1 + \left\lfloor \frac{a}{b} \right\rfloor,$$</span> but finding <span class="math-container" id="q_289">$a \downarrow \downarrow \downarrow b$</span> may be intractable as a result.</p>
Tags hyperoperation,ackermann-function
_number B.34
Formula_Id q_290
Latex \int e^{x^2} dx
Title When does a function NOT have an antiderivative?
Question <p>I know this question may sound naïve but why can't we write <span class="math-container" id="q_290">$\int e^{x^2} dx$</span> as <span class="math-container" id="q_291">$\int e^{2x} dx$</span>? The former does not have an antiderivative, while the latter has.</p>  <p>In light of this question, what are sufficient conditions for a function NOT to have an antiderivative. That is, do we need careful examination of a function to say it does not have an antiderivative or is there any way that once you see the function, you can right away say it does not have an antiderivative?</p>
Tags integration
_number B.35
Formula_Id q_294
Latex \lnot P \to A_1 \to\ ... \ \to A_n \to P
Title Proof by contradiction, status of initial assumption after the proof is complete.
Question <p>First of all I'd like to say that I have looked for the answers to my specific question and have not found it in the existing topics.</p>  <p>The question is fairly simple. Say, we need to  prove statement P by the method of contradiction. <strong>Assuming</strong> that <span class="math-container" id="q_292">$\lnot P$</span> holds, using the list of statements proven earlier to hold or derived by us during the proof, we arrive to P being <span class="math-container" id="q_293">$true$</span>.</p>  <p><span class="math-container" id="q_294">$$\lnot P  \to A_1 \to\ ... \ \to A_n \to P$$</span> <span class="math-container" id="q_295">$$\lnot P  \to  P \iff \lnot(\lnot P) \lor P \iff P $$</span></p>  <p>We can therefore add P to the list of our proven statements, because it was derived. Most of the proofs contain something in the lines of "<em>the obtained contradiction proves that our initial assumption (<span class="math-container" id="q_296">$\lnot P$</span>) was wrong</em> and so <span class="math-container" id="q_297">$P$</span> holds".</p>  <p>What I don't understand is, if the initial assumption (<span class="math-container" id="q_298">$\lnot P$</span>) is thus proven to be <em>false</em>, then why can we be sure that anything derived from it holds (in particular, that P holds)? On the other hand, if it cannot be derived then the assumption (<span class="math-container" id="q_299">$\lnot P$</span>) can in fact be true. </p>  <p>Can someone explain why this type of argument cannot be used?</p>
Tags logic,proof-writing
_number B.36
Formula_Id q_305
Latex f\circ g = g \circ f
Title Non trivial examples of <span class="math-container" id="q_300">$f\circ g = g \circ f$</span> but <span class="math-container" id="q_301">$f^{-1} \neq g$</span> and <span class="math-container" id="q_302">$f\neq\mathrm{id}\neq g$</span>.
Question <p>Are there real-valued functions <span class="math-container" id="q_303">$f$</span> and <span class="math-container" id="q_304">$g$</span> which are neither each other's inverses, the identity, nor linear, yet exhibit the behaviour <span class="math-container" id="q_305">$$f\circ g = g \circ f?$$</span></p>  <p>Examples such as <span class="math-container" id="q_306">$f(x) = 2x$</span> and <span class="math-container" id="q_307">$g(x)=3x$</span> are "trivial" in this sense. </p>  <p>Moreover, given a function <span class="math-container" id="q_308">$f$</span>, can one go about obtaining an example of a function <span class="math-container" id="q_309">$g$</span> which commutes with <span class="math-container" id="q_310">$f$</span>? </p>  <p>I suppose this would be similar to fixed point iteration? E.g. if <span class="math-container" id="q_311">$f\colon\mathbb R\smallsetminus\{1\}\to\mathbb R$</span> is defined by <span class="math-container" id="q_312">$f(x) = 2x/(1-x)$</span>, I would need a function <span class="math-container" id="q_313">$g$</span> such that  <span class="math-container" id="q_314">$$g(x) = f^{-1}\circ g\circ f =\frac{g(\frac{2x}{1-x})}{2+g(\frac{2x}{1-x})},$$</span> so maybe choosing an appropriate "starting function" <span class="math-container" id="q_315">$g_0$</span> and finding a fixed point of <span class="math-container" id="q_316">$g_{n+1} \mapsto f^{-1}\circ g_n\circ f$</span> would be a possible strategy, but I can't seem to find a suitable <span class="math-container" id="q_317">$g_0$</span>.</p>
Tags real-analysis,functional-analysis,functions
_number B.37
Formula_Id q_319
Latex q, r: a = bq + r
Title Uses of Axiom of Choice
Question <p>I am a first-year maths student but I occasionally drift away from our taught material. Some years ago I saw the ZFC axioms for the first time, but now that I am in college, and although the stuff I've been taught so far is nowhere near ZFC (in terms of difficulty), it happened to me that we use the axiom of choice all the time in every module, even if we don't know it by name yet.</p>  <p>For example, in the proof that, for every non-negative integers <span class="math-container" id="q_318">$a, b$</span>, there exist integers <span class="math-container" id="q_319">$q, r: a = bq + r$</span> (with the known restrictions on r), and the proof starts like this: <span class="math-container" id="q_320">$Choose$</span> the largest integer <span class="math-container" id="q_321">$q : qb &lt;= a$</span>... blah blah blah.</p>  <p>Is it the axiom of choice that allows us to execute this simple yet so important step? </p>  <p>And a couple more questions: Can you name some other simple proofs, theorems, results etc for which the axiom of choice is essential?</p>  <p>Also, I've read that AOC has long been a topic of dispute for mathematicians, and that even today, some people do not accept it. Are there any alternative axiomatic systems that work equally well without needing AOC? Thanks!</p>
Tags set-theory
_number B.38
Formula_Id q_329
Latex a(x)y+b(x)y'+c(x)y"+d(x)y'''+...+q(x)=0
Title What is the meaning of the term "linear"
Question <p><span class="math-container" id="q_328">$a_1x_1+a_2x_2+a_3x_3+...+a_nx_n=$</span> is called a linear equation because it represents the equation of a line in an n dimensional space. So "linear" comes from the word "line".Basically there should not be any higher power of x failing which the graph of the function will not be a straight line.</p>  <p>simillarly</p>  <p><span class="math-container" id="q_329">$a(x)y+b(x)y'+c(x)y"+d(x)y'''+...+q(x)=0$</span> is also called linear differential equation because all the derivatives have power equal to 1 which is similar to the above definition of a linear equation.</p>  <p>A function f is called linear if: <span class="math-container" id="q_330">$f(x+y)=f(x)+f(y)$</span> and <span class="math-container" id="q_331">$f(cx)=cf(x)$</span>. Here c is a constant. In this definition of linearity of function "<span class="math-container" id="q_332">$f$</span>" what does the word linear means? How does it relate to a straight line?</p>  <p>Finally what does the term linear means in case of linear vector spaces? Where is the reference to a straight line?</p>  <p>So, whether linear is just a word used in different contexts? Does it have different meaning in different situation? Or linearity refers to some relation to a straight line? At Least please explain how the linearity of function f and linear vector space relate to the equation of a line.</p>
Tags linear-algebra
_number B.40
Formula_Id q_336
Latex \sum_{r=1}^n (-1)^{(n-r)} {n \choose r}(r)^m
Title Confusion in how to find number of onto functions if two sets are given
Question <p>In the book it is given if <strong>A</strong> and <strong>B</strong> are two finite sets containing <span class="math-container" id="q_333">$m$</span> and <span class="math-container" id="q_334">$n$</span> elements, respectively, then the number of onto functions from <strong>A</strong> to <strong>B</strong> will be if </p>  <p><span class="math-container" id="q_335">$n \leq m$</span> </p>  <p><span class="math-container" id="q_336">$$\sum_{r=1}^n (-1)^{(n-r)} {n \choose r}(r)^m $$</span></p>  <p>well I can't understand it and I am aware with combinations.</p>
Tags combinatorics,functions,combinations
_number B.41
Formula_Id q_340
Latex \sum_{n\geq1}\frac1{n^2+1}=\frac{\pi\coth\pi-1}2
Title Prove <span class="math-container" id="q_339">$\sum_{n\geq1}\frac1{n^2+1}=\frac{\pi\coth\pi-1}2$</span>
Question <p>I am trying to prove  <span class="math-container" id="q_340">$$\sum_{n\geq1}\frac1{n^2+1}=\frac{\pi\coth\pi-1}2$$</span> Letting <span class="math-container" id="q_341">$$S=\sum_{n\geq1}\frac1{n^2+1}$$</span> we recall the Fourier series for the exponential function <span class="math-container" id="q_342">$$e^x=\frac{\sinh\pi}\pi+\frac{2\sinh\pi}\pi\sum_{n\geq1}\frac{(-1)^n}{n^2+1}(\cos nx-n\sin nx)$$</span> Plugging in <span class="math-container" id="q_343">$x=\pi$</span> <span class="math-container" id="q_344">$$e^\pi=\frac{\sinh\pi}\pi+\frac{2\sinh\pi}\pi\sum_{n\geq1}\frac{(-1)^n}{n^2+1}(\cos n\pi-n\sin n\pi)$$</span> <span class="math-container" id="q_345">$$e^\pi=\frac{\sinh\pi}\pi+\frac{2\sinh\pi}\pi\sum_{n\geq1}\frac{(-1)^n}{n^2+1}((-1)^n-n\cdot0)$$</span> <span class="math-container" id="q_346">$$e^\pi=\frac{\sinh\pi}\pi+\frac{2\sinh\pi}\pi S$$</span> <span class="math-container" id="q_347">$$S=\frac{\pi e^\pi}{2\sinh\pi}-\frac12$$</span> But that is nowhere near to correct. What did I do wrong, and how do can I prove the identity? Thanks.</p>
Tags real-analysis,sequences-and-series
_number B.43
Formula_Id q_351
Latex ( \mathscr{M}_{2\times2}(\mathbb{Q}) , \times )
Title For <span class="math-container" id="q_348">$A,B \in \mathscr{M}_{2\times2}(\mathbb{Q}) $</span> of finite order, show that <span class="math-container" id="q_349">$AB$</span> has infinite order
Question <p>Let <span class="math-container" id="q_350">$G$</span> be the group <span class="math-container" id="q_351">$ ( \mathscr{M}_{2\times2}(\mathbb{Q}) , \times ) $</span> of nonsingular matrices.</p>  <p>Let <span class="math-container" id="q_352">$ A = \left ( \begin{matrix}  0 &amp; -1 \\   1 &amp; 0  \end{matrix} \right ) $</span>, the order of <span class="math-container" id="q_353">$A$</span> is <span class="math-container" id="q_354">$4$</span>;</p>  <p>Let <span class="math-container" id="q_355">$ B = \left ( \begin{matrix}  0 &amp; 1 \\   -1 &amp; -1  \end{matrix} \right ) $</span>, the order of <span class="math-container" id="q_356">$B$</span> is <span class="math-container" id="q_357">$3$</span>.</p>  <p>Show that <span class="math-container" id="q_358">$AB$</span> has infinite order.</p>  <hr>  <p>The only reasoning possible here is by contradiction as <span class="math-container" id="q_359">$G$</span> is not abelian. And so I tried, but I got stuck before any concrete development.</p>  <p>Any hints are welcome, Thanks.</p>
Tags matrices,group-theory,cyclic-groups
_number B.44
Formula_Id q_362
Latex \sin(x) , \sin(2x) , \sin(3x) ,...,\sin(nx)
Title How to prove that {<span class="math-container" id="q_360">$\sin(x) , \sin(2x) , \sin(3x) ,...,\sin(nx)$</span>} is independent in <span class="math-container" id="q_361">$\mathbb{R}$</span>?
Question <blockquote>   <p>Prove that {<span class="math-container" id="q_362">$\sin(x) , \sin(2x) , \sin(3x) ,...,\sin(nx)$</span>} is independent in <span class="math-container" id="q_363">$\mathbb{R}$</span></p> </blockquote>  <p>my trial :</p>  <p>we know that the Wronsekian shouldn't be <span class="math-container" id="q_364">$0$</span> to get the trivial solution and thus they are independent. its not trivial to show that <span class="math-container" id="q_365">$ W \not = 0$</span></p>  <p>W = </p>  <p><span class="math-container" id="q_366">$\begin{vmatrix} (1)\sin(x) &amp; (1)\sin(2x) &amp; (1)\sin(3x) &amp;  ... &amp;   (1)\sin(nx) \\ (1)\cos(x) &amp; (2)\cos(2x) &amp; (3)\cos(3x) &amp;  ... &amp;   (n)\cos(nx) \\  -(1)^2\sin(x) &amp; -(2)^2\sin(2x) &amp; -(3)^2\sin(3x) &amp;  ... &amp;   -(n)^2\sin(nx) \\ -(1)^3\cos(x) &amp; -(2)^3\cos(2x) &amp; -(3)^3\cos(3x) &amp;  ... &amp;   -(n)^3\cos(nx) \\ \end{vmatrix}$</span></p>  <p>and so on. it looks like Vandermonde matrix but i cant prove that and so we conclude that its <span class="math-container" id="q_367">$W\not =0$</span></p>
Tags ordinary-differential-equations
_number B.45
Formula_Id q_370
Latex \int x^k f(x) dx=0
Title Suppose <span class="math-container" id="q_368">$f$</span> is a Lebesgue integrable function on<span class="math-container" id="q_369">$[0,1]$</span> which satisfies <span class="math-container" id="q_370">$\int x^k f(x) dx=0$</span> , prove <span class="math-container" id="q_371">$f=0 \text{ a.e.}$</span>
Question <p>Suppose <span class="math-container" id="q_372">$f$</span> is an indefinitely differentiable real valued function on<span class="math-container" id="q_373">$[0,1]$</span> which satisfies <span class="math-container" id="q_374">$\int_0^1 x^k f(x)\, dx=0$</span> for <span class="math-container" id="q_375">$k=\{0,1,2,3,.. .\}$</span>, prove <span class="math-container" id="q_376">$f=0$</span> .<br> My attempt :<br> To prove this assertion , it suffice to prove <span class="math-container" id="q_377">$$\int_0^1 f^2 (x) \,dx=0$$</span>  Then by approximation theorem , we can find a polynomial such that for every <span class="math-container" id="q_378">$\epsilon \gt 0$</span> <span class="math-container" id="q_379">$$\int_0^1 f^2 (x) \,dx= \int_0^1 (f(x)-\sum_{n=0}^N a_n x^n)f(x) ,dx+\int_0^1 \sum_{n=0}^N f(x) a_n x^n\,dx \le \epsilon M$$</span> where <span class="math-container" id="q_380">$M=\sup_{x \in [0,1]}|f(x)|$</span> .<br> It seems that we do not need the condition that <span class="math-container" id="q_381">$f$</span> is indefinitely differentiable , if <span class="math-container" id="q_382">$f$</span> is continuous then the conclution may hold .<br> My question :<br> a) Suppose <span class="math-container" id="q_383">$f$</span> is a Lebesgue integrable real valued function on<span class="math-container" id="q_384">$[0,1]$</span> which satisfies <span class="math-container" id="q_385">$\int_0^1 x^k f(x)\ dx=0$</span> for <span class="math-container" id="q_386">$k=\{0,1,2,3,.. .\}$</span>, can we prove <span class="math-container" id="q_387">$f=0$</span> except on a set of measure <span class="math-container" id="q_388">$0$</span> ?     </p>  <p>b) If a) is not true , suppose <span class="math-container" id="q_389">$f$</span> is a Riemann integrable real valued function on<span class="math-container" id="q_390">$[0,1]$</span> which satisfies <span class="math-container" id="q_391">$\int_0^1 x^k f(x)\ dx=0$</span> for <span class="math-container" id="q_392">$k=\{0,1,2,3,.. .\}$</span>, can we prove <span class="math-container" id="q_393">$f=0$</span> except on a set of measure <span class="math-container" id="q_394">$0$</span> ?  </p>  <p>EDIT:<br> To prove <span class="math-container" id="q_395">$f=0 \text{  a.e.}$</span> , it suffice to prove all the fourier coefficient of <span class="math-container" id="q_396">$f$</span> equal to <span class="math-container" id="q_397">$0$</span> , then  <span class="math-container" id="q_398">$$\int_0^1 f(x) \cos(2 \pi nx) \,dx \le \int_0^1 |f(x)||\cos(2 \pi nx)-\sum_{n=0}^{N}a_n x^n| \le \epsilon||f||_{L^1}$$</span> and the proof is complete.</p>
Tags real-analysis
_number B.46
Formula_Id q_402
Latex rq \equiv 1 \bmod p
Title Prove that for a given prime <span class="math-container" id="q_399">$p$</span> and each <span class="math-container" id="q_400">$0 < r < p-1$</span>, there exists a <span class="math-container" id="q_401">$q$</span> such that <span class="math-container" id="q_402">$rq \equiv 1 \bmod p$</span>
Question <p>Prove that for a given prime <span class="math-container" id="q_403">$p$</span> and each <span class="math-container" id="q_404">$0 &lt; r &lt; p-1$</span>, there exists a <span class="math-container" id="q_405">$q$</span> such that </p>  <p><span class="math-container" id="q_406">$$rq \equiv 1 \bmod p$$</span></p>  <p>I've only taken one intro number theory course (years ago), and this just popped up in a computer science class (homework). I was assuming that this proof would be elementary since my current class in an algorithm cours, but after the few basic attempts I've tried it didn't look promising. Here's a couple approaches I thought of:</p>  <hr>  <p>(<em>reverse engineer</em>)</p>  <p>To arrive at the conclusion we would need</p>  <p><span class="math-container" id="q_407">$$rq - 1 = kp$$</span></p>  <p>for some <span class="math-container" id="q_408">$k$</span>. A little manipulation:</p>  <p><span class="math-container" id="q_409">$$qr - kp = 1$$</span></p>  <p>That looks familiar, but I can't see anything from it.</p>  <hr>  <p>(<em>sum on <span class="math-container" id="q_410">$r$</span></em>)</p>  <p><span class="math-container" id="q_411">$$\sum_{r=1}^{p-2} r = \frac{(p-2)(p-1)}{2} = p\frac{p - 3}{2} + 1 \equiv 1 \bmod p$$</span></p>  <p>which looks good but I don't know how to incorporate <span class="math-container" id="q_412">$r$</span> int0 the final equality.  </p>  <hr>  <p>(<em>Wilson's Theorem—proved by Lagrange</em>)  </p>  <p>I vaguely recall this theorem, but I was looking at it in an old book and it wasn't easy to see how we arrived there. Anyways, <span class="math-container" id="q_413">$p$</span> is prime <em>iff</em> <span class="math-container" id="q_414">$$(p-1)! \equiv -1 \bmod p$$</span></p>  <p>Here the <span class="math-container" id="q_415">$r$</span> multiplier is built in to the factorial expression so I was thinking of adding <span class="math-container" id="q_416">$2$</span> to either side</p>  <p><span class="math-container" id="q_417">$$(p-1)! + 2 \equiv 1 \bmod p$$</span></p>  <p>which is a dead end (pretty sure). But then I was thinking, maybe multiplying Wilson't Thm by <span class="math-container" id="q_418">$(p+1)$</span>? Then getting</p>  <p><span class="math-container" id="q_419">$$(p+1)(p-1)! = -(p+1) \bmod p$$</span></p>  <p>which I think results in</p>  <p><span class="math-container" id="q_420">$$(p+1)(p-1)! = 1 \bmod p$$</span></p>  <p>of which <span class="math-container" id="q_421">$r$</span> is a multiple and <span class="math-container" id="q_422">$q$</span> is obvious. But I'm not sure if that's valid.</p>
Tags elementary-number-theory,proof-verification
_number B.47
Formula_Id q_427
Latex (x+y)^k \geq x^k + y^k
Title Hints for showing that if <span class="math-container" id="q_423">$x,y \geq 0$</span>, then <span class="math-container" id="q_424">$(x+y)^k \geq x^k + y^k$</span> for all <span class="math-container" id="q_425">$k \geq 1$</span>
Question <p>I'm trying to show the following:</p>  <blockquote>   <blockquote>     <p>If <span class="math-container" id="q_426">$x,y \geq 0$</span>, then <span class="math-container" id="q_427">$(x+y)^k \geq x^k + y^k$</span> for all <span class="math-container" id="q_428">$k \in \mathbb{R_{\geq 1}}$</span></p>   </blockquote> </blockquote>  <p>So, far I've tried a few things, but nothing seems to stick. Clearly <span class="math-container" id="q_429">$x \leq x+ y$</span> and since both sides of the inequality are positive, the inequality <span class="math-container" id="q_430">$x^k \leq (x + y)^k$</span> will hold for <span class="math-container" id="q_431">$k \geq 1$</span>. Similarly, <span class="math-container" id="q_432">$y^k \leq (x+y)^k$</span>. Adding both inequalities together, we obtain: <span class="math-container" id="q_433">$x^k + y^k \leq 2(x+y)^k$</span>. While this is close, of course, it is not what we want to show.</p>  <p>Alternatively, I was thinking that if we fix <span class="math-container" id="q_434">$x,y \geq 0$</span>, let <span class="math-container" id="q_435">$f(k) = (x+y)^k$</span>, and let <span class="math-container" id="q_436">$g(k) = x^k + y^k$</span>, we can show using the binomial theorem that <span class="math-container" id="q_437">$(x+y)^r \geq x^r + y^r$</span> for all <span class="math-container" id="q_438">$r \in \mathbb{Z^+}$</span>. Then, if we can show both <span class="math-container" id="q_439">$f$</span> and <span class="math-container" id="q_440">$g$</span> intersect only when <span class="math-container" id="q_441">$k =1 $</span>, we might have better luck proving the statement since then we would have <span class="math-container" id="q_442">$(x+y)^r &gt; x^r + y^r$</span> for all positive integers <span class="math-container" id="q_443">$r \geq 2$</span>. A real number <span class="math-container" id="q_444">$m \notin \mathbb{Z}$</span> with the property that <span class="math-container" id="q_445">$f(m) = g(m)$</span> could not then exist since then that would violate the fact that <span class="math-container" id="q_446">$k=1$</span> is the only intersection. We could then invoke the continuity of <span class="math-container" id="q_447">$f$</span> and <span class="math-container" id="q_448">$g$</span> together with the fact that <span class="math-container" id="q_449">$(x+y)^r &gt; x^r + y^r$</span> for all positive integers <span class="math-container" id="q_450">$r \geq 2$</span> to obtain our result. Of course, all of this is dependent on rigorously showing that <span class="math-container" id="q_451">$f$</span> and <span class="math-container" id="q_452">$g$</span> intersect only when <span class="math-container" id="q_453">$k=1$</span>.</p>  <p>Otherwise, I'm running low on ideas. Any hints would be greatly appreciated.</p>
Tags real-analysis,functions,inequality
_number B.48
Formula_Id q_459
Latex \sum{\frac{1}{n^{2+\cos{n}}}}
Title Divergent series <span class="math-container" id="q_459">$\sum{\frac{1}{n^{2+\cos{n}}}}$</span>
Question <p>Bonjour.  Show that  <span class="math-container" id="q_460">$$\sum{\frac{1}{n^{2+\cos{n}}}}$$</span> is a divergent serie. <span class="math-container" id="q_461">$$\\$$</span></p>  <p>My main problem is: If <span class="math-container" id="q_462">$\epsilon$</span> is “infinitely small positive real number” define <span class="math-container" id="q_463">$A_{\epsilon}$</span> as the set of all <span class="math-container" id="q_464">$n, |2+\cos n|\leq 1+\epsilon$</span> <span class="math-container" id="q_465">$(n \in A_{\epsilon}\iff-1\leq \cos n \leq -1+\epsilon)$</span>. The divergence should come from the sum over <span class="math-container" id="q_466">$A_{\epsilon}$</span> but I have no idea to how to handle this. <span class="math-container" id="q_467">$$\\$$</span></p>
Tags real-analysis,integration,sequences-and-series,analysis
_number B.50
Formula_Id q_472
Latex (1+x)^n+(1+x)^{n+1}\frac{1}{2}+(1+x)^{n+2}\frac{1}{2^2}+\cdots\cdots +(1+x)^{2n}\frac{1}{2^n}.
Title Sum of series having binomial coefficients
Question <p>Prove that <span class="math-container" id="q_468">$\displaystyle \sum_{r=0}^n {n+r\choose r} \frac{1}{2^{r}}= 2^{n}$</span></p>  <p>what i try</p>  <p><span class="math-container" id="q_469">$$\binom{n}{n}+\binom{n+1}{1}\frac{1}{2}+\binom{n+2}{2}\frac{1}{2^2}+\binom{n+3}{3}\frac{1}{2^3}+\cdots +\binom{n+n}{n}\frac{1}{2^n}$$</span></p>  <p><span class="math-container" id="q_470">$$\binom{n}{n}+\binom{n+1}{n}\frac{1}{2}+\binom{n+2}{n}\frac{1}{2^2}+\binom{n+3}{n}\frac{1}{2^3}+\cdots +\binom{n+n}{n}\frac{1}{2^n}.$$</span></p>  <p>coefficient of <span class="math-container" id="q_471">$x^n$</span> in </p>  <p><span class="math-container" id="q_472">$$(1+x)^n+(1+x)^{n+1}\frac{1}{2}+(1+x)^{n+2}\frac{1}{2^2}+\cdots\cdots +(1+x)^{2n}\frac{1}{2^n}.$$</span></p>  <p>How do i solve ithelp me plesse</p>
Tags binomial-coefficients
_number B.51
Formula_Id q_479
Latex n=n_1n_2...n_k+1
Title Prove <span class="math-container" id="q_473">$\forall n\in\mathbb{N}$</span>, <span class="math-container" id="q_474">$\exists m\in\mathbb{N}$</span> s.t. <span class="math-container" id="q_475">$m>n$</span> and <span class="math-container" id="q_476">$m$</span> is prime
Question <p>There are two parts I am having trouble getting started.</p>  <p>A. Prove that <span class="math-container" id="q_477">$n_1, n_2,...,n_k\in\mathbb{N}$</span> are each at least <span class="math-container" id="q_478">$2$</span> then <span class="math-container" id="q_479">$n=n_1n_2...n_k+1$</span> is not divisible by any numbers <span class="math-container" id="q_480">$n_1, n_2,...,n_k$</span>. </p>  <p>B. Prove that the truth of the negation leads to a contradiction. (Use theorem: For all <span class="math-container" id="q_481">$a,b\in\mathbb{N}$</span> there exist a unique quotient <span class="math-container" id="q_482">$q$</span> and remainder <span class="math-container" id="q_483">$r$</span> in <span class="math-container" id="q_484">$\mathbb{Z^+}$</span> such that we have both <span class="math-container" id="q_485">$a=qb+r$</span> and <span class="math-container" id="q_486">$0\leq r&lt;q$</span>.)</p>  <p>For part A, I started with, given <span class="math-container" id="q_487">$k\in\mathbb{N}$</span> and <span class="math-container" id="q_488">$n_1, n_2,...,n_k\geq1$</span>, I'll show that <span class="math-container" id="q_489">$\forall i$</span>, <span class="math-container" id="q_490">$n_i \nmid n=n_1n_2...n_k+1$</span> to set it up, but I'm not sure how to actually go about starting it.</p>  <p>For part B, I know that the negation is <span class="math-container" id="q_491">$\exists n\in\mathbb{N}$</span> s.t. <span class="math-container" id="q_492">$\forall m\in\mathbb{N}$</span> either <span class="math-container" id="q_493">$m\leq n$</span> or <span class="math-container" id="q_494">$m$</span> is not prime, but again I'm not sure what I should do to start the proof or exactly how to incorporate that theorem.</p>
Tags proof-writing,prime-numbers
_number B.52
Formula_Id q_501
Latex AB = 1 \Rightarrow BA = 1
Title Show that one-sided inverse of a square matrix is a true inverse
Question <p>We know that for a group element <span class="math-container" id="q_495">$g\in G$</span>, <span class="math-container" id="q_496">$gh=1$</span> does not necessarily mean that <span class="math-container" id="q_497">$hg = 1$</span>. In the case for matrices (linear maps between vector spaces), it is also true that <span class="math-container" id="q_498">$AB = 1 \nRightarrow BA = 1$</span>. This happens when the <span class="math-container" id="q_499">$A$</span> and <span class="math-container" id="q_500">$B$</span> are not square matrices (in which case they do not even form a group under multiplication).</p>  <p>However if we restrict the square matrices, <span class="math-container" id="q_501">$AB = 1 \Rightarrow BA = 1$</span>. What is simple proof of this that avoids chasing the entries, and makes use simply the vector space structure of linear transformations?</p>  <p>(In fact if we could prove this, I think this might imply that for a group to have one-sided(but not two-sided) inverses, it has to be infinite, since every finite group admits a finite dimensional representation.</p>
Tags linear-algebra,group-theory
_number B.53
Formula_Id q_503
Latex P(N) = (S|S ⊆ N)
Title By using a diagonal argument, show that the powerset <span class="math-container" id="q_502">$P(N) = (S|S ? N)$</span> is uncountable.
Question <p>Any tips or solutions for this one?</p>  <p>By using a diagonal argument, show that the powerset <span class="math-container" id="q_503">$P(N) = (S|S ? N)$</span> is uncountable.</p>
Tags discrete-mathematics,elementary-set-theory
_number B.54
Formula_Id q_504
Latex \frac{1}{\sqrt{-1}}=\sqrt{-1}
Title <span class="math-container" id="q_504">$\frac{1}{\sqrt{-1}}=\sqrt{-1}$</span>?
Question <p>I have trouble to comprehend what my mistake is in the following calculation:</p>  <p>If we set <span class="math-container" id="q_505">$\sqrt{-1}$</span> to be the new number with the property that <span class="math-container" id="q_506">$(\sqrt{-1})^2 = -1$</span> then I can write <span class="math-container" id="q_507">$$\frac{1}{\sqrt{-1}}=\sqrt{\frac{1}{-1}}=\sqrt{-1}.$$</span></p>  <p>But we also have (and I know this is the correct result) <span class="math-container" id="q_508">$$ \frac{1}{\sqrt{-1}}\cdot\frac{\sqrt{-1}}{\sqrt{-1}}=\frac{\sqrt{-1}}{-1}=-\sqrt{-1}$$</span></p>  <p>What am I missing? Thanks.</p>
Tags complex-numbers,definition
_number B.55
Formula_Id q_510
Latex \exists p\ \bigl(\text{$p$ is prime } \rightarrow \forall x \text{ ($x$ is prime)}\bigr)
Title A curious logical formula involving prime numbers
Question <p>Let <span class="math-container" id="q_509">$S$</span> be a nonempty set of natural numbers. Is the following formula <span class="math-container" id="q_510">$$ \exists p\ \bigl(\text{$p$ is prime } \rightarrow \forall x  \text{ ($x$ is prime)}\bigr)  $$</span> true or false on <span class="math-container" id="q_511">$S$</span>? I know the answer to this question, but what would be the shortest way to arrive to the conclusion using some deduction system?</p>
Tags logic,first-order-logic
_number B.56
Formula_Id q_514
Latex f : B \to \mathbb{R}^m
Title Preimage of continuous one-to-one function on connected domain is not continuous.
Question <p>I know that given <span class="math-container" id="q_512">$B$</span>, a compact subset of <span class="math-container" id="q_513">$\mathbb{R}^n$</span>, and <span class="math-container" id="q_514">$f : B \to \mathbb{R}^m$</span>, a continuous injective (one-to-one) function, <span class="math-container" id="q_515">$f^{-1}$</span> is continuous on <span class="math-container" id="q_516">$f(B)$</span>. (This true).</p>  <p>I also know that image <span class="math-container" id="q_517">$f(X)$</span> of a connected subset <span class="math-container" id="q_518">$X$</span> is connected under a continuous function.</p>  <p>Now let <span class="math-container" id="q_519">$X$</span> be a connected (non-compact) subset of <span class="math-container" id="q_520">$\mathbb{R}^n$</span>, and <span class="math-container" id="q_521">$f : X \to \mathbb{R}^m$</span> be a continuous injective (one-to-one) function. I am trying (and struggling) to provide a counterexample in which mapping <span class="math-container" id="q_522">$f^{-1} : f(X) \mapsto X$</span> is not continuous on <span class="math-container" id="q_523">$f(X)$</span>. (A rigorous, parametrized example).</p>  <p>Thank you in advance!</p>
Tags real-analysis,general-topology,analysis,continuity,metric-spaces
_number B.57
Formula_Id q_525
Latex 3\arcsin \frac{1}{4} + \arccos \frac {11}{16} = \frac {\pi}{2}
Title Prove that <span class="math-container" id="q_524">$3\arcsin \frac{1}{4} + \arccos \frac {11}{16} = \frac {\pi}{2}$</span>
Question <p>Can someone help me with this exercise? I honestly don't know where to start and how to prove it. You don't have to answer it fully, just give me a hint or something. Thank you in advance.</p>  <blockquote>   <p>Exercise 1. Prove that  <span class="math-container" id="q_525">$3\arcsin \frac{1}{4} + \arccos \frac {11}{16} = \frac {\pi}{2}$</span></p> </blockquote>  <p>Thanks.</p>
Tags trigonometry
_number B.58
Formula_Id q_528
Latex \sum_{d|n}{\phi(d)}=n
Title Multiple proofs of <span class="math-container" id="q_526">$\sum_{d|n}{\phi(d)}=n$</span>
Question <p>I am looking for multiple proofs of that statement: here <span class="math-container" id="q_527">$\phi(n)$</span> denotes the Euler’s totient  <span class="math-container" id="q_528">$$\sum_{d|n}{\phi(d)}=n$$</span> </p>  <p>Here’s one: </p>  <p>By unique factorisation theorem: <span class="math-container" id="q_529">$n=\prod_{k=1}^{m}{p_k^{\alpha_k}}$</span> and <span class="math-container" id="q_530">$d=\prod_{k=1}^{m}{p_k^{\beta_k}}$</span> where <span class="math-container" id="q_531">$0\leq \beta_k\leq \alpha_k$</span> so: </p>  <p><span class="math-container" id="q_532">$\begin{align} \sum_{d|n}{\phi(d)}&amp;=\sum_{0\leq \beta_k\leq \alpha_k}{\phi\left(\prod_{k=1}^{m}{p_k^{\beta_k}}\right)}\\ &amp;= \sum_{0\leq \beta_k\leq \alpha_k}{\prod_{k=1}^{m}\phi({p_k^{\beta_k})}}\\ &amp;=\sum_{0\leq \beta_k\leq \alpha_k}{\prod_{k=1}^{m}{(p_k^{\beta_k}-p_k^{\beta_k-1}})}\\ &amp;=\prod_{k=1}^{m}{\sum_{0\leq \beta_k\leq \alpha_k}{(p_k^{\beta_k}-p_k^{\beta_k-1}}})\\ &amp;= \prod_{k=1}^{m}{p_k^{\alpha_k}}\\ &amp;=n. \end{align}$</span></p>
Tags group-theory,number-theory,alternative-proof,big-list
_number B.59
Formula_Id q_535
Latex \lim_{n\rightarrow \infty } a_{n}
Title Limiting value of a sequence when n tends to infinity
Question <p>Q) Let, <span class="math-container" id="q_533">$a_{n} \;=\; \left ( 1-\frac{1}{\sqrt{2}} \right ) ... \left ( 1- \frac{1}{\sqrt{n+1}} \right )$</span> , <span class="math-container" id="q_534">$n \geq 1$</span>. Then <span class="math-container" id="q_535">$\lim_{n\rightarrow \infty } a_{n}$</span></p>  <p>(A) equals <span class="math-container" id="q_536">$1$</span></p>  <p>(B) does not exist</p>  <p>(C) equals <span class="math-container" id="q_537">$\frac{1}{\sqrt{\pi }}$</span></p>  <p>(D) equals <span class="math-container" id="q_538">$0$</span>  </p>  <p><strong>My Approach</strong> :- I am not getting a particular direction or any procedure to simplify <span class="math-container" id="q_539">$a_{n}$</span> and find its value when n tends to infinity.<br> So, I tried like this simple way to substitute values and trying to find the limiting value :-<br> <span class="math-container" id="q_540">$\left ( 1-\frac{1}{\sqrt{1+1}} \right ) * \left ( 1-\frac{1}{\sqrt{2+1}} \right )*\left ( 1-\frac{1}{\sqrt{3+1}} \right )*\left ( 1-\frac{1}{\sqrt{4+1}} \right )*\left ( 1-\frac{1}{\sqrt{5+1}} \right )*\left ( 1-\frac{1}{\sqrt{6+1}} \right )*\left ( 1-\frac{1}{\sqrt{7+1}} \right )*\left ( 1-\frac{1}{\sqrt{8+1}} \right )*.........*\left ( 1-\frac{1}{\sqrt{n+1}}    \right )$</span>   </p>  <p>=<span class="math-container" id="q_541">$(0.293)*(0.423)*(0.5)*(0.553)*(0.622)*(0.647)*(0.667)* ....$</span> =0.009*...</p>  <p>So, here value is tending to zero. I think option <span class="math-container" id="q_542">$(D)$</span> is correct.<br> I have tried like this<br> <span class="math-container" id="q_543">$\left ( \frac{\sqrt{2}-1}{\sqrt{2}} \right )*\left ( \frac{\sqrt{3}-1}{\sqrt{3}} \right )*\left ( \frac{\sqrt{4}-1}{\sqrt{4}} \right )*.......\left ( \frac{\sqrt{(n+1)}-1}{\sqrt{n+1}} \right )$</span><br> = <span class="math-container" id="q_544">$\left ( \frac{(\sqrt{2}-1)*(\sqrt{3}-1)*(\sqrt{4}-1)*.......*(\sqrt{n+1}-1)}{{\sqrt{(n+1)!}}} \right )$</span><br> Now, again I stuck how to simplify further and find the value for which <span class="math-container" id="q_545">$a_{n}$</span> converges when <span class="math-container" id="q_546">$n$</span> tends to infinity . Please help if there is any procedure to solve this question. </p>
Tags calculus,sequences-and-series,limits,products
_number B.60
Formula_Id q_561
Latex |\mathbb{Q}| = |\mathbb{Z}|
Title Prove that the cardinality of the set of rational numbers and the set of integers is equal
Question <p>I just learned about cardinality in my discrete class a few days ago, and this is in the homework. This is all fairly confusing to me, and I'm not entirely sure where to even start. Here's the full question:</p>  <p>Let <span class="math-container" id="q_559">$\mathbb{Q}$</span> denote the set of rational numbers and <span class="math-container" id="q_560">$\mathbb{Z}$</span> denote the set of integers. Prove that <span class="math-container" id="q_561">$|\mathbb{Q}| = |\mathbb{Z}|$</span>.</p>  <p>I thought about saying that every element in <span class="math-container" id="q_562">$\mathbb{Q}$</span> can be written as some element in <span class="math-container" id="q_563">$\mathbb{Z} \times \mathbb{Z}$</span>, but I still don't know how to prove that that is a bijection, or even how to prove that <span class="math-container" id="q_564">$|\mathbb{Z} \times \mathbb{Z}| = |\mathbb{Z}|$</span>.</p>  <p>Any help would be greatly appreciated.</p>
Tags discrete-mathematics
_number B.62
Formula_Id q_569
Latex \text{lcm}(n_1,n_2)=\frac{n_1 n_2}{\gcd(n_1,n_2)}
Title <span class="math-container" id="q_565">$\gcd$</span> and <span class="math-container" id="q_566">$\text{lcm}$</span> of more than <span class="math-container" id="q_567">$2$</span> positive integers
Question <p>For any two positive integers <span class="math-container" id="q_568">${n_1,n_2}$</span>, the relationship between their greatest common divisor and their least common multiple is given by</p>  <p><span class="math-container" id="q_569">$$\text{lcm}(n_1,n_2)=\frac{n_1 n_2}{\gcd(n_1,n_2)}$$</span></p>  <p>If I have a set of <span class="math-container" id="q_570">$r$</span> positive integers <span class="math-container" id="q_571">${n_1,n_2,n_3,...,n_r}$</span>, does the same relationship hold? Is it true that</p>  <p><span class="math-container" id="q_572">$$\text{lcm}(n_1,n_2,n_3,...,n_r)=\frac{\prod_{i=1}^r n_i}{gcd(n_1,n2,n_3,...,n_r)}$$</span></p>  <p>I feel like this should be easy to prove, but I'm struggling to get a handle on it.</p>
Tags proof-explanation,greatest-common-divisor,least-common-multiple
_number B.63
Formula_Id q_574
Latex f([a, b]) \subset [a, b]
Title Suppose that f : [a, b] ? R is continuous and that f([a, b]) ? [a, b]. Prove that there exists a point c ? [a, b] satisfying f(c) = c.
Question <p>Suppose that <span class="math-container" id="q_573">$f : [a, b] \to \mathbb{R}$</span> is continuous and that <span class="math-container" id="q_574">$f([a, b]) \subset [a, b]$</span>. Prove that there exists a point <span class="math-container" id="q_575">$c \in [a, b]$</span> satisfying <span class="math-container" id="q_576">$f(c) = c$</span>. </p>  <p>(If either <span class="math-container" id="q_577">$f(a) = a$</span> or <span class="math-container" id="q_578">$f(b) = b$</span> there is nothing left to show, so you might as well assume that <span class="math-container" id="q_579">$f(a) = a$</span> and <span class="math-container" id="q_580">$f(b) = b$</span>. Since <span class="math-container" id="q_581">$f$</span> takes its values in <span class="math-container" id="q_582">$[a, b]$</span> this is the same as assuming that <span class="math-container" id="q_583">$f(a) &gt; a$</span> and <span class="math-container" id="q_584">$f(b) &lt; b$</span>.)</p>  <p>So far, I have:</p>  <p>Pf. Assume <span class="math-container" id="q_585">$f(a)&gt;a$</span> and <span class="math-container" id="q_586">$f(b)&lt; b$</span>.  Let <span class="math-container" id="q_587">$x, y \in [a,b]$</span> such that <span class="math-container" id="q_588">$f(a)=x$</span> and <span class="math-container" id="q_589">$f(b)=y$</span> which means <span class="math-container" id="q_590">$f[a,b]=[x,y]$</span>. Notice <span class="math-container" id="q_591">$[x,y]\subset [a, b]$</span>. Since f is continuous on <span class="math-container" id="q_592">$[x,y]$</span>, there exists some <span class="math-container" id="q_593">$c \in [a,b]$</span> such that <span class="math-container" id="q_594">$x$</span> is less than or equal to <span class="math-container" id="q_595">$c$</span> is less than or equal to <span class="math-container" id="q_596">$y$</span>...</p>  <p>This is where I am stuck because I don't think I can just assume by Intermediate Value Theorem that some <span class="math-container" id="q_597">$f(c)=c$</span>?</p>
Tags real-analysis,continuity,proof-explanation
_number B.64
Formula_Id q_604
Latex t\lambda\le e^{t\lambda-1}\tag2.
Title How can we show that <span class="math-container" id="q_598">$e^{-2\lambda t}\lambda^2\le\frac1{e^2t^2}$</span> for all <span class="math-container" id="q_599">$\lambda,t\ge0$</span>?
Question <p>How can we show that <span class="math-container" id="q_600">$$e^{-2\lambda t}\lambda^2\le\frac1{e^2t^2}\tag1$$</span> for all <span class="math-container" id="q_601">$\lambda,t\ge0$</span>?</p>  <p>Applying <span class="math-container" id="q_602">$\ln$</span> to both sides yields that <span class="math-container" id="q_603">$(1)$</span> should be equivalent to <span class="math-container" id="q_604">$$t\lambda\le e^{t\lambda-1}\tag2.$$</span> So, if I did no mistake, it should suffice to show <span class="math-container" id="q_605">$x\le e^{x-1}$</span> for all <span class="math-container" id="q_606">$x\ge0$</span>. How can we do this?</p>
Tags calculus,inequality,exponential-function
_number B.65
Formula_Id q_612
Latex (x^TAh)^T = h^TA^Tx
Title if <span class="math-container" id="q_607">$x,h \in \mathbb{R}^d$</span> and <span class="math-container" id="q_608">$A \in \mathbb{R}^{d\times d}$</span> is it possible to justify that <span class="math-container" id="q_609">$(x^TAh)^T = h^TA^Tx$</span>?
Question <p>if <span class="math-container" id="q_610">$x,h \in \mathbb{R}^d$</span> and <span class="math-container" id="q_611">$A \in \mathbb{R}^{d\times d}$</span></p>  <p>is it possible to justify that <span class="math-container" id="q_612">$(x^TAh)^T = h^TA^Tx$</span>?</p>
Tags linear-algebra,transpose
_number B.66
Formula_Id q_616
Latex \det{\begin{bmatrix}A&amp;B\\O&amp;C\end{bmatrix}}=\det(A)\det(C)
Title Combination of matrixes
Question <p>If A is a <span class="math-container" id="q_613">$k\times k$</span> matrix,B is a <span class="math-container" id="q_614">$k\times l$</span> matrix and C is a <span class="math-container" id="q_615">$l\times l$</span> matrix prove that:</p>  <p><span class="math-container" id="q_616">$\det{\begin{bmatrix}A&amp;B\\O&amp;C\end{bmatrix}}=\det(A)\det(C)$</span></p>  <p>O is the matrix that all it's elements are equal to zero.</p>  <p>I know some rules for calculating determinants but I don't know how to begin in this question.</p>
Tags calculus,determinant
_number B.67
Formula_Id q_620
Latex a^n+1
Title Prove <span class="math-container" id="q_617">$a^n+1$</span> is divisible by <span class="math-container" id="q_618">$a + 1$</span> if <span class="math-container" id="q_619">$n$</span> is odd
Question <p>Prove <span class="math-container" id="q_620">$a^n+1$</span> is divisible by <span class="math-container" id="q_621">$a + 1$</span> if <span class="math-container" id="q_622">$n$</span> is odd:</p>  <p>We know <span class="math-container" id="q_623">$a$</span> cannot be <span class="math-container" id="q_624">$-1$</span> and the <span class="math-container" id="q_625">$n \in \mathbb{N}$</span>. Since <span class="math-container" id="q_626">$n$</span> must be odd, we can rewrite <span class="math-container" id="q_627">$n$</span> as <span class="math-container" id="q_628">$2k+1$</span>. Now we assume it holds for prove that it holds for the next term.</p>  <p><span class="math-container" id="q_629">$$a^{2(k+1)+1}+1$$</span> <span class="math-container" id="q_630">$$=a^{2k+3}+1$$</span> <span class="math-container" id="q_631">$$=a^3\cdot a^{2k}+1$$</span> <span class="math-container" id="q_632">$$=(a^3+1)\cdot a^{2k} -a^{2k}+1$$</span></p>  <p>Im not sure on what to do next. Since <span class="math-container" id="q_633">$a^{2k}$</span> means that the exponential term will be even and thus you cant use the fact that <span class="math-container" id="q_634">$a^n+1$</span> is divisible by <span class="math-container" id="q_635">$a + 1$</span> if <span class="math-container" id="q_636">$n$</span> is odd.    </p>
Tags polynomials,induction,divisibility
_number B.68
Formula_Id q_637
Latex \ {s \choose s} + {s+1 \choose s} +...+ {n \choose s} = {n+1 \choose s+1}
Title Induction with two variable parameters
Question <p>So I was assigned this homework problem: <span class="math-container" id="q_637">$$\ {s \choose s} + {s+1 \choose s} +...+ {n \choose s} = {n+1 \choose s+1}$$</span> for all s and all <span class="math-container" id="q_638">$n \geq s$</span> I've tried to email both my professor and my TA and their explanations seem contradictory. My professor responded saying the statement I need to prove is "The formula is correct for <span class="math-container" id="q_639">$0 \leq s \leq n$</span>." Whereas my TA told me I need to use induction on both variables and I'm not sure how to do that. Any help is appreciated!</p>
Tags combinatorics
_number B.69
Formula_Id q_643
Latex \begin{equation} \sum_{j=0}^{N-1}\cos\left(l\frac{\left(2j+1\right)\pi}{2N} \right)=0 \end{equation}
Title Proving <span class="math-container" id="q_640">$\sum_{j=0}^{N-1}\cos\frac{\left(2j+1\right)\pi}{2N}=0$</span>
Question <p>Let <span class="math-container" id="q_641">$l\in\mathbb{Z}$</span> and <span class="math-container" id="q_642">$N\in\mathbb{N}$</span>. I need to prove the following: <span class="math-container" id="q_643">$\begin{equation} \sum_{j=0}^{N-1}\cos\left(l\frac{\left(2j+1\right)\pi}{2N} \right)=0 \end{equation}$</span> I tried to use Euler formula and then sum the first <span class="math-container" id="q_644">$N$</span> terms of the geometric serie I get, but it didn't work. Any ideas?</p>
Tags trigonometry,summation
_number B.70
Formula_Id q_646
Latex 1^2 + 2^2 + .... + n^2 = \frac{n(n+1)(2n+1)}{6}
Title Show, with induction that <span class="math-container" id="q_645">$1^2 + 2^2 + .... + n^2 = \frac{n(n+1)(2n+1)}{6}$</span>
Question <p>Show, with induction that</p>  <p><span class="math-container" id="q_646">$1^2 + 2^2 + .... + n^2 = \frac{n(n+1)(2n+1)}{6}$</span></p>  <p>My attempt</p>  <p><strong>Case 1:</strong> n = 1</p>  <p><span class="math-container" id="q_647">$LHS = 1^2$</span> </p>  <p><span class="math-container" id="q_648">$RHS = \frac{(1+1)(2+1)}{6} = \frac{2*3}{6} = 1$</span></p>  <p><strong>Case 2:</strong> n = p</p>  <p><span class="math-container" id="q_649">$LHS_{p} = 1^2 + 2^2 + ... + p^2$</span></p>  <p><span class="math-container" id="q_650">$RHS_{p} = \frac{p(p+1)(2p+1)}{6}$</span></p>  <p><strong>Case 3:</strong> n = p + 1</p>  <p><span class="math-container" id="q_651">$LHS_{p+1} = 1^2+2^2+....+p^2+(p+1)^2$</span></p>  <p><span class="math-container" id="q_652">$RHS_{p+1} = \frac{(p+1)((p+1)+1)(2(p+1)+1)}{6}$</span></p>  <p>Now to show this with induction I think i need to show that</p>  <p><span class="math-container" id="q_653">$RHS_{p+1} = RHS_{p} + (p+1)^2$</span></p>  <p><span class="math-container" id="q_654">$RHS_{p+1} = \frac{p(p+1)(2p+1)}{6} + (p+1)^2$</span></p>  <p>So I need to rewrite </p>  <p><span class="math-container" id="q_655">$RHS_{p+1} = \frac{(p+1)((p+1)+1)(2(p+1)+1)}{6} $</span>to be equal to <span class="math-container" id="q_656">$\frac{p(p+1)(2p+1)}{6} + (p+1)^2$</span>  Anyone see how I can do that? Or got any other solution?</p>
Tags induction
_number B.71
Formula_Id q_663
Latex \binom{n}{0}^2 + \binom{n}{1}^2 + ... + \binom{n}{n}^2 = \binom{2n}{n}
Title Help on proof of <span class="math-container" id="q_661">$\binom{n}{0}^2 + \binom{n}{1}^2 + ... + \binom{n}{n}^2 = \binom{2n}{n}$</span>
Question <p>The proof is required to be made through the binomial theorem. I will expose the demonstration I was tought, and forward my questions after exposing it. You'll see question marks like this one <strong>(?-n)</strong> on points I don't quite understand, where <span class="math-container" id="q_662">$n$</span> is the numeration of the mark. This are the doubts I have about the demonstration, the which I hope someone can clarify.</p>  <p>Prove that <span class="math-container" id="q_663">$\binom{n}{0}^2 + \binom{n}{1}^2 + ... + \binom{n}{n}^2 = \binom{2n}{n}$</span>.</p>  <p>We will use the following equality, and call it <span class="math-container" id="q_664">$P$</span>: </p>  <p><span class="math-container" id="q_665">$(1+x)^n(1+x)^n=(1+x)^{2n}$</span>    <strong>(?-1)</strong></p>  <p>The result will be proved finding the <span class="math-container" id="q_666">$x^n$</span> coefficient of both terms of this equality <strong>(?-2)</strong>.</p>  <p>According to the binomial theorem, the left-hand side of this equation is the product of two factors, both equal to</p>  <p><span class="math-container" id="q_667">$\binom{n}{0}1+\binom{n}{1}x+...+\binom{n}{r}x^r+...+\binom{n}{n}x^n$</span></p>  <p>When both factors multiply, a term on <span class="math-container" id="q_668">$x^n$</span> is obtained when a term of the first factor has some <span class="math-container" id="q_669">$x^i$</span> and the term of the second factor has some <span class="math-container" id="q_670">$x^{n-i}$</span>. Therefor the coefficients of <span class="math-container" id="q_671">$x^n$</span> are</p>  <p><span class="math-container" id="q_672">$\binom{n}{0}\binom{n}{n}+\binom{n}{1}\binom{n}{n-1}+\binom{n}{2}\binom{n}{n-2}+...\binom{n}{n}\binom{n}{0}$</span> .</p>  <p>Since <span class="math-container" id="q_673">$\binom{n}{n-r}=\binom{n}{r}$</span>, the previous summation is equal to <span class="math-container" id="q_674">$\binom{n}{0}^2 + \binom{n}{1}^2 + ... + \binom{n}{n}^2$</span>. So the left hand side of the equation we are asked to proove is a coefficient of <span class="math-container" id="q_675">$x^n$</span>. When we expand the right-hand side of the equation <span class="math-container" id="q_676">$P$</span>, we find that <span class="math-container" id="q_677">$\binom{2n}{n}$</span> is a coefficient of <span class="math-container" id="q_678">$x^n$</span>. Therefore <strong>(?-3)</strong> the left-hand side of the equation we were asked to prove is in deed equal to <span class="math-container" id="q_679">$\binom{2n}{n}$</span>. In conclussion,</p>  <p><span class="math-container" id="q_680">$\binom{n}{0}^2 + \binom{n}{1}^2 + ... + \binom{n}{n}^2 = \binom{2n}{n}$</span>.</p>  <p>This was all the demonstration. My doubt one <strong>(?-1)</strong> goes about where the heck does this equation come from? How would I know what equation to come up with if requested to prove a different equality?</p>  <p>Doubt two <strong>(?-2)</strong> goes about why would the solution of the first equation would have anything to do with finding the <span class="math-container" id="q_681">$x^n$</span> coefficients of the one I just made up (see doubt one).</p>  <p>Doubt three <strong>(?-3)</strong> goes about why demonstrating that <span class="math-container" id="q_682">$a$</span> is a coefficient of <span class="math-container" id="q_683">$x^n$</span> on the left hand side of the equation I made up, and that <span class="math-container" id="q_684">$b$</span> is a coefficient of <span class="math-container" id="q_685">$x^n$</span> on the right-hand side of this equation as well, would prove my original equation, the one I was supposed to prove on the first place?</p>  <p>I know there are many doubts here, I hope you guys can help me. Sorry for the long post, it's a long demonstration.</p>
Tags discrete-mathematics,binomial-coefficients,binomial-theorem
_number B.73
Formula_Id q_690
Latex f(x)=x+\dfrac{1}{x}
Title Show that the image of the function <span class="math-container" id="q_686">$f:(0,\infty)\rightarrow \mathbb{R}$</span>, <span class="math-container" id="q_687">$f(x)=x+\dfrac{1}{x}$</span> is the interval <span class="math-container" id="q_688">$[2,\infty)$</span>.
Question <ul> <li>Show that the image of the function <span class="math-container" id="q_689">$f:(0,\infty)\rightarrow \mathbb{R}$</span>, <span class="math-container" id="q_690">$f(x)=x+\dfrac{1}{x}$</span> is the interval <span class="math-container" id="q_691">$[2,\infty)$</span>.</li> </ul>  <p>If <span class="math-container" id="q_692">$x=1$</span>, then <span class="math-container" id="q_693">$f(1)=2$</span>. So how can I show that the mage of the function is the interval <span class="math-container" id="q_694">$[2,\infty)$</span>?</p>
Tags functions,elementary-set-theory
_number B.74
Formula_Id q_698
Latex \lim_{u\to \infty} \frac{u^m}{e^u} = 0
Title Prove that for each integer <span class="math-container" id="q_695">$m$</span>, <span class="math-container" id="q_696">$ \lim_{u\to \infty} \frac{u^m}{e^u} = 0 $</span>
Question <p>I'm unsure how to show  that for each integer <span class="math-container" id="q_697">$m$</span>, <span class="math-container" id="q_698">$ \lim_{u\to \infty} \frac{u^m}{e^u} = 0 $</span>.   Looking at the solutions it starts with <span class="math-container" id="q_699">$e^u$</span> <span class="math-container" id="q_700">$&gt;$</span> <span class="math-container" id="q_701">$\frac{u^{m+1}}{(m+1)!}$</span> but not sure how this is a logical step.</p>
Tags real-analysis,calculus,limits
_number B.75
Formula_Id q_707
Latex \bigcup_{i \in \mathbb{N}}(a_i+b_i\mathbb{Z})=\mathbb{Z}
Title Covering <span class="math-container" id="q_702">$\mathbb{Z}$</span> by arithmetic progressions
Question <p>I am solving problems from an old exam (in topology, but I've translated the problem into more algebraic terms). The problem is the following:</p>  <blockquote>   <p>Let <span class="math-container" id="q_703">$a+b\mathbb{Z}=\{z\in \mathbb{Z}\mid z = a+bk \text{ for some  }k\in \mathbb{Z}\}$</span> where <span class="math-container" id="q_704">$a\in \mathbb{Z}$</span> and <span class="math-container" id="q_705">$b\in  \mathbb{Z}-\{0\}$</span>. Suppose we have a collection of such sets   <span class="math-container" id="q_706">$\{a_i+b_i\mathbb{Z}\mid i \in \mathbb{N}\}$</span>   satisfying:</p>      <p><span class="math-container" id="q_707">$$\bigcup_{i \in \mathbb{N}}(a_i+b_i\mathbb{Z})=\mathbb{Z}$$</span></p>      <p>Show whether it is always possible to extract a finite   <span class="math-container" id="q_708">$I\subset \mathbb{N}$</span> s.t.</p>      <p><span class="math-container" id="q_709">$$\bigcup_{i \in I}(a_i+b_i\mathbb{Z})=\mathbb{Z}$$</span></p> </blockquote>  <p>Unfortunately, I seem to have forgotten a lot of my elementary algebra... Nevertheless, I have attempted something:</p>  <p>Let <span class="math-container" id="q_710">$\{p_k\}=\{2,3,5,\dots\}$</span> be the set of primes. We can construct: <span class="math-container" id="q_711">$$\left(\bigcup_{k\in \mathbb{N}}(0+p_k\mathbb{Z})\right)\cup (-1+\ell_1 \mathbb{Z})\cup (1+\ell_2\mathbb{Z})=\mathbb{Z}$$</span></p>  <p>for some appropriate non-negative integers <span class="math-container" id="q_712">$\ell_1,\ell_2$</span>. We could for instance pick <span class="math-container" id="q_713">$\ell_1=\ell_2=5$</span>. Suppose there is a finite sub-collection <span class="math-container" id="q_714">$\{0+p_{k_j}\}$</span>, <span class="math-container" id="q_715">$j=1,\dots,n$</span> s.t. </p>  <p><span class="math-container" id="q_716">$$\left(\bigcup_{1\leq j\leq n}(0+p_{k_j}\mathbb{Z})\right)\cup (-1+5 \mathbb{Z})\cup (1+5\mathbb{Z})$$</span></p>  <p>Now, assume <span class="math-container" id="q_717">$p$</span> is some prime s.t. <span class="math-container" id="q_718">$p&gt;\max\{p_{k_1},\dots,p_{k_n}\}$</span>, then clearly <span class="math-container" id="q_719">$p\notin \bigcup_{1\leq j\leq n}(0+p_{k_j}\mathbb{Z})$</span>. But here I run into a problem. I want <span class="math-container" id="q_720">$p\notin(-1+5 \mathbb{Z})\cup (1+5\mathbb{Z})$</span>. That is, I want <span class="math-container" id="q_721">$5\nmid p-1$</span> and <span class="math-container" id="q_722">$5\nmid p+1$</span>. This is of course possible if <span class="math-container" id="q_723">$p$</span> is a prime with a <span class="math-container" id="q_724">$7$</span> as its last digit. However, this approach means I have to prove that there are infinitely many primes ending on a <span class="math-container" id="q_725">$7$</span>, which seems like a silly thing to prove for a simple problem like this. Surely, there is a nicer way of solving this?</p>  <p><strong>EDIT</strong>: I am particularily interested in a solution not relying on topology, and whether a solution like my attempted solution works.</p>
Tags general-topology,elementary-number-theory
_number B.76
Formula_Id q_727
Latex (- 1) (- 1) = 1
Title Show that the relation <span class="math-container" id="q_726">$(- 1) (- 1) = 1$</span> is a consequence of the distributive law
Question <blockquote>   <p>Show that the relation <span class="math-container" id="q_727">$(- 1) (- 1) =  1$</span> is a consequence of the distributive law.</p> </blockquote>  <p>This question is the first problem from 'Number Theory for Beginners" by Andre Weil. I cannot get the point from where to begin. I tried using <span class="math-container" id="q_728">$1\cdot 1 = 1$</span> and <span class="math-container" id="q_729">$ 1\cdot x = x $</span>, but couldn't get somewhere. Can you help me just with a hint? I would be willing to work up from there.</p>
Tags elementary-number-theory
_number B.77
Formula_Id q_739
Latex \displaystyle \left \vert{ \frac {e^{-ixu}-1}{u}}\right\vert \le \vert x \vert
Title Inequality with complex exponential
Question <p>Rudin in Real and Complex Analysis uses this in a proof near the beginning of chapter 9:</p>  <blockquote>   <p><span class="math-container" id="q_739">$\displaystyle \left \vert{ \frac {e^{-ixu}-1}{u}}\right\vert \le \vert x \vert$</span> for all real <span class="math-container" id="q_740">$u \ne 0$</span></p> </blockquote>  <p>Why is this true?</p>  <p>Edit: I believe <span class="math-container" id="q_741">$x$</span> is real</p>
Tags inequality,exponential-function,fourier-transform
_number B.79
Formula_Id q_743
Latex \emptyset, \{1\}, \{2\}, \{1, 2\}, \{3\}, \{1, 3\}, \{2, 3\}, \{1, 2, 3\}, \{4\}, \ldots
Title Why does this proof that the set of all finite subsets of N is a countable set not work for the set of all subsets of N?
Question <p>I found this proof in a StackExchange thread and found it pretty understandable and simple:</p>  <p>"The other answers give some sort of formula, like you were trying to do.  But, the simplest way to see that the set of all finite subsets of <span class="math-container" id="q_742">$\mathbb{N}$</span> is countable is probably the following.</p>  <p>If you can list out the elements of a set, with one coming first, then the next, and so on, then that shows the set is countable.  There is an easy pattern to see here.  Just start out with the least elements.</p>  <p><span class="math-container" id="q_743">$$\emptyset, \{1\}, \{2\}, \{1, 2\}, \{3\}, \{1, 3\}, \{2, 3\}, \{1, 2, 3\}, \{4\}, \ldots$$</span></p>  <p>In other words, first comes <span class="math-container" id="q_744">$\{1\}$</span>, then comes <span class="math-container" id="q_745">$\{2\}$</span>.  Each time you introduce a new integer, <span class="math-container" id="q_746">$n$</span>, list all the subsets of <span class="math-container" id="q_747">$[n] = \{1, 2, \ldots, n\}$</span> that contain <span class="math-container" id="q_748">$n$</span> (the ones that don't contain <span class="math-container" id="q_749">$n$</span> have already showed up).  Therefore, all subsets of <span class="math-container" id="q_750">$[n]$</span> show up in the first <span class="math-container" id="q_751">$2^{n}$</span> elements of this sequence."</p>  <p>I understand how it applies for finite subsets of N, but I cant really pinpoint of why it would not apply to a set of all subsets of N. We could continue this scheme for ever, couldnt we?  I assume that I think in a wrong way about infinity but I am not quite sure. Any help is greatly appreciated!</p>
Tags analysis,elementary-set-theory,proof-explanation
_number B.80
Formula_Id q_754
Latex \{1, \cdots, n\}
Title Any infinite set contains a countable subset. Why is my proof wrong? (Axiom of Choice)
Question <p>Let <span class="math-container" id="q_752">$M$</span> be an infinite set.  </p>  <blockquote>   <p>Proposition 1:<br>   For any <span class="math-container" id="q_753">$n \in \mathbb{N}$</span>, there exists an injection from <span class="math-container" id="q_754">$\{1, \cdots, n\}$</span> to <span class="math-container" id="q_755">$M$</span>.    </p> </blockquote>  <p>(1)<br> Since <span class="math-container" id="q_756">$M \neq \emptyset$</span>, there exists <span class="math-container" id="q_757">$x \in M$</span>.<br> Define <span class="math-container" id="q_758">$f(1)$</span> as <span class="math-container" id="q_759">$f(1) := x$</span>.<br> <span class="math-container" id="q_760">$f$</span> is an injection from <span class="math-container" id="q_761">$\{1\}$</span> to <span class="math-container" id="q_762">$M$</span>.<br> (2)<br> Suppose that there exists an injection <span class="math-container" id="q_763">$f$</span> from <span class="math-container" id="q_764">$\{1, \cdots, n\}$</span> to <span class="math-container" id="q_765">$M$</span>.<br> Since <span class="math-container" id="q_766">$M$</span> is an infinite set, <span class="math-container" id="q_767">$M - \{f(1), \cdots, f(n)\}$</span> is not an empty set.<br> So, there exists <span class="math-container" id="q_768">$x \in M - \{f(1), \cdots, f(n)\}$</span>.<br> Define <span class="math-container" id="q_769">$g(1), \cdots, g(n+1)$</span> as <span class="math-container" id="q_770">$g(1) := f(1), \cdots, g(n):=f(n)$</span> and <span class="math-container" id="q_771">$g(n+1) := x$</span>.<br> Obviously, <span class="math-container" id="q_772">$g$</span> is an injection from <span class="math-container" id="q_773">$\{1, \cdots, n+1\}$</span> to <span class="math-container" id="q_774">$M$</span>.  </p>  <p>Let <span class="math-container" id="q_775">$n_1$</span> be an arbitrary natural number.<br> If I wanna calculate <span class="math-container" id="q_776">$h(n_1)$</span>, then I get an injection <span class="math-container" id="q_777">$g$</span> from <span class="math-container" id="q_778">$\{1, \cdots, n_1\}$</span> to <span class="math-container" id="q_779">$M$</span> by Proposition 1.<br> And I return <span class="math-container" id="q_780">$g(n_1)$</span> as the value of <span class="math-container" id="q_781">$h(n_1)$</span>.<br> And I store the pairs <span class="math-container" id="q_782">$(1, g(1)), \cdots, (n_1, g(n_1))$</span> to my database.  </p>  <p>If I wanna calculate <span class="math-container" id="q_783">$h(n_2)$</span> for <span class="math-container" id="q_784">$n_2 \leq n_1$</span>, then I search my database and I get the value <span class="math-container" id="q_785">$g(n_2)$</span> from my database and I return <span class="math-container" id="q_786">$g(n_2)$</span> as the value of <span class="math-container" id="q_787">$h(n_2)$</span>.</p>  <p>If I wanna calculate <span class="math-container" id="q_788">$h(n_3)$</span> for <span class="math-container" id="q_789">$n_3 &gt; n_1$</span>, then I add the pairs <span class="math-container" id="q_790">$(n_1 + 1, g(n_1+1)), \cdots, (n_3, g(n_3))$</span> to my database by Proposition 1 and I return <span class="math-container" id="q_791">$g(n_3)$</span> as the value of <span class="math-container" id="q_792">$h(n_3)$</span>.  </p>  <p>I can calculate <span class="math-container" id="q_793">$h(n)$</span> for any <span class="math-container" id="q_794">$n \in \mathbb{N}$</span>.  </p>  <p>From above, we get an injection <span class="math-container" id="q_795">$h : \mathbb{N} \to M$</span>.  </p>  <p>Why is my proof wrong?</p>  <p>By the way.<br> Suppose that a man wanna know if I have an injection <span class="math-container" id="q_796">$h : \mathbb{N} \to M$</span> or not.<br> Then how can the man know if I have  an injection <span class="math-container" id="q_797">$h : \mathbb{N} \to M$</span> or not?</p>
Tags elementary-set-theory,axiom-of-choice
_number B.81
Formula_Id q_807
Latex A = \displaystyle\int_{0}^{2\pi}{g(x)\cdot \cos(x)\space\mathrm{d}x}
Title All definite integrals evaluate to 0 using periodic functions.
Question <p>I know that my reasoning is incorrect, I just don't know where I went wrong. I did discuss this with my Maths teacher, and even she could not find what I did wrong.</p>  <p>Let us begin by assuming a function, <span class="math-container" id="q_798">$f(x)$</span> that is continuous and has an antiderivative in the interval <span class="math-container" id="q_799">$[0, 2\pi]$</span>. Let <span class="math-container" id="q_800">$A$</span> be the area under the curve for <span class="math-container" id="q_801">$f(x)$</span> in the interval <span class="math-container" id="q_802">$[0, 2\pi]$</span></p>  <p><span class="math-container" id="q_803">$A = \displaystyle \int_{0}^{2\pi}{f(x)\space\mathrm{d}x}$</span></p>  <p>Now there must exist a function, <span class="math-container" id="q_804">$g(x)$</span> such that:  </p>  <p><span class="math-container" id="q_805">$f(x) = g(x)\cdot \cos(x)$</span></p>  <p>Substituting the value of <span class="math-container" id="q_806">$f(x)$</span>:</p>  <p><span class="math-container" id="q_807">$A = \displaystyle\int_{0}^{2\pi}{g(x)\cdot \cos(x)\space\mathrm{d}x}$</span></p>  <p>Using t substitution:<br> Let <span class="math-container" id="q_808">$t = \sin(x)$</span><br> Then: <span class="math-container" id="q_809">$\mathrm{d}t = \cos(x)\space\mathrm{d}x$</span><br> And:  <span class="math-container" id="q_810">$x = \arcsin(t)$</span></p>  <p>Changing the limits:<br> <span class="math-container" id="q_811">$t = \sin(x)$</span><br> <span class="math-container" id="q_812">$0$</span> becomes <span class="math-container" id="q_813">$\sin(0) = 0$</span><br> <span class="math-container" id="q_814">$2\pi$</span> becomes <span class="math-container" id="q_815">$\sin(2\pi) = 0$</span></p>  <p>Substituting in the definite integral:</p>  <p><span class="math-container" id="q_816">$A = \displaystyle \int_{0}^{0}{g(\arcsin(t))\space\mathrm{d}t}$</span></p>  <p>But Definite Integral where the lower and upper bounds are the same is <span class="math-container" id="q_817">$0$</span>.<br> So:</p>  <p><span class="math-container" id="q_818">$A = 0$</span>, which is not possible.</p>  <p>Thanks for the help.</p>
Tags calculus,integration,trigonometry,definite-integrals,inverse-function
_number B.82
Formula_Id q_819
Latex 1, 1+\frac{1}{2}, 1+\frac{1}{2}+\frac{1}{3}, 1+\frac{1}{2}+\frac{1}{3}+\frac{1}{4}, . . .
Title Is the sequence of sums of inverse of natural numbers bounded?
Question <p>I'm reading through Spivak Ch.22 (Infinite Sequences) right now. He mentioned in the written portion that it's often not a trivial matter to determine the boundedness of sequences. With that in mind, he gave us a sequence to chew on before we learn more about boundedness. That sequence is:</p>  <p><span class="math-container" id="q_819">$$1, 1+\frac{1}{2}, 1+\frac{1}{2}+\frac{1}{3}, 1+\frac{1}{2}+\frac{1}{3}+\frac{1}{4}, . . .$$</span></p>  <p>I know that a sequence is bounded above if there is a number <span class="math-container" id="q_820">$M$</span> such that <span class="math-container" id="q_821">$a_n\leq M$</span> for all <span class="math-container" id="q_822">$n$</span>. Any hints here?</p>
Tags calculus,sequences-and-series,harmonic-numbers
_number B.83
Formula_Id q_825
Latex I=&lt;p,x&gt;
Title Is the ideal generated by <span class="math-container" id="q_823">${4,x}$</span> a principal ideal in <span class="math-container" id="q_824">$Z[x]$</span>?
Question <p>I've : <span class="math-container" id="q_825">$I=&lt;p,x&gt;$</span> is not a principal ideal in <span class="math-container" id="q_826">$Z[x]$</span> where p is prime. My question :   Is <span class="math-container" id="q_827">$I=&lt;p,x&gt;$</span> a principal ideal in <span class="math-container" id="q_828">$Z[x]$</span> where p is not a prime? More particularly, is the ideal generated by <span class="math-container" id="q_829">${4,x}$</span> a principal ideal in <span class="math-container" id="q_830">$Z[x]$</span> ?</p>
Tags abstract-algebra,ring-theory,ideals,principal-ideal-domains
_number B.84
Formula_Id q_840
Latex p_n = \frac{1}{2}p_{n-1}
Title Expected number of steps for a bug to reach position <span class="math-container" id="q_831">$N$</span>
Question <p>A bug starts at time <span class="math-container" id="q_832">$0$</span> at position <span class="math-container" id="q_833">$0$</span>. At each step, the bug either moves to the right by <span class="math-container" id="q_834">$1$</span> step <span class="math-container" id="q_835">$(+1)$</span> with probability <span class="math-container" id="q_836">$1/2$</span>, or returns to the origin with probability <span class="math-container" id="q_837">$1/2$</span>. What is the expected number of steps for this bug to reach position <span class="math-container" id="q_838">$N$</span>?</p>  <p>I tried to first find the possibility that this bug reaches <span class="math-container" id="q_839">$N$</span> as the number of steps goes to infinity. The recurrence equation I find is <span class="math-container" id="q_840">$$p_n = \frac{1}{2}p_{n-1}$$</span>, where <span class="math-container" id="q_841">$p_n$</span> is the possibility for the bug starting at position <span class="math-container" id="q_842">$n$</span> to reach <span class="math-container" id="q_843">$N$</span>. We also have the boundary condition <span class="math-container" id="q_844">$p_N = 1$</span>. Then we see that <span class="math-container" id="q_845">$p_{N-1}=2$</span>, and that <span class="math-container" id="q_846">$p_0 = 2^N$</span>, which doesn't make sense at all because it is greater than <span class="math-container" id="q_847">$1$</span>. I think I should sort out the value of probability first, and think about the number of expected steps later.</p>  <p>I'm sure there is something wrong with the recurrence equation, but what's wrong about it?</p>
Tags markov-chains,random-walk
_number B.85
Formula_Id q_849
Latex \sum_{k=0}^{n}k\cdot \left(\begin{array}{l}{n}\\{k}\end{array}\right)=O\left( 2 ^ {n\log _{3}n}\right)?
Title Is it true that <span class="math-container" id="q_848">$\sum_{k=0}^{n}k\cdot \left(\begin{array}{l}{n}\\{k}\end{array}\right)=O\left(2 ^ {n\log _{3}n}\right)?$</span>
Question <p><strong>Problem</strong>: Is it true that <span class="math-container" id="q_849">$\sum_{k=0}^{n}k\cdot \left(\begin{array}{l}{n}\\{k}\end{array}\right)=O\left( 2 ^ {n\log _{3}n}\right)?$</span></p>  <p><strong>My start of solution</strong>: <span class="math-container" id="q_850">$$\sum_{k=0}^{n}k\cdot \left(\begin{array}{l}{n}\\{k}\end{array}\right)\leq \sum_{k=0}^{n}k\cdot \left(\begin{array}{l}{n}\\{\lfloor \frac{n}{2}\rfloor}\end{array}\right)\leq \frac{n\cdot(n+1)}{2}\cdot \left(\begin{array}{l}{n}\\{\lfloor \frac{n}{2}\rfloor}\end{array}\right)\leq n(n+1)! \leq nn^n \leq n^{n+1}$$</span></p>  <p>I think this upper bound is way too large and I can't seem to find a solution.</p>
Tags combinatorics,elementary-number-theory,discrete-mathematics
_number B.86
Formula_Id q_854
Latex \forall n \in \Bbb{N} : \big(\sum_{i=1}^{n}a_{i}\big) \big(\sum_{i=1}^{n} \frac{1}{a_{i}}\big) \ge n^2
Title Is it true that <span class="math-container" id="q_851">$\forall n \in \Bbb{N} : (\sum_{i=1}^{n} a_{i} ) (\sum_{i=1}^{n} \frac{1}{a_{i}} ) \ge n^2$</span> , if all <span class="math-container" id="q_852">$a_{i}$</span> are positive?
Question <blockquote>   <p><strong>If  <span class="math-container" id="q_853">$\forall i \in \Bbb{N}: a_{i} \in \Bbb{R}^+$</span> , is it true that <span class="math-container" id="q_854">$\forall n \in \Bbb{N} : \big(\sum_{i=1}^{n}a_{i}\big) \big(\sum_{i=1}^{n}  \frac{1}{a_{i}}\big) \ge n^2$</span> ?</strong></p> </blockquote>  <p>I have been able to prove that this holds for <span class="math-container" id="q_855">$n=1$</span> , <span class="math-container" id="q_856">$n=2$</span>, and <span class="math-container" id="q_857">$n=3$</span> using the following lemma:</p>  <blockquote>   <p><strong>Lemma 1:</strong>  Let <span class="math-container" id="q_858">$a,b \in \Bbb{R}^+$</span>. If <span class="math-container" id="q_859">$ab =1$</span> then <span class="math-container" id="q_860">$a+b \ge 2$</span></p> </blockquote>  <p>For example, the case for <span class="math-container" id="q_861">$n=3$</span> can be proven like this:</p>  <p>Let <span class="math-container" id="q_862">$a,b,c \in \Bbb{R}^+$</span>. Then we have:</p>  <p><span class="math-container" id="q_863">$(a+b+c)\big(\frac{1}{a} + \frac{1}{b} + \frac{1}{c}\big) = 1 + \frac{a}{b} + \frac{a}{c} + \frac{b}{a} + 1 + \frac{b}{c} + \frac{c}{a} + \frac{c}{b}  + 1 $</span></p>  <p><span class="math-container" id="q_864">$= 3 + \big(\frac{a}{b}  + \frac{b}{a}\big) + \big(\frac{a}{c} + \frac{c}{a}\big) + \big(\frac{b}{c} + \frac{c}{b}\big) $</span></p>  <p>By lemma 1, <span class="math-container" id="q_865">$\big(\frac{a}{b}  + \frac{b}{a}\big) \ge 2$</span>,  <span class="math-container" id="q_866">$ \big(\frac{a}{c} + \frac{c}{a}\big) \ge 2$</span> and  <span class="math-container" id="q_867">$\big(\frac{b}{c} + \frac{c}{b}\big) \ge 2$</span> , therefore:</p>  <p><span class="math-container" id="q_868">$3 + \big(\frac{a}{b}  + \frac{b}{a}\big) + \big(\frac{a}{c} + \frac{c}{a}\big) + \big(\frac{b}{c} + \frac{c}{b}\big) \ge 3 + 2 + 2 +2 = 9 = 3^2 \ \blacksquare $</span></p>  <p>However I'm not sure the generalized version for all natural <span class="math-container" id="q_869">$n$</span> is true. I can't come up with a counterexample and when I try to prove it by induction I get stuck.</p>  <p>Here is my attempt:</p>  <p>Let <span class="math-container" id="q_870">$P(n)::\big(\sum_{i=1}^{n}a_{i}\big) \big(\sum_{i=1}^{n} \frac{1}{a_{i}}\big) \ge n^2$</span></p>  <p><strong>Base case:</strong> <span class="math-container" id="q_871">$\big(\sum_{i=1}^{1}a_{i}\big) \big(\sum_{i=1}^{1} \frac{1}{a_{i}}\big) = a_{1} \frac{1}{a_{1}} = 1 = 1^2$</span> , so <span class="math-container" id="q_872">$P(1)$</span> is true.</p>  <p><strong>Inductive hypothesis:</strong>  I assume <span class="math-container" id="q_873">$P(n)$</span> is true.</p>  <p><strong>Inductive step:</strong></p>  <p><span class="math-container" id="q_874">$$\left(\sum_{i=1}^{n+1}a_{i}\right) \left(\sum_{i=1}^{n+1} \frac{1}{a_{i}}\right) = \left[\left(\sum_{i=1}^{n}a_{i}\right) + a_{n+1}\right] \left[\left(\sum_{i=1}^{n} \frac{1}{a_{i}}\right) + \frac{1}{a_{n+1}}\right]$$</span></p>  <p><span class="math-container" id="q_875">$$=\left(\sum_{i=1}^{n}a_{i}\right) \left[\left(\sum_{i=1}^{n} \frac{1}{a_{i}}\right) + \frac{1}{a_{n+1}}\right] + a_{n+1} \left[\left(\sum_{i=1}^{n} \frac{1}{a_{i}}\right) + \frac{1}{a_{n+1}}\right]$$</span></p>  <p><span class="math-container" id="q_876">$$=\left(\sum_{i=1}^{n}a_{i}\right)\left(\sum_{i=1}^{n} \frac{1}{a_{i}}\right) +\left(\sum_{i=1}^{n}a_{i}\right) \frac{1}{a_{n+1}} + a_{n+1} \left(\sum_{i=1}^{n} \frac{1}{a_{i}}\right) +a_{n+1} \frac{1}{a_{n+1}}$$</span></p>  <p><span class="math-container" id="q_877">$$=\left(\sum_{i=1}^{n}a_{i}\right)\left(\sum_{i=1}^{n} \frac{1}{a_{i}}\right) +\left(\sum_{i=1}^{n}a_{i}\right) \frac{1}{a_{n+1}} + a_{n+1} \left(\sum_{i=1}^{n} \frac{1}{a_{i}}\right) +1 $$</span></p>  <p><span class="math-container" id="q_878">$$\underbrace{\ge}_{IH} n^2 + \left(\sum_{i=1}^{n}a_{i}\right) \frac{1}{a_{n+1}} + a_{n+1} \left(\sum_{i=1}^{n} \frac{1}{a_{i}}\right) + 1$$</span></p>  <p>And here I don't know what to do with the <span class="math-container" id="q_879">$\big( \sum_{i=1}^{n}a_{i} \big) \frac{1}{a_{n+1}} + a_{n+1} \big(\sum_{i=1}^{n} \frac{1}{a_{i}}\big)$</span> term.</p>  <p><strong>Is this inequality true? If it is, how can I prove it? If it isn't, can anyone show me a counterexample?</strong></p>
Tags algebra-precalculus,inequality
_number B.87
Formula_Id q_883
Latex \mathbb{Z}[x]
Title Is the polynomial <span class="math-container" id="q_880">$x^4+10x^2+1$</span> reducible over <span class="math-container" id="q_881">$\mathbb{Z}[x]$</span>?
Question <p>Is the polynomial  <span class="math-container" id="q_882">$x^4+10x^2+1$</span>  reducible over  <span class="math-container" id="q_883">$\mathbb{Z}[x]$</span>?</p>
Tags abstract-algebra,ring-theory,field-theory,irreducible-polynomials
_number B.88
Formula_Id q_884
Latex A^{2} + B^{2} = C^{2} + D^{2}
Title Parametrization of pythagorean-like equation
Question <p>Is there any known complete parametrization of the Diophantine equation <span class="math-container" id="q_884">$$ A^{2} + B^{2} = C^{2} + D^{2} $$</span> where <span class="math-container" id="q_885">$A, B, C, D$</span> are (positive) rational numbers, or equivalently, integers?</p>
Tags number-theory,diophantine-equations
_number B.89
Formula_Id q_894
Latex A^{-1}A=\mathbb I_n
Title Question on the definition of an Inverse matrix
Question <p>By definition, if <span class="math-container" id="q_886">$A$</span> is a <span class="math-container" id="q_887">$ n \times n $</span> matrix, an inverse of <span class="math-container" id="q_888">$A$</span> is an <span class="math-container" id="q_889">$ n \times n $</span> matrix <span class="math-container" id="q_890">$A^{-1}$</span> with the property that:</p>  <p><span class="math-container" id="q_891">$$ A^{-1}A=\mathbb I_n \ \ \land \ \ AA^{-1}=\mathbb I_n \ \ \ \ (1)$$</span> </p>  <p>where <span class="math-container" id="q_892">$ \mathbb I_n $</span> is the <span class="math-container" id="q_893">$ n \times n $</span> identity matrix.</p>  <p>Are there any cases where <span class="math-container" id="q_894">$ A^{-1}A=\mathbb I_n$</span> but <span class="math-container" id="q_895">$AA^{-1} \neq \mathbb I_n$</span> or the other way around (and thus making (1) a false statement) ? </p>
Tags linear-algebra,matrices
_number B.90
Formula_Id q_931
Latex P(X,Y) = (X^{p-1}-1)XY - 1
Title Is there a principal maximal ideal in <span class="math-container" id="q_901">$\mathbb F_q[X,Y]$</span>?
Question <blockquote>   <p>Given an infinite field <span class="math-container" id="q_902">$K$</span>, one can prove that any maximal ideal of <span class="math-container" id="q_903">$K[X,Y]$</span> can't be principal. In fact, every non-principal prime ideal is a maximal ideal, and can be generated by two polynomials.</p>      <p>I am wondering whether the same result holds in <span class="math-container" id="q_904">$\mathbb F_q[X,Y]$</span>. Can we find a principal ideal <span class="math-container" id="q_905">$I = (P(X,Y))$</span> for some irreducible polynomial <span class="math-container" id="q_906">$P$</span> that is a maximal ideal ?  </p> </blockquote>  <p>Such a polynomial <span class="math-container" id="q_907">$P$</span> must have positive degrees in both <span class="math-container" id="q_908">$X$</span> and <span class="math-container" id="q_909">$Y$</span>. Indeed, given an irreducible polynomial <span class="math-container" id="q_910">$Q(X)$</span> in only one variable, the quotient  <span class="math-container" id="q_911">$$\mathbb F_q[X,Y]/(Q(X))\cong (\mathbb F_q[X]/(Q(X)))[Y]$$</span> is a ring of polynomials over a field and as such, can never be a field.  </p>  <p>Moreover, such a polynomial <span class="math-container" id="q_912">$P$</span> must have the form  <span class="math-container" id="q_913">$$P(X,Y) = \sum_{i=0}^n P_i(X)Y^i$$</span> where <span class="math-container" id="q_914">$d\geq 1$</span>, the <span class="math-container" id="q_915">$P_i$</span>'s are polynomials in <span class="math-container" id="q_916">$X$</span> and <span class="math-container" id="q_917">$P_n$</span> is a nonzero polynomial that vanishes identically on <span class="math-container" id="q_918">$\mathbb F_q$</span>, thus <span class="math-container" id="q_919">$P_n$</span> must be divisible by <span class="math-container" id="q_920">$\prod_{\alpha \in \mathbb F_q} (X-\alpha)$</span>.<br> Indeed, if there were some <span class="math-container" id="q_921">$\alpha \in \mathbb F_q$</span> such that <span class="math-container" id="q_922">$P_n(\alpha) \not = 0$</span>, then the ideal <span class="math-container" id="q_923">$I:=(P,X-\alpha)$</span> would contain <span class="math-container" id="q_924">$(P)$</span> strictly. If <span class="math-container" id="q_925">$(P)$</span> were to be maximal, <span class="math-container" id="q_926">$I$</span> would be the whole of <span class="math-container" id="q_927">$\mathbb F_q[X,Y]$</span>. Writing down the fact that <span class="math-container" id="q_928">$1\in I$</span> and evaluating <span class="math-container" id="q_929">$X=\alpha$</span> would leads us to the conclusion <span class="math-container" id="q_930">$n=\deg_Y(P)=0$</span>, which is absurd.  </p>  <p>This is all I could infer so far. With respect to the above, I tried looking at <span class="math-container" id="q_931">$P(X,Y) = (X^{p-1}-1)XY - 1$</span> or <span class="math-container" id="q_932">$P(X,Y) = (X^{p-1}-1)XY - X - 1$</span> in <span class="math-container" id="q_933">$\mathbb F_p[X,Y]$</span> for some prime number <span class="math-container" id="q_934">$p$</span> but I have trouble determining whether the quotient is a field or not.  </p>  <p>Would somebody know the answer of the problem, and according to it, give a proof or a counter-example ? Thank you very much in advance.</p>
Tags polynomials,commutative-algebra,field-theory,finite-fields
_number B.92
Formula_Id q_940
Latex det(xI - AB) = det(xI - BA)
Title Characteristic Polynomial <span class="math-container" id="q_935">$AB =$</span> characteristic polynomial <span class="math-container" id="q_936">$ BA$</span>?
Question <p>Let <span class="math-container" id="q_937">$A,B$</span> matrix on <span class="math-container" id="q_938">$\mathbb{R}$</span> size <span class="math-container" id="q_939">$nxn$</span>. How can I prove that <span class="math-container" id="q_940">$det(xI - AB) = det(xI - BA)$</span> if <span class="math-container" id="q_941">$A$</span> and <span class="math-container" id="q_942">$B$</span> are singular matrix</p>
Tags linear-algebra,polynomials
_number B.93
Formula_Id q_944
Latex 2^{\aleph_{0} }
Title Are there <span class="math-container" id="q_943">$2^{\aleph_{0} }$</span> sets of natural numbers such that each two have finite intersection
Question <p><strong>Question:</strong> Are there <span class="math-container" id="q_944">$2^{\aleph_{0} }$</span> sets of natural numbers such that each two have finite intersection.</p>  <p>From what I've read about infinite families, I need to ignore those who have the properpty <span class="math-container" id="q_945">$P$</span>.</p>  <p>Property <span class="math-container" id="q_946">$P$</span>: the family is threadless, but whenever we take finitely many sets from the family, those sets have infinite intersection.</p>  <p>and probably find ones with property <span class="math-container" id="q_947">$T$</span>.</p>  <p>Property <span class="math-container" id="q_948">$T$</span>: the family is threadless, and it is an “almost-tower”: Whenever you pick two sets in the family, one of them is almost-contained in the other. - probably meaning that we can find such an intersection which is finite, because members are contained in each other.</p>  <p>Then I thought..</p>  <p>To find them, I should think of rational numbers instead of natural numbers, remembering that rational numbers can be paired up with natural ones, so solving this problem for families of rational numbers is the same as solving it for families of natural numbers. Now, I need to consider various ways of defining the real numbers.</p>  <p>And I'm stuck.. any help is appreciated.</p>
Tags cardinals,rational-numbers,natural-numbers
_number B.94
Formula_Id q_954
Latex \lim_{x\to 0}\frac{\sin x}x=1
Title Length of convex paths and bounding <span class="math-container" id="q_949">$\sin x$</span>
Question <p>The problem:</p>  <blockquote>   <p>Defining <span class="math-container" id="q_950">$\sin x$</span> as the leg <span class="math-container" id="q_951">$b$</span> of a right triangle with <span class="math-container" id="q_952">$\angle B=x$</span> (in radians) and hypotenuse <span class="math-container" id="q_953">$1$</span>, prove that <span class="math-container" id="q_954">$$\lim_{x\to 0}\frac{\sin x}x=1$$</span></p> </blockquote>  <p>(The motivation is to find the derivative of <span class="math-container" id="q_955">$\sin x$</span> in a elementary, "pre-Taylor" and "pre-series" context).</p>  <p>I have seen many times a proof that it is based in the fact that the length of the arc <span class="math-container" id="q_956">$x$</span> satisfies <span class="math-container" id="q_957">$$\sin x&lt;x&lt;\tan x$$</span> or sometimes <span class="math-container" id="q_958">$$\sin x&lt;x&lt;\sin x+1-\cos x$$</span> The lower bound is clear because the <span class="math-container" id="q_959">$\sin x$</span> is the length of a straight segment and <span class="math-container" id="q_960">$x$</span> is the length of a curved segment with the same endpoints. But I find that the upper bound is based on this intuitive fact:</p>  <blockquote>   <p>If <span class="math-container" id="q_961">$F$</span> and <span class="math-container" id="q_962">$G$</span> are two convex subsets of <span class="math-container" id="q_963">$\Bbb R^2$</span> and   <span class="math-container" id="q_964">$F\subset G$</span>, then <span class="math-container" id="q_965">$|\partial F|&lt;|\partial G|$</span>.</p> </blockquote>  <p>(<span class="math-container" id="q_966">$|\partial F|$</span> is the length of the boundary of <span class="math-container" id="q_967">$F$</span>).</p>  <p>But I haven't ever seen a proof of that. I tried it myself but I get stuck trying to bound <span class="math-container" id="q_968">$$\int_s^t\sqrt{1+y'(u)^2}du$$</span> provided for example that <span class="math-container" id="q_969">$y''$</span> is negative and <span class="math-container" id="q_970">$y(s)=y(t)$</span>. I realize that <span class="math-container" id="q_971">$y'$</span> can't be bounded (note for example <span class="math-container" id="q_972">$y=\sqrt{1-x^2}$</span>, <span class="math-container" id="q_973">$-1\le x\le 1$</span>).</p>  <p><strong>Question</strong></p>  <p>Is it possible to justify the derivative of <span class="math-container" id="q_974">$\sin x$</span> or the inequality about the lengths of bounds? Is there any calculus text with this approach (or similar) to define trigonometical functions?</p>
Tags real-analysis,derivatives,trigonometry,reference-request,arc-length
_number B.95
Formula_Id q_977
Latex \sum_i \frac{a_i}{a_i+a_{i+1}+a_{i+2}+\cdots}
Title Let <span class="math-container" id="q_975">$\sum_i a_i$</span> be a convergent sum with positive <span class="math-container" id="q_976">$a_i$</span>. Does <span class="math-container" id="q_977">$\sum_i \frac{a_i}{a_i+a_{i+1}+a_{i+2}+\cdots}$</span> always diverge?
Question <p>Let <span class="math-container" id="q_978">$\sum_i a_i$</span> be a convergent sum, with all <span class="math-container" id="q_979">$a_i$</span> positive.  Let <span class="math-container" id="q_980">$s_n=\sum_{i=n}^{\infty}a_i$</span>. </p>  <blockquote>   <p>Does <span class="math-container" id="q_981">$\sum_i  \frac{a_i}{s_i}$</span> always diverge?</p> </blockquote>  <p>I've tried a few examples such as <span class="math-container" id="q_982">$a_i= r^i$</span> (geometric series) and <span class="math-container" id="q_983">$a_i=1/i^2$</span> and it seems to always diverge.</p>
Tags sequences-and-series,analysis
_number B.96
Formula_Id q_997
Latex lcm(b_1,...,b_m)
Title what is the dimension of <span class="math-container" id="q_984">$\mathbb{R}$</span> at a vector space over there field <span class="math-container" id="q_985">$\mathbb{Q}$</span>?
Question <p>If we look at <span class="math-container" id="q_986">$\mathbb{C}$</span> as a vector space over <span class="math-container" id="q_987">$\mathbb{R}$</span> it's dimension will be <span class="math-container" id="q_988">$2$</span>, because <span class="math-container" id="q_989">$\mathbb{C} = span\{1,i\}$</span>. <br/> A question I thought of is what would be the dimension of <span class="math-container" id="q_990">$\mathbb{R}$</span> as a vector space over <span class="math-container" id="q_991">$\mathbb{Q}$</span>? <br/> I feel like the answer should be infinity, because if the dimension was finite, say <span class="math-container" id="q_992">$n$</span>, then for every <span class="math-container" id="q_993">$m := n+1$</span> real numbers  <span class="math-container" id="q_994">$x_1,...x_m$</span> there was a linear combination with rational coefficients that gives <span class="math-container" id="q_995">$0$</span>: <span class="math-container" id="q_996">$\frac{a_1}{b_1}x_1+...+\frac{a_m}{b_m}x_m = 0$</span>. Multiplying by <span class="math-container" id="q_997">$lcm(b_1,...,b_m)$</span> we get that for every <span class="math-container" id="q_998">$m$</span> real numbers there is a linear combination with natural coefficients that gives <span class="math-container" id="q_999">$0$</span>. That feels false, how do you prove it?</p>
Tags linear-algebra,vector-spaces
_number B.97
Formula_Id q_1010
Latex R(e^{2\pi ix})=e^{2\pi i (x+\alpha)}.
Title If <span class="math-container" id="q_1000">$R:S^{1}\rightarrow S^{1}$</span> is a irrational rotation, <span class="math-container" id="q_1001">$\{R^{n}([x])\}$</span> is dense in <span class="math-container" id="q_1002">$S^{1}$</span> for all points.
Question <p>Let <span class="math-container" id="q_1003">$\alpha$</span> a irrational number, and <span class="math-container" id="q_1004">$R:S^{1}\rightarrow S^{1}$</span> the irrational rotation, i.e., <span class="math-container" id="q_1005">$[x]\rightarrow[x+\alpha]$</span>. I need to prove that, for all <span class="math-container" id="q_1006">$[x]\in S^{1}$</span>, the set <span class="math-container" id="q_1007">$\{R^{n}([x])\}$</span> is dense in <span class="math-container" id="q_1008">$S^{1}$</span>.</p>  <p>First, I can write <span class="math-container" id="q_1009">$R$</span> by</p>  <p><span class="math-container" id="q_1010">$$R(e^{2\pi ix})=e^{2\pi i (x+\alpha)}.$$</span></p>  <p>So, I can write <span class="math-container" id="q_1011">$R^{n}(x)$</span>, <span class="math-container" id="q_1012">$n\in\mathbb{Z}$</span> by</p>  <p><span class="math-container" id="q_1013">$$R^{n}(e^{2\pi i x})=e^{2\pi i(x+n\alpha)} $$</span></p>  <p>I need to prove that, for all <span class="math-container" id="q_1014">$e^{2\pi i x}\in S^{1}$</span> and for all <span class="math-container" id="q_1015">$[y]=e^{2\pi i y}\in S^{1}$</span>, every neighborhood <span class="math-container" id="q_1016">$V$</span> of <span class="math-container" id="q_1017">$[y]$</span> contains a point <span class="math-container" id="q_1018">$[z_{y}]=e^{2\pi i z}$</span> such that <span class="math-container" id="q_1019">$[z_y]=R^{n}([x])$</span> for some <span class="math-container" id="q_1020">$n\in\mathbb{Z}$</span>. That is,</p>  <p><span class="math-container" id="q_1021">$e^{2\pi i z}=e^{2\pi i (x+\alpha n)}\Rightarrow 2\pi iz=2\pi i(x+\alpha n)+2ik\pi\;\textrm{for some}\;k\in\mathbb{Z}\Rightarrow z=x+\alpha n+k.$</span></p>  <p>Is my way correct? If it does, how can I proceed now? If it doesn't, what I need to do?</p>  <p>I don't think this question is duplicate. I'm showing my attempt to proof, that is different from other proofs.</p>
Tags general-topology,circles,rotations,irrational-numbers
_number B.98