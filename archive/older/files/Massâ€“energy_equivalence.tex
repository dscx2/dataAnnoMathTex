{{redirect2|E{{=}}MC2|E{{=}}mc2}}
{{short description|A physical law that mass and energy are proportionate measures of the same underlying property of an object}}
{{Special relativity sidebar |consequences}}
[[File:E=mc²-explication.svg|thumb|236px|{{math|1=''E'' = ''mc''{{smallsup|2}}}} explained]]
{{General physics}}
In [[physics]], '''mass–energy equivalence''' is the principle that anything having [[mass]] has an equivalent amount of [[energy]] and vice versa, with these fundamental quantities directly relating to one another by [[Albert Einstein]]'s famous formula:<ref name="famous" />

<math display="block" qid=Q35875>E=mc^2</math>

This formula states that the equivalent energy ({{math|''E''}}) can be calculated as the mass ({{math|''m''}}) multiplied by the [[speed of light]] ({{math|''c''}} = ~{{val|3|e=8}} m/s) squared. Similarly, anything having energy exhibits a corresponding mass {{math|''m''}} given by its energy {{math|''E''}} divided by the speed of light squared {{math|''c''<sup>2</sup>}}. Because the speed of light is a large number in everyday units, the formula implies that even an everyday object at rest with a modest amount of mass has a very large amount of energy intrinsically. [[Chemical reaction]]s, [[nuclear reaction]]s, and other [[energy transformation]]s may cause a [[Physical system|system]] to lose some of its energy content to the environment (and thus some corresponding mass), releasing it as the [[radiant energy]] of [[light]] or as [[thermal energy]] for example.

Mass–energy equivalence arose originally from [[special relativity]] as a paradox described by [[Henri Poincaré]].<ref name=action>{{Citation
| author=Poincaré, H. | year=1900 | title=La théorie de Lorentz et le principe de réaction | journal=Archives Néerlandaises des Sciences Exactes et Naturelles | volume =5  | pages =252–278| title-link=s:fr:La théorie de Lorentz et le principe de réaction }}. See also the [http://www.physicsinsights.org/poincare-1900.pdf English translation]</ref> Einstein proposed it on 21 November 1905, in the paper ''Does the inertia of a body depend upon its energy-content?'', one of his [[Annus Mirabilis papers|''Annus Mirabilis'' (Miraculous Year) papers]].<ref name=inertia>{{Citation | author=Einstein, A. | year=1905 | title=Ist die Trägheit eines Körpers von seinem Energieinhalt abhängig? | journal=Annalen der Physik | volume=18 | issue=13 | pages=639–643 | doi=10.1002/andp.19053231314|bibcode = 1905AnP...323..639E | url=https://zenodo.org/record/1424057 }}. See also the [http://www.fourmilab.ch/etexts/einstein/E_mc2/www/ English translation.]</ref> Einstein was the first to propose that the equivalence of mass and energy is a general principle and a consequence of the [[Spacetime symmetries|symmetries of space and time]].

A consequence of the mass–energy equivalence is that if a body is stationary, it still has some internal or intrinsic energy, called its [[Intrinsic mass#Rest energy|rest energy]], corresponding to its [[rest mass]].  When the body is in motion, its total energy is greater than its rest energy, and equivalently its ''total mass'' (also called [[relativistic mass]] in this context) is greater than its rest mass.  This rest mass is also called the intrinsic or [[invariant mass]] because it remains the same regardless of this motion, even for the extreme speeds or gravity considered in special and [[general relativity]].

The mass–energy formula also serves to [[Conversion of units|convert]] [[:Category:Units of mass|units of mass]] to [[:Category:Units of energy|units of energy]] (and vice versa), no matter what [[Systems of measurement|system of measurement units]] is used.

==Nomenclature==
The formula was initially written in many different notations, and its interpretation and justification was further developed in several steps.<ref name=jammer2 /><ref name=hecht>{{Citation | author=Hecht, Eugene | date=2011 | title=How Einstein confirmed E0=mc2 | journal=American Journal of Physics | volume =79 | issue=6 | pages =591–600 | doi=10.1119/1.3549223|bibcode = 2011AmJPh..79..591H }}</ref>
In "Does the inertia of a body depend upon its energy content?" (1905), Einstein used {{math|''V''}} to mean the speed of light in a vacuum and {{math|''L''}} to mean the [[energy]] lost by a body in the form of [[radiation]].<ref name="inertia"/> Consequently, the equation {{math|1=''E'' = ''mc''<sup>2</sup>}} was not originally written as a formula but as a sentence in German saying that "if a body gives off the energy {{math|''L''}} in the form of radiation, its mass diminishes by {{math|{{sfrac|''L''|''V''<sup>2</sup>}}}}." A remark placed above it informed that the equation was approximated by neglecting "magnitudes of fourth and higher orders" of a [[Series (mathematics)|series expansion]].<ref>See the sentence on the last page 641 of the original German edition, above the equation 
{{math|1=''K''<sub>0</sub> − ''K''<sub>1</sub> = {{sfrac|''L''|''V''<sup>2</sup>}} {{sfrac|''v''<sup>2</sup>|2}}}}. See also the sentence above the last equation in the English translation, {{math|1=''K''<sub>0</sub> − ''K''<sub>1</sub> = {{sfrac|1|2}}({{sfrac|''L''|''c''<sup>2</sup>}})''v''<sup>2</sup>}}, and the comment on the symbols used in ''About this edition'' that follows the translation.</ref>

In May 1907, Einstein explained that the expression for energy {{math|''ε''}} of a moving mass point assumes the simplest form, when its expression for the state of rest is chosen to be {{math|1=''ε''<sub>0</sub> = ''μV''<sup>2</sup>}} (where {{math|''μ''}} is the mass), which is in agreement with the "principle of the equivalence of mass and energy". In addition, Einstein used the formula {{math|1=''μ'' = {{sfrac|''E''<sub>0</sub>|''V''<sup>2</sup>}}}}, with {{math|''E''<sub>0</sub>}} being the energy of a system of mass points, to describe the energy and mass increase of that system when the velocity of the differently moving mass points is increased.<ref>{{Citation|doi=10.1002/andp.19073280713|author=Einstein, Albert|date=1907|title=Über die vom Relativitätsprinzip geforderte Trägheit der Energie|journal=Annalen der Physik|volume=328|issue=7|pages=371–384|bibcode = 1907AnP...328..371E |url=http://www.physik.uni-augsburg.de/annalen/history/einstein-papers/1907_23_371-384.pdf}}</ref>

In June 1907, [[Max Planck]] rewrote Einstein's mass–energy relationship as {{math|1=''M'' = {{sfrac|''E''<sub>0</sub> + ''pV''<sub>0</sub>|''c''<sup>2</sup>}}}}, where {{math|''p''}} is the pressure and {{math|''V''}} the volume to express the relation between mass, its ''latent energy'', and thermodynamic energy within the body.<ref>{{Citation|author=Planck, Max|date=1907|title=Zur Dynamik bewegter Systeme|journal=Sitzungsberichte der Königlich-Preussischen Akademie der Wissenschaften, Berlin|volume=Erster Halbband|issue=29|pages=542–570|url=https://archive.org/details/sitzungsberichte1907deutsch|bibcode=1908AnP...331....1P|doi=10.1002/andp.19083310602}}
:English Wikisource translation: [[s:Translation:On the Dynamics of Moving Systems|On the Dynamics of Moving Systems]]</ref> Subsequently, in October 1907, this was rewritten as {{math|1=''M''<sub>0</sub> = {{sfrac|''E''<sub>0</sub>|''c''<sup>2</sup>}}}} and given a quantum interpretation by [[Johannes Stark]], who assumed its validity and correctness (''Gültigkeit'').<ref>{{Citation |first=J. |last=Stark |title=Elementarquantum der Energie, Modell der negativen und der positiven Elekrizität |journal=Physikalische Zeitschrift |page=881 |volume=24 |issue=8 |date=1907|url=https://archive.org/details/physikalischeze00unkngoog }}</ref>

In December 1907, Einstein expressed the equivalence in the form {{math|1=''M'' = ''μ'' + {{sfrac|''E''<sub>0</sub>|''c''<sup>2</sup>}}}} and concluded: "A mass {{math|''μ''}} is equivalent, as regards inertia, to a quantity of energy {{math|''μc<sup>2</sup>''}}. [...] It appears far more natural to consider every inertial mass as a store of energy."<ref>{{Citation|author=Einstein, Albert|date=1908|title=Über das Relativitätsprinzip und die aus demselben gezogenen Folgerungen|journal=Jahrbuch der Radioaktivität und Elektronik|volume=4|pages=411–462|url=http://www.soso.ch/wissen/hist/SRT/E-1907.pdf|bibcode = 1908JRE.....4..411E }}</ref><ref>{{Citation|author=Schwartz, H. M.|date=1977|title=Einstein's comprehensive 1907 essay on relativity, part II|journal=American Journal of Physics|volume=45|pages=811–817|doi=10.1119/1.11053|issue=9|bibcode = 1977AmJPh..45..811S }}</ref>

In 1909, [[Gilbert N. Lewis]] and [[Richard C. Tolman]] used two variations of the formula: {{math|1=''m'' = {{sfrac|''E''|''c''<sup>2</sup>}}}} and {{math|1=''m''<sub>0</sub> = {{sfrac|''E''<sub>0</sub>|''c''<sup>2</sup>}}}}, with {{mvar|E}} being the relativistic energy (the energy of an object when the object is moving), {{math|''E''<sub>0</sub>}} is the rest energy (the energy when not moving), {{mvar|m}} is the [[relativistic mass]] (the rest mass and the extra mass gained when moving), and {{math|''m''<sub>0</sub>}} is the [[invariant mass|rest mass]] (the mass when not moving).<ref>{{Citation |author=Lewis, Gilbert N. |author2=Tolman, Richard C. |lastauthoramp=yes |date=1909|title=The Principle of Relativity, and Non-Newtonian Mechanics|journal=Proceedings of the American Academy of Arts and Sciences|volume=44|pages=709–726|doi=10.2307/20022495|issue=25|title-link=s:The Principle of Relativity, and Non-Newtonian Mechanics |jstor=20022495 }}</ref> The same relations in different notation were used by [[Hendrik Lorentz]] in 1913 (published 1914), though he placed the energy on the left-hand side: {{math|1=''ε'' = ''Mc''<sup>2</sup>}} and {{math|1=''ε''<sub>0</sub> = ''mc''<sup>2</sup>}}, with {{mvar|ε}} being the total energy (rest energy plus kinetic energy) of a moving material point, {{math|''ε''<sub>0</sub>}} its rest energy, {{mvar|M}} the relativistic mass, and {{mvar|m}} the invariant (or rest) mass.<ref>{{Citation
|author=Lorentz, Hendrik Antoon|date=1914|title=Das Relativitätsprinzip. Drei Vorlesungen gehalten in Teylers Stiftung zu Haarlem (1913)|publisher=B.G. Teubner
|location=Leipzig and Berlin|title-link=s:de:Das Relativitätsprinzip (Lorentz)}}</ref>

In 1911, [[Max von Laue]] gave a more comprehensive proof of {{math|1=''M''<sub>0</sub> = {{sfrac|''E''<sub>0</sub>|''c''<sup>2</sup>}}}} from the [[stress–energy tensor]],<ref>{{Citation|author=Laue, Max von|title=Zur Dynamik der Relativitätstheorie|journal=Annalen der Physik|volume=340|issue=8|doi=10.1002/andp.19113400808|date=1911|pages=524–542|bibcode = 1911AnP...340..524L }}
:English Wikisource translation: [[s:Translation:On the Dynamics of the Theory of Relativity|On the Dynamics of the Theory of Relativity]]</ref> which was later (1918) generalized by [[Felix Klein]].<ref>{{Citation|author=Klein, Felix|title=Über die Integralform der Erhaltungssätze und die Theorie der räumlich-geschlossenen Welt|journal=Göttinger Nachrichten|date=1918|pages=394–423|url=http://gdz.sub.uni-goettingen.de/dms/load/img/?PPN=PPN243240503&DMDID=DMDLOG_0051}}</ref>

Einstein returned to the topic once again after [[World War II]] and this time he wrote {{math|1=''E'' = ''mc''<sup>2</sup>}} in the title of his article<ref>A.Einstein {{math|1=E = mc<sup>2</sup>}}'': the most urgent problem of our time'' Science illustrated, vol. 1 no. 1, April issue, pp. 16–17, 1946 (item 417 in the "Bibliography"</ref> intended as an explanation for a general reader by analogy.<ref>M.C.Shields ''Bibliography of the Writings of Albert Einstein to May 1951'' in Albert Einstein: Philosopher-Scientist by Paul Arthur Schilpp (Editor) [https://www.questia.com/PM.qst?a=o&d=84000079 Albert Einstein Philosopher – Scientist]</ref>

==Conservation of mass and energy==
{{Main|Conservation of energy|Conservation of mass}}
Mass and energy can be seen as two names (and two measurement units) for the same underlying, conserved physical quantity.<ref>"Einstein was unequivocally against the traditional idea of conservation of mass. He had concluded that mass and energy were essentially one and the same; 'inert[ial] mass is simply latent energy.'[ref...]. He made his position known publicly time and again[ref...]...", Eugene Hecht, "Einstein on mass and energy." Am. J. Phys., Vol. 77, No. 9, September 2009, [http://www.stat.physik.uni-potsdam.de/~pikovsky/teaching/stud_seminar/einstein.pdf online].</ref> Thus, the laws of [[conservation of energy]] and [[Mass in special relativity|conservation of (total) mass]] are equivalent and both hold true.<ref>"There followed also the principle of the equivalence of mass and energy, with the laws of conservation of mass and energy becoming one and the same.",  Albert Einstein, "Considerations Concerning the Fundaments of Theoretical Physics", Science, Washington, DC, vol. 91, no. 2369, May 24th, 1940 [http://promo.aaas.org/kn_marketing/pdfs/Science_1940_0524.pdf scanned image online]</ref> Einstein elaborated in a 1946 essay that "the principle of the conservation of mass [...] proved inadequate in the face of the special theory of relativity.  It was therefore merged with the energy [[conservation law|conservation]] principle—just as, about 60 years before, the principle of the [[conservation of mechanical energy]] had been combined with the principle of the conservation of heat [thermal energy].  We might say that the principle of the conservation of energy, having previously swallowed up that of the conservation of heat, now proceeded to swallow that of the conservation of mass—and holds the field alone."<ref>{{cite book|url= https://books.google.com/books?id=SYPbH6xCbUMC&pg=PA14 |first=Albert |last=Einstein |title=The Theory of Relativity (And Other Essays) |publisher=Citadel Press |year=1950 |p=14|isbn=9780806517650 }}</ref>

If the conservation of mass law is interpreted as conservation of [[rest mass|''rest'' mass]], it does not hold true in special relativity.  The ''rest'' energy (equivalently, rest mass) of a particle can be converted, not "to energy" (it already ''is'' energy (mass)), but rather to ''other'' forms of energy (mass) that require motion, such as [[kinetic energy]], [[thermal energy]], or [[radiant energy]]. Similarly, kinetic or radiant energy can be converted to other kinds of particles that have rest energy (rest mass).  In the transformation process, neither the total amount of mass nor the total amount of energy changes, since both properties are connected via a simple constant.<ref>In F. Fernflores. The Equivalence of Mass and Energy. Stanford Encyclopedia of Philosophy. [http://plato.stanford.edu/entries/equivME/#2.1]</ref><ref name="A. Wheeler, 1992. pp. 248">E. F. Taylor and J. A. Wheeler, ''Spacetime Physics'', W.H. Freeman and Co., NY. 1992. {{ISBN|0-7167-2327-1}}, see pp. 248–9 for discussion of mass remaining constant after detonation of nuclear bombs, until heat is allowed to escape.</ref> This view requires that if either energy or (total) mass disappears from a system, it is always found that both have simply moved to another place, where they are both measurable as an increase of both energy and mass that corresponds to the loss in the first system.

===Fast-moving objects and systems of objects===
When an object is pushed in the direction of motion, it gains [[momentum]] and energy, but when the object is already traveling near the speed of light, it cannot move much faster, no matter how much energy it absorbs. Its momentum and energy continue to increase without bounds, whereas its speed approaches (but never reaches) a constant value—the speed of light. This implies that in relativity the momentum of an object cannot be a constant times the [[velocity]], nor can the [[Kinetic energy#Kinetic energy of rigid bodies|kinetic energy]] be a constant times the square of the velocity.

A property called the [[relativistic mass]] is defined as the ratio of the momentum of an object to its velocity.<ref>Note that the relativistic mass, in contrast to the rest mass {{math|''m''<sub>0</sub>}}, is not a relativistic invariant, and that the velocity <math>\,v=dx^{(4)}/dt</math> is not a Minkowski four-vector, in contrast to the quantity <math>\tilde v=dx^{(4)}/d\tau</math>, where <math> d\tau =dt\cdot\sqrt{1-(v^2/c^2)}</math> is the differential of the [[proper time]]. However, the energy–momentum four-vector <math> p^{(4)}=m_0\cdot dx^{(4)}/d\tau</math> is a genuine Minkowski four-vector, and the intrinsic origin of the square root in the definition of the relativistic mass is the distinction between {{mvar|dτ}} and {{mvar|dt}}.</ref> Relativistic mass depends on the motion of the object, so that different observers in relative motion see different values for it. If the object is moving slowly, the relativistic mass is nearly equal to the [[rest mass]] and both are nearly equal to the usual Newtonian mass. If the object is moving quickly, the relativistic mass is greater than the rest mass by an amount equal to the mass associated with the [[kinetic energy]] of the object. As the object approaches the speed of light, the relativistic mass grows infinitely, because the kinetic energy grows infinitely and this energy is associated with mass.

The relativistic mass is always equal to the total energy (rest energy plus kinetic energy) divided by {{math|''c''<sup>2</sup>}}.<ref name="tipler">{{citation |title=Modern Physics |author1=Paul Allen Tipler |author2=Ralph A. Llewellyn |pages=87–88 |url=https://books.google.com/?id=tpU18JqcSNkC&lpg=PP1&pg=PA87#v=onepage&q= |isbn=0-7167-4345-0 |publisher=W. H. Freeman and Company |date=January 2003}}</ref> Because the relativistic mass is exactly proportional to the energy, relativistic mass and relativistic energy are nearly synonyms; the only difference between them is the [[unit of measurement|units]]. If length and time are measured in [[natural units]], the speed of light is equal to 1, and even this difference disappears. Then mass and energy have the same units and are always equal, so it is redundant to speak about relativistic mass, because it is just another name for the energy. This is why physicists usually reserve the useful short word "mass" to mean rest mass, or [[invariant mass]], and not relativistic mass.

The relativistic mass of a moving object is larger than the relativistic mass of an object that is not moving, because a moving object has extra kinetic energy. The ''rest mass'' of an object is defined as the mass of an object when it is at rest, so that the rest mass is always the same, independent of the motion of the observer: it is the same in all [[inertial frame]]s.

For things and systems made up of many parts, like an [[atomic nucleus]], [[planet]], or [[star]], the relativistic mass is the sum of the relativistic masses (or energies) of the parts, because energies are additive in isolated systems. This is not true in open systems, however, if energy is subtracted. For example, if a system is [[Binding energy#Mass-energy relation|''bound'']] by attractive forces, and the energy gained due to the forces of attraction in excess of the work done is removed from the system, then mass is lost with this removed energy. For example, the mass of an atomic nucleus is less than the total mass of the protons and neutrons that make it up, but this is only true after this energy from binding has been removed in the form of a gamma ray (which in this system, carries away the mass of the energy of binding). This mass decrease is also equivalent to the energy required to break up the nucleus into individual protons and neutrons (in this case, work and mass would need to be supplied). Similarly, the mass of the solar system is slightly less than the sum of the individual masses of the sun and planets.

For a system of particles going off in different directions, the [[invariant mass]] of the system is the analog of the rest mass, and is the same for all observers, even those in relative motion. It is defined as the total energy (divided by {{math|''c''<sup>2</sup>}}) in the [[center of mass frame]] (where by definition, the system total momentum is zero). A simple example of an object with moving parts but zero total momentum is a container of gas. In this case, the mass of the container is given by its total energy (including the kinetic energy of the gas molecules), since the system total energy and invariant mass are the same in any reference frame where the momentum is zero, and such a reference frame is also the only frame in which the object can be weighed. In a similar way, the theory of special relativity posits that the thermal energy in all objects (including solids) contributes to their total masses and weights, even though this energy is present as the kinetic and potential energies of the atoms in the object, and it (in a similar way to the gas) is not seen in the rest masses of the atoms that make up the object.

In a similar manner, even photons (light quanta), if trapped in a container space (as a [[photon gas]] or [[thermal radiation]]), would contribute a mass associated with their energy to the container. Such an extra mass, in theory, could be weighed in the same way as any other type of rest mass. This is true in special relativity theory, even though individually photons have no rest mass. The property that trapped energy ''in any form'' adds weighable mass to systems that have no net momentum is one of the characteristic and notable consequences of relativity. It has no counterpart in classical Newtonian physics, in which radiation, light, heat, and kinetic energy never exhibit weighable mass under any circumstances.

Just as the relativistic mass of an isolated system is conserved through time, so also is its invariant mass.This property allows the conservation of all types of mass in systems, and also conservation of all types of mass in reactions where matter is destroyed (annihilated), leaving behind the energy that was associated with it (which is now in non-material form, rather than material form). Matter may appear and disappear in various reactions, but mass and energy are both unchanged in this process.

==Applicability of the strict formula==

As is noted above, two different definitions of mass have been used in special relativity, and also two different definitions of energy. The simple equation <math>E = mc^2</math> is not generally applicable to all these types of mass and energy, except in the special  case that the total additive momentum is zero for the system under consideration. In such a case, which is always guaranteed when observing the system from either its [[center of mass frame]] or its [[center of momentum frame]], <math>E = mc^2</math> is always true for any type of mass and energy that are chosen. Thus, for example, in the center of mass frame, the total energy of an object or system is equal to its rest mass times <math>c^2</math>, a useful equality. This is the relationship used for the container of gas in the previous example. It is ''not'' true in other reference frames where the center of mass is in motion. In these systems or for such an object, its total energy depends on both its rest (or invariant) mass, and its (total) momentum.<ref>Relativity DeMystified, D. McMahon, Mc Graw Hill (USA), 2006, {{ISBN|0-07-145545-0}}</ref>

In inertial reference frames other than the rest frame or center of mass frame, the equation <math>E = mc^2</math> remains true if the energy is the relativistic energy ''and'' the mass is the relativistic mass. It is also correct if the energy is the rest or invariant energy (also the minimum energy), ''and'' the mass is the rest mass, or the invariant mass. However, connection of the '''total or relativistic energy''' (<math>E_r</math>) with the '''rest or invariant mass''' (<math>m_0</math>) requires consideration of the system's total momentum, in systems and reference frames where the total momentum (of [[Magnitude (mathematics)#Euclidean vectors|magnitude]] {{math|''p''}}) has a non-zero value. The formula then required to connect the two different kinds of mass and energy, is the extended version of Einstein's equation, called the relativistic [[energy–momentum relation]]:<ref>Dynamics and Relativity, J.R. Forshaw, A.G. Smith, Wiley, 2009, {{ISBN|978-0-470-01460-8}}</ref>

:<math>\begin{align}
  E_r^2 - |\vec{p} \,|^2 c^2 &= m_0^2 c^4 \\
              E_r^2 - (pc)^2 &= (m_0 c^2)^2
\end{align}</math>

or

:<math>E_r = \sqrt{ (m_0 c^2)^2 + (pc)^2 } \,\!</math>

Here the <math>(pc)^2</math> term represents the square of the [[Euclidean norm]] (total vector length) of the various momentum vectors in the system, which reduces to the square of the simple momentum magnitude, if only a single particle is considered. This equation reduces to <math>E = mc^2</math> when the momentum term is zero. For photons where <math>m_0 = 0</math>, the equation reduces to <math>E_r = pc</math>.

==Meanings of the strict formula==
{{More citations needed|section|date=February 2016}}
[[File:E equals m plus c square at Taipei101.jpg|thumb|right|The mass–energy equivalence formula was displayed on [[Taipei 101]] during the event of the [[World Year of Physics 2005]].]]

Mass–energy equivalence states that any object has a certain energy, even when it is stationary. In [[Newtonian mechanics]], a motionless body has no [[kinetic energy]], and it may or may not have other amounts of internal stored energy, like [[chemical energy]] or [[thermal energy]], in addition to any [[potential energy]] it may have from its position in a [[field (physics)|field of force]]. In Newtonian mechanics, all of these energies are much smaller than the mass of the object times the speed of light squared.

In relativity, all the energy that moves with an object (that is, all the energy present in the object's rest frame) contributes to the total mass of the body, which measures how much it resists acceleration. Each bit of potential and kinetic energy makes a proportional contribution to the mass. As noted above, even if a box of ideal mirrors "contains" light, then the individually massless photons still contribute to the total mass of the box, by the amount of their energy divided by {{math|''c''<sup>2</sup>}}.<ref>
{{Citation
 |title=Mechanics
 |edition=2
 |first1=H. S.
 |last1=Hans
 |first2=S. P.
 |last2=Puri
 |publisher=Tata McGraw-Hill
 |date=2003
 |isbn=0-07-047360-9
 |page=433
 |url=https://books.google.com/books?id=hrBe52GPHrYC
}},
[https://books.google.com/books?id=hrBe52GPHrYC&pg=PA433 Chapter 12 page 433]
</ref>

In [[Theory of relativity|relativity]], removing energy is removing mass, and for an observer in the center of mass frame, the formula {{math|1=''m'' = {{sfrac|''E''|''c''<sup>2</sup>}}}} indicates how much mass is lost when energy is removed. In a nuclear reaction, the mass of the atoms that come out is less than the mass of the atoms that go in, and the difference in mass shows up as heat and light with the same relativistic mass as the difference (and also the same [[invariant mass]] in the center of mass frame of the system). In this case, the {{mvar|E}} in the formula is the energy released and removed, and the mass {{mvar|m}} is how much the mass decreases. In the same way, when any sort of energy is added to an isolated system, the increase in the mass is equal to the added energy divided by {{math|''c''<sup>2</sup>}}. For example, when water is heated it gains about {{val|1.11|e=-17|u=kg}} of mass for every joule of heat added to the water.

An object moves with different speed in different frames, depending on the motion of the observer, so the kinetic energy in both Newtonian mechanics and relativity is ''frame dependent''. This means that the amount of relativistic energy, and therefore the amount of relativistic mass, that an object is measured to have depends on the observer. The ''rest mass'' is defined as the mass that an object has when it is not moving (or when an inertial frame is chosen such that it is not moving). The term also applies to the invariant mass of systems when the system as a whole is not "moving" (has no net momentum). The rest and invariant masses are the smallest possible value of the mass of the object or system. They also are conserved quantities, so long as the system is isolated. Because of the way they are calculated, the effects of moving observers are subtracted, so these quantities do not change with the motion of the observer.

The rest mass is almost never additive: the rest mass of an object is not the sum of the rest masses of its parts. The rest mass of an object is the total energy of all the parts, including kinetic energy, as measured by an observer that sees the center of the mass of the object to be standing still. The rest mass adds up only if the parts are standing still and do not attract or repel, so that they do not have any extra kinetic or potential energy. The other possibility is that they have a positive kinetic energy and a negative potential energy that exactly cancels.

===Binding energy and the "mass defect"===
{{main|Binding energy|Mass defect}}
{{more citations needed section|date=July 2013}}

Whenever any type of energy is removed from a system, the mass associated with the energy is also removed, and the system therefore loses mass. This mass defect in the system may be simply calculated as {{math|1=Δ''m'' = {{sfrac|Δ''E''|''c''<sup>2</sup>}}}}, and this was the form of the equation historically first presented by Einstein in 1905. However, use of this formula in such circumstances has led to the false idea that mass has been "converted" to energy. This may be particularly the case when the energy (and mass) removed from the system is associated with the ''binding energy'' of the system. In such cases, the binding energy is observed as a "mass defect" or deficit in the new system.

The fact that the released energy is not easily weighed in many such cases, may cause its mass to be neglected as though it no longer existed. This circumstance has encouraged the false idea of conversion of ''mass'' to energy, rather than the correct idea that the binding energy of such systems is relatively large, and exhibits a measurable mass, which is removed when the binding energy is removed. {{citation needed|date=April 2015}}.

The difference between the rest mass of a bound system and of the unbound parts is the [[binding energy]] of the system, if this energy has been removed after binding. For example, a water molecule weighs a little less than two free hydrogen atoms and an oxygen atom. The minuscule mass difference is the energy needed to split the molecule into three individual atoms (divided by {{math|''c''<sup>2</sup>}}), which was given off as heat when the molecule formed (this heat had mass). Likewise, a stick of dynamite in theory weighs a little bit more than the fragments after the explosion, but this is true only so long as the fragments are cooled and the heat removed. In this case the mass difference is the energy/heat that is released when the dynamite explodes, and when this heat escapes, the mass associated with it escapes, only to be deposited in the surroundings, which absorb the heat (so that total mass is conserved).

Such a change in mass may only happen when the system is open, and the energy and mass escapes. Thus, if a stick of dynamite is blown up in a hermetically sealed chamber, the mass of the chamber and fragments, the heat, sound, and light would still be equal to the original mass of the chamber and dynamite. If sitting on a scale, the weight and mass would not change. This would in theory also happen even with a nuclear bomb, if it could be kept in an ideal box of infinite strength, which did not rupture or pass radiation.<ref name="A. Wheeler, 1992. pp. 248"/> Thus, a 21.5&nbsp;[[TNT equivalent|kiloton]] ({{val|9|e=13|u=joule}}) nuclear bomb produces about one gram of heat and electromagnetic radiation, but the mass of this energy would not be detectable in an exploded bomb in an ideal box sitting on a scale; instead, the contents of the box would be heated to millions of degrees without changing total mass and weight. If then, however, a transparent window (passing only electromagnetic radiation) were opened in such an ideal box after the explosion, and a beam of X-rays and other lower-energy light allowed to escape the box, it would eventually be found to weigh one gram less than it had before the explosion. This weight loss and mass loss would happen as the box was cooled by this process, to room temperature. However, any surrounding mass that absorbed the X-rays (and other "heat") would ''gain'' this gram of mass from the resulting heating, so the mass "loss" would represent merely its relocation. Thus, no mass (or, in the case of a nuclear bomb, no matter) would be "converted" to energy in such a process. Mass and energy, as always, would both be separately conserved.

===Massless particles===
Massless particles have zero rest mass. Their relativistic mass is simply their relativistic energy, divided by {{math|''c''<sup>2</sup>}}, or {{math|1=''m''{{ssub|rel}} = {{Sfrac|''E''|''c''<sup>2</sup>}}}}.<ref>
{{Citation
 |title=Basic relativity
 |edition=2
 |first1=Richard A.
 |last1=Mould
 |publisher=Springer
 |date=2002
 |isbn=0-387-95210-1
 |page=126
 |url=https://books.google.com/books?id=lfGE-wyJYIUC
}},
[https://books.google.com/books?id=lfGE-wyJYIUC&pg=PA126 Chapter 5 page 126]
</ref><ref>
{{Citation
 |title=Introduction to electromagnetic theory: a modern perspective
 |first1=Tail L.
 |last1=Chow
 |publisher=Jones & Bartlett Learning
 |date=2006
 |isbn=0-7637-3827-1
 |page=392
 |url=https://books.google.com/books?id=dpnpMhw1zo8C
}},
[https://books.google.com/books?id=dpnpMhw1zo8C&pg=PA392 Chapter 10 page 392]
</ref> The energy for photons is {{math|1=''E'' = ''hf''}}, where {{mvar|h}} is [[Planck constant|Planck's constant]] and {{mvar|f}} is the photon frequency. This frequency and thus the relativistic energy are frame-dependent.

If an observer runs away from a photon in the direction the photon travels from a source, and it catches up with the observer—when the photon catches up, the observer sees it as having less energy than it had at the source. The faster the observer is traveling with regard to the source when the photon catches up, the less energy the photon has. As an observer approaches the speed of light with regard to the source, the photon looks redder and redder, by [[relativistic Doppler effect]] (the Doppler shift is the relativistic formula), and the energy of a very long-wavelength photon approaches zero. This is because the photon is ''massless''—the rest mass of a photon is zero.

===Massless particles contribute rest mass and invariant mass to systems===
Two photons moving in different directions cannot both be made to have arbitrarily small total energy by changing frames, or by moving toward or away from them. The reason is that in a two-photon system, the energy of one photon is decreased by chasing after it, but the energy of the other increases with the same shift in observer motion. Two photons not moving in the same direction comprise an [[inertial frame]] where the combined energy is smallest, but not zero. This is called the [[center of mass]] frame or the [[center of momentum]] frame; these terms are almost synonyms (the center of mass frame is the special case of a center of momentum frame where the center of mass is put at the origin). The most that chasing a pair of photons can accomplish to decrease their energy is to put the observer in a frame where the photons have equal energy and are moving directly away from each other. In this frame, the observer is now moving in the same direction and speed as the center of mass of the two photons. The total momentum of the photons is now zero, since their momenta are equal and opposite. In this frame the two photons, as a system, have a mass equal to their total energy divided by {{math|''c''<sup>2</sup>}}. This mass is called the [[invariant mass]] of the pair of photons together. It is the smallest mass and energy the system may be seen to have, by any observer. It is only the invariant mass of a two-photon system that can be used to make a single particle with the same rest mass.

If the photons are formed by the collision of a particle and an antiparticle, the invariant mass is the same as the total energy of the particle and antiparticle (their rest energy plus the kinetic energy), in the center of mass frame, where they automatically move in equal and opposite directions (since they have equal momentum in this frame). If the photons are formed by the disintegration of a ''single'' particle with a well-defined rest mass, like the neutral [[pion]], the invariant mass of the photons is equal to rest mass of the pion. In this case, the center of mass frame for the pion is just the frame where the pion is at rest, and the center of mass does not change after it disintegrates into two photons. After the two photons are formed, their center of mass is still moving the same way the pion did, and their total energy in this frame adds up to the mass energy of the pion. Thus, by calculating the [[invariant mass]] of pairs of photons in a particle detector, pairs can be identified that were probably produced by pion disintegration.

A similar calculation illustrates that the invariant mass of systems is conserved, even when massive particles (particles with rest mass) within the system are converted to massless particles (such as photons). In such cases, the photons contribute invariant mass to the system, even though they individually have no invariant mass or rest mass. Thus, an electron and positron (each of which has rest mass) may undergo [[electron–positron annihilation|annihilation]] with each other to produce two photons, each of which is massless (has no rest mass). However, in such circumstances, no system mass is lost. Instead, the system of both photons moving away from each other has an invariant mass, which acts like a rest mass for any system in which the photons are trapped, or that can be weighed. Thus, not only the quantity of relativistic mass, but also the quantity of invariant mass does not change in transformations between "matter" (electrons and positrons) and energy (photons).

===Relation to gravity===
In physics, there are two distinct concepts of [[mass]]: the gravitational mass and the inertial mass. The gravitational mass is the quantity that determines the strength of the [[gravitational field]] generated by an object, as well as the gravitational force acting on the object when it is immersed in a gravitational field produced by other bodies. The inertial mass, on the other hand, quantifies how much an object accelerates if a given force is applied to it. The mass–energy equivalence in special relativity refers to the inertial mass. However, already in the context of Newton gravity, the Weak [[Equivalence Principle]] is postulated: the gravitational and the inertial mass of every object are the same. Thus, the mass–energy equivalence, combined with the Weak Equivalence Principle, results in the prediction that all forms of energy contribute to the gravitational field generated by an object. This observation is one of the pillars of the [[general theory of relativity]].

The above prediction, that all forms of energy interact gravitationally, has been subject to experimental tests. The first observation testing this prediction was made in 1919.<ref>{{Citation |last=Dyson|first =F.W. |author2= Eddington, A.S. |author3=Davidson, C.R. |lastauthoramp=yes |date=1920 |title=A Determination of the Deflection of Light by the Sun's Gravitational Field, from Observations Made at the Solar eclipse of May 29, 1919|journal= [[Philosophical Transactions of the Royal Society A: Mathematical, Physical and Engineering Sciences|Phil. Trans. Roy. Soc. A]]|volume=220|issue=571–581|pages= 291–333|bibcode=1920RSPTA.220..291D|doi=10.1098/rsta.1920.0009|url =https://zenodo.org/record/1432106 }}</ref> During a [[Solar eclipse of May 29, 1919|solar eclipse]], [[Arthur Eddington]] observed that the light from stars passing close to the Sun was bent. The effect is due to the gravitational attraction of light by the Sun. The observation confirmed that the energy carried by light indeed is equivalent to a gravitational mass. Another seminal experiment, the [[Pound–Rebka experiment]], was performed in 1960.<ref>{{Citation|last=Pound| first=R. V.| author2 = Rebka Jr. G. A. | date= April 1, 1960| title=Apparent weight of photons| journal=[[Physical Review Letters]]| volume=4| issue=7|  pages=337–341| doi = 10.1103/PhysRevLett.4.337| bibcode=1960PhRvL...4..337P}}</ref> In this test a beam of light was emitted from the top of a tower and detected at the bottom. The frequency of the light detected was higher than the light emitted. This result confirms that the energy of photons increases when they fall in the gravitational field of the Earth. The energy, and therefore the gravitational mass, of photons is proportional to their frequency as stated by the [[Planck constant|Planck's relation]].

==Application to nuclear physics==
[[File:USS Enterprise (CVAN-65), USS Long Beach (CGN-9) and USS Bainbridge (DLGN-25) underway in the Mediterranean Sea during Operation Sea Orbit, in 1964.jpg|thumb|right|Task Force One, the world's first nuclear-powered task force. {{USS|Enterprise|CVN-65|2}}, {{USS|Long Beach|CGN-9|2}} and {{USS|Bainbridge|CGN-25|2}} in formation in the Mediterranean, 18 June 1964. ''Enterprise'' crew members are spelling out Einstein's mass–energy equivalence formula {{math|1=''E'' = ''mc''<sup>2</sup>}} on the flight deck.]]

[[Max Planck]] pointed out that the mass–energy equivalence formula implied{{how|date=November 2019}} that bound systems would have a mass less than the sum of their constituents, once the binding energy had been allowed to escape. However, Planck was thinking about chemical reactions, where the binding energy is too small to measure. Einstein suggested that radioactive materials such as [[radium]] would provide a test of the theory, but even though a large amount of energy is released per atom in radium, due to the [[half-life]] of the substance (1602 years), only a small fraction of radium atoms decay over an experimentally measurable period of time.

Once the nucleus was discovered, experimenters realized that the very high binding energies of the atomic nuclei should allow calculation of their binding energies, simply from mass differences. But it was not until the discovery of the [[neutron]] in 1932, and the measurement of the neutron mass, that this calculation could actually be performed (see [[nuclear binding energy]] for example calculation). A little while later, the [[Cockcroft–Walton accelerator]] produced the first [[Nuclear transmutation|transmutation]] reaction ({{nowrap|{{su|p=7|b=3}}Li + {{su|p=1|b=1}}p → 2 {{su|p=4|b=2}}He}}), verifying Einstein's formula to an accuracy of ±0.5%.{{citation needed|date=November 2019}}
In 2005, Rainville et al. published a direct test of the energy-equivalence of mass lost in the binding energy of a neutron to atoms of particular isotopes of silicon and sulfur, by comparing the mass lost to the energy of the emitted gamma ray associated with the neutron capture. The binding mass-loss agreed with the gamma ray energy to a precision of ±0.00004%, the most accurate test of {{math|1=''E'' = ''mc''<sup>2</sup>}} to date.<ref name="rainville">Rainville, S. et al. World Year of Physics: A direct test of {{math|1=E = mc<sup>2</sup>}}. ''Nature'' 438, 1096–1097 (22 December 2005) | {{doi|10.1038/4381096a}}; Published online 21 December 2005.</ref>

The mass–energy equivalence formula was used in the understanding of [[nuclear fission]] reactions, and implies the great amount of energy that can be released by a [[nuclear fission]] [[chain reaction]], used in both [[nuclear weapon]]s and [[nuclear power]]. By measuring the mass of different [[atomic nuclei]] and subtracting from that number the total mass of the [[protons]] and [[neutrons]] as they would weigh separately, one gets the exact [[binding energy]] available in an [[atomic nucleus]]. This is used to calculate the energy released in any [[nuclear reaction]], as the difference in the total mass of the nuclei that enter and exit the reaction.

==Practical examples==
Einstein used the [[Centimeter gram second system of units|CGS]] system of units (centimeters, grams, seconds, dynes, and ergs), but the formula is independent of the system of units. In [[natural units]], the numerical value of the speed of light is set to equal 1, and the formula expresses an equality of numerical values: {{math|1=''E'' = ''m''}}. In the [[International System of Units|SI]] system (expressing the ratio {{math|{{sfrac|''E''|''m''}}}} in [[joules]] per kilogram using the value of {{math|''c''}} in [[metre per second|meters per second]]):<ref>{{cite book |title=Megawatts and Megatons: The Future of Nuclear Power and Nuclear Weapons |edition=illustrated |first1=Richard L. |last1=Garwin |first2=Georges |last2=Charpak |publisher=University of Chicago Press |year=2002 |isbn=978-0-226-28427-9 |page=17 |url=https://books.google.com/books?id=1YgBR6shTckC}} [https://books.google.com/books?id=1YgBR6shTckC&pg=PA17 Extract of page 17]</ref>

:{{math|1={{Sfrac|''E''|''m''}} =}} {{math|1=''c''<sup>2</sup> = ({{val|299792458|u=m/s}})<sup>2</sup> =}} {{math|{{val|89875517873681764|u=J/kg}}}} (≈&nbsp;9.0 × 10<sup>16</sup> joules per kilogram).

So the energy equivalent of one kilogram of mass is
*89.9&nbsp;[[joules|petajoules]]
*25.0&nbsp;billion [[kilowatt-hour]]s (≈&nbsp;25,000&nbsp;[[GW·h]])
*21.5&nbsp;trillion [[calorie|kilocalories]] (≈&nbsp;21&nbsp;Pcal)<ref name="Conversion">Conversions used: 1956 International (Steam) Table (IT) values where one calorie ≡&nbsp;4.1868&nbsp;J and one BTU ≡&nbsp;1055.05585262&nbsp;J. Weapons designers' conversion value of one&nbsp;gram TNT ≡ 1000&nbsp;calories used.</ref>
*85.2&nbsp;trillion [[British thermal unit|BTUs]]<ref name="Conversion"/>
*0.0852 [[Quad (unit)|quads]]

or the energy released by combustion of the following:
*21 500&nbsp;[[kiloton]]s of [[TNT equivalent|TNT-equivalent]] energy (≈&nbsp;21&nbsp;Mt)<ref name="Conversion"/>
*{{val|2630000000}} [[litre]]s or {{val|695000000}} US [[gallon]]s of automotive [[Gasoline#Energy content (high and low heating value)|gasoline]]

Any time energy is generated, the process can be evaluated from an {{math|1=''E'' = ''mc''<sup>2</sup>}} perspective. For instance, the "[[Fat Man|Gadget]]"-style bomb used in the [[Trinity test]] and the [[bombing of Nagasaki]] had an explosive yield equivalent to 21&nbsp;kt of TNT. About 1&nbsp;kg of the approximately 6.15&nbsp;kg of plutonium in each of these bombs fissioned into lighter elements totaling almost exactly one gram less, after cooling. The electromagnetic radiation and kinetic energy (thermal and blast energy) released in this explosion carried the missing one gram of mass.<ref>The 6.2&nbsp;kg core comprised 0.8% gallium by weight. Also, about 20% of the Gadget's yield was due to fast fissioning in its natural uranium tamper. This resulted in 4.1&nbsp;moles of plutonium fissioning with 180&nbsp;MeV per atom actually contributing prompt kinetic energy to the explosion. Note too that the term ''"Gadget"-style'' is used here instead of "Fat Man" because this general design of bomb was very rapidly upgraded to a more efficient one requiring only 5&nbsp;kg of the plutonium–gallium alloy.{{citation needed|date=August 2012}}</ref> This occurs because nuclear [[binding energy]] is released whenever elements with more than 62 nucleons fission.{{citation needed|date=August 2012}}

Another example is [[hydroelectricity|hydroelectric generation]]. The electrical energy produced by [[Grand Coulee Dam]]'s [[Water turbine|turbines]] every 3.7&nbsp;hours represents one gram of mass. This mass passes to electrical devices (such as lights in cities) powered by the generators, where it appears as a gram of heat and light.<ref>Assuming the dam is generating at its peak capacity of 6,809&nbsp;MW.{{citation needed|date=August 2012}}</ref> Turbine designers look at their equations in terms of pressure, torque, and RPM. However, Einstein's equations show that all energy has mass, and thus the electrical energy produced by a dam's generators, and the resulting heat and light, all retain their mass—which is equivalent to the energy. The potential energy—and equivalent mass—represented by the waters of the [[Columbia River]] as it descends to the Pacific Ocean would be converted to heat due to [[viscous friction]] and the [[turbulence]] of white water rapids and waterfalls were it not for the dam and its generators. This heat would remain as mass on site at the water, were it not for the equipment that converted some of this potential and kinetic energy into electrical energy, which can move from place to place (taking mass with it).{{citation needed|date=August 2012}}

Whenever energy is added to a system, the system gains mass, as shown when the equation is rearranged:
* A spring's mass increases whenever it is put into compression or tension. Its added mass arises from the added potential energy stored within it, which is bound in the stretched chemical (electron) bonds linking the atoms within the spring.
* Raising the temperature of an object (increasing its heat energy) increases its mass. For example, consider the world's primary mass standard for the kilogram, made of platinum/iridium. If its temperature is allowed to change by 1&nbsp;°C, its mass changes by 1.5 picograms (1&nbsp;pg = {{val|1|e=-12|u=g}}).<ref>Assuming a 90/10 alloy of Pt/Ir by weight, a {{math|''C<sub>p</sub>''}} of 25.9 for Pt and 25.1 for Ir, a Pt-dominated average {{math|''C<sub>p</sub>''}} of 25.8, 5.134&nbsp;moles of metal, and 132&nbsp;J⋅K<sup>−1</sup> for the prototype. A variation of ±1.5&nbsp;picograms is of course, much smaller than the actual uncertainty in the mass of the international prototype, which is ±2&nbsp;micrograms.</ref>
* A spinning ball weighs more than a ball that is not spinning. Its increase of mass is exactly the equivalent of the mass of energy of rotation, which is itself the sum of the kinetic energies of all the moving parts of the ball. For example, [[the Earth]] itself is more massive due to its rotation, than it would be with no rotation. This [[rotational energy]] ({{val|2.14|e=29|u=J}}) represents 2.38 billion metric tons of added mass.<ref>InfraNet Lab (2008-12-07). Harnessing the Energy from the Earth's Rotation. Article on Earth rotation energy. Divided by c^2. InfraNet Lab, 7 December 2008. Retrieved from {{cite web |url=http://infranetlab.org/blog/harnessing-energy-earth%E2%80%99s-rotation |title=Archived copy |accessdate=2015-03-26 |url-status=dead |archiveurl=https://web.archive.org/web/20150402111115/http://infranetlab.org/blog/harnessing-energy-earth%E2%80%99s-rotation |archivedate=2015-04-02 }}</ref>

Note that no net mass or energy is really created or lost in any of these examples and scenarios. Mass/energy simply moves from one place to another. These are some examples of the ''transfer'' of energy and mass in accordance with the ''principle of mass–energy conservation''.{{citation needed|date=August 2012}}

==Efficiency==
{{more citations needed|date=October 2019}}
Although mass cannot be converted to energy,<ref name="A. Wheeler, 1992. pp. 248"/> in some reactions matter particles (which contain a form of rest energy) can be destroyed and the energy released can be converted to other types of energy that are more usable and obvious as forms of energy—such as light and energy of motion (heat, etc.). However, the total amount of energy and mass does not change in such a transformation. Even when particles are not destroyed, a certain fraction of the ill-defined "matter" in ordinary objects can be destroyed, and its associated energy liberated and made available as the more dramatic energies of light and heat, even though no identifiable real particles are destroyed, and even though (again) the total energy is unchanged (as also the total mass). Such conversions between types of energy (resting to active energy) happen in nuclear weapons, in which the protons and neutrons in atomic nuclei lose a small fraction of their average mass, but this mass loss is not due to the destruction of any protons or neutrons (or even, in general, lighter particles like electrons). Also the mass is not destroyed, but simply removed from the system in the form of heat and light from the reaction.

In nuclear reactions, typically only a small fraction of the total mass–energy of the bomb converts into the mass–energy of heat, light, radiation, and motion—which are "active" forms that can be used. When an atom fissions, it loses only about 0.1% of its mass (which escapes from the system and does not disappear), and additionally, in a bomb or reactor not all the atoms can fission. In a modern fission-based atomic bomb, the efficiency is only about 40%, so only 40% of the fissionable atoms actually fission, and only about 0.03% of the fissile core mass appears as energy in the end. In nuclear fusion, more of the mass is released as usable energy, roughly 0.3%. But in a fusion bomb, the bomb mass is partly casing and non-reacting components, so that in practicality, again (coincidentally) no more than about 0.03% of the total mass of the entire weapon is released as usable energy (which, again, retains the "missing" mass). See [[nuclear weapon yield]] for practical details of this ratio in modern nuclear weapons.

In theory, it should be possible to destroy matter and convert all of the rest-energy associated with matter into heat and light (which would of course have the same mass), but none of the theoretically known methods are practical. One way to convert all the energy within matter into usable energy is to annihilate matter with [[antimatter]]. But [[baryon asymmetry|antimatter is rare in our universe]], and must be made first. Due to inefficient mechanisms of production, making antimatter always requires far more usable energy than would be released when it was annihilated.

Since most of the mass of ordinary objects resides in protons and neutrons, converting all the energy of ordinary matter into more useful energy requires that the protons and neutrons be converted to lighter particles, or particles with no rest-mass at all. In the [[Standard Model]] of particle physics, the [[baryon number|number of protons plus neutrons]] is nearly exactly conserved. Still, [[Gerard 't Hooft]] showed that there is a process that converts protons and neutrons to antielectrons and neutrinos.<ref>G. 't Hooft, "Computation of the quantum effects due to a four-dimensional pseudoparticle", Physical Review D14:3432–3450 (1976).</ref> This is the weak SU(2) [[instanton]] proposed by Belavin Polyakov Schwarz and Tyupkin.<ref>A. Belavin, A. M. Polyakov, A. Schwarz, Yu. Tyupkin, "Pseudoparticle Solutions to Yang Mills Equations", Physics Letters 59B:85 (1975).</ref> This process, can in principle destroy matter and convert all the energy of matter into neutrinos and usable energy, but it is normally extraordinarily slow. Later it became clear that this process happens at a fast rate at very high temperatures,<ref>F. Klinkhammer, [[Nicholas Manton|N. Manton]], "A Saddle Point Solution in the Weinberg Salam Theory", Physical Review D 30:2212.</ref> since then, instanton-like configurations are copiously produced from [[statistical mechanics|thermal fluctuations]]. The temperature required is so high that it would only have been reached shortly after the [[Big Bang]].

Many extensions of the standard model contain [[magnetic monopole]]s, and in some models of [[grand unification theory|grand unification]], these monopoles catalyze [[proton decay]], a process known as the [[Callan-Rubakov effect]].<ref>Rubakov V. A. "Monopole Catalysis of Proton Decay", Reports on Progress in Physics 51:189–241 (1988).</ref> This process would be an efficient mass–energy conversion at ordinary temperatures, but it requires making [[Magnetic monopole|monopoles]] and anti-monopoles first. The energy required to produce monopoles is believed to be enormous, but magnetic charge is conserved, so that the lightest monopole is stable. All these properties are deduced in theoretical models—magnetic monopoles have never been observed, nor have they been produced in any experiment so far.

A third known method of total matter–energy "conversion" (which again in practice only means conversion of one type of energy into a different type of energy), is using gravity, specifically black holes. [[Stephen Hawking]] theorized<ref>S.W. Hawking "Black Holes Explosions?" ''Nature'' 248:30 (1974).</ref> that black holes radiate thermally with no regard to how they are formed. So, it is theoretically possible to throw matter into a black hole and use the emitted heat to generate power. According to the theory of [[Hawking radiation]], however, the black hole used radiates at a higher rate the smaller it is, producing usable powers at only small black hole masses, where usable may for example be something greater than the local background radiation. It is also worth noting that the ambient irradiated power would change with the mass of the black hole, increasing as the mass of the black hole decreases, or decreasing as the mass increases, at a rate where power is proportional to the inverse square of the mass. In a "practical" scenario, mass and energy could be dumped into the black hole to regulate this growth, or keep its size, and thus power output, near constant. This could result from the fact that mass and energy are lost from the hole with its thermal radiation.

==Background==

===Mass–velocity relationship===
In developing [[special relativity]], Einstein found that the [[kinetic energy#Relativistic kinetic energy of rigid bodies|kinetic energy]] of a moving body is
:<math>E_k =  m_0 c^2( \gamma -1 ) = m_0 c^2\left(\frac{1}{\sqrt{1-\frac{v^2}{c^2}}} - 1\right),</math>

with {{math|''v''}} the [[velocity]], {{math|''m''{{sub|0}}}} the rest mass, and {{math|''γ''}} the [[Lorentz factor]].

He included the second term on the right to make sure that for small velocities the energy would be the same as in classical mechanics, thus satisfying the [[correspondence principle]]:
:<math>E_k = \frac{1}{2}m_0 v^2 + \cdots </math>

Without this second term, there would be an additional contribution in the energy when the particle is not moving.

Einstein found that the [[Momentum#Modern definitions of momentum|total momentum]] of a moving particle is:
:<math>P = \frac{m_0 v}{\sqrt{1-\frac{v^2}{c^2}}}. </math>

It is this quantity that is conserved in collisions. The ratio of the momentum to the velocity is the [[relativistic mass]], {{mvar|m}}.
:<math>m = \frac{m_0}{\sqrt{1-\frac{v^2}{c^2}}}</math>

And the relativistic mass and the relativistic kinetic energy are related by the formula:
:<math>E_k = m c^2 - m_0 c^2. \,</math>

Einstein wanted to omit the unnatural second term on the right-hand side, whose only purpose is to make the energy at rest zero, and to declare that the particle has a total energy, which obeys:
:<math> E = m c^2 \,</math>

which is a sum of the rest energy {{math|''m''<sub>0</sub>''c''<sup>2</sup>}} and the kinetic energy. This total energy is mathematically more elegant, and fits better with the momentum in relativity. But to come to this conclusion, Einstein needed to think carefully about collisions. This expression for the energy implied that matter at rest has a huge amount of energy, and it is not clear whether this energy is physically real, or just a mathematical artifact with no physical meaning.

In a collision process where all the rest-masses are the same at the beginning as at the end, either expression for the energy is conserved. The two expressions only differ by a constant that is the same at the beginning and at the end of the collision. Still, by analyzing the situation where particles are thrown off a heavy central particle, it is easy to see that the inertia of the central particle is reduced by the total energy emitted. This allowed Einstein to conclude that the inertia of a heavy particle is increased or diminished according to the energy it absorbs or emits.

===Relativistic mass===
{{Main|Mass in special relativity}}

After Einstein first made his proposal, it became clear that the word mass can have two different meanings. Some denote the ''relativistic mass'' with an explicit index:
:<math>m_{\mathrm{rel}} = \frac{m_0}{\sqrt{1-\frac{v^2}{c^2}}} .</math>

This mass is the ratio of momentum to velocity, and it is also the relativistic energy divided by {{math|1=''c''<sup>2</sup>}} (it is not Lorentz-invariant, in contrast to <math>m_0</math>). The equation {{math|1=''E'' = ''m''<sub>rel</sub>''c''<sup>2</sup>}} holds for moving objects. When the velocity is small, the relativistic mass and the rest mass are almost exactly the same.
* {{math|1=''E'' = ''mc''<sup>2</sup>}} either means {{math|1=''E'' = ''m''<sub>0</sub>''c''<sup>2</sup>}} for an object at rest, or {{math|1=''E'' = ''m''<sub>rel</sub>''c''<sup>2</sup>}} when the object is moving.

Also Einstein (following [[Hendrik Lorentz]] and [[Max Abraham]]) used velocity- and direction-dependent mass concepts ([[Mass in special relativity#Early developments: transverse and longitudinal mass|longitudinal and transverse mass]]) in his 1905 electrodynamics paper and in another paper in 1906.<ref>{{Citation | author=Einstein, A. | date=1905 | title=Zur Elektrodynamik bewegter Körper | journal=Annalen der Physik | volume=17 | pages=891–921 | doi=10.1002/andp.19053221004 | url=http://www.physik.uni-augsburg.de/annalen/history/papers/1905_17_891-921.pdf | bibcode=1905AnP...322..891E | issue=10 | url-status=dead | archiveurl=https://web.archive.org/web/20080227120546/http://www.physik.uni-augsburg.de/annalen/history/papers/1905_17_891-921.pdf | archivedate=2008-02-27 }}. [http://www.fourmilab.ch/etexts/einstein/specrel/www/ English translation.]</ref><ref>{{Citation | author=Einstein, A. | date=1906 | title=Über eine Methode zur Bestimmung des Verhältnisses der transversalen und longitudinalen Masse des Elektrons | journal=Annalen der Physik | volume=21 | pages=583–586 | doi=10.1002/andp.19063261310 | url=http://www.physik.uni-augsburg.de/annalen/history/papers/1906_21_583-586.pdf | bibcode=1906AnP...326..583E | issue=13 | url-status=dead | archiveurl=https://web.archive.org/web/20080229114212/http://www.physik.uni-augsburg.de/annalen/history/papers/1906_21_583-586.pdf | archivedate=2008-02-29 }}</ref>
However, in his first paper on {{math|1=''E'' = ''mc''<sup>2</sup>}} (1905), he treated {{mvar|m}} as what would now be called the ''rest mass''.<ref name="inertia" /> Some claim that (in later years) he did not like the idea of "relativistic mass".<ref name=Okun>See e.g. Lev B.Okun, ''The concept of Mass'', Physics Today '''42''' (6), June 1969, p. 31–36, http://www.physicstoday.org/vol-42/iss-6/vol42no6p31_36.pdf</ref>&nbsp; When modern physicists say "mass", they are usually talking about rest mass, since if they meant "relativistic mass", they would just say "energy".

Considerable debate has ensued over the use of the concept "relativistic mass" and the connection of "mass" in relativity to "mass" in Newtonian dynamics. For example, one view is that only rest mass is a viable concept and is a property of the particle; while relativistic mass is a conglomeration of particle properties and properties of spacetime. A perspective that avoids this debate, due to Kjell Vøyenli, is that the Newtonian concept of mass as a particle property and the relativistic concept of mass have to be viewed as embedded in their own theories and as having no precise connection.<ref name="Jammer">{{Citation |title=Concepts of mass in contemporary physics and philosophy |author=Max Jammer |url=https://books.google.com/?id=jujK1bn4QUQC&pg=PA51 |page=51 |isbn=0-691-01017-X |publisher=Princeton University Press |date=1999 }}</ref><ref name="Vøyenli">

{{Citation |journal=Foundations of Physics |doi=10.1007/BF00708670 |title= The classical and relativistic concepts of mass |date=1976 |author1= Eriksen, Erik |author2=Vøyenli, Kjell |volume=6 |issue=1 |pages=115–124 |publisher=Springer |bibcode=1976FoPh....6..115E}}</ref>

===Low speed expansion===
We can rewrite the expression {{math|1=''E'' = ''γm''<sub>0</sub>''c''<sup>2</sup>}} as a [[Taylor series]]:
:<math>E = m_0 c^2 \left[1 + \frac{1}{2} \left(\frac{v}{c}\right)^2 + \frac{3}{8} \left(\frac{v}{c}\right)^4 + \frac{5}{16} \left(\frac{v}{c}\right)^6 + \ldots \right]. </math>

For speeds much smaller than the speed of light, higher-order terms in this expression get smaller and smaller because {{math|{{sfrac|''v''|''c''}}}} is small. For low speeds we can ignore all but the first two terms:
:<math>E \approx m_0 c^2 + \frac{1}{2} m_0 v^2 . </math>

The total energy is a sum of the rest energy and the [[classical mechanics|Newtonian]] [[kinetic energy]].

The classical energy equation ignores both the {{math|''m''<sub>0</sub>''c''<sup>2</sup>}} part, and the high-speed corrections. This is appropriate, because all the high-order corrections are small. Since only ''changes'' in energy affect the behavior of objects, whether we include the {{math|''m''<sub>0</sub>''c''<sup>2</sup>}} part makes no difference, since it is constant. For the same reason, it is possible to subtract the rest energy from the total energy in relativity. By considering the emission of energy in different frames, Einstein could show that the rest energy has a real physical meaning.

The higher-order terms are extra corrections to Newtonian mechanics, and become important at higher speeds. The Newtonian equation is only a low-speed approximation, but an extraordinarily good one. All of the calculations used in putting astronauts on the moon, for example, could have been done using Newton's equations without any of the higher-order corrections.{{citation needed|date=May 2014}}
The total mass energy equivalence should also include the rotational and vibrational kinetic energies as well as the linear kinetic energy at low speeds.

==History==
While Einstein was the first to have correctly deduced the mass–energy equivalence formula, he was not the first to have related energy with mass. But nearly all previous authors thought that the energy that contributes to mass comes only from electromagnetic fields.<ref name="jann">{{Citation |author1 =Jannsen, M. |author2 =Mecklenburg, M. | date=2007 | title=From classical to relativistic mechanics: Electromagnetic models of the electron. | editor=V. F. Hendricks| journal=Interactions: Mathematics, Physics and Philosophy | pages=65–134 | place=Dordrecht | publisher=Springer | url =http://www.tc.umn.edu/~janss011/ |display-editors=etal}}</ref><ref name="whit">{{Citation |author =Whittaker, E.T. | date=1951–1953 | title= 2. Edition: A History of the theories of aether and electricity, vol. 1: The classical theories / vol. 2: The modern theories 1900–1926 | place=London | publisher=Nelson}}</ref><ref name="mill">{{Citation|author=Miller, Arthur I.|date=1981|title=Albert Einstein's special theory of relativity. Emergence (1905) and early interpretation (1905–1911)|place=Reading|publisher=Addison–Wesley|isbn=0-201-04679-2|url-access=registration|url=https://archive.org/details/alberteinsteinss0000mill}}</ref><ref name="darr">{{Citation | author=Darrigol, O. | title=The Genesis of the theory of relativity | date=2005 | journal=Séminaire Poincaré | volume=1 | pages=1–22| url=http://www.bourbaphy.fr/darrigol2.pdf| doi=10.1007/3-7643-7436-5_1| isbn=978-3-7643-7435-8 }}</ref>

===Newton: matter and light===
In 1717 [[Isaac Newton]] speculated that light particles and matter particles were interconvertible in "Query 30" of the ''[[Opticks]]'', where he asks:

{{quote|Are not the gross bodies and light convertible into one another, and may not bodies receive much of their activity from the particles of light which enter their composition?}}

===Swedenborg: matter composed of "pure and total motion"===
In 1734 the Swedish scientist and theologian [[Emanuel Swedenborg]] in his ''[[The Principia (book)|Principia]]'' theorized that all matter is ultimately composed of dimensionless points of "pure and total motion".  He described this motion as being without force, direction or speed, but having the potential for force, direction and speed everywhere within it.<ref>{{Citation|last=Swedenborg|first=Emanuel|title=Principia Rerum Naturalia|date=1734|location=Leipzig|language=Latin|chapter=De Simplici Mundi vel Puncto naturali|page=32}}</ref><ref>{{Citation|last=Swedenborg|first=Emanuel|title=The Principia; or The First Principles of Natural Things|date=1845|publisher=W. Newbery|location=London|pages=55–57|others=Translated by [[Augustus Clissold]]}}</ref>

===Electromagnetic mass===
{{Main|Electromagnetic mass}}
There were many attempts in the 19th and the beginning of the 20th century—like those of [[J. J. Thomson]] (1881), [[Oliver Heaviside]] (1888), and [[George Frederick Charles Searle]] (1897), [[Wilhelm Wien]] (1900), [[Max Abraham]] (1902), [[Hendrik Antoon Lorentz]] (1904) — to understand how the mass of a charged object depends on the electrostatic field.<ref name="jann"/><ref name="whit"/> This concept was called [[electromagnetic mass]], and was considered as being dependent on velocity and direction as well. Lorentz (1904) gave the following expressions for longitudinal and transverse electromagnetic mass:
:<math>m_{L}=\frac{m_{0}}{\left(\sqrt{1-\frac{v^{2}}{c^{2}}}\right)^{3}},\quad m_{T}=\frac{m_{0}}{\sqrt{1-\frac{v^{2}}{c^{2}}}} </math>,

where
:<math>m_{0}=\frac{4}{3}\frac{E_{em}}{c^{2}}</math>

===Radiation pressure and inertia===
{{Main|Electromagnetic mass#Inertia of energy and radiation paradoxes}}
Another way of deriving some sort of electromagnetic mass was based on the concept of [[radiation pressure]]. In 1900, [[Henri Poincaré]] associated electromagnetic radiation energy with a "fictitious fluid" having momentum and mass<ref name=action /> 
:<math>m_{em}=\frac{E_{em}}{c^2}\,.</math>

By that, Poincaré tried to save the [[center of mass]] theorem in Lorentz's theory, though his treatment led to radiation paradoxes.<ref name="darr" />

[[Friedrich Hasenöhrl]] showed in 1904, that electromagnetic [[cavity radiation]] contributes the "apparent mass"
:<math>m_{0}=\frac{4}{3}\frac{E_{em}}{c^{2}}</math>

to the cavity's mass. He argued that this implies mass dependence on temperature as well.<ref>{{Cite web|author=Philip Ball|title=Did Einstein discover {{math|1=''E'' = ''mc''<sup>2</sup>}}?|publisher=[[Physics World]]|date=Aug 23, 2011|url=http://physicsworld.com/cws/article/news/46941}}</ref>

===Einstein: mass–energy equivalence===
[[Albert Einstein]] did not formulate exactly the formula {{math|1=''E'' = ''mc''<sup>2</sup>}} in his 1905 [[Annus Mirabilis Papers|''Annus Mirabilis'' paper]] "Does the Inertia of an object Depend Upon Its Energy Content?";<ref name="inertia" /> rather, the paper states that if a body gives off the energy {{mvar|L}} in the form of radiation, its mass diminishes by {{math|{{sfrac|''L''|''c''<sup>2</sup>}}}}. (Here, "radiation" means [[electromagnetic radiation]], or light, and mass means the ordinary Newtonian mass of a slow-moving object.) This formulation relates only a change {{math|Δ''m''}} in mass to a change {{mvar|L}} in energy without requiring the absolute relationship.

Objects with zero mass presumably have zero energy, so the extension that all mass is proportional to energy is obvious from this result. In 1905, even the hypothesis that changes in energy are accompanied by changes in mass was untested. Not until the discovery of the first type of antimatter (the [[positron]] in 1932) was it found that all of the mass of pairs of resting particles could be converted to radiation.

====The first derivation by Einstein (1905)====
Already in his relativity paper "On the electrodynamics of moving bodies", Einstein derived the correct expression for the kinetic energy of particles:
:<math>E_{k}=mc^{2}\left(\frac{1}{\sqrt{1-\frac{v^{2}}{c^{2}}}}-1\right)</math>.

Now the question remained open as to which formulation applies to bodies at rest. This was tackled by Einstein in his paper "Does the inertia of a body depend upon its energy content?", where he used a body emitting two light pulses in opposite directions, having energies of {{math|''E''<sub>0</sub>}} before and {{math|''E''<sub>1</sub>}} after the emission as seen in its rest frame. As seen from a moving frame, this becomes {{math|''H''<sub>0</sub>}} and {{math|''H''<sub>1</sub>}}. Einstein obtained:
:<math>\left(H_{0}-E_{0}\right)-\left(H_{1}-E_{1}\right)=E\left(\frac{1}{\sqrt{1-\frac{v^{2}}{c^{2}}}}-1\right)</math>

then he argued that {{math|''H'' − ''E''}} can only differ from the kinetic energy {{math|''K''}} by an additive constant, which gives
:<math>K_{0}-K_{1}=E\left(\frac{1}{\sqrt{1-\frac{v^{2}}{c^{2}}}}-1\right)</math>

Neglecting effects higher than third order in {{math|{{sfrac|''v''|''c''}}}} after a [[Taylor series]] expansion of the right side of this gives:
:<math>K_{0}-K_{1}=\frac{E}{c^{2}}\frac{v^{2}}{2}.</math>

Einstein concluded that the emission reduces the body's mass by {{math|{{sfrac|''E''|''c''<sup>2</sup>}}}}, and that the mass of a body is a measure of its energy content.

The correctness of Einstein's 1905 derivation of {{math|1=''E'' = ''mc''<sup>2</sup>}} was criticized by [[Max Planck]] (1907), who argued that it is only valid to first approximation. Another criticism was formulated by [[Herbert Ives]] (1952) and [[Max Jammer]] (1961), asserting that Einstein's derivation is based on [[begging the question]].<ref name=jammer2>{{Citation|author=Jammer, Max|title=Concepts of Mass in Classical and Modern Physics|location=New York|publisher=Dover|origyear=1961|year=1997|isbn=0-486-29998-8}}</ref><ref>{{Citation | author=Ives, Herbert E. | date=1952 | title=Derivation of the mass–energy relation | journal=Journal of the Optical Society of America | volume= 42 | issue=8 | pages =540–543 | doi=10.1364/JOSA.42.000540}}</ref>
On the other hand, [[John Stachel]] and [[Roberto Torretti]] (1982) argued that Ives' criticism was wrong, and that Einstein's derivation was correct.<ref>{{Citation | author1=Stachel, John |author2=Torretti, Roberto | date=1982 | title=Einstein's first derivation of mass–energy equivalence | journal=American Journal of Physics | volume =50 | issue=8 | pages =760–763 | doi=10.1119/1.12764|bibcode = 1982AmJPh..50..760S }}</ref>
Hans Ohanian (2008) agreed with Stachel/Torretti's criticism of Ives, though he argued that Einstein's derivation was wrong for other reasons.<ref>{{Citation | author=Ohanian, Hans | date=2008 | title=Did Einstein prove E=mc2? | journal=Studies in History and Philosophy of Science Part B | volume =40 | issue=2 | pages =167–173 | doi=10.1016/j.shpsb.2009.03.002|arxiv=0805.1400| bibcode=2009SHPMP..40..167O }}</ref> For a recent review, see Hecht (2011).<ref name=hecht />

====Alternative version====
An alternative version of Einstein's [[thought experiment]] was proposed by [[Fritz Rohrlich]] (1990), who based his reasoning on the [[Doppler effect]].<ref>{{Citation | author=Rohrlich, Fritz | date=1990 | title=An elementary derivation of {{math|1=''E'' = ''mc''<sup>2</sup>}} | journal=American Journal of Physics | volume= 58 | issue=4 | pages =348–349 | doi=10.1119/1.16168|bibcode = 1990AmJPh..58..348R }}</ref>
Like Einstein, he considered a body at rest with mass {{mvar|M}}. If the body is examined in a frame moving with nonrelativistic velocity {{mvar|v}}, it is no longer at rest and in the moving frame it has momentum {{math|1=''P'' = ''Mv''}}. Then he supposed the body emits two pulses of light to the left and to the right, each carrying an equal amount of energy {{math|{{sfrac|''E''|2}}}}. In its rest frame, the object remains at rest after the emission since the two beams are equal in strength and carry opposite momentum.

However, if the same process is considered in a frame that moves with velocity {{math|''v''}} to the left, the pulse moving to the left is [[redshift]]ed, while the pulse moving to the right is [[blue shift]]ed. The blue light carries more momentum than the red light, so that the momentum of the light in the moving frame is not balanced: the light is carrying some net momentum to the right.

The object has not changed its velocity before or after the emission. Yet in this frame it has lost some right-momentum to the light. The only way it could have lost momentum is by losing mass. This also solves Poincaré's radiation paradox, discussed above.

The velocity is small, so the right-moving light is blueshifted by an amount equal to the nonrelativistic [[Doppler shift]] factor {{math|1 − {{sfrac|''v''|''c''}}}}. The momentum of the light is its energy divided by {{mvar|c}}, and it is increased by a factor of {{math|{{sfrac|''v''|''c''}}}}. So the right-moving light is carrying an extra momentum {{math|Δ''P''}} given by:
:<math> \Delta P = {v \over c}{E \over 2c} .</math>

The left-moving light carries a little less momentum, by the same amount {{math|Δ''P''}}. So the total right-momentum in the light is twice {{math|Δ''P''}}. This is the right-momentum that the object lost.
:<math> 2\Delta P = v {E\over c^2} .</math>

The momentum of the object in the moving frame after the emission is reduced to this amount:
:<math> P' = Mv - 2\Delta P = \left(M - {E\over c^2}\right)v .</math>

So the change in the object's mass is equal to the total energy lost divided by {{math|''c''<sup>2</sup>}}. Since any emission of energy can be carried out by a two step process, where first the energy is emitted as light and then the light is converted to some other form of energy, any emission of energy is accompanied by a loss of mass. Similarly, by considering absorption, a gain in energy is accompanied by a gain in mass.

====Relativistic center-of-mass theorem (1906)====
Like Poincaré, Einstein concluded in 1906 that the inertia of electromagnetic energy is a necessary condition for the center-of-mass theorem to hold. On this occasion, Einstein referred to Poincaré's 1900 paper and wrote:<ref>{{Citation|author=Einstein, A. |date=1906 |title=Das Prinzip von der Erhaltung der Schwerpunktsbewegung und die Trägheit der Energie |journal=Annalen der Physik |volume=20 |pages=627–633 |doi=10.1002/andp.19063250814 |url=http://www.physik.uni-augsburg.de/annalen/history/papers/1906_20_627-633.pdf |bibcode=1906AnP...325..627E |issue=8 |url-status=dead |archiveurl=https://web.archive.org/web/20060318060830/http://www.physik.uni-augsburg.de/annalen/history/papers/1906_20_627-633.pdf |archivedate=2006-03-18 }}</ref>

{{quote|Although the merely formal considerations, which we will need for the proof, are already mostly contained in a work by H. Poincaré<sup>2</sup>, for the sake of clarity I will not rely on that work.<ref>Einstein 1906: Trotzdem die einfachen formalen Betrachtungen, die zum Nachweis dieser Behauptung durchgeführt werden müssen, in der Hauptsache bereits in einer Arbeit von H. Poincaré enthalten sind<sup>2</sup>, werde ich mich doch der Übersichtlichkeit halber nicht auf jene Arbeit stützen.</ref>}}

In Einstein's more physical, as opposed to formal or mathematical, point of view, there was no need for fictitious masses. He could avoid the ''[[perpetual motion|perpetuum mobile]]'' problem because, on the basis of the mass–energy equivalence, he could show that the transport of inertia that accompanies the emission and absorption of radiation solves the problem. Poincaré's rejection of the principle of action–reaction can be avoided through Einstein's {{math|1=''E'' = ''mc''<sup>2</sup>}}, because mass conservation appears as a special case of the [[energy conservation law]].

===Others===
During the nineteenth century there were several speculative attempts to show that mass and energy were proportional in various [[Aether theories|ether theories]].<ref>Helge Kragh, "Fin-de-Siècle Physics: A World Picture in Flux" in ''Quantum Generations: A History of Physics in the Twentieth Century'' (Princeton, NJ: Princeton University Press, 1999.</ref> In 1873 [[Nikolay Umov]] pointed out a relation between mass and energy for ether in the form of {{math|1=''Е'' = ''kmc''<sup>2</sup>}}, where {{math|0.5 ≤ ''k'' ≤ 1}}.<ref>''Умов Н. А.'' Избранные сочинения. М. — Л., 1950. (Russian)</ref> The writings of [[Samuel Tolver Preston]],<ref>Preston, S. T., Physics of the Ether, E. & F. N. Spon, London, (1875).</ref><ref>Bjerknes: [http://itis.volta.alessandria.it/episteme/ep6/ep6-bjerk1.htm S. Tolver Preston's Explosive Idea ''E'' = ''mc''<sup>2</sup>.] {{webarchive|url=https://web.archive.org/web/20081012102907/http://itis.volta.alessandria.it/episteme/ep6/ep6-bjerk1.htm |date=2008-10-12 }}</ref> and a 1903 paper by [[Olinto De Pretto]],<ref name="math">MathPages: [http://www.mathpages.com/rr/s8-08/8-08.htm Who Invented Relativity?]</ref><ref>De Pretto, O. ''Reale Instituto Veneto Di Scienze, Lettere Ed Arti'', LXIII, II, 439–500, reprinted in Bartocci.</ref> presented a mass–energy relation. Bartocci (1999) observed that there were only [[six degrees of separation|three degrees of separation]] linking De Pretto to Einstein, concluding that Einstein was probably aware of De Pretto's work.<ref>Umberto Bartocci, ''Albert Einstein e Olinto De Pretto—La vera storia della formula più famosa del mondo'', editore Andromeda, Bologna, 1999.</ref>

Preston and De Pretto, following [[Le Sage's theory of gravitation|Le Sage]], imagined that the universe was filled with an ether of tiny particles that always move at speed {{mvar|c}}. Each of these particles has a kinetic energy of {{math|''mc''<sup>2</sup>}} up to a small numerical factor. The nonrelativistic kinetic energy formula did not always include the traditional factor of {{sfrac|2}}, since [[Gottfried Leibniz|Leibniz]] introduced kinetic energy without it, and the {{sfrac|2}} is largely conventional in prerelativistic physics.<ref>{{Citation| author=Prentiss, J.J.| title=Why is the energy of motion proportional to the square of the velocity? | journal=American Journal of Physics | volume=73| issue =  8 |date=August 2005 | page=705| doi=10.1119/1.1927550|bibcode = 2005AmJPh..73..701P }}</ref> By assuming that every particle has a mass that is the sum of the masses of the ether particles, the authors concluded that all matter contains an amount of kinetic energy either given by {{math|1=''E'' = ''mc''<sup>2</sup>}} or {{math|1=2''E'' = ''mc''<sup>2</sup>}} depending on the convention. A particle ether was usually considered unacceptably speculative science at the time,<ref>John Worrall, review of the book ''Conceptions of Ether. Studies in the History of Ether Theories'' by Cantor and Hodges, The British Journal of the Philosophy of Science vol 36, no 1, March 1985, p. 84. The article contrasts a particle ether with a wave-carrying ether, the latter ''was'' acceptable.</ref> and since these authors did not formulate relativity, their reasoning is completely different from that of Einstein, who used relativity to change frames.

Independently, [[Gustave Le Bon]] in 1905 speculated that atoms could release large amounts of latent energy, reasoning from an all-encompassing qualitative philosophy of physics.<ref>Le Bon: [http://www.rexresearch.com/lebonfor/evforp1.htm#p1b3ch2 The Evolution of Forces.]</ref><ref>Bizouard: [http://www.annales.org/archives/x/poincaBizouard.pdf Poincaré ''E'' = ''mc''<sup>2</sup> l'équation de Poincaré, Einstein et Planck.]</ref>

===Radioactivity and nuclear energy===
It was quickly noted after the discovery of [[radioactivity]] in 1897, that the total energy due to radioactive processes is about one ''million times'' greater than that involved in any known molecular change. However, it raised the question where this energy is coming from. After eliminating the idea of absorption and emission of some sort of [[Le Sage's theory of gravitation|Lesagian ether particles]], the existence of a huge amount of latent energy, stored within matter, was proposed by [[Ernest Rutherford]] and [[Frederick Soddy]] in 1903. Rutherford also suggested that this internal energy is stored within normal matter as well. He went on to speculate in 1904:<ref>{{Citation|last=Rutherford|first= Ernest |date=1904|title=Radioactivity |publisher=University Press|location=Cambridge|pages=336–338|url=https://archive.org/details/radioactivity00ruthrich}}</ref><ref>{{Citation|last=Heisenberg|first= Werner |date=1958|title=Physics And Philosophy: The Revolution In Modern Science |publisher=Harper & Brothers|location=New York|pages=118–119|url=https://archive.org/details/physicsandphilos010613mbp}}</ref>

{{quote|If it were ever found possible to control at will the rate of disintegration of the radio-elements, an enormous amount of energy could be obtained from a small quantity of matter.}}

Einstein's equation is in no way an explanation of the large energies released in radioactive decay (this comes from the powerful [[nuclear force]]s involved; forces that were still unknown in 1905). In any case, the enormous energy released from radioactive decay (which had been measured by Rutherford) was much more easily measured than the (still small) change in the gross mass of materials as a result. Einstein's equation, by theory, can give these energies by measuring mass differences before and after reactions, but in practice, these mass differences in 1905 were still too small to be measured in bulk. Prior to this, the ease of measuring radioactive decay energies with a calorimeter was thought possibly likely to allow measurement of changes in mass difference, as a check on Einstein's equation itself. Einstein mentions in his 1905 paper that mass–energy equivalence might perhaps be tested with radioactive decay, which releases enough energy (the quantitative amount known roughly by 1905) to possibly be "weighed," when missing from the system (having been given off as heat). However, radioactivity seemed to proceed at its own unalterable (and quite slow, for radioactives known then) pace, and even when simple nuclear reactions became possible using proton bombardment, the idea that these great amounts of usable energy could be liberated at will with any practicality, proved difficult to substantiate. Rutherford was reported in 1933 to have declared that this energy could not be exploited efficiently: "Anyone who expects a source of power from the transformation of the atom is talking [[moonshine]]."<ref>"We might in these processes obtain very much more energy than the proton supplied, but on the average we could not expect to obtain energy in this way. It was a very poor and inefficient way of producing energy, and anyone who looked for a source of power in the transformation of the atoms was talking moonshine. But the subject was scientifically interesting because it gave insight into the atoms." [http://archive.timesonline.co.uk/ ''The Times'' archives], September 12, 1933, "The British association—breaking down the atom"</ref>

[[File:Einstein - Time Magazine - July 1, 1946.jpg|right|thumb|The popular connection between Einstein, {{math|1=''E'' = ''mc''<sup>2</sup>}}, and the [[nuclear weapon|atomic bomb]] was prominently indicated on the cover of ''[[Time (magazine)|Time]]'' magazine in July 1946 by the writing of the equation on the [[mushroom cloud]].]]

This situation changed dramatically in 1932 with the discovery of the neutron and its mass, allowing mass differences for single [[nuclide]]s and their reactions to be calculated directly, and compared with the sum of masses for the particles that made up their composition. In 1933, the energy released from the reaction of lithium-7 plus protons giving rise to 2 alpha particles (as noted above by Rutherford), allowed Einstein's equation to be tested to an error of ±0.5%. However, scientists still did not see such reactions as a practical source of power, due to the energy cost of accelerating reaction particles.

After the very public demonstration of huge energies released from [[nuclear fission]] after the [[atomic bombings of Hiroshima and Nagasaki]] in 1945, the equation {{math|1=''E'' = ''mc''<sup>2</sup>}} became directly linked in the public eye with the power and peril of [[nuclear weapon]]s. The equation was featured as early as page 2 of the [[Smyth Report]], the official 1945 release by the US government on the development of the atomic bomb, and by 1946 the equation was linked closely enough with Einstein's work that the cover of ''[[Time (magazine)|Time]]'' magazine prominently featured a picture of Einstein next to an image of a [[mushroom cloud]] emblazoned with the equation.<ref>[http://www.time.com/time/covers/0,16641,19460701,00.html Cover.] ''Time'' magazine, July 1, 1946.</ref> Einstein himself had only a minor role in the [[Manhattan Project]]: he had [[Einstein–Szilárd letter|cosigned a letter]] to the U.S. President in 1939 urging funding for research into atomic energy, warning that an atomic bomb was theoretically possible. The letter persuaded Roosevelt to devote a significant portion of the wartime budget to atomic research. Without a security clearance, Einstein's only scientific contribution was an analysis of an [[isotope separation]] method in theoretical terms. It was inconsequential, on account of Einstein not being given sufficient information (for security reasons) to fully work on the problem.<ref>Isaacson, ''Einstein: His Life and Universe''.</ref>

While {{math|1=''E'' = ''mc''<sup>2</sup>}} is useful for understanding the amount of energy potentially released in a fission reaction, it was not strictly necessary to develop the weapon, once the fission process was known, and its energy measured at 200&nbsp;MeV (which was directly possible, using a quantitative Geiger counter, at that time). As the physicist and Manhattan Project participant [[Robert Serber]] put it: "Somehow the popular notion took hold long ago that Einstein's theory of relativity, in particular his famous equation {{math|1=''E'' = ''mc''<sup>2</sup>}}, plays some essential role in the theory of fission. Albert Einstein had a part in alerting the United States government to the possibility of building an atomic bomb, but his theory of relativity is not required in discussing fission. The theory of fission is what physicists call a non-relativistic theory, meaning that relativistic effects are too small to affect the dynamics of the fission process significantly."<ref>Robert Serber, ''The Los Alamos Primer: The First Lectures on How to Build an Atomic Bomb'' (University of California Press, 1992), page 7. Note that the quotation is taken from Serber's 1992 version, and is not in the original 1943 [[Los Alamos Primer]] of the same name.</ref> However the association between {{math|1=''E'' = ''mc''<sup>2</sup>}} and nuclear energy has since stuck, and because of this association, and its simple expression of the ideas of Albert Einstein himself, it has become "the world's most famous equation".<ref name="famous">{{cite book |title=E=mc^2: A Biography of the World's Most Famous Equation |edition=illustrated |first1=David |last1=Bodanis |publisher=Bloomsbury Publishing |year=2009 |isbn=978-0-8027-1821-1 |page= |url=https://books.google.com/books?id=8TX2tFLZ7gYC }}</ref>

While Serber's view of the strict lack of need to use mass–energy equivalence in designing the atomic bomb is correct, it does not take into account the pivotal role this relationship played in making the fundamental leap to the initial hypothesis that large atoms were energetically ''allowed'' to split into approximately equal parts (before this energy was in fact measured). In late 1938, [[Lise Meitner]] and [[Otto Robert Frisch]]—while on a winter walk during which they solved the meaning of Hahn's experimental results and introduced the idea that would be called atomic fission—directly used Einstein's equation to help them understand the quantitative energetics of the reaction that overcame the "surface tension-like" forces that hold the nucleus together, and allowed the fission fragments to separate to a configuration from which their charges could force them into an energetic ''fission''. To do this, they used ''packing fraction'', or nuclear [[binding energy]] values for elements, which Meitner had memorized. These, together with use of {{math|1=''E'' = ''mc''<sup>2</sup>}} allowed them to realize on the spot that the basic fission process was energetically possible:

<blockquote> ...We walked up and down in the snow, I on skis and she on foot. ...and gradually the idea took shape... explained by Bohr's idea that the nucleus is like a liquid drop; such a drop might elongate and divide itself... We knew there were strong forces that would resist, ..just as surface tension. But nuclei differed from ordinary drops. At this point we both sat down on a tree trunk and started to calculate on scraps of paper. ...the Uranium nucleus might indeed be a very wobbly, unstable drop, ready to divide itself... But, ...when the two drops separated they would be driven apart by electrical repulsion, about 200 MeV in all. Fortunately Lise Meitner remembered how to compute the masses of nuclei... and worked out that the two nuclei formed... would be lighter by about one-fifth the mass of a proton. Now whenever mass disappears energy is created, according to Einstein's formula {{math|1=''E'' = ''mc''<sup>2</sup>}}, and... the mass was just equivalent to 200 MeV; it all fitted!<ref>[http://homepage.mac.com/dtrapp/people/Meitnerium.html A quote from Frisch about the discovery day. Accessed April 4, 2009.] {{webarchive|url=https://web.archive.org/web/20081229192550/http://homepage.mac.com/dtrapp/people/Meitnerium.html |date=December 29, 2008 }}</ref><ref>{{Citation | last=Sime | first=Ruth | title= Lise Meitner: A Life in Physics | publisher = University of California Press | series = California Studies in the History of Science | volume = 13 | date = 1996 | location = Berkeley | pages = 236–237 | isbn = 0-520-20860-9 }}</ref></blockquote>

==See also==
{{Portal|Physics}}
{{cmn|colwidth=30em|
* [[Energy density]]
* [[Index of energy articles]]
* [[Index of wave articles]]
* [[Outline of energy]]
}}

==References==
{{reflist|30em}}

==External links==
{{Wikisourcepar|Relativity: The Special and General Theory}}
{{Commons category|Einstein formula}}
* [http://www.mathpages.com/home/kmath600/kmath600.htm Einstein on the Inertia of Energy] – MathPages
*{{Citation|first=Ronald C.|last=Lasky|title=What is the significance of E = mc<sup>2</sup>? And what does it mean?|date= April 23, 2007|publisher=[[Scientific American]]|url=http://www.sciam.com/article.cfm?id=significance-e-mc-2-means|work = | pages = | accessdate = | language = }}
* [http://profmattstrassler.com/articles-and-posts/particle-physics-basics/mass-energy-matter-etc/mass-and-energy/ Mass and Energy] – Conversations About Science with Theoretical Physicist Matt Strassler
* [http://imagine.gsfc.nasa.gov/docs/ask_astro/answers/970724a.html Ask an Astrophysicist | Energy–Matter Conversion], [[NASA]], 1997
* [http://plato.stanford.edu/entries/equivME The Equivalence of Mass and Energy] – Entry in the ''Stanford Encyclopedia of Philosophy''
*Gail Wilson (May 2014) [http://www3.imperial.ac.uk/newsandeventspggrp/imperialcollege/newssummary/news_16-5-2014-15-32-44 Scientists discover how to turn light into matter after 80-year quest] [[Imperial College]]
* {{cite web|last=Merrifield|first=Michael|title=E=mc<sup>2</sup> – Mass–Energy Equivalence|url=http://www.sixtysymbols.com/videos/emc2.htm|work=Sixty Symbols|publisher=[[Brady Haran]] for the [[University of Nottingham]]|author2=Copeland, Ed |author3=Bowley, Roger }}
* {{cite journal|doi=10.1119/1.3160671|quote=Early on, Einstein embraced the idea of a speed-dependent mass but changed his mind in 1906 and thereafter carefully avoided that notion entirely. He shunned, and explicitly rejected, what later came to be known as 'relativistic mass'. ... He consistently related the rest energy of a system to its invariant inertial mass.|title=Einstein on mass and energy|journal=American Journal of Physics|volume=77|issue=9|pages=799–806|year=2009|last1=Hecht|first1=Eugene|citeseerx=10.1.1.205.7995|bibcode=2009AmJPh..77..799H}}

{{Einstein}}
{{Relativity}}

{{DEFAULTSORT:Mass-Energy Equivalence}}
[[Category:1905 introductions]]
[[Category:1905 in science]]
[[Category:1905 in Germany]]
[[Category:Albert Einstein]]
[[Category:Energy (physics)]]
[[Category:Equations]]
[[Category:Mass]]
[[Category:Special relativity]]
