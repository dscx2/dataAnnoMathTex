<!DOCTYPE html><html>
<head>
<title>System description of HAT</title>
<!--Generated on Sun Dec  6 04:28:21 2020 by LaTeXML (version 0.8.5) http://dlmf.nist.gov/LaTeXML/.-->

<meta http-equiv="Content-Type" content="text/html; charset=UTF-8">
<meta name="keywords" lang="en" content="instrumentation: miscellaneous – telescopes –
techniques: photometric – stars: variables – methods: data analysis">
</head>
<body>
<div class="ltx_page_main">
<div class="ltx_page_content">
<article class="ltx_document ltx_authors_multiline">
<h1 class="ltx_title ltx_title_document">System description and first light-curves of HAT, an autonomous
observatory for variability search</h1>
<div class="ltx_authors">
<span class="ltx_creator ltx_role_author">
<span class="ltx_personname">G. Á. Bakos<span id="id8.1.id1" class="ltx_note ltx_role_affiliation"><sup class="ltx_note_mark">1</sup><span class="ltx_note_outer"><span class="ltx_note_content"><sup class="ltx_note_mark">1</sup><span class="ltx_note_type">affiliation: </span>Predoctoral Fellow, Smithsonian Astrophysical Observatory</span></span></span> <span id="id9.2.id2" class="ltx_note ltx_role_affiliation"><sup class="ltx_note_mark">2</sup><span class="ltx_note_outer"><span class="ltx_note_content"><sup class="ltx_note_mark">2</sup><span class="ltx_note_type">affiliation: </span>Konkoly Observatory, Budapest, H-1525, P.O. Box 67</span></span></span> 
</span><span class="ltx_author_notes"><span>
<span class="ltx_contact ltx_role_affiliation">Harvard-Smithsonian Center for Astrophysics
<br class="ltx_break">60 Garden street,
Cambridge, MA02138
</span>
<span class="ltx_contact ltx_role_email"><a href="mailto:gbakos@cfa.harvard.edu">gbakos@cfa.harvard.edu</a>
</span></span></span></span>
<span class="ltx_creator ltx_role_author">
<span class="ltx_personname">J. Lázár, I. Papp, P. Sári
</span><span class="ltx_author_notes"><span>
<span class="ltx_contact ltx_role_affiliation">Hungarian Astronomical Association, H-1461 Budapest, P.O.Box 219
</span>
<span class="ltx_contact ltx_role_email"><a href="mailto:jlazar,ipapp,%20psari@mcse.hu">jlazar,ipapp, psari@mcse.hu</a>
</span></span></span></span>
<span class="ltx_creator ltx_role_author">
<span class="ltx_personname">E. M. Green
</span><span class="ltx_author_notes"><span>
<span class="ltx_contact ltx_role_affiliation">Steward Observatory, University of Arizona, Tucson, AZ 85721
</span>
<span class="ltx_contact ltx_role_email"><a href="mailto:bgreen@as.arizona.edu">bgreen@as.arizona.edu</a>
</span></span></span></span>
</div>

<div class="ltx_abstract">
<h6 class="ltx_title ltx_title_abstract">Abstract</h6>
<p id="id7.4" class="ltx_p">Having been operational at Kitt Peak for more than a year, the
prototype of the Hungarian Automated Telescope (HAT-1) has been used
for all-sky variability search of the northern hemisphere. The small
autonomous observatory is recording brightness of stars in the range of
<math id="id4.1.m1.1" class="ltx_Math" alttext="\rm I_{c}\approx 6\--13^{m}" display="inline"><semantics id="id4.1.m1.1a"><mrow id="id4.1.m1.1.1" xref="id4.1.m1.1.1.cmml"><msub id="id4.1.m1.1.1.2" xref="id4.1.m1.1.1.2.cmml"><mi mathvariant="normal" id="id4.1.m1.1.1.2.2" xref="id4.1.m1.1.1.2.2.cmml">I</mi><mi mathvariant="normal" id="id4.1.m1.1.1.2.3" xref="id4.1.m1.1.1.2.3.cmml">c</mi></msub><mo id="id4.1.m1.1.1.1" xref="id4.1.m1.1.1.1.cmml">≈</mo><mrow id="id4.1.m1.1.1.3" xref="id4.1.m1.1.1.3.cmml"><mn id="id4.1.m1.1.1.3.2" xref="id4.1.m1.1.1.3.2.cmml">6</mn><mo id="id4.1.m1.1.1.3.1" xref="id4.1.m1.1.1.3.1.cmml">-</mo><msup id="id4.1.m1.1.1.3.3" xref="id4.1.m1.1.1.3.3.cmml"><mn id="id4.1.m1.1.1.3.3.2" xref="id4.1.m1.1.1.3.3.2.cmml">13</mn><mi mathvariant="normal" id="id4.1.m1.1.1.3.3.3" xref="id4.1.m1.1.1.3.3.3.cmml">m</mi></msup></mrow></mrow><annotation-xml encoding="MathML-Content" id="id4.1.m1.1b"><apply id="id4.1.m1.1.1.cmml" xref="id4.1.m1.1.1"><approx id="id4.1.m1.1.1.1.cmml" xref="id4.1.m1.1.1.1"></approx><apply id="id4.1.m1.1.1.2.cmml" xref="id4.1.m1.1.1.2"><csymbol cd="ambiguous" id="id4.1.m1.1.1.2.1.cmml" xref="id4.1.m1.1.1.2">subscript</csymbol><ci id="id4.1.m1.1.1.2.2.cmml" xref="id4.1.m1.1.1.2.2">I</ci><ci id="id4.1.m1.1.1.2.3.cmml" xref="id4.1.m1.1.1.2.3">c</ci></apply><apply id="id4.1.m1.1.1.3.cmml" xref="id4.1.m1.1.1.3"><minus id="id4.1.m1.1.1.3.1.cmml" xref="id4.1.m1.1.1.3.1"></minus><cn type="integer" id="id4.1.m1.1.1.3.2.cmml" xref="id4.1.m1.1.1.3.2">6</cn><apply id="id4.1.m1.1.1.3.3.cmml" xref="id4.1.m1.1.1.3.3"><csymbol cd="ambiguous" id="id4.1.m1.1.1.3.3.1.cmml" xref="id4.1.m1.1.1.3.3">superscript</csymbol><cn type="integer" id="id4.1.m1.1.1.3.3.2.cmml" xref="id4.1.m1.1.1.3.3.2">13</cn><ci id="id4.1.m1.1.1.3.3.3.cmml" xref="id4.1.m1.1.1.3.3.3">m</ci></apply></apply></apply></annotation-xml><annotation encoding="application/x-tex" id="id4.1.m1.1c">\rm I_{c}\approx 6\--13^{m}</annotation><annotation encoding="application/x-llamapun" id="id4.1.m1.1d">roman_I start_POSTSUBSCRIPT roman_c end_POSTSUBSCRIPT ≈ 6 - 13 start_POSTSUPERSCRIPT roman_m end_POSTSUPERSCRIPT</annotation></semantics></math> with a telephoto lens and its
<math id="id5.2.m2.1" class="ltx_Math" alttext="9\arcdeg\times 9\arcdeg" display="inline"><semantics id="id5.2.m2.1a"><mrow id="id5.2.m2.1.1" xref="id5.2.m2.1.1.cmml"><mrow id="id5.2.m2.1.1.2" xref="id5.2.m2.1.1.2.cmml"><mrow id="id5.2.m2.1.1.2.2" xref="id5.2.m2.1.1.2.2.cmml"><mn id="id5.2.m2.1.1.2.2.2" xref="id5.2.m2.1.1.2.2.2.cmml">9</mn><mo id="id5.2.m2.1.1.2.2.1" xref="id5.2.m2.1.1.2.2.1.cmml">⁢</mo><mi mathvariant="normal" id="id5.2.m2.1.1.2.2.3" xref="id5.2.m2.1.1.2.2.3.cmml">°</mi></mrow><mo id="id5.2.m2.1.1.2.1" xref="id5.2.m2.1.1.2.1.cmml">×</mo><mn id="id5.2.m2.1.1.2.3" xref="id5.2.m2.1.1.2.3.cmml">9</mn></mrow><mo id="id5.2.m2.1.1.1" xref="id5.2.m2.1.1.1.cmml">⁢</mo><mi mathvariant="normal" id="id5.2.m2.1.1.3" xref="id5.2.m2.1.1.3.cmml">°</mi></mrow><annotation-xml encoding="MathML-Content" id="id5.2.m2.1b"><apply id="id5.2.m2.1.1.cmml" xref="id5.2.m2.1.1"><times id="id5.2.m2.1.1.1.cmml" xref="id5.2.m2.1.1.1"></times><apply id="id5.2.m2.1.1.2.cmml" xref="id5.2.m2.1.1.2"><times id="id5.2.m2.1.1.2.1.cmml" xref="id5.2.m2.1.1.2.1"></times><apply id="id5.2.m2.1.1.2.2.cmml" xref="id5.2.m2.1.1.2.2"><times id="id5.2.m2.1.1.2.2.1.cmml" xref="id5.2.m2.1.1.2.2.1"></times><cn type="integer" id="id5.2.m2.1.1.2.2.2.cmml" xref="id5.2.m2.1.1.2.2.2">9</cn><ci id="id5.2.m2.1.1.2.2.3.cmml" xref="id5.2.m2.1.1.2.2.3">°</ci></apply><cn type="integer" id="id5.2.m2.1.1.2.3.cmml" xref="id5.2.m2.1.1.2.3">9</cn></apply><ci id="id5.2.m2.1.1.3.cmml" xref="id5.2.m2.1.1.3">°</ci></apply></annotation-xml><annotation encoding="application/x-tex" id="id5.2.m2.1c">9\arcdeg\times 9\arcdeg</annotation><annotation encoding="application/x-llamapun" id="id5.2.m2.1d">9 ° × 9 °</annotation></semantics></math> field of view (FOV), yielding a data rate of
<math id="id6.3.m3.1" class="ltx_Math" alttext="\sim 10^{6}" display="inline"><semantics id="id6.3.m3.1a"><mrow id="id6.3.m3.1.1" xref="id6.3.m3.1.1.cmml"><mi id="id6.3.m3.1.1.2" xref="id6.3.m3.1.1.2.cmml"></mi><mo id="id6.3.m3.1.1.1" xref="id6.3.m3.1.1.1.cmml">∼</mo><msup id="id6.3.m3.1.1.3" xref="id6.3.m3.1.1.3.cmml"><mn id="id6.3.m3.1.1.3.2" xref="id6.3.m3.1.1.3.2.cmml">10</mn><mn id="id6.3.m3.1.1.3.3" xref="id6.3.m3.1.1.3.3.cmml">6</mn></msup></mrow><annotation-xml encoding="MathML-Content" id="id6.3.m3.1b"><apply id="id6.3.m3.1.1.cmml" xref="id6.3.m3.1.1"><csymbol cd="latexml" id="id6.3.m3.1.1.1.cmml" xref="id6.3.m3.1.1.1">similar-to</csymbol><csymbol cd="latexml" id="id6.3.m3.1.1.2.cmml" xref="id6.3.m3.1.1.2">absent</csymbol><apply id="id6.3.m3.1.1.3.cmml" xref="id6.3.m3.1.1.3"><csymbol cd="ambiguous" id="id6.3.m3.1.1.3.1.cmml" xref="id6.3.m3.1.1.3">superscript</csymbol><cn type="integer" id="id6.3.m3.1.1.3.2.cmml" xref="id6.3.m3.1.1.3.2">10</cn><cn type="integer" id="id6.3.m3.1.1.3.3.cmml" xref="id6.3.m3.1.1.3.3">6</cn></apply></apply></annotation-xml><annotation encoding="application/x-tex" id="id6.3.m3.1c">\sim 10^{6}</annotation><annotation encoding="application/x-llamapun" id="id6.3.m3.1d">∼ 10 start_POSTSUPERSCRIPT 6 end_POSTSUPERSCRIPT</annotation></semantics></math> photometric measurements per night. We give brief hardware
and software description of the system, controlled by a single PC
running RealTime Linux operating system (OS). We overview site-specific
details, and quantify the astrometric and photometric capabilities of
HAT. As a demonstration of system performance we give a sample of 60
short period variables in a single selected field, all bright, with
<math id="id7.4.m4.1" class="ltx_Math" alttext="\rm I&lt;13^{m}" display="inline"><semantics id="id7.4.m4.1a"><mrow id="id7.4.m4.1.1" xref="id7.4.m4.1.1.cmml"><mi mathvariant="normal" id="id7.4.m4.1.1.2" xref="id7.4.m4.1.1.2.cmml">I</mi><mo id="id7.4.m4.1.1.1" xref="id7.4.m4.1.1.1.cmml">&lt;</mo><msup id="id7.4.m4.1.1.3" xref="id7.4.m4.1.1.3.cmml"><mn id="id7.4.m4.1.1.3.2" xref="id7.4.m4.1.1.3.2.cmml">13</mn><mi mathvariant="normal" id="id7.4.m4.1.1.3.3" xref="id7.4.m4.1.1.3.3.cmml">m</mi></msup></mrow><annotation-xml encoding="MathML-Content" id="id7.4.m4.1b"><apply id="id7.4.m4.1.1.cmml" xref="id7.4.m4.1.1"><lt id="id7.4.m4.1.1.1.cmml" xref="id7.4.m4.1.1.1"></lt><ci id="id7.4.m4.1.1.2.cmml" xref="id7.4.m4.1.1.2">I</ci><apply id="id7.4.m4.1.1.3.cmml" xref="id7.4.m4.1.1.3"><csymbol cd="ambiguous" id="id7.4.m4.1.1.3.1.cmml" xref="id7.4.m4.1.1.3">superscript</csymbol><cn type="integer" id="id7.4.m4.1.1.3.2.cmml" xref="id7.4.m4.1.1.3.2">13</cn><ci id="id7.4.m4.1.1.3.3.cmml" xref="id7.4.m4.1.1.3.3">m</ci></apply></apply></annotation-xml><annotation encoding="application/x-tex" id="id7.4.m4.1c">\rm I&lt;13^{m}</annotation><annotation encoding="application/x-llamapun" id="id7.4.m4.1d">roman_I &lt; 13 start_POSTSUPERSCRIPT roman_m end_POSTSUPERSCRIPT</annotation></semantics></math>, of which only 14 were known before. Depending on the
observing strategy, search for extrasolar planet transits is also a
feasible observing program. We conclude with a short discussion on
future directions. Further information can be found at the HAT
homepage: <span id="id7.4.1" class="ltx_text ltx_font_typewriter">http://www-cfa.harvard.edu/~gbakos/HAT/</span>.</p>
</div>
<div class="ltx_keywords">instrumentation: miscellaneous – telescopes –
techniques: photometric – stars: variables – methods: data analysis
</div>
<span id="id1" class="ltx_note ltx_role_slugcomment"><sup class="ltx_note_mark">†</sup><span class="ltx_note_outer"><span class="ltx_note_content"><sup class="ltx_note_mark">†</sup><span class="ltx_note_type">slugcomment: </span>DRAFT</span></span></span><span id="id2" class="ltx_ERROR undefined">{comment}</span>
<div id="p1" class="ltx_para">
<p id="p1.1" class="ltx_p">Title:
System description and first light-curves of HAT, an autonomous
observatory for variability search</p>
</div>
<div id="p2" class="ltx_para">
<p id="p2.1" class="ltx_p">Authors:
Gaspar A. Bakos (CfA and Konkoly Obs.),
Jozsef Lazar (Hungarian Astronomical Association: HAA),
Istvan Papp (HAA),
Pal Sari (HAA),
Elizabeth M. Green (Steward Obs., Univ. of Arizona)</p>
</div>
<div id="p3" class="ltx_para">
<p id="p3.1" class="ltx_p">Comments:
Submitted to PASP, 18 pages, 8 figures.
See http://www-cfa.harvard.edu/ gbakos/HAT/Papers/ for more information
and hi-resolution figures.</p>
</div>
<div id="p4" class="ltx_para">
<p id="p4.1" class="ltx_p">Abstract:
Having been operational at Kitt Peak for more than a year, the
prototype of the Hungarian Automated Telescope (HAT-1) has been used
for all-sky variability search of the northern hemisphere. The small
autonomous observatory is recording brightness of stars in the range of
I 6–13m with a telephoto lens and its 9x9 degree field of view,
yielding a data rate of  10^6 photometric measurements per night. We
give brief hardware and software description of the system, controlled
by a single PC running RealTime Linux operating system. We
overview site-specific details, and quantify the astrometric and
photometric capabilities of HAT. As a demonstration of system
performance we give a sample of 60 short period variables in a single
selected field, all bright, with I ¡ 13m, of which only 14 were known
before. Depending on the observing strategy, search for extrasolar
planet transits is also a feasible observing program. We conclude with
a short discussion on future directions. Further information can be
found at the HAT homepage: http://www-cfa.harvard.edu/ gbakos/HAT/</p>
</div>
<section id="S1" class="ltx_section">
<h2 class="ltx_title ltx_title_section">
<span class="ltx_tag ltx_tag_section">1 </span>Introduction</h2>

<div id="S1.p1" class="ltx_para">
<p id="S1.p1.1" class="ltx_p">The idea of automating ground based observational astronomy goes back
more than two decades. Minimizing manpower can assure uniform and
massive data flow with low budget and the absence of human mistakes.</p>
</div>
<div id="S1.p2" class="ltx_para">
<p id="S1.p2.1" class="ltx_p">The increasing number of robotic telescopes<span id="footnote1" class="ltx_note ltx_role_footnote"><sup class="ltx_note_mark">1</sup><span class="ltx_note_outer"><span class="ltx_note_content"><sup class="ltx_note_mark">1</sup><span class="ltx_tag ltx_tag_note">1</span>
See eg. http://alpha.uni-sw.gwdg.de/~hessman/MONET</span></span></span>
(capable of computer controlled multiple observations) have been used
for a broad range of projects, such as astrometry:
Carlsberg Meridian Telescope <cite class="ltx_cite ltx_citemacro_citep">(Helmer &amp; Morrison, <a href="#bib.bib16" title="" class="ltx_ref">1985</a>)</cite>;
photoelectric photometry of pre-selected targets:
Fairborn Observatory <cite class="ltx_cite ltx_citemacro_citep">(Boyd et al., <a href="#bib.bib8" title="" class="ltx_ref">1984</a>)</cite>;
supernova search: KAIT <cite class="ltx_cite ltx_citemacro_citep">(Richmond, Treffers, &amp; Filippenko, <a href="#bib.bib44" title="" class="ltx_ref">1993</a>)</cite>;
GRB follow-up: ROTSE <cite class="ltx_cite ltx_citemacro_citep">(Akerlof et al., <a href="#bib.bib4" title="" class="ltx_ref">2000</a>; Smith et al., <a href="#bib.bib43" title="" class="ltx_ref">2002</a>)</cite>
and LOTIS <cite class="ltx_cite ltx_citemacro_citep">(Park et al., <a href="#bib.bib33" title="" class="ltx_ref">1997</a>, <a href="#bib.bib34" title="" class="ltx_ref">2001</a>)</cite>,
exo-planet searches: STARE <cite class="ltx_cite ltx_citemacro_citep">(Brown &amp; Charbonneau, <a href="#bib.bib9" title="" class="ltx_ref">1999</a>)</cite>,
Vulcan <cite class="ltx_cite ltx_citemacro_citep">(Borucki et al., <a href="#bib.bib7" title="" class="ltx_ref">2001</a>)</cite>;
and asteroid searches: TAOS <cite class="ltx_cite ltx_citemacro_citep">(Chen, <a href="#bib.bib13" title="" class="ltx_ref">1999</a>)</cite>.
The variety of targets is usually narrow, looking only for
specific timescales, light-curve shapes and intensity ranges.</p>
</div>
<div id="S1.p3" class="ltx_para">
<p id="S1.p3.1" class="ltx_p">Although most projects concentrating on special targets gain a huge
amount of photometric data, only few of them are capable of presenting
their by-products to the astronomical community, e.g. OGLE
<cite class="ltx_cite ltx_citemacro_citep">(e.g.  Woźniak, <a href="#bib.bib50" title="" class="ltx_ref">2002</a>)</cite>, MACHO <cite class="ltx_cite ltx_citemacro_citep">(Allsman &amp; Axelrod, <a href="#bib.bib3" title="" class="ltx_ref">2001</a>)</cite> and ROTSE
<cite class="ltx_cite ltx_citemacro_citep">(Akerlof et al., <a href="#bib.bib4" title="" class="ltx_ref">2000</a>)</cite>.</p>
</div>
<div id="S1.p4" class="ltx_para">
<p id="S1.p4.3" class="ltx_p">Initiated by ideas of <cite class="ltx_cite ltx_citemacro_citet">Paczyński (<a href="#bib.bib31" title="" class="ltx_ref">1997</a>)</cite>, the All-Sky Automated Survey’s
<cite class="ltx_cite ltx_citemacro_citep">(ASAS; Pojmański, <a href="#bib.bib39" title="" class="ltx_ref">1997</a>)</cite> approach is different, in that the final goal
has been photometric monitoring of <span id="S1.p4.3.1" class="ltx_text ltx_font_italic">all bright stars</span> in a major
part of the southern sky down to <math id="S1.p4.1.m1.1" class="ltx_Math" alttext="\rm I\approx 14^{m}" display="inline"><semantics id="S1.p4.1.m1.1a"><mrow id="S1.p4.1.m1.1.1" xref="S1.p4.1.m1.1.1.cmml"><mi mathvariant="normal" id="S1.p4.1.m1.1.1.2" xref="S1.p4.1.m1.1.1.2.cmml">I</mi><mo id="S1.p4.1.m1.1.1.1" xref="S1.p4.1.m1.1.1.1.cmml">≈</mo><msup id="S1.p4.1.m1.1.1.3" xref="S1.p4.1.m1.1.1.3.cmml"><mn id="S1.p4.1.m1.1.1.3.2" xref="S1.p4.1.m1.1.1.3.2.cmml">14</mn><mi mathvariant="normal" id="S1.p4.1.m1.1.1.3.3" xref="S1.p4.1.m1.1.1.3.3.cmml">m</mi></msup></mrow><annotation-xml encoding="MathML-Content" id="S1.p4.1.m1.1b"><apply id="S1.p4.1.m1.1.1.cmml" xref="S1.p4.1.m1.1.1"><approx id="S1.p4.1.m1.1.1.1.cmml" xref="S1.p4.1.m1.1.1.1"></approx><ci id="S1.p4.1.m1.1.1.2.cmml" xref="S1.p4.1.m1.1.1.2">I</ci><apply id="S1.p4.1.m1.1.1.3.cmml" xref="S1.p4.1.m1.1.1.3"><csymbol cd="ambiguous" id="S1.p4.1.m1.1.1.3.1.cmml" xref="S1.p4.1.m1.1.1.3">superscript</csymbol><cn type="integer" id="S1.p4.1.m1.1.1.3.2.cmml" xref="S1.p4.1.m1.1.1.3.2">14</cn><ci id="S1.p4.1.m1.1.1.3.3.cmml" xref="S1.p4.1.m1.1.1.3.3">m</ci></apply></apply></annotation-xml><annotation encoding="application/x-tex" id="S1.p4.1.m1.1c">\rm I\approx 14^{m}</annotation><annotation encoding="application/x-llamapun" id="S1.p4.1.m1.1d">roman_I ≈ 14 start_POSTSUPERSCRIPT roman_m end_POSTSUPERSCRIPT</annotation></semantics></math>. Using a fully
automated but inexpensive system consisting of an amateur-class CCD, a
small telephoto lens and an equatorial mount, ASAS presented catalogues
of 4000 bright variables from a <math id="S1.p4.2.m2.1" class="ltx_Math" alttext="300\sq\arcdeg" display="inline"><semantics id="S1.p4.2.m2.1a"><mrow id="S1.p4.2.m2.1.1" xref="S1.p4.2.m2.1.1.cmml"><mn id="S1.p4.2.m2.1.1.2" xref="S1.p4.2.m2.1.1.2.cmml">300</mn><mo id="S1.p4.2.m2.1.1.1" xref="S1.p4.2.m2.1.1.1.cmml">⁢</mo><mi mathvariant="normal" id="S1.p4.2.m2.1.1.3" xref="S1.p4.2.m2.1.1.3.cmml">□</mi><mo id="S1.p4.2.m2.1.1.1a" xref="S1.p4.2.m2.1.1.1.cmml">⁢</mo><mi mathvariant="normal" id="S1.p4.2.m2.1.1.4" xref="S1.p4.2.m2.1.1.4.cmml">°</mi></mrow><annotation-xml encoding="MathML-Content" id="S1.p4.2.m2.1b"><apply id="S1.p4.2.m2.1.1.cmml" xref="S1.p4.2.m2.1.1"><times id="S1.p4.2.m2.1.1.1.cmml" xref="S1.p4.2.m2.1.1.1"></times><cn type="integer" id="S1.p4.2.m2.1.1.2.cmml" xref="S1.p4.2.m2.1.1.2">300</cn><ci id="S1.p4.2.m2.1.1.3.cmml" xref="S1.p4.2.m2.1.1.3">□</ci><ci id="S1.p4.2.m2.1.1.4.cmml" xref="S1.p4.2.m2.1.1.4">°</ci></apply></annotation-xml><annotation encoding="application/x-tex" id="S1.p4.2.m2.1c">300\sq\arcdeg</annotation><annotation encoding="application/x-llamapun" id="S1.p4.2.m2.1d">300 □ °</annotation></semantics></math> area of the southern
sky, <span id="S1.p4.3.2" class="ltx_text ltx_font_italic">96% being new discoveries</span> <cite class="ltx_cite ltx_citemacro_citep">(Pojmański, <a href="#bib.bib40" title="" class="ltx_ref">1998</a>, <a href="#bib.bib41" title="" class="ltx_ref">2000</a>)</cite>. The upgraded
ASAS-3 will produce an order of magnitude increase in the data
flow.<span id="footnote2" class="ltx_note ltx_role_footnote"><sup class="ltx_note_mark">2</sup><span class="ltx_note_outer"><span class="ltx_note_content"><sup class="ltx_note_mark">2</sup><span class="ltx_tag ltx_tag_note">2</span>http://www.astrouw.edu.pl/~gp/asas/asas_asas3.html</span></span></span>
The incompleteness of our knowledge on bright variable stars was
reinforced by <cite class="ltx_cite ltx_citemacro_citet">Akerlof et al. (<a href="#bib.bib4" title="" class="ltx_ref">2000</a>)</cite> who discovered 1781 new variables in a
<math id="S1.p4.3.m3.1" class="ltx_Math" alttext="2000\sq\arcdeg" display="inline"><semantics id="S1.p4.3.m3.1a"><mrow id="S1.p4.3.m3.1.1" xref="S1.p4.3.m3.1.1.cmml"><mn id="S1.p4.3.m3.1.1.2" xref="S1.p4.3.m3.1.1.2.cmml">2000</mn><mo id="S1.p4.3.m3.1.1.1" xref="S1.p4.3.m3.1.1.1.cmml">⁢</mo><mi mathvariant="normal" id="S1.p4.3.m3.1.1.3" xref="S1.p4.3.m3.1.1.3.cmml">□</mi><mo id="S1.p4.3.m3.1.1.1a" xref="S1.p4.3.m3.1.1.1.cmml">⁢</mo><mi mathvariant="normal" id="S1.p4.3.m3.1.1.4" xref="S1.p4.3.m3.1.1.4.cmml">°</mi></mrow><annotation-xml encoding="MathML-Content" id="S1.p4.3.m3.1b"><apply id="S1.p4.3.m3.1.1.cmml" xref="S1.p4.3.m3.1.1"><times id="S1.p4.3.m3.1.1.1.cmml" xref="S1.p4.3.m3.1.1.1"></times><cn type="integer" id="S1.p4.3.m3.1.1.2.cmml" xref="S1.p4.3.m3.1.1.2">2000</cn><ci id="S1.p4.3.m3.1.1.3.cmml" xref="S1.p4.3.m3.1.1.3">□</ci><ci id="S1.p4.3.m3.1.1.4.cmml" xref="S1.p4.3.m3.1.1.4">°</ci></apply></annotation-xml><annotation encoding="application/x-tex" id="S1.p4.3.m3.1c">2000\sq\arcdeg</annotation><annotation encoding="application/x-llamapun" id="S1.p4.3.m3.1d">2000 □ °</annotation></semantics></math> area.</p>
</div>
<div id="S1.p5" class="ltx_para">
<p id="S1.p5.1" class="ltx_p">Why is general variability study of bright objects important? Several
answers can be found in <cite class="ltx_cite ltx_citemacro_citet">Paczyński (<a href="#bib.bib31" title="" class="ltx_ref">1997</a>, <a href="#bib.bib32" title="" class="ltx_ref">2000</a>)</cite>, and others can be added.
Variable stars are essential for testing stellar structure and
evolution theories, examining galactic structure or establishing the
extragalactic distance scale. Only bright variables are within the
range of high resolution spectroscopy, parallax and proper motion
measurements. Our knowledge of issues related to variable stars
(e.g. distance scale) can be refined by the combination of detailed
study of close-by, bright objects and of equidistant, homogeneous
samples (e.g. OGLE – Galactic bulge, LMC, SMC). Serious
incompleteness at the bright end affects all conclusions. A systematic,
well-calibrated survey presents clean, statistically valuable samples
with well defined limits for different subtypes of variable objects. A
reliable database with sufficient and ever-growing time-span of
light-curves can be used as an archive, for e.g., correlating optical
variability with X-ray observations, made by satellites. It can be a
valuable input to schedule big-telescope and space-mission
observations, where telescope time is limited, or prior and longer-term
data on field variables is necessary (e.g. the Kepler mission),
furthermore, it can provide them with a real-time alert system of rare
events. Such events can be nova explosions, helium flash of a star
<cite class="ltx_cite ltx_citemacro_citep">(Sakurai’s object: Nakano &amp; Kushida, <a href="#bib.bib29" title="" class="ltx_ref">1996</a>; Duerbeck et al., <a href="#bib.bib15" title="" class="ltx_ref">2000</a>)</cite>,
super-outbursts of dwarf-novae <cite class="ltx_cite ltx_citemacro_citep">(WZ Sge: Ishioka et al., <a href="#bib.bib18" title="" class="ltx_ref">2001</a>)</cite>.</p>
</div>
<div id="S1.p6" class="ltx_para">
<p id="S1.p6.1" class="ltx_p">To mention specific examples, observational data is scarce for spotted
red subgiant variables (RS Cvn, FK Com), which are crucial in
understanding the stellar magnetic cycles. Detached eclipsing binaries
(through their stellar mass, radius and luminosity determination) can
be perfect distance and age indicators, if nearby systems are properly
calibrated. Samples of such objects in the solar neighborhood are
sparse <cite class="ltx_cite ltx_citemacro_citep">(Paczyński, <a href="#bib.bib31" title="" class="ltx_ref">1997</a>)</cite>, partly due to their short and narrow eclipses, and
lack of observational data (see Fig. <a href="#S1.F1" title="Figure 1 ‣ 1 Introduction ‣ System description and first light-curves of HAT, an autonomous observatory for variability search" class="ltx_ref"><span class="ltx_text ltx_ref_tag">1</span></a> for our
light-curve of a <span id="S1.p6.1.1" class="ltx_text ltx_font_italic">semi</span>-detached binary). Bright contact binaries
exhibit similar incompleteness, although long-term observations could
reveal interesting phenomena, such as the transition to semi-detached
state.</p>
</div>
<div id="S1.p7" class="ltx_para">
<p id="S1.p7.3" class="ltx_p">The Hipparcos Space Astrometry Mission presented us with discovery of a
few thousand new bright variables, and the Hertzsprung-Russell (HR)
diagram was described in terms of luminosity stability at the
millimagnitude level <cite class="ltx_cite ltx_citemacro_citep">(Perryman et al., <a href="#bib.bib37" title="" class="ltx_ref">1997</a>)</cite>. However, Hipparcos observed
only selected stars (120000 or <math id="S1.p7.1.m1.1" class="ltx_Math" alttext="3/\sq\arcdeg" display="inline"><semantics id="S1.p7.1.m1.1a"><mrow id="S1.p7.1.m1.1.1" xref="S1.p7.1.m1.1.1.cmml"><mrow id="S1.p7.1.m1.1.1.2" xref="S1.p7.1.m1.1.1.2.cmml"><mn id="S1.p7.1.m1.1.1.2.2" xref="S1.p7.1.m1.1.1.2.2.cmml">3</mn><mo id="S1.p7.1.m1.1.1.2.1" xref="S1.p7.1.m1.1.1.2.1.cmml">/</mo><mi mathvariant="normal" id="S1.p7.1.m1.1.1.2.3" xref="S1.p7.1.m1.1.1.2.3.cmml">□</mi></mrow><mo id="S1.p7.1.m1.1.1.1" xref="S1.p7.1.m1.1.1.1.cmml">⁢</mo><mi mathvariant="normal" id="S1.p7.1.m1.1.1.3" xref="S1.p7.1.m1.1.1.3.cmml">°</mi></mrow><annotation-xml encoding="MathML-Content" id="S1.p7.1.m1.1b"><apply id="S1.p7.1.m1.1.1.cmml" xref="S1.p7.1.m1.1.1"><times id="S1.p7.1.m1.1.1.1.cmml" xref="S1.p7.1.m1.1.1.1"></times><apply id="S1.p7.1.m1.1.1.2.cmml" xref="S1.p7.1.m1.1.1.2"><divide id="S1.p7.1.m1.1.1.2.1.cmml" xref="S1.p7.1.m1.1.1.2.1"></divide><cn type="integer" id="S1.p7.1.m1.1.1.2.2.cmml" xref="S1.p7.1.m1.1.1.2.2">3</cn><ci id="S1.p7.1.m1.1.1.2.3.cmml" xref="S1.p7.1.m1.1.1.2.3">□</ci></apply><ci id="S1.p7.1.m1.1.1.3.cmml" xref="S1.p7.1.m1.1.1.3">°</ci></apply></annotation-xml><annotation encoding="application/x-tex" id="S1.p7.1.m1.1c">3/\sq\arcdeg</annotation><annotation encoding="application/x-llamapun" id="S1.p7.1.m1.1d">3 / □ °</annotation></semantics></math>), and the variable star
sample is further limited by the <math id="S1.p7.2.m2.1" class="ltx_Math" alttext="\sim 110" display="inline"><semantics id="S1.p7.2.m2.1a"><mrow id="S1.p7.2.m2.1.1" xref="S1.p7.2.m2.1.1.cmml"><mi id="S1.p7.2.m2.1.1.2" xref="S1.p7.2.m2.1.1.2.cmml"></mi><mo id="S1.p7.2.m2.1.1.1" xref="S1.p7.2.m2.1.1.1.cmml">∼</mo><mn id="S1.p7.2.m2.1.1.3" xref="S1.p7.2.m2.1.1.3.cmml">110</mn></mrow><annotation-xml encoding="MathML-Content" id="S1.p7.2.m2.1b"><apply id="S1.p7.2.m2.1.1.cmml" xref="S1.p7.2.m2.1.1"><csymbol cd="latexml" id="S1.p7.2.m2.1.1.1.cmml" xref="S1.p7.2.m2.1.1.1">similar-to</csymbol><csymbol cd="latexml" id="S1.p7.2.m2.1.1.2.cmml" xref="S1.p7.2.m2.1.1.2">absent</csymbol><cn type="integer" id="S1.p7.2.m2.1.1.3.cmml" xref="S1.p7.2.m2.1.1.3">110</cn></apply></annotation-xml><annotation encoding="application/x-tex" id="S1.p7.2.m2.1c">\sim 110</annotation><annotation encoding="application/x-llamapun" id="S1.p7.2.m2.1d">∼ 110</annotation></semantics></math> epochs per star on average
and cut-off at <math id="S1.p7.3.m3.1" class="ltx_Math" alttext="\rm I\approx 9^{m}" display="inline"><semantics id="S1.p7.3.m3.1a"><mrow id="S1.p7.3.m3.1.1" xref="S1.p7.3.m3.1.1.cmml"><mi mathvariant="normal" id="S1.p7.3.m3.1.1.2" xref="S1.p7.3.m3.1.1.2.cmml">I</mi><mo id="S1.p7.3.m3.1.1.1" xref="S1.p7.3.m3.1.1.1.cmml">≈</mo><msup id="S1.p7.3.m3.1.1.3" xref="S1.p7.3.m3.1.1.3.cmml"><mn id="S1.p7.3.m3.1.1.3.2" xref="S1.p7.3.m3.1.1.3.2.cmml">9</mn><mi mathvariant="normal" id="S1.p7.3.m3.1.1.3.3" xref="S1.p7.3.m3.1.1.3.3.cmml">m</mi></msup></mrow><annotation-xml encoding="MathML-Content" id="S1.p7.3.m3.1b"><apply id="S1.p7.3.m3.1.1.cmml" xref="S1.p7.3.m3.1.1"><approx id="S1.p7.3.m3.1.1.1.cmml" xref="S1.p7.3.m3.1.1.1"></approx><ci id="S1.p7.3.m3.1.1.2.cmml" xref="S1.p7.3.m3.1.1.2">I</ci><apply id="S1.p7.3.m3.1.1.3.cmml" xref="S1.p7.3.m3.1.1.3"><csymbol cd="ambiguous" id="S1.p7.3.m3.1.1.3.1.cmml" xref="S1.p7.3.m3.1.1.3">superscript</csymbol><cn type="integer" id="S1.p7.3.m3.1.1.3.2.cmml" xref="S1.p7.3.m3.1.1.3.2">9</cn><ci id="S1.p7.3.m3.1.1.3.3.cmml" xref="S1.p7.3.m3.1.1.3.3">m</ci></apply></apply></annotation-xml><annotation encoding="application/x-tex" id="S1.p7.3.m3.1c">\rm I\approx 9^{m}</annotation><annotation encoding="application/x-llamapun" id="S1.p7.3.m3.1d">roman_I ≈ 9 start_POSTSUPERSCRIPT roman_m end_POSTSUPERSCRIPT</annotation></semantics></math>.</p>
</div>
<div id="S1.p8" class="ltx_para">
<p id="S1.p8.1" class="ltx_p">Mapping the location of the large variety of pulsating variables on the
HR diagram is still far from being complete. Addressing phenomena, such
as long-term period and amplitude modulations (e.g. Blazhko effect of
RR Lyrae), evolution of the pulsational status of a star, is possible
only by long-term and <span id="S1.p8.1.1" class="ltx_text ltx_font_italic">homogeneous</span> observations. Given the huge
data-flow, interesting phenomena are expected to emerge, for instance
further observational evidence for chaos in W Vir and RV Tau stars
<cite class="ltx_cite ltx_citemacro_citep">(Buchler, Serre &amp; Kolláth, <a href="#bib.bib10" title="" class="ltx_ref">1995</a>)</cite>, triple-mode variables <cite class="ltx_cite ltx_citemacro_citep">(GSC 40181807: Antipin, <a href="#bib.bib5" title="" class="ltx_ref">1997</a>)</cite>,
Cepheids which stop pulsating <cite class="ltx_cite ltx_citemacro_citep">(V19 in M33: Macri, Sasselov, &amp; Stanek, <a href="#bib.bib28" title="" class="ltx_ref">2001</a>)</cite> or strong
amplitude modulation of Cepheids <cite class="ltx_cite ltx_citemacro_citep">(V473 Lyr: Burki et al., <a href="#bib.bib12" title="" class="ltx_ref">1986</a>)</cite>.
Long-term monitoring of semi-regular and Mira variables is needed to
disentangle multiperiodicity and systematic amplitude variations
<cite class="ltx_cite ltx_citemacro_citep">(e.g.  Kiss et al., <a href="#bib.bib21" title="" class="ltx_ref">2000</a>)</cite>. The sample of the recently established
<math id="S1.p8.1.m1.1" class="ltx_Math" alttext="\gamma" display="inline"><semantics id="S1.p8.1.m1.1a"><mi id="S1.p8.1.m1.1.1" xref="S1.p8.1.m1.1.1.cmml">γ</mi><annotation-xml encoding="MathML-Content" id="S1.p8.1.m1.1b"><ci id="S1.p8.1.m1.1.1.cmml" xref="S1.p8.1.m1.1.1">𝛾</ci></annotation-xml><annotation encoding="application/x-tex" id="S1.p8.1.m1.1c">\gamma</annotation><annotation encoding="application/x-llamapun" id="S1.p8.1.m1.1d">italic_γ</annotation></semantics></math> Dor subtype (oscillations in non-radial gravity-mode) consists
of only a few dozen stars. One example of the possibilities is our
HAT-1 light-curve of the triple-mode pulsator AC And (Fig. <a href="#S1.F1" title="Figure 1 ‣ 1 Introduction ‣ System description and first light-curves of HAT, an autonomous observatory for variability search" class="ltx_ref"><span class="ltx_text ltx_ref_tag">1</span></a>).</p>
</div>
<figure id="S1.F1" class="ltx_figure"><img src="x1.png" id="S1.F1.g1" class="ltx_graphics" width="676" height="718" alt="Phased light-curves of Algol-type, semi-detached eclipsing
binary TT And, and triple-mode pulsator AC And from HAT observations
(See §">
<figcaption class="ltx_caption"><span class="ltx_tag ltx_tag_figure">Figure 1: </span>Phased light-curves of Algol-type, semi-detached eclipsing
binary TT And, and triple-mode pulsator AC And from HAT observations
(See §<a href="#S7" title="7 Photometric precision of HAT ‣ System description and first light-curves of HAT, an autonomous observatory for variability search" class="ltx_ref"><span class="ltx_text ltx_ref_tag">7</span></a>). “Chaotic” appearance of the AC
light-curve is due to its triple-mode behavior. Phase was computed from
its longest period.
</figcaption>
</figure>
<div id="S1.p9" class="ltx_para">
<p id="S1.p9.1" class="ltx_p">While the number of robotic telescopes is around a hundred, there are
only a few completely autonomous observatories, where the human
supervision is eliminated, and all auxiliary appliances (dome, weather
station, etc.) are under a reliable computer control. These
observatories can be installed to remote sites with adequate
astro-climate and infrastructure (electricity, Internet) without need
of an on-site observer, and daily maintenance. The station can be
monitored via Internet, and operation is not a bottleneck any more.</p>
</div>
<div id="S1.p10" class="ltx_para">
<p id="S1.p10.1" class="ltx_p">HAT is such a small autonomous observatory intended to carry out a
northern counterpart of ASAS, i.e., a variability study of the northern
sky. HAT was developed and constructed by one professional and three
amateur astronomers<span id="footnote3" class="ltx_note ltx_role_footnote"><sup class="ltx_note_mark">3</sup><span class="ltx_note_outer"><span class="ltx_note_content"><sup class="ltx_note_mark">3</sup><span class="ltx_tag ltx_tag_note">3</span>
G. Bakos (astronomical considerations and software),
J. Lázár (software development; www.xperts.hu),
I. Papp (electronic design) and
P. Sári (mechanical engineering)</span></span></span>
in Hungary, and has been fully operational since May, 2001 at Steward
Observatory, Kitt Peak, Arizona. HAT is controlled by a single Linux PC
without human supervision.</p>
</div>
<div id="S1.p11" class="ltx_para">
<p id="S1.p11.13" class="ltx_p">The 180mm focal length and 65mm aperture of the telephoto lens, and the
<math id="S1.p11.1.m1.1" class="ltx_Math" alttext="\rm 2K\times 2K" display="inline"><semantics id="S1.p11.1.m1.1a"><mrow id="S1.p11.1.m1.1.1" xref="S1.p11.1.m1.1.1.cmml"><mrow id="S1.p11.1.m1.1.1.2" xref="S1.p11.1.m1.1.1.2.cmml"><mrow id="S1.p11.1.m1.1.1.2.2" xref="S1.p11.1.m1.1.1.2.2.cmml"><mn id="S1.p11.1.m1.1.1.2.2.2" xref="S1.p11.1.m1.1.1.2.2.2.cmml">2</mn><mo id="S1.p11.1.m1.1.1.2.2.1" xref="S1.p11.1.m1.1.1.2.2.1.cmml">⁢</mo><mi mathvariant="normal" id="S1.p11.1.m1.1.1.2.2.3" xref="S1.p11.1.m1.1.1.2.2.3.cmml">K</mi></mrow><mo id="S1.p11.1.m1.1.1.2.1" xref="S1.p11.1.m1.1.1.2.1.cmml">×</mo><mn id="S1.p11.1.m1.1.1.2.3" xref="S1.p11.1.m1.1.1.2.3.cmml">2</mn></mrow><mo id="S1.p11.1.m1.1.1.1" xref="S1.p11.1.m1.1.1.1.cmml">⁢</mo><mi mathvariant="normal" id="S1.p11.1.m1.1.1.3" xref="S1.p11.1.m1.1.1.3.cmml">K</mi></mrow><annotation-xml encoding="MathML-Content" id="S1.p11.1.m1.1b"><apply id="S1.p11.1.m1.1.1.cmml" xref="S1.p11.1.m1.1.1"><times id="S1.p11.1.m1.1.1.1.cmml" xref="S1.p11.1.m1.1.1.1"></times><apply id="S1.p11.1.m1.1.1.2.cmml" xref="S1.p11.1.m1.1.1.2"><times id="S1.p11.1.m1.1.1.2.1.cmml" xref="S1.p11.1.m1.1.1.2.1"></times><apply id="S1.p11.1.m1.1.1.2.2.cmml" xref="S1.p11.1.m1.1.1.2.2"><times id="S1.p11.1.m1.1.1.2.2.1.cmml" xref="S1.p11.1.m1.1.1.2.2.1"></times><cn type="integer" id="S1.p11.1.m1.1.1.2.2.2.cmml" xref="S1.p11.1.m1.1.1.2.2.2">2</cn><ci id="S1.p11.1.m1.1.1.2.2.3.cmml" xref="S1.p11.1.m1.1.1.2.2.3">K</ci></apply><cn type="integer" id="S1.p11.1.m1.1.1.2.3.cmml" xref="S1.p11.1.m1.1.1.2.3">2</cn></apply><ci id="S1.p11.1.m1.1.1.3.cmml" xref="S1.p11.1.m1.1.1.3">K</ci></apply></annotation-xml><annotation encoding="application/x-tex" id="S1.p11.1.m1.1c">\rm 2K\times 2K</annotation><annotation encoding="application/x-llamapun" id="S1.p11.1.m1.1d">2 roman_K × 2 roman_K</annotation></semantics></math> CCD yields a wide FOV: <math id="S1.p11.2.m2.1" class="ltx_Math" alttext="9\arcdeg\times 9\arcdeg" display="inline"><semantics id="S1.p11.2.m2.1a"><mrow id="S1.p11.2.m2.1.1" xref="S1.p11.2.m2.1.1.cmml"><mrow id="S1.p11.2.m2.1.1.2" xref="S1.p11.2.m2.1.1.2.cmml"><mrow id="S1.p11.2.m2.1.1.2.2" xref="S1.p11.2.m2.1.1.2.2.cmml"><mn id="S1.p11.2.m2.1.1.2.2.2" xref="S1.p11.2.m2.1.1.2.2.2.cmml">9</mn><mo id="S1.p11.2.m2.1.1.2.2.1" xref="S1.p11.2.m2.1.1.2.2.1.cmml">⁢</mo><mi mathvariant="normal" id="S1.p11.2.m2.1.1.2.2.3" xref="S1.p11.2.m2.1.1.2.2.3.cmml">°</mi></mrow><mo id="S1.p11.2.m2.1.1.2.1" xref="S1.p11.2.m2.1.1.2.1.cmml">×</mo><mn id="S1.p11.2.m2.1.1.2.3" xref="S1.p11.2.m2.1.1.2.3.cmml">9</mn></mrow><mo id="S1.p11.2.m2.1.1.1" xref="S1.p11.2.m2.1.1.1.cmml">⁢</mo><mi mathvariant="normal" id="S1.p11.2.m2.1.1.3" xref="S1.p11.2.m2.1.1.3.cmml">°</mi></mrow><annotation-xml encoding="MathML-Content" id="S1.p11.2.m2.1b"><apply id="S1.p11.2.m2.1.1.cmml" xref="S1.p11.2.m2.1.1"><times id="S1.p11.2.m2.1.1.1.cmml" xref="S1.p11.2.m2.1.1.1"></times><apply id="S1.p11.2.m2.1.1.2.cmml" xref="S1.p11.2.m2.1.1.2"><times id="S1.p11.2.m2.1.1.2.1.cmml" xref="S1.p11.2.m2.1.1.2.1"></times><apply id="S1.p11.2.m2.1.1.2.2.cmml" xref="S1.p11.2.m2.1.1.2.2"><times id="S1.p11.2.m2.1.1.2.2.1.cmml" xref="S1.p11.2.m2.1.1.2.2.1"></times><cn type="integer" id="S1.p11.2.m2.1.1.2.2.2.cmml" xref="S1.p11.2.m2.1.1.2.2.2">9</cn><ci id="S1.p11.2.m2.1.1.2.2.3.cmml" xref="S1.p11.2.m2.1.1.2.2.3">°</ci></apply><cn type="integer" id="S1.p11.2.m2.1.1.2.3.cmml" xref="S1.p11.2.m2.1.1.2.3">9</cn></apply><ci id="S1.p11.2.m2.1.1.3.cmml" xref="S1.p11.2.m2.1.1.3">°</ci></apply></annotation-xml><annotation encoding="application/x-tex" id="S1.p11.2.m2.1c">9\arcdeg\times 9\arcdeg</annotation><annotation encoding="application/x-llamapun" id="S1.p11.2.m2.1d">9 ° × 9 °</annotation></semantics></math> on
the sky. Our typical exposure times allow us to study variability of the
<math id="S1.p11.3.m3.1" class="ltx_Math" alttext="\sim 20000" display="inline"><semantics id="S1.p11.3.m3.1a"><mrow id="S1.p11.3.m3.1.1" xref="S1.p11.3.m3.1.1.cmml"><mi id="S1.p11.3.m3.1.1.2" xref="S1.p11.3.m3.1.1.2.cmml"></mi><mo id="S1.p11.3.m3.1.1.1" xref="S1.p11.3.m3.1.1.1.cmml">∼</mo><mn id="S1.p11.3.m3.1.1.3" xref="S1.p11.3.m3.1.1.3.cmml">20000</mn></mrow><annotation-xml encoding="MathML-Content" id="S1.p11.3.m3.1b"><apply id="S1.p11.3.m3.1.1.cmml" xref="S1.p11.3.m3.1.1"><csymbol cd="latexml" id="S1.p11.3.m3.1.1.1.cmml" xref="S1.p11.3.m3.1.1.1">similar-to</csymbol><csymbol cd="latexml" id="S1.p11.3.m3.1.1.2.cmml" xref="S1.p11.3.m3.1.1.2">absent</csymbol><cn type="integer" id="S1.p11.3.m3.1.1.3.cmml" xref="S1.p11.3.m3.1.1.3">20000</cn></apply></annotation-xml><annotation encoding="application/x-tex" id="S1.p11.3.m3.1c">\sim 20000</annotation><annotation encoding="application/x-llamapun" id="S1.p11.3.m3.1d">∼ 20000</annotation></semantics></math> objects per field brighter than <math id="S1.p11.4.m4.1" class="ltx_Math" alttext="\rm I_{c}\approx 13^{m}" display="inline"><semantics id="S1.p11.4.m4.1a"><mrow id="S1.p11.4.m4.1.1" xref="S1.p11.4.m4.1.1.cmml"><msub id="S1.p11.4.m4.1.1.2" xref="S1.p11.4.m4.1.1.2.cmml"><mi mathvariant="normal" id="S1.p11.4.m4.1.1.2.2" xref="S1.p11.4.m4.1.1.2.2.cmml">I</mi><mi mathvariant="normal" id="S1.p11.4.m4.1.1.2.3" xref="S1.p11.4.m4.1.1.2.3.cmml">c</mi></msub><mo id="S1.p11.4.m4.1.1.1" xref="S1.p11.4.m4.1.1.1.cmml">≈</mo><msup id="S1.p11.4.m4.1.1.3" xref="S1.p11.4.m4.1.1.3.cmml"><mn id="S1.p11.4.m4.1.1.3.2" xref="S1.p11.4.m4.1.1.3.2.cmml">13</mn><mi mathvariant="normal" id="S1.p11.4.m4.1.1.3.3" xref="S1.p11.4.m4.1.1.3.3.cmml">m</mi></msup></mrow><annotation-xml encoding="MathML-Content" id="S1.p11.4.m4.1b"><apply id="S1.p11.4.m4.1.1.cmml" xref="S1.p11.4.m4.1.1"><approx id="S1.p11.4.m4.1.1.1.cmml" xref="S1.p11.4.m4.1.1.1"></approx><apply id="S1.p11.4.m4.1.1.2.cmml" xref="S1.p11.4.m4.1.1.2"><csymbol cd="ambiguous" id="S1.p11.4.m4.1.1.2.1.cmml" xref="S1.p11.4.m4.1.1.2">subscript</csymbol><ci id="S1.p11.4.m4.1.1.2.2.cmml" xref="S1.p11.4.m4.1.1.2.2">I</ci><ci id="S1.p11.4.m4.1.1.2.3.cmml" xref="S1.p11.4.m4.1.1.2.3">c</ci></apply><apply id="S1.p11.4.m4.1.1.3.cmml" xref="S1.p11.4.m4.1.1.3"><csymbol cd="ambiguous" id="S1.p11.4.m4.1.1.3.1.cmml" xref="S1.p11.4.m4.1.1.3">superscript</csymbol><cn type="integer" id="S1.p11.4.m4.1.1.3.2.cmml" xref="S1.p11.4.m4.1.1.3.2">13</cn><ci id="S1.p11.4.m4.1.1.3.3.cmml" xref="S1.p11.4.m4.1.1.3.3">m</ci></apply></apply></annotation-xml><annotation encoding="application/x-tex" id="S1.p11.4.m4.1c">\rm I_{c}\approx 13^{m}</annotation><annotation encoding="application/x-llamapun" id="S1.p11.4.m4.1d">roman_I start_POSTSUBSCRIPT roman_c end_POSTSUBSCRIPT ≈ 13 start_POSTSUPERSCRIPT roman_m end_POSTSUPERSCRIPT</annotation></semantics></math> with few
percent precision (<math id="S1.p11.5.m5.1" class="ltx_Math" alttext="\rm 0.01^{m}\--0.05^{m}" display="inline"><semantics id="S1.p11.5.m5.1a"><mrow id="S1.p11.5.m5.1.1" xref="S1.p11.5.m5.1.1.cmml"><msup id="S1.p11.5.m5.1.1.2" xref="S1.p11.5.m5.1.1.2.cmml"><mn id="S1.p11.5.m5.1.1.2.2" xref="S1.p11.5.m5.1.1.2.2.cmml">0.01</mn><mi mathvariant="normal" id="S1.p11.5.m5.1.1.2.3" xref="S1.p11.5.m5.1.1.2.3.cmml">m</mi></msup><mo id="S1.p11.5.m5.1.1.1" xref="S1.p11.5.m5.1.1.1.cmml">-</mo><msup id="S1.p11.5.m5.1.1.3" xref="S1.p11.5.m5.1.1.3.cmml"><mn id="S1.p11.5.m5.1.1.3.2" xref="S1.p11.5.m5.1.1.3.2.cmml">0.05</mn><mi mathvariant="normal" id="S1.p11.5.m5.1.1.3.3" xref="S1.p11.5.m5.1.1.3.3.cmml">m</mi></msup></mrow><annotation-xml encoding="MathML-Content" id="S1.p11.5.m5.1b"><apply id="S1.p11.5.m5.1.1.cmml" xref="S1.p11.5.m5.1.1"><minus id="S1.p11.5.m5.1.1.1.cmml" xref="S1.p11.5.m5.1.1.1"></minus><apply id="S1.p11.5.m5.1.1.2.cmml" xref="S1.p11.5.m5.1.1.2"><csymbol cd="ambiguous" id="S1.p11.5.m5.1.1.2.1.cmml" xref="S1.p11.5.m5.1.1.2">superscript</csymbol><cn type="float" id="S1.p11.5.m5.1.1.2.2.cmml" xref="S1.p11.5.m5.1.1.2.2">0.01</cn><ci id="S1.p11.5.m5.1.1.2.3.cmml" xref="S1.p11.5.m5.1.1.2.3">m</ci></apply><apply id="S1.p11.5.m5.1.1.3.cmml" xref="S1.p11.5.m5.1.1.3"><csymbol cd="ambiguous" id="S1.p11.5.m5.1.1.3.1.cmml" xref="S1.p11.5.m5.1.1.3">superscript</csymbol><cn type="float" id="S1.p11.5.m5.1.1.3.2.cmml" xref="S1.p11.5.m5.1.1.3.2">0.05</cn><ci id="S1.p11.5.m5.1.1.3.3.cmml" xref="S1.p11.5.m5.1.1.3.3">m</ci></apply></apply></annotation-xml><annotation encoding="application/x-tex" id="S1.p11.5.m5.1c">\rm 0.01^{m}\--0.05^{m}</annotation><annotation encoding="application/x-llamapun" id="S1.p11.5.m5.1d">0.01 start_POSTSUPERSCRIPT roman_m end_POSTSUPERSCRIPT - 0.05 start_POSTSUPERSCRIPT roman_m end_POSTSUPERSCRIPT</annotation></semantics></math>), and few minute to one year
time-resolution. As an extreme of the possible observing tactics, HAT
is capable of recording the brightness of <span id="S1.p11.13.1" class="ltx_text ltx_font_italic">every locally and
seasonally</span> visible star in the range of <math id="S1.p11.6.m6.1" class="ltx_Math" alttext="\rm I_{c}\approx 6\--13^{m}" display="inline"><semantics id="S1.p11.6.m6.1a"><mrow id="S1.p11.6.m6.1.1" xref="S1.p11.6.m6.1.1.cmml"><msub id="S1.p11.6.m6.1.1.2" xref="S1.p11.6.m6.1.1.2.cmml"><mi mathvariant="normal" id="S1.p11.6.m6.1.1.2.2" xref="S1.p11.6.m6.1.1.2.2.cmml">I</mi><mi mathvariant="normal" id="S1.p11.6.m6.1.1.2.3" xref="S1.p11.6.m6.1.1.2.3.cmml">c</mi></msub><mo id="S1.p11.6.m6.1.1.1" xref="S1.p11.6.m6.1.1.1.cmml">≈</mo><mrow id="S1.p11.6.m6.1.1.3" xref="S1.p11.6.m6.1.1.3.cmml"><mn id="S1.p11.6.m6.1.1.3.2" xref="S1.p11.6.m6.1.1.3.2.cmml">6</mn><mo id="S1.p11.6.m6.1.1.3.1" xref="S1.p11.6.m6.1.1.3.1.cmml">-</mo><msup id="S1.p11.6.m6.1.1.3.3" xref="S1.p11.6.m6.1.1.3.3.cmml"><mn id="S1.p11.6.m6.1.1.3.3.2" xref="S1.p11.6.m6.1.1.3.3.2.cmml">13</mn><mi mathvariant="normal" id="S1.p11.6.m6.1.1.3.3.3" xref="S1.p11.6.m6.1.1.3.3.3.cmml">m</mi></msup></mrow></mrow><annotation-xml encoding="MathML-Content" id="S1.p11.6.m6.1b"><apply id="S1.p11.6.m6.1.1.cmml" xref="S1.p11.6.m6.1.1"><approx id="S1.p11.6.m6.1.1.1.cmml" xref="S1.p11.6.m6.1.1.1"></approx><apply id="S1.p11.6.m6.1.1.2.cmml" xref="S1.p11.6.m6.1.1.2"><csymbol cd="ambiguous" id="S1.p11.6.m6.1.1.2.1.cmml" xref="S1.p11.6.m6.1.1.2">subscript</csymbol><ci id="S1.p11.6.m6.1.1.2.2.cmml" xref="S1.p11.6.m6.1.1.2.2">I</ci><ci id="S1.p11.6.m6.1.1.2.3.cmml" xref="S1.p11.6.m6.1.1.2.3">c</ci></apply><apply id="S1.p11.6.m6.1.1.3.cmml" xref="S1.p11.6.m6.1.1.3"><minus id="S1.p11.6.m6.1.1.3.1.cmml" xref="S1.p11.6.m6.1.1.3.1"></minus><cn type="integer" id="S1.p11.6.m6.1.1.3.2.cmml" xref="S1.p11.6.m6.1.1.3.2">6</cn><apply id="S1.p11.6.m6.1.1.3.3.cmml" xref="S1.p11.6.m6.1.1.3.3"><csymbol cd="ambiguous" id="S1.p11.6.m6.1.1.3.3.1.cmml" xref="S1.p11.6.m6.1.1.3.3">superscript</csymbol><cn type="integer" id="S1.p11.6.m6.1.1.3.3.2.cmml" xref="S1.p11.6.m6.1.1.3.3.2">13</cn><ci id="S1.p11.6.m6.1.1.3.3.3.cmml" xref="S1.p11.6.m6.1.1.3.3.3">m</ci></apply></apply></apply></annotation-xml><annotation encoding="application/x-tex" id="S1.p11.6.m6.1c">\rm I_{c}\approx 6\--13^{m}</annotation><annotation encoding="application/x-llamapun" id="S1.p11.6.m6.1d">roman_I start_POSTSUBSCRIPT roman_c end_POSTSUBSCRIPT ≈ 6 - 13 start_POSTSUPERSCRIPT roman_m end_POSTSUPERSCRIPT</annotation></semantics></math> in
every second day.
Our limiting magnitude and photometric precision (See
§<a href="#S7" title="7 Photometric precision of HAT ‣ System description and first light-curves of HAT, an autonomous observatory for variability search" class="ltx_ref"><span class="ltx_text ltx_ref_tag">7</span></a>) corresponds to the following detection <span id="S1.p11.13.2" class="ltx_text ltx_font_italic">cut-offs</span> for a few selected variability types (ranges indicate that
the distance limit depends on the luminosity within the type):
<math id="S1.p11.7.m7.1" class="ltx_Math" alttext="\gamma" display="inline"><semantics id="S1.p11.7.m7.1a"><mi id="S1.p11.7.m7.1.1" xref="S1.p11.7.m7.1.1.cmml">γ</mi><annotation-xml encoding="MathML-Content" id="S1.p11.7.m7.1b"><ci id="S1.p11.7.m7.1.1.cmml" xref="S1.p11.7.m7.1.1">𝛾</ci></annotation-xml><annotation encoding="application/x-tex" id="S1.p11.7.m7.1c">\gamma</annotation><annotation encoding="application/x-llamapun" id="S1.p11.7.m7.1d">italic_γ</annotation></semantics></math> Doradus stars (<math id="S1.p11.8.m8.1" class="ltx_Math" alttext="\rm\langle 700pc\rangle" display="inline"><semantics id="S1.p11.8.m8.1a"><mrow id="S1.p11.8.m8.1.1.1" xref="S1.p11.8.m8.1.1.2.cmml"><mo stretchy="false" id="S1.p11.8.m8.1.1.1.2" xref="S1.p11.8.m8.1.1.2.1.cmml">⟨</mo><mrow id="S1.p11.8.m8.1.1.1.1" xref="S1.p11.8.m8.1.1.1.1.cmml"><mn id="S1.p11.8.m8.1.1.1.1.2" xref="S1.p11.8.m8.1.1.1.1.2.cmml">700</mn><mo id="S1.p11.8.m8.1.1.1.1.1" xref="S1.p11.8.m8.1.1.1.1.1.cmml">⁢</mo><mi mathvariant="normal" id="S1.p11.8.m8.1.1.1.1.3" xref="S1.p11.8.m8.1.1.1.1.3.cmml">p</mi><mo id="S1.p11.8.m8.1.1.1.1.1a" xref="S1.p11.8.m8.1.1.1.1.1.cmml">⁢</mo><mi mathvariant="normal" id="S1.p11.8.m8.1.1.1.1.4" xref="S1.p11.8.m8.1.1.1.1.4.cmml">c</mi></mrow><mo stretchy="false" id="S1.p11.8.m8.1.1.1.3" xref="S1.p11.8.m8.1.1.2.1.cmml">⟩</mo></mrow><annotation-xml encoding="MathML-Content" id="S1.p11.8.m8.1b"><apply id="S1.p11.8.m8.1.1.2.cmml" xref="S1.p11.8.m8.1.1.1"><csymbol cd="latexml" id="S1.p11.8.m8.1.1.2.1.cmml" xref="S1.p11.8.m8.1.1.1.2">delimited-⟨⟩</csymbol><apply id="S1.p11.8.m8.1.1.1.1.cmml" xref="S1.p11.8.m8.1.1.1.1"><times id="S1.p11.8.m8.1.1.1.1.1.cmml" xref="S1.p11.8.m8.1.1.1.1.1"></times><cn type="integer" id="S1.p11.8.m8.1.1.1.1.2.cmml" xref="S1.p11.8.m8.1.1.1.1.2">700</cn><ci id="S1.p11.8.m8.1.1.1.1.3.cmml" xref="S1.p11.8.m8.1.1.1.1.3">p</ci><ci id="S1.p11.8.m8.1.1.1.1.4.cmml" xref="S1.p11.8.m8.1.1.1.1.4">c</ci></apply></apply></annotation-xml><annotation encoding="application/x-tex" id="S1.p11.8.m8.1c">\rm\langle 700pc\rangle</annotation><annotation encoding="application/x-llamapun" id="S1.p11.8.m8.1d">⟨ 700 roman_p roman_c ⟩</annotation></semantics></math>),
<math id="S1.p11.9.m9.1" class="ltx_Math" alttext="\delta" display="inline"><semantics id="S1.p11.9.m9.1a"><mi id="S1.p11.9.m9.1.1" xref="S1.p11.9.m9.1.1.cmml">δ</mi><annotation-xml encoding="MathML-Content" id="S1.p11.9.m9.1b"><ci id="S1.p11.9.m9.1.1.cmml" xref="S1.p11.9.m9.1.1">𝛿</ci></annotation-xml><annotation encoding="application/x-tex" id="S1.p11.9.m9.1c">\delta</annotation><annotation encoding="application/x-llamapun" id="S1.p11.9.m9.1d">italic_δ</annotation></semantics></math> Scuti (<math id="S1.p11.10.m10.1" class="ltx_Math" alttext="\rm 1\--2.5kpc" display="inline"><semantics id="S1.p11.10.m10.1a"><mrow id="S1.p11.10.m10.1.1" xref="S1.p11.10.m10.1.1.cmml"><mn id="S1.p11.10.m10.1.1.2" xref="S1.p11.10.m10.1.1.2.cmml">1</mn><mo id="S1.p11.10.m10.1.1.1" xref="S1.p11.10.m10.1.1.1.cmml">-</mo><mrow id="S1.p11.10.m10.1.1.3" xref="S1.p11.10.m10.1.1.3.cmml"><mn id="S1.p11.10.m10.1.1.3.2" xref="S1.p11.10.m10.1.1.3.2.cmml">2.5</mn><mo id="S1.p11.10.m10.1.1.3.1" xref="S1.p11.10.m10.1.1.3.1.cmml">⁢</mo><mi id="S1.p11.10.m10.1.1.3.3" xref="S1.p11.10.m10.1.1.3.3.cmml">kpc</mi></mrow></mrow><annotation-xml encoding="MathML-Content" id="S1.p11.10.m10.1b"><apply id="S1.p11.10.m10.1.1.cmml" xref="S1.p11.10.m10.1.1"><minus id="S1.p11.10.m10.1.1.1.cmml" xref="S1.p11.10.m10.1.1.1"></minus><cn type="integer" id="S1.p11.10.m10.1.1.2.cmml" xref="S1.p11.10.m10.1.1.2">1</cn><apply id="S1.p11.10.m10.1.1.3.cmml" xref="S1.p11.10.m10.1.1.3"><times id="S1.p11.10.m10.1.1.3.1.cmml" xref="S1.p11.10.m10.1.1.3.1"></times><cn type="float" id="S1.p11.10.m10.1.1.3.2.cmml" xref="S1.p11.10.m10.1.1.3.2">2.5</cn><ci id="S1.p11.10.m10.1.1.3.3.cmml" xref="S1.p11.10.m10.1.1.3.3">kpc</ci></apply></apply></annotation-xml><annotation encoding="application/x-tex" id="S1.p11.10.m10.1c">\rm 1\--2.5kpc</annotation><annotation encoding="application/x-llamapun" id="S1.p11.10.m10.1d">1 - 2.5 roman_kpc</annotation></semantics></math>),
RR Lyrae (<math id="S1.p11.11.m11.1" class="ltx_Math" alttext="\rm 3kpc" display="inline"><semantics id="S1.p11.11.m11.1a"><mrow id="S1.p11.11.m11.1.1" xref="S1.p11.11.m11.1.1.cmml"><mn id="S1.p11.11.m11.1.1.2" xref="S1.p11.11.m11.1.1.2.cmml">3</mn><mo id="S1.p11.11.m11.1.1.1" xref="S1.p11.11.m11.1.1.1.cmml">⁢</mo><mi mathvariant="normal" id="S1.p11.11.m11.1.1.3" xref="S1.p11.11.m11.1.1.3.cmml">k</mi><mo id="S1.p11.11.m11.1.1.1a" xref="S1.p11.11.m11.1.1.1.cmml">⁢</mo><mi mathvariant="normal" id="S1.p11.11.m11.1.1.4" xref="S1.p11.11.m11.1.1.4.cmml">p</mi><mo id="S1.p11.11.m11.1.1.1b" xref="S1.p11.11.m11.1.1.1.cmml">⁢</mo><mi mathvariant="normal" id="S1.p11.11.m11.1.1.5" xref="S1.p11.11.m11.1.1.5.cmml">c</mi></mrow><annotation-xml encoding="MathML-Content" id="S1.p11.11.m11.1b"><apply id="S1.p11.11.m11.1.1.cmml" xref="S1.p11.11.m11.1.1"><times id="S1.p11.11.m11.1.1.1.cmml" xref="S1.p11.11.m11.1.1.1"></times><cn type="integer" id="S1.p11.11.m11.1.1.2.cmml" xref="S1.p11.11.m11.1.1.2">3</cn><ci id="S1.p11.11.m11.1.1.3.cmml" xref="S1.p11.11.m11.1.1.3">k</ci><ci id="S1.p11.11.m11.1.1.4.cmml" xref="S1.p11.11.m11.1.1.4">p</ci><ci id="S1.p11.11.m11.1.1.5.cmml" xref="S1.p11.11.m11.1.1.5">c</ci></apply></annotation-xml><annotation encoding="application/x-tex" id="S1.p11.11.m11.1c">\rm 3kpc</annotation><annotation encoding="application/x-llamapun" id="S1.p11.11.m11.1d">3 roman_k roman_p roman_c</annotation></semantics></math>), Cepheids (<math id="S1.p11.12.m12.1" class="ltx_Math" alttext="\rm 10\--150kpc" display="inline"><semantics id="S1.p11.12.m12.1a"><mrow id="S1.p11.12.m12.1.1" xref="S1.p11.12.m12.1.1.cmml"><mn id="S1.p11.12.m12.1.1.2" xref="S1.p11.12.m12.1.1.2.cmml">10</mn><mo id="S1.p11.12.m12.1.1.1" xref="S1.p11.12.m12.1.1.1.cmml">-</mo><mrow id="S1.p11.12.m12.1.1.3" xref="S1.p11.12.m12.1.1.3.cmml"><mn id="S1.p11.12.m12.1.1.3.2" xref="S1.p11.12.m12.1.1.3.2.cmml">150</mn><mo id="S1.p11.12.m12.1.1.3.1" xref="S1.p11.12.m12.1.1.3.1.cmml">⁢</mo><mi mathvariant="normal" id="S1.p11.12.m12.1.1.3.3" xref="S1.p11.12.m12.1.1.3.3.cmml">k</mi><mo id="S1.p11.12.m12.1.1.3.1a" xref="S1.p11.12.m12.1.1.3.1.cmml">⁢</mo><mi mathvariant="normal" id="S1.p11.12.m12.1.1.3.4" xref="S1.p11.12.m12.1.1.3.4.cmml">p</mi><mo id="S1.p11.12.m12.1.1.3.1b" xref="S1.p11.12.m12.1.1.3.1.cmml">⁢</mo><mi mathvariant="normal" id="S1.p11.12.m12.1.1.3.5" xref="S1.p11.12.m12.1.1.3.5.cmml">c</mi></mrow></mrow><annotation-xml encoding="MathML-Content" id="S1.p11.12.m12.1b"><apply id="S1.p11.12.m12.1.1.cmml" xref="S1.p11.12.m12.1.1"><minus id="S1.p11.12.m12.1.1.1.cmml" xref="S1.p11.12.m12.1.1.1"></minus><cn type="integer" id="S1.p11.12.m12.1.1.2.cmml" xref="S1.p11.12.m12.1.1.2">10</cn><apply id="S1.p11.12.m12.1.1.3.cmml" xref="S1.p11.12.m12.1.1.3"><times id="S1.p11.12.m12.1.1.3.1.cmml" xref="S1.p11.12.m12.1.1.3.1"></times><cn type="integer" id="S1.p11.12.m12.1.1.3.2.cmml" xref="S1.p11.12.m12.1.1.3.2">150</cn><ci id="S1.p11.12.m12.1.1.3.3.cmml" xref="S1.p11.12.m12.1.1.3.3">k</ci><ci id="S1.p11.12.m12.1.1.3.4.cmml" xref="S1.p11.12.m12.1.1.3.4">p</ci><ci id="S1.p11.12.m12.1.1.3.5.cmml" xref="S1.p11.12.m12.1.1.3.5">c</ci></apply></apply></annotation-xml><annotation encoding="application/x-tex" id="S1.p11.12.m12.1c">\rm 10\--150kpc</annotation><annotation encoding="application/x-llamapun" id="S1.p11.12.m12.1d">10 - 150 roman_k roman_p roman_c</annotation></semantics></math>), Miras and semiregular
variables (<math id="S1.p11.13.m13.1" class="ltx_Math" alttext="\rm\langle 60kpc\rangle" display="inline"><semantics id="S1.p11.13.m13.1a"><mrow id="S1.p11.13.m13.1.1.1" xref="S1.p11.13.m13.1.1.2.cmml"><mo stretchy="false" id="S1.p11.13.m13.1.1.1.2" xref="S1.p11.13.m13.1.1.2.1.cmml">⟨</mo><mrow id="S1.p11.13.m13.1.1.1.1" xref="S1.p11.13.m13.1.1.1.1.cmml"><mn id="S1.p11.13.m13.1.1.1.1.2" xref="S1.p11.13.m13.1.1.1.1.2.cmml">60</mn><mo id="S1.p11.13.m13.1.1.1.1.1" xref="S1.p11.13.m13.1.1.1.1.1.cmml">⁢</mo><mi mathvariant="normal" id="S1.p11.13.m13.1.1.1.1.3" xref="S1.p11.13.m13.1.1.1.1.3.cmml">k</mi><mo id="S1.p11.13.m13.1.1.1.1.1a" xref="S1.p11.13.m13.1.1.1.1.1.cmml">⁢</mo><mi mathvariant="normal" id="S1.p11.13.m13.1.1.1.1.4" xref="S1.p11.13.m13.1.1.1.1.4.cmml">p</mi><mo id="S1.p11.13.m13.1.1.1.1.1b" xref="S1.p11.13.m13.1.1.1.1.1.cmml">⁢</mo><mi mathvariant="normal" id="S1.p11.13.m13.1.1.1.1.5" xref="S1.p11.13.m13.1.1.1.1.5.cmml">c</mi></mrow><mo stretchy="false" id="S1.p11.13.m13.1.1.1.3" xref="S1.p11.13.m13.1.1.2.1.cmml">⟩</mo></mrow><annotation-xml encoding="MathML-Content" id="S1.p11.13.m13.1b"><apply id="S1.p11.13.m13.1.1.2.cmml" xref="S1.p11.13.m13.1.1.1"><csymbol cd="latexml" id="S1.p11.13.m13.1.1.2.1.cmml" xref="S1.p11.13.m13.1.1.1.2">delimited-⟨⟩</csymbol><apply id="S1.p11.13.m13.1.1.1.1.cmml" xref="S1.p11.13.m13.1.1.1.1"><times id="S1.p11.13.m13.1.1.1.1.1.cmml" xref="S1.p11.13.m13.1.1.1.1.1"></times><cn type="integer" id="S1.p11.13.m13.1.1.1.1.2.cmml" xref="S1.p11.13.m13.1.1.1.1.2">60</cn><ci id="S1.p11.13.m13.1.1.1.1.3.cmml" xref="S1.p11.13.m13.1.1.1.1.3">k</ci><ci id="S1.p11.13.m13.1.1.1.1.4.cmml" xref="S1.p11.13.m13.1.1.1.1.4">p</ci><ci id="S1.p11.13.m13.1.1.1.1.5.cmml" xref="S1.p11.13.m13.1.1.1.1.5">c</ci></apply></apply></annotation-xml><annotation encoding="application/x-tex" id="S1.p11.13.m13.1c">\rm\langle 60kpc\rangle</annotation><annotation encoding="application/x-llamapun" id="S1.p11.13.m13.1d">⟨ 60 roman_k roman_p roman_c ⟩</annotation></semantics></math>). The limits are based on the
luminosity of the sources, and – especially at the larger values –
overestimate true detection cut-offs, as they do not take into account
reddening and crowding.</p>
</div>
<div id="S1.p12" class="ltx_para">
<p id="S1.p12.1" class="ltx_p">With the above specifications, HAT is also suitable for exo-planet
search via transits, which is also included in our program. However, we
concentrate on a broader range of variabilities: a large fraction of
the sky is monitored sparsely, and few selected fields frequently so as
to have preliminary results on short-period changes.</p>
</div>
<div id="S1.p13" class="ltx_para">
<p id="S1.p13.1" class="ltx_p">We expect that our survey will contribute to most of the aforementioned
issues related to variability search (public archive, alert system),
and supplement the incomplete variable classes by new discoveries.</p>
</div>
<div id="S1.p14" class="ltx_para">
<p id="S1.p14.2" class="ltx_p">Our current data rate is roughly <math id="S1.p14.1.m1.1" class="ltx_Math" alttext="10^{6}" display="inline"><semantics id="S1.p14.1.m1.1a"><msup id="S1.p14.1.m1.1.1" xref="S1.p14.1.m1.1.1.cmml"><mn id="S1.p14.1.m1.1.1.2" xref="S1.p14.1.m1.1.1.2.cmml">10</mn><mn id="S1.p14.1.m1.1.1.3" xref="S1.p14.1.m1.1.1.3.cmml">6</mn></msup><annotation-xml encoding="MathML-Content" id="S1.p14.1.m1.1b"><apply id="S1.p14.1.m1.1.1.cmml" xref="S1.p14.1.m1.1.1"><csymbol cd="ambiguous" id="S1.p14.1.m1.1.1.1.cmml" xref="S1.p14.1.m1.1.1">superscript</csymbol><cn type="integer" id="S1.p14.1.m1.1.1.2.cmml" xref="S1.p14.1.m1.1.1.2">10</cn><cn type="integer" id="S1.p14.1.m1.1.1.3.cmml" xref="S1.p14.1.m1.1.1.3">6</cn></apply></annotation-xml><annotation encoding="application/x-tex" id="S1.p14.1.m1.1c">10^{6}</annotation><annotation encoding="application/x-llamapun" id="S1.p14.1.m1.1d">10 start_POSTSUPERSCRIPT 6 end_POSTSUPERSCRIPT</annotation></semantics></math> photometric measurement, or two
Gigabytes of raw data per night. Typically a few percent of the sources
are variable, i.e., variable light-curves are supplemented by <math id="S1.p14.2.m2.1" class="ltx_Math" alttext="\sim 20000" display="inline"><semantics id="S1.p14.2.m2.1a"><mrow id="S1.p14.2.m2.1.1" xref="S1.p14.2.m2.1.1.cmml"><mi id="S1.p14.2.m2.1.1.2" xref="S1.p14.2.m2.1.1.2.cmml"></mi><mo id="S1.p14.2.m2.1.1.1" xref="S1.p14.2.m2.1.1.1.cmml">∼</mo><mn id="S1.p14.2.m2.1.1.3" xref="S1.p14.2.m2.1.1.3.cmml">20000</mn></mrow><annotation-xml encoding="MathML-Content" id="S1.p14.2.m2.1b"><apply id="S1.p14.2.m2.1.1.cmml" xref="S1.p14.2.m2.1.1"><csymbol cd="latexml" id="S1.p14.2.m2.1.1.1.cmml" xref="S1.p14.2.m2.1.1.1">similar-to</csymbol><csymbol cd="latexml" id="S1.p14.2.m2.1.1.2.cmml" xref="S1.p14.2.m2.1.1.2">absent</csymbol><cn type="integer" id="S1.p14.2.m2.1.1.3.cmml" xref="S1.p14.2.m2.1.1.3">20000</cn></apply></annotation-xml><annotation encoding="application/x-tex" id="S1.p14.2.m2.1c">\sim 20000</annotation><annotation encoding="application/x-llamapun" id="S1.p14.2.m2.1d">∼ 20000</annotation></semantics></math> data points each night.</p>
</div>
<div id="S1.p15" class="ltx_para">
<p id="S1.p15.1" class="ltx_p">HAT is monitoring only a fraction of the northern sky, but given the
fact that it is an off-the-shelf system there is perspective for
installation of new units in the near future.</p>
</div>
<div id="S1.p16" class="ltx_para">
<p id="S1.p16.1" class="ltx_p">The paper is arranged as follows: §<a href="#S2" title="2 Hardware System ‣ System description and first light-curves of HAT, an autonomous observatory for variability search" class="ltx_ref"><span class="ltx_text ltx_ref_tag">2</span></a> gives an overview on
the hardware, §<a href="#S3" title="3 Software System ‣ System description and first light-curves of HAT, an autonomous observatory for variability search" class="ltx_ref"><span class="ltx_text ltx_ref_tag">3</span></a> describes our software environment, §<a href="#S4" title="4 Pointing precision of HAT ‣ System description and first light-curves of HAT, an autonomous observatory for variability search" class="ltx_ref"><span class="ltx_text ltx_ref_tag">4</span></a> quantifies the pointing precision of HAT, §<a href="#S5" title="5 Installation of HAT-1 on Kitt Peak ‣ System description and first light-curves of HAT, an autonomous observatory for variability search" class="ltx_ref"><span class="ltx_text ltx_ref_tag">5</span></a> and §<a href="#S6" title="6 Observations from Kitt Peak ‣ System description and first light-curves of HAT, an autonomous observatory for variability search" class="ltx_ref"><span class="ltx_text ltx_ref_tag">6</span></a> summarize our site-specific
installation at Kitt Peak and observations in the past one year, §<a href="#S7" title="7 Photometric precision of HAT ‣ System description and first light-curves of HAT, an autonomous observatory for variability search" class="ltx_ref"><span class="ltx_text ltx_ref_tag">7</span></a> estimates our photometric precision, §<a href="#S8" title="8 Summary and future directions ‣ System description and first light-curves of HAT, an autonomous observatory for variability search" class="ltx_ref"><span class="ltx_text ltx_ref_tag">8</span></a>
gives summary and future directions.</p>
</div>
</section>
<section id="S2" class="ltx_section">
<h2 class="ltx_title ltx_title_section">
<span class="ltx_tag ltx_tag_section">2 </span>Hardware System</h2>

<div id="S2.p1" class="ltx_para">
<p id="S2.p1.1" class="ltx_p">The HAT hardware consists of an equatorial telescope mount, enclosure
(dome), CCD, a telephoto lens and a PC. Several devices are attached to
the dome and telescope, such as rain-detector, photosensor,
lens-heating and domeflat lights. The PC is protected from inclement
weather by a close-by heated room (“warmroom”). Two parallel port
cables with lightning protection connect the dome to the PC, and are
responsible for driving the telescope and dome devices. Cables for the
CCD depend on the specific setup; in our case a serial line for a Meade
Pictor CCD (for testing purposes) and a custom data cable-pair for an
Apogee AP10 CCD run from the PC to the telescope. A cable in a separate
conduit carries 110V AC to the dome.</p>
</div>
<section id="S2.SS1" class="ltx_subsection">
<h3 class="ltx_title ltx_title_subsection">
<span class="ltx_tag ltx_tag_subsection">2.1 </span>The Robotic Mount</h3>

<div id="S2.SS1.p1" class="ltx_para">
<p id="S2.SS1.p1.1" class="ltx_p">Requirements of a mount for massive variability search are: reliability
(for <math id="S2.SS1.p1.1.m1.1" class="ltx_Math" alttext="10^{5}\--10^{6}" display="inline"><semantics id="S2.SS1.p1.1.m1.1a"><mrow id="S2.SS1.p1.1.m1.1.1" xref="S2.SS1.p1.1.m1.1.1.cmml"><msup id="S2.SS1.p1.1.m1.1.1.2" xref="S2.SS1.p1.1.m1.1.1.2.cmml"><mn id="S2.SS1.p1.1.m1.1.1.2.2" xref="S2.SS1.p1.1.m1.1.1.2.2.cmml">10</mn><mn id="S2.SS1.p1.1.m1.1.1.2.3" xref="S2.SS1.p1.1.m1.1.1.2.3.cmml">5</mn></msup><mo id="S2.SS1.p1.1.m1.1.1.1" xref="S2.SS1.p1.1.m1.1.1.1.cmml">-</mo><msup id="S2.SS1.p1.1.m1.1.1.3" xref="S2.SS1.p1.1.m1.1.1.3.cmml"><mn id="S2.SS1.p1.1.m1.1.1.3.2" xref="S2.SS1.p1.1.m1.1.1.3.2.cmml">10</mn><mn id="S2.SS1.p1.1.m1.1.1.3.3" xref="S2.SS1.p1.1.m1.1.1.3.3.cmml">6</mn></msup></mrow><annotation-xml encoding="MathML-Content" id="S2.SS1.p1.1.m1.1b"><apply id="S2.SS1.p1.1.m1.1.1.cmml" xref="S2.SS1.p1.1.m1.1.1"><minus id="S2.SS1.p1.1.m1.1.1.1.cmml" xref="S2.SS1.p1.1.m1.1.1.1"></minus><apply id="S2.SS1.p1.1.m1.1.1.2.cmml" xref="S2.SS1.p1.1.m1.1.1.2"><csymbol cd="ambiguous" id="S2.SS1.p1.1.m1.1.1.2.1.cmml" xref="S2.SS1.p1.1.m1.1.1.2">superscript</csymbol><cn type="integer" id="S2.SS1.p1.1.m1.1.1.2.2.cmml" xref="S2.SS1.p1.1.m1.1.1.2.2">10</cn><cn type="integer" id="S2.SS1.p1.1.m1.1.1.2.3.cmml" xref="S2.SS1.p1.1.m1.1.1.2.3">5</cn></apply><apply id="S2.SS1.p1.1.m1.1.1.3.cmml" xref="S2.SS1.p1.1.m1.1.1.3"><csymbol cd="ambiguous" id="S2.SS1.p1.1.m1.1.1.3.1.cmml" xref="S2.SS1.p1.1.m1.1.1.3">superscript</csymbol><cn type="integer" id="S2.SS1.p1.1.m1.1.1.3.2.cmml" xref="S2.SS1.p1.1.m1.1.1.3.2">10</cn><cn type="integer" id="S2.SS1.p1.1.m1.1.1.3.3.cmml" xref="S2.SS1.p1.1.m1.1.1.3.3">6</cn></apply></apply></annotation-xml><annotation encoding="application/x-tex" id="S2.SS1.p1.1.m1.1c">10^{5}\--10^{6}</annotation><annotation encoding="application/x-llamapun" id="S2.SS1.p1.1.m1.1d">10 start_POSTSUPERSCRIPT 5 end_POSTSUPERSCRIPT - 10 start_POSTSUPERSCRIPT 6 end_POSTSUPERSCRIPT</annotation></semantics></math> actuation), pointing to an accuracy of few orders of
magnitude smaller than the FOV, ability to recover from awkward
positions, never losing orientation, and relatively quick slew-time.</p>
</div>
<div id="S2.SS1.p2" class="ltx_para">
<p id="S2.SS1.p2.1" class="ltx_p">Our mount started as a replica of Gregorz Pojmański’s instrument for
all-sky monitoring <cite class="ltx_cite ltx_citemacro_citep">(Pojmański, <a href="#bib.bib39" title="" class="ltx_ref">1997</a>)</cite>, who kindly shared his plans with our
group. The main mechanical concept of the mount is left unchanged,
i.e., it is a backlash-free friction drive of a horseshoe structure.
Most of the details and dimensions, however, were re-designed. The
mount is machined from forged aluminum alloy, to minimize the
possibility of slow warpage, which could cause uncertainties in the
pointing. All parts are anodized, to yield a tough and resistant
surface.</p>
</div>
<div id="S2.SS1.p3" class="ltx_para">
<p id="S2.SS1.p3.1" class="ltx_p">The base of HAT is a <math id="S2.SS1.p3.1.m1.1" class="ltx_Math" alttext="\rm\sim 200mm\oslash" display="inline"><semantics id="S2.SS1.p3.1.m1.1a"><mrow id="S2.SS1.p3.1.m1.1b"><mo id="S2.SS1.p3.1.m1.1.1" xref="S2.SS1.p3.1.m1.1.1.cmml">∼</mo><mn id="S2.SS1.p3.1.m1.1.2" xref="S2.SS1.p3.1.m1.1.2.cmml">200</mn><mi mathvariant="normal" id="S2.SS1.p3.1.m1.1.3" xref="S2.SS1.p3.1.m1.1.3.cmml">m</mi><mi mathvariant="normal" id="S2.SS1.p3.1.m1.1.4" xref="S2.SS1.p3.1.m1.1.4.cmml">m</mi><mo id="S2.SS1.p3.1.m1.1.5" xref="S2.SS1.p3.1.m1.1.5.cmml">⊘</mo></mrow><annotation-xml encoding="MathML-Content" id="S2.SS1.p3.1.m1.1c"><cerror id="S2.SS1.p3.1.m1.1d"><csymbol cd="ambiguous" id="S2.SS1.p3.1.m1.1e">fragments</csymbol><csymbol cd="latexml" id="S2.SS1.p3.1.m1.1.1.cmml" xref="S2.SS1.p3.1.m1.1.1">similar-to</csymbol><cn type="integer" id="S2.SS1.p3.1.m1.1.2.cmml" xref="S2.SS1.p3.1.m1.1.2">200</cn><csymbol cd="unknown" id="S2.SS1.p3.1.m1.1.3.cmml" xref="S2.SS1.p3.1.m1.1.3">m</csymbol><csymbol cd="unknown" id="S2.SS1.p3.1.m1.1.4.cmml" xref="S2.SS1.p3.1.m1.1.4">m</csymbol><ci id="S2.SS1.p3.1.m1.1.5.cmml" xref="S2.SS1.p3.1.m1.1.5">⊘</ci></cerror></annotation-xml><annotation encoding="application/x-tex" id="S2.SS1.p3.1.m1.1f">\rm\sim 200mm\oslash</annotation><annotation encoding="application/x-llamapun" id="S2.SS1.p3.1.m1.1g">∼ 200 roman_m roman_m ⊘</annotation></semantics></math> (diameter) disc, which can
be mounted on a pier (see Fig. <a href="#S2.F2" title="Figure 2 ‣ 2.1 The Robotic Mount ‣ 2 Hardware System ‣ System description and first light-curves of HAT, an autonomous observatory for variability search" class="ltx_ref"><span class="ltx_text ltx_ref_tag">2</span></a> for details of the
mount). A rectangular steel plate can smoothly rotate on it, and can be
fine-tuned by two adjusting screws, thus enabling the azimuth setting
during polar axis adjustments.</p>
</div>
<div id="S2.SS1.p4" class="ltx_para">
<p id="S2.SS1.p4.2" class="ltx_p">The base frame of the telescope is held to the steel plate by a fixed
screw and another screw freely running in altitude in a groove, for
both sides. It can be gently adjusted in altitude <math id="S2.SS1.p4.1.m1.1" class="ltx_Math" alttext="\pm 15\arcdeg" display="inline"><semantics id="S2.SS1.p4.1.m1.1a"><mrow id="S2.SS1.p4.1.m1.1.1" xref="S2.SS1.p4.1.m1.1.1.cmml"><mo id="S2.SS1.p4.1.m1.1.1.1" xref="S2.SS1.p4.1.m1.1.1.1.cmml">±</mo><mrow id="S2.SS1.p4.1.m1.1.1.2" xref="S2.SS1.p4.1.m1.1.1.2.cmml"><mn id="S2.SS1.p4.1.m1.1.1.2.2" xref="S2.SS1.p4.1.m1.1.1.2.2.cmml">15</mn><mo id="S2.SS1.p4.1.m1.1.1.2.1" xref="S2.SS1.p4.1.m1.1.1.2.1.cmml">⁢</mo><mi mathvariant="normal" id="S2.SS1.p4.1.m1.1.1.2.3" xref="S2.SS1.p4.1.m1.1.1.2.3.cmml">°</mi></mrow></mrow><annotation-xml encoding="MathML-Content" id="S2.SS1.p4.1.m1.1b"><apply id="S2.SS1.p4.1.m1.1.1.cmml" xref="S2.SS1.p4.1.m1.1.1"><csymbol cd="latexml" id="S2.SS1.p4.1.m1.1.1.1.cmml" xref="S2.SS1.p4.1.m1.1.1.1">plus-or-minus</csymbol><apply id="S2.SS1.p4.1.m1.1.1.2.cmml" xref="S2.SS1.p4.1.m1.1.1.2"><times id="S2.SS1.p4.1.m1.1.1.2.1.cmml" xref="S2.SS1.p4.1.m1.1.1.2.1"></times><cn type="integer" id="S2.SS1.p4.1.m1.1.1.2.2.cmml" xref="S2.SS1.p4.1.m1.1.1.2.2">15</cn><ci id="S2.SS1.p4.1.m1.1.1.2.3.cmml" xref="S2.SS1.p4.1.m1.1.1.2.3">°</ci></apply></apply></annotation-xml><annotation encoding="application/x-tex" id="S2.SS1.p4.1.m1.1c">\pm 15\arcdeg</annotation><annotation encoding="application/x-llamapun" id="S2.SS1.p4.1.m1.1d">± 15 °</annotation></semantics></math>
(default center is <math id="S2.SS1.p4.2.m2.1" class="ltx_Math" alttext="45\arcdeg" display="inline"><semantics id="S2.SS1.p4.2.m2.1a"><mrow id="S2.SS1.p4.2.m2.1.1" xref="S2.SS1.p4.2.m2.1.1.cmml"><mn id="S2.SS1.p4.2.m2.1.1.2" xref="S2.SS1.p4.2.m2.1.1.2.cmml">45</mn><mo id="S2.SS1.p4.2.m2.1.1.1" xref="S2.SS1.p4.2.m2.1.1.1.cmml">⁢</mo><mi mathvariant="normal" id="S2.SS1.p4.2.m2.1.1.3" xref="S2.SS1.p4.2.m2.1.1.3.cmml">°</mi></mrow><annotation-xml encoding="MathML-Content" id="S2.SS1.p4.2.m2.1b"><apply id="S2.SS1.p4.2.m2.1.1.cmml" xref="S2.SS1.p4.2.m2.1.1"><times id="S2.SS1.p4.2.m2.1.1.1.cmml" xref="S2.SS1.p4.2.m2.1.1.1"></times><cn type="integer" id="S2.SS1.p4.2.m2.1.1.2.cmml" xref="S2.SS1.p4.2.m2.1.1.2">45</cn><ci id="S2.SS1.p4.2.m2.1.1.3.cmml" xref="S2.SS1.p4.2.m2.1.1.3">°</ci></apply></annotation-xml><annotation encoding="application/x-tex" id="S2.SS1.p4.2.m2.1c">45\arcdeg</annotation><annotation encoding="application/x-llamapun" id="S2.SS1.p4.2.m2.1d">45 °</annotation></semantics></math>) with a spindle, much like the azimuth
setting, and can be secured with the freely running screw. Inclinations
outside these limits can be easily set by tilting the mounting of the
base plate.</p>
</div>
<figure id="S2.F2" class="ltx_figure"><img src="x2.png" id="S2.F2.g1" class="ltx_graphics" width="676" height="809" alt="HAT mount drawing without the instrument mounting plate, CCD
and lens. Labels on the image:
A: main azimuth disc,
B: rectangular base frame,
C: altitude groove and adjusting spindle,
D: RA bearing-housing and polar telescope hole,
E: RA loose roller,
F: RA friction-drive,
G: horseshoe,
H: horseshoe holding arm,
I: RA and Dec stepper motors,
J: RA driving mechanism: sprockets and cogwheels,
K: Dec driving mechanism,
L: Dec discs,
M: Dec lateral bars,
N: counterweights,
O: proxy sensors.
">
<figcaption class="ltx_caption"><span class="ltx_tag ltx_tag_figure">Figure 2: </span>HAT mount drawing without the instrument mounting plate, CCD
and lens. Labels on the image:
A: main azimuth disc,
B: rectangular base frame,
C: altitude groove and adjusting spindle,
D: RA bearing-housing and polar telescope hole,
E: RA loose roller,
F: RA friction-drive,
G: horseshoe,
H: horseshoe holding arm,
I: RA and Dec stepper motors,
J: RA driving mechanism: sprockets and cogwheels,
K: Dec driving mechanism,
L: Dec discs,
M: Dec lateral bars,
N: counterweights,
O: proxy sensors.
</figcaption>
</figure>
<div id="S2.SS1.p5" class="ltx_para">
<p id="S2.SS1.p5.3" class="ltx_p">The horseshoe is supported at three points: the bearing-housing, a
loose roller, and the right ascension friction-drive. The horseshoe
outer diameter is <math id="S2.SS1.p5.1.m1.1" class="ltx_Math" alttext="\rm\sim 500mm" display="inline"><semantics id="S2.SS1.p5.1.m1.1a"><mrow id="S2.SS1.p5.1.m1.1.1" xref="S2.SS1.p5.1.m1.1.1.cmml"><mi id="S2.SS1.p5.1.m1.1.1.2" xref="S2.SS1.p5.1.m1.1.1.2.cmml"></mi><mo id="S2.SS1.p5.1.m1.1.1.1" xref="S2.SS1.p5.1.m1.1.1.1.cmml">∼</mo><mrow id="S2.SS1.p5.1.m1.1.1.3" xref="S2.SS1.p5.1.m1.1.1.3.cmml"><mn id="S2.SS1.p5.1.m1.1.1.3.2" xref="S2.SS1.p5.1.m1.1.1.3.2.cmml">500</mn><mo id="S2.SS1.p5.1.m1.1.1.3.1" xref="S2.SS1.p5.1.m1.1.1.3.1.cmml">⁢</mo><mi mathvariant="normal" id="S2.SS1.p5.1.m1.1.1.3.3" xref="S2.SS1.p5.1.m1.1.1.3.3.cmml">m</mi><mo id="S2.SS1.p5.1.m1.1.1.3.1a" xref="S2.SS1.p5.1.m1.1.1.3.1.cmml">⁢</mo><mi mathvariant="normal" id="S2.SS1.p5.1.m1.1.1.3.4" xref="S2.SS1.p5.1.m1.1.1.3.4.cmml">m</mi></mrow></mrow><annotation-xml encoding="MathML-Content" id="S2.SS1.p5.1.m1.1b"><apply id="S2.SS1.p5.1.m1.1.1.cmml" xref="S2.SS1.p5.1.m1.1.1"><csymbol cd="latexml" id="S2.SS1.p5.1.m1.1.1.1.cmml" xref="S2.SS1.p5.1.m1.1.1.1">similar-to</csymbol><csymbol cd="latexml" id="S2.SS1.p5.1.m1.1.1.2.cmml" xref="S2.SS1.p5.1.m1.1.1.2">absent</csymbol><apply id="S2.SS1.p5.1.m1.1.1.3.cmml" xref="S2.SS1.p5.1.m1.1.1.3"><times id="S2.SS1.p5.1.m1.1.1.3.1.cmml" xref="S2.SS1.p5.1.m1.1.1.3.1"></times><cn type="integer" id="S2.SS1.p5.1.m1.1.1.3.2.cmml" xref="S2.SS1.p5.1.m1.1.1.3.2">500</cn><ci id="S2.SS1.p5.1.m1.1.1.3.3.cmml" xref="S2.SS1.p5.1.m1.1.1.3.3">m</ci><ci id="S2.SS1.p5.1.m1.1.1.3.4.cmml" xref="S2.SS1.p5.1.m1.1.1.3.4">m</ci></apply></apply></annotation-xml><annotation encoding="application/x-tex" id="S2.SS1.p5.1.m1.1c">\rm\sim 500mm</annotation><annotation encoding="application/x-llamapun" id="S2.SS1.p5.1.m1.1d">∼ 500 roman_m roman_m</annotation></semantics></math> with three arms attached to it in
<math id="S2.SS1.p5.2.m2.1" class="ltx_Math" alttext="120\arcdeg" display="inline"><semantics id="S2.SS1.p5.2.m2.1a"><mrow id="S2.SS1.p5.2.m2.1.1" xref="S2.SS1.p5.2.m2.1.1.cmml"><mn id="S2.SS1.p5.2.m2.1.1.2" xref="S2.SS1.p5.2.m2.1.1.2.cmml">120</mn><mo id="S2.SS1.p5.2.m2.1.1.1" xref="S2.SS1.p5.2.m2.1.1.1.cmml">⁢</mo><mi mathvariant="normal" id="S2.SS1.p5.2.m2.1.1.3" xref="S2.SS1.p5.2.m2.1.1.3.cmml">°</mi></mrow><annotation-xml encoding="MathML-Content" id="S2.SS1.p5.2.m2.1b"><apply id="S2.SS1.p5.2.m2.1.1.cmml" xref="S2.SS1.p5.2.m2.1.1"><times id="S2.SS1.p5.2.m2.1.1.1.cmml" xref="S2.SS1.p5.2.m2.1.1.1"></times><cn type="integer" id="S2.SS1.p5.2.m2.1.1.2.cmml" xref="S2.SS1.p5.2.m2.1.1.2">120</cn><ci id="S2.SS1.p5.2.m2.1.1.3.cmml" xref="S2.SS1.p5.2.m2.1.1.3">°</ci></apply></annotation-xml><annotation encoding="application/x-tex" id="S2.SS1.p5.2.m2.1c">120\arcdeg</annotation><annotation encoding="application/x-llamapun" id="S2.SS1.p5.2.m2.1d">120 °</annotation></semantics></math> spacings, forming a cone with pitch angle of
<math id="S2.SS1.p5.3.m3.1" class="ltx_Math" alttext="\sim 60\arcdeg" display="inline"><semantics id="S2.SS1.p5.3.m3.1a"><mrow id="S2.SS1.p5.3.m3.1.1" xref="S2.SS1.p5.3.m3.1.1.cmml"><mi id="S2.SS1.p5.3.m3.1.1.2" xref="S2.SS1.p5.3.m3.1.1.2.cmml"></mi><mo id="S2.SS1.p5.3.m3.1.1.1" xref="S2.SS1.p5.3.m3.1.1.1.cmml">∼</mo><mrow id="S2.SS1.p5.3.m3.1.1.3" xref="S2.SS1.p5.3.m3.1.1.3.cmml"><mn id="S2.SS1.p5.3.m3.1.1.3.2" xref="S2.SS1.p5.3.m3.1.1.3.2.cmml">60</mn><mo id="S2.SS1.p5.3.m3.1.1.3.1" xref="S2.SS1.p5.3.m3.1.1.3.1.cmml">⁢</mo><mi mathvariant="normal" id="S2.SS1.p5.3.m3.1.1.3.3" xref="S2.SS1.p5.3.m3.1.1.3.3.cmml">°</mi></mrow></mrow><annotation-xml encoding="MathML-Content" id="S2.SS1.p5.3.m3.1b"><apply id="S2.SS1.p5.3.m3.1.1.cmml" xref="S2.SS1.p5.3.m3.1.1"><csymbol cd="latexml" id="S2.SS1.p5.3.m3.1.1.1.cmml" xref="S2.SS1.p5.3.m3.1.1.1">similar-to</csymbol><csymbol cd="latexml" id="S2.SS1.p5.3.m3.1.1.2.cmml" xref="S2.SS1.p5.3.m3.1.1.2">absent</csymbol><apply id="S2.SS1.p5.3.m3.1.1.3.cmml" xref="S2.SS1.p5.3.m3.1.1.3"><times id="S2.SS1.p5.3.m3.1.1.3.1.cmml" xref="S2.SS1.p5.3.m3.1.1.3.1"></times><cn type="integer" id="S2.SS1.p5.3.m3.1.1.3.2.cmml" xref="S2.SS1.p5.3.m3.1.1.3.2">60</cn><ci id="S2.SS1.p5.3.m3.1.1.3.3.cmml" xref="S2.SS1.p5.3.m3.1.1.3.3">°</ci></apply></apply></annotation-xml><annotation encoding="application/x-tex" id="S2.SS1.p5.3.m3.1c">\sim 60\arcdeg</annotation><annotation encoding="application/x-llamapun" id="S2.SS1.p5.3.m3.1d">∼ 60 °</annotation></semantics></math>. The apex of this cone is the bearing housing of the RA
axis with a high-precision ball-bearing inside. As the horseshoe is cut
out from a larger piece of plate, relaxation of stress can change its
shape, which affects tracking of astronomical objects. Thus, the whole
structure is machined on a precision lathe as a piece later.</p>
</div>
<div id="S2.SS1.p6" class="ltx_para">
<p id="S2.SS1.p6.4" class="ltx_p">The horseshoe is constrained to the rollers by strong springs holding
an arm and a small bearing running in the circular groove in the inner
side of the horseshoe. The tight contact is crucial for proper
positioning, as the not perfectly balanced horseshoe is friction
driven, thus – especially at the extreme positions – torque is
needed even for constant motion. The right ascension is driven by a
five-phase stepper motor via sprockets and cogwheels and a
stainless-steel roller with combined gear-ratio of <math id="S2.SS1.p6.1.m1.1" class="ltx_Math" alttext="1" display="inline"><semantics id="S2.SS1.p6.1.m1.1a"><mn id="S2.SS1.p6.1.m1.1.1" xref="S2.SS1.p6.1.m1.1.1.cmml">1</mn><annotation-xml encoding="MathML-Content" id="S2.SS1.p6.1.m1.1b"><cn type="integer" id="S2.SS1.p6.1.m1.1.1.cmml" xref="S2.SS1.p6.1.m1.1.1">1</cn></annotation-xml><annotation encoding="application/x-tex" id="S2.SS1.p6.1.m1.1c">1</annotation><annotation encoding="application/x-llamapun" id="S2.SS1.p6.1.m1.1d">1</annotation></semantics></math> microstep of the
motor <math id="S2.SS1.p6.2.m2.1" class="ltx_Math" alttext="=1\arcsec" display="inline"><semantics id="S2.SS1.p6.2.m2.1a"><mrow id="S2.SS1.p6.2.m2.1.1" xref="S2.SS1.p6.2.m2.1.1.cmml"><mi id="S2.SS1.p6.2.m2.1.1.2" xref="S2.SS1.p6.2.m2.1.1.2.cmml"></mi><mo id="S2.SS1.p6.2.m2.1.1.1" xref="S2.SS1.p6.2.m2.1.1.1.cmml">=</mo><mrow id="S2.SS1.p6.2.m2.1.1.3" xref="S2.SS1.p6.2.m2.1.1.3.cmml"><mn id="S2.SS1.p6.2.m2.1.1.3.2" xref="S2.SS1.p6.2.m2.1.1.3.2.cmml">1</mn><mo id="S2.SS1.p6.2.m2.1.1.3.1" xref="S2.SS1.p6.2.m2.1.1.3.1.cmml">⁢</mo><mi mathvariant="normal" id="S2.SS1.p6.2.m2.1.1.3.3" xref="S2.SS1.p6.2.m2.1.1.3.3.cmml">″</mi></mrow></mrow><annotation-xml encoding="MathML-Content" id="S2.SS1.p6.2.m2.1b"><apply id="S2.SS1.p6.2.m2.1.1.cmml" xref="S2.SS1.p6.2.m2.1.1"><eq id="S2.SS1.p6.2.m2.1.1.1.cmml" xref="S2.SS1.p6.2.m2.1.1.1"></eq><csymbol cd="latexml" id="S2.SS1.p6.2.m2.1.1.2.cmml" xref="S2.SS1.p6.2.m2.1.1.2">absent</csymbol><apply id="S2.SS1.p6.2.m2.1.1.3.cmml" xref="S2.SS1.p6.2.m2.1.1.3"><times id="S2.SS1.p6.2.m2.1.1.3.1.cmml" xref="S2.SS1.p6.2.m2.1.1.3.1"></times><cn type="integer" id="S2.SS1.p6.2.m2.1.1.3.2.cmml" xref="S2.SS1.p6.2.m2.1.1.3.2">1</cn><ci id="S2.SS1.p6.2.m2.1.1.3.3.cmml" xref="S2.SS1.p6.2.m2.1.1.3.3">″</ci></apply></apply></annotation-xml><annotation encoding="application/x-tex" id="S2.SS1.p6.2.m2.1c">=1\arcsec</annotation><annotation encoding="application/x-llamapun" id="S2.SS1.p6.2.m2.1d">= 1 ″</annotation></semantics></math> (arcsec) movement of the horseshoe. Slipping is
further minimized by acceleration and deceleration of the mount in
<math id="S2.SS1.p6.3.m3.1" class="ltx_Math" alttext="\sim 50\--100" display="inline"><semantics id="S2.SS1.p6.3.m3.1a"><mrow id="S2.SS1.p6.3.m3.1.1" xref="S2.SS1.p6.3.m3.1.1.cmml"><mi id="S2.SS1.p6.3.m3.1.1.2" xref="S2.SS1.p6.3.m3.1.1.2.cmml"></mi><mo id="S2.SS1.p6.3.m3.1.1.1" xref="S2.SS1.p6.3.m3.1.1.1.cmml">∼</mo><mrow id="S2.SS1.p6.3.m3.1.1.3" xref="S2.SS1.p6.3.m3.1.1.3.cmml"><mn id="S2.SS1.p6.3.m3.1.1.3.2" xref="S2.SS1.p6.3.m3.1.1.3.2.cmml">50</mn><mo id="S2.SS1.p6.3.m3.1.1.3.1" xref="S2.SS1.p6.3.m3.1.1.3.1.cmml">-</mo><mn id="S2.SS1.p6.3.m3.1.1.3.3" xref="S2.SS1.p6.3.m3.1.1.3.3.cmml">100</mn></mrow></mrow><annotation-xml encoding="MathML-Content" id="S2.SS1.p6.3.m3.1b"><apply id="S2.SS1.p6.3.m3.1.1.cmml" xref="S2.SS1.p6.3.m3.1.1"><csymbol cd="latexml" id="S2.SS1.p6.3.m3.1.1.1.cmml" xref="S2.SS1.p6.3.m3.1.1.1">similar-to</csymbol><csymbol cd="latexml" id="S2.SS1.p6.3.m3.1.1.2.cmml" xref="S2.SS1.p6.3.m3.1.1.2">absent</csymbol><apply id="S2.SS1.p6.3.m3.1.1.3.cmml" xref="S2.SS1.p6.3.m3.1.1.3"><minus id="S2.SS1.p6.3.m3.1.1.3.1.cmml" xref="S2.SS1.p6.3.m3.1.1.3.1"></minus><cn type="integer" id="S2.SS1.p6.3.m3.1.1.3.2.cmml" xref="S2.SS1.p6.3.m3.1.1.3.2">50</cn><cn type="integer" id="S2.SS1.p6.3.m3.1.1.3.3.cmml" xref="S2.SS1.p6.3.m3.1.1.3.3">100</cn></apply></apply></annotation-xml><annotation encoding="application/x-tex" id="S2.SS1.p6.3.m3.1c">\sim 50\--100</annotation><annotation encoding="application/x-llamapun" id="S2.SS1.p6.3.m3.1d">∼ 50 - 100</annotation></semantics></math> discrete velocity steps in an interval (“ramping”),
both specified by the user at the software control. The steel wires of
the sprockets prevent them from stretching, and yield a backlash-free
drive. The maximal slew speed is <math id="S2.SS1.p6.4.m4.1" class="ltx_Math" alttext="\rm\sim 2\arcdeg/sec" display="inline"><semantics id="S2.SS1.p6.4.m4.1a"><mrow id="S2.SS1.p6.4.m4.1.1" xref="S2.SS1.p6.4.m4.1.1.cmml"><mi id="S2.SS1.p6.4.m4.1.1.2" xref="S2.SS1.p6.4.m4.1.1.2.cmml"></mi><mo id="S2.SS1.p6.4.m4.1.1.1" xref="S2.SS1.p6.4.m4.1.1.1.cmml">∼</mo><mrow id="S2.SS1.p6.4.m4.1.1.3" xref="S2.SS1.p6.4.m4.1.1.3.cmml"><mrow id="S2.SS1.p6.4.m4.1.1.3.2" xref="S2.SS1.p6.4.m4.1.1.3.2.cmml"><mn id="S2.SS1.p6.4.m4.1.1.3.2.2" xref="S2.SS1.p6.4.m4.1.1.3.2.2.cmml">2</mn><mo id="S2.SS1.p6.4.m4.1.1.3.2.1" xref="S2.SS1.p6.4.m4.1.1.3.2.1.cmml">⁢</mo><mi mathvariant="normal" id="S2.SS1.p6.4.m4.1.1.3.2.3" xref="S2.SS1.p6.4.m4.1.1.3.2.3.cmml">°</mi></mrow><mo id="S2.SS1.p6.4.m4.1.1.3.1" xref="S2.SS1.p6.4.m4.1.1.3.1.cmml">/</mo><mi id="S2.SS1.p6.4.m4.1.1.3.3" xref="S2.SS1.p6.4.m4.1.1.3.3.cmml">sec</mi></mrow></mrow><annotation-xml encoding="MathML-Content" id="S2.SS1.p6.4.m4.1b"><apply id="S2.SS1.p6.4.m4.1.1.cmml" xref="S2.SS1.p6.4.m4.1.1"><csymbol cd="latexml" id="S2.SS1.p6.4.m4.1.1.1.cmml" xref="S2.SS1.p6.4.m4.1.1.1">similar-to</csymbol><csymbol cd="latexml" id="S2.SS1.p6.4.m4.1.1.2.cmml" xref="S2.SS1.p6.4.m4.1.1.2">absent</csymbol><apply id="S2.SS1.p6.4.m4.1.1.3.cmml" xref="S2.SS1.p6.4.m4.1.1.3"><divide id="S2.SS1.p6.4.m4.1.1.3.1.cmml" xref="S2.SS1.p6.4.m4.1.1.3.1"></divide><apply id="S2.SS1.p6.4.m4.1.1.3.2.cmml" xref="S2.SS1.p6.4.m4.1.1.3.2"><times id="S2.SS1.p6.4.m4.1.1.3.2.1.cmml" xref="S2.SS1.p6.4.m4.1.1.3.2.1"></times><cn type="integer" id="S2.SS1.p6.4.m4.1.1.3.2.2.cmml" xref="S2.SS1.p6.4.m4.1.1.3.2.2">2</cn><ci id="S2.SS1.p6.4.m4.1.1.3.2.3.cmml" xref="S2.SS1.p6.4.m4.1.1.3.2.3">°</ci></apply><ci id="S2.SS1.p6.4.m4.1.1.3.3.cmml" xref="S2.SS1.p6.4.m4.1.1.3.3">sec</ci></apply></apply></annotation-xml><annotation encoding="application/x-tex" id="S2.SS1.p6.4.m4.1c">\rm\sim 2\arcdeg/sec</annotation><annotation encoding="application/x-llamapun" id="S2.SS1.p6.4.m4.1d">∼ 2 ° / roman_sec</annotation></semantics></math>.</p>
</div>
<div id="S2.SS1.p7" class="ltx_para">
<p id="S2.SS1.p7.3" class="ltx_p">The declination axis is made up of two <math id="S2.SS1.p7.1.m1.1" class="ltx_Math" alttext="\rm 200mm\oslash" display="inline"><semantics id="S2.SS1.p7.1.m1.1a"><mrow id="S2.SS1.p7.1.m1.1b"><mn id="S2.SS1.p7.1.m1.1.1" xref="S2.SS1.p7.1.m1.1.1.cmml">200</mn><mi mathvariant="normal" id="S2.SS1.p7.1.m1.1.2" xref="S2.SS1.p7.1.m1.1.2.cmml">m</mi><mi mathvariant="normal" id="S2.SS1.p7.1.m1.1.3" xref="S2.SS1.p7.1.m1.1.3.cmml">m</mi><mo id="S2.SS1.p7.1.m1.1.4" xref="S2.SS1.p7.1.m1.1.4.cmml">⊘</mo></mrow><annotation-xml encoding="MathML-Content" id="S2.SS1.p7.1.m1.1c"><cerror id="S2.SS1.p7.1.m1.1d"><csymbol cd="ambiguous" id="S2.SS1.p7.1.m1.1e">fragments</csymbol><cn type="integer" id="S2.SS1.p7.1.m1.1.1.cmml" xref="S2.SS1.p7.1.m1.1.1">200</cn><csymbol cd="unknown" id="S2.SS1.p7.1.m1.1.2.cmml" xref="S2.SS1.p7.1.m1.1.2">m</csymbol><csymbol cd="unknown" id="S2.SS1.p7.1.m1.1.3.cmml" xref="S2.SS1.p7.1.m1.1.3">m</csymbol><ci id="S2.SS1.p7.1.m1.1.4.cmml" xref="S2.SS1.p7.1.m1.1.4">⊘</ci></cerror></annotation-xml><annotation encoding="application/x-tex" id="S2.SS1.p7.1.m1.1f">\rm 200mm\oslash</annotation><annotation encoding="application/x-llamapun" id="S2.SS1.p7.1.m1.1g">200 roman_m roman_m ⊘</annotation></semantics></math> discs
attached by bearings to the inner side of the horseshoe, and connected
by four lateral bars, two of which hold the instrument-mounting plate.
The axis is driven in a similar manner to the RA, i.e., by a 5-phase
stepper, sprocket and cogwheels, but the final resolution is <math id="S2.SS1.p7.2.m2.1" class="ltx_Math" alttext="\rm 5\arcsec/microstep" display="inline"><semantics id="S2.SS1.p7.2.m2.1a"><mrow id="S2.SS1.p7.2.m2.1.1" xref="S2.SS1.p7.2.m2.1.1.cmml"><mrow id="S2.SS1.p7.2.m2.1.1.2" xref="S2.SS1.p7.2.m2.1.1.2.cmml"><mn id="S2.SS1.p7.2.m2.1.1.2.2" xref="S2.SS1.p7.2.m2.1.1.2.2.cmml">5</mn><mo id="S2.SS1.p7.2.m2.1.1.2.1" xref="S2.SS1.p7.2.m2.1.1.2.1.cmml">⁢</mo><mi mathvariant="normal" id="S2.SS1.p7.2.m2.1.1.2.3" xref="S2.SS1.p7.2.m2.1.1.2.3.cmml">″</mi></mrow><mo id="S2.SS1.p7.2.m2.1.1.1" xref="S2.SS1.p7.2.m2.1.1.1.cmml">/</mo><mi id="S2.SS1.p7.2.m2.1.1.3" xref="S2.SS1.p7.2.m2.1.1.3.cmml">microstep</mi></mrow><annotation-xml encoding="MathML-Content" id="S2.SS1.p7.2.m2.1b"><apply id="S2.SS1.p7.2.m2.1.1.cmml" xref="S2.SS1.p7.2.m2.1.1"><divide id="S2.SS1.p7.2.m2.1.1.1.cmml" xref="S2.SS1.p7.2.m2.1.1.1"></divide><apply id="S2.SS1.p7.2.m2.1.1.2.cmml" xref="S2.SS1.p7.2.m2.1.1.2"><times id="S2.SS1.p7.2.m2.1.1.2.1.cmml" xref="S2.SS1.p7.2.m2.1.1.2.1"></times><cn type="integer" id="S2.SS1.p7.2.m2.1.1.2.2.cmml" xref="S2.SS1.p7.2.m2.1.1.2.2">5</cn><ci id="S2.SS1.p7.2.m2.1.1.2.3.cmml" xref="S2.SS1.p7.2.m2.1.1.2.3">″</ci></apply><ci id="S2.SS1.p7.2.m2.1.1.3.cmml" xref="S2.SS1.p7.2.m2.1.1.3">microstep</ci></apply></annotation-xml><annotation encoding="application/x-tex" id="S2.SS1.p7.2.m2.1c">\rm 5\arcsec/microstep</annotation><annotation encoding="application/x-llamapun" id="S2.SS1.p7.2.m2.1d">5 ″ / roman_microstep</annotation></semantics></math> and maximal speed is <math id="S2.SS1.p7.3.m3.1" class="ltx_Math" alttext="\rm\sim 5\arcdeg/sec" display="inline"><semantics id="S2.SS1.p7.3.m3.1a"><mrow id="S2.SS1.p7.3.m3.1.1" xref="S2.SS1.p7.3.m3.1.1.cmml"><mi id="S2.SS1.p7.3.m3.1.1.2" xref="S2.SS1.p7.3.m3.1.1.2.cmml"></mi><mo id="S2.SS1.p7.3.m3.1.1.1" xref="S2.SS1.p7.3.m3.1.1.1.cmml">∼</mo><mrow id="S2.SS1.p7.3.m3.1.1.3" xref="S2.SS1.p7.3.m3.1.1.3.cmml"><mrow id="S2.SS1.p7.3.m3.1.1.3.2" xref="S2.SS1.p7.3.m3.1.1.3.2.cmml"><mn id="S2.SS1.p7.3.m3.1.1.3.2.2" xref="S2.SS1.p7.3.m3.1.1.3.2.2.cmml">5</mn><mo id="S2.SS1.p7.3.m3.1.1.3.2.1" xref="S2.SS1.p7.3.m3.1.1.3.2.1.cmml">⁢</mo><mi mathvariant="normal" id="S2.SS1.p7.3.m3.1.1.3.2.3" xref="S2.SS1.p7.3.m3.1.1.3.2.3.cmml">°</mi></mrow><mo id="S2.SS1.p7.3.m3.1.1.3.1" xref="S2.SS1.p7.3.m3.1.1.3.1.cmml">/</mo><mi id="S2.SS1.p7.3.m3.1.1.3.3" xref="S2.SS1.p7.3.m3.1.1.3.3.cmml">sec</mi></mrow></mrow><annotation-xml encoding="MathML-Content" id="S2.SS1.p7.3.m3.1b"><apply id="S2.SS1.p7.3.m3.1.1.cmml" xref="S2.SS1.p7.3.m3.1.1"><csymbol cd="latexml" id="S2.SS1.p7.3.m3.1.1.1.cmml" xref="S2.SS1.p7.3.m3.1.1.1">similar-to</csymbol><csymbol cd="latexml" id="S2.SS1.p7.3.m3.1.1.2.cmml" xref="S2.SS1.p7.3.m3.1.1.2">absent</csymbol><apply id="S2.SS1.p7.3.m3.1.1.3.cmml" xref="S2.SS1.p7.3.m3.1.1.3"><divide id="S2.SS1.p7.3.m3.1.1.3.1.cmml" xref="S2.SS1.p7.3.m3.1.1.3.1"></divide><apply id="S2.SS1.p7.3.m3.1.1.3.2.cmml" xref="S2.SS1.p7.3.m3.1.1.3.2"><times id="S2.SS1.p7.3.m3.1.1.3.2.1.cmml" xref="S2.SS1.p7.3.m3.1.1.3.2.1"></times><cn type="integer" id="S2.SS1.p7.3.m3.1.1.3.2.2.cmml" xref="S2.SS1.p7.3.m3.1.1.3.2.2">5</cn><ci id="S2.SS1.p7.3.m3.1.1.3.2.3.cmml" xref="S2.SS1.p7.3.m3.1.1.3.2.3">°</ci></apply><ci id="S2.SS1.p7.3.m3.1.1.3.3.cmml" xref="S2.SS1.p7.3.m3.1.1.3.3">sec</ci></apply></apply></annotation-xml><annotation encoding="application/x-tex" id="S2.SS1.p7.3.m3.1c">\rm\sim 5\arcdeg/sec</annotation><annotation encoding="application/x-llamapun" id="S2.SS1.p7.3.m3.1d">∼ 5 ° / roman_sec</annotation></semantics></math>.</p>
</div>
<div id="S2.SS1.p8" class="ltx_para">
<p id="S2.SS1.p8.2" class="ltx_p">The maximal dimension or a telescope which fits our mount is
<math id="S2.SS1.p8.1.m1.1" class="ltx_Math" alttext="\rm\sim 200mm\oslash\times 400mm(l)" display="inline"><semantics id="S2.SS1.p8.1.m1.1a"><mrow id="S2.SS1.p8.1.m1.1b"><mo id="S2.SS1.p8.1.m1.1.1" xref="S2.SS1.p8.1.m1.1.1.cmml">∼</mo><mn id="S2.SS1.p8.1.m1.1.2" xref="S2.SS1.p8.1.m1.1.2.cmml">200</mn><mi mathvariant="normal" id="S2.SS1.p8.1.m1.1.3" xref="S2.SS1.p8.1.m1.1.3.cmml">m</mi><mi mathvariant="normal" id="S2.SS1.p8.1.m1.1.4" xref="S2.SS1.p8.1.m1.1.4.cmml">m</mi><mo id="S2.SS1.p8.1.m1.1.5" xref="S2.SS1.p8.1.m1.1.5.cmml">⊘</mo><mo id="S2.SS1.p8.1.m1.1.6" xref="S2.SS1.p8.1.m1.1.6.cmml">×</mo><mn id="S2.SS1.p8.1.m1.1.7" xref="S2.SS1.p8.1.m1.1.7.cmml">400</mn><mi mathvariant="normal" id="S2.SS1.p8.1.m1.1.8" xref="S2.SS1.p8.1.m1.1.8.cmml">m</mi><mi mathvariant="normal" id="S2.SS1.p8.1.m1.1.9" xref="S2.SS1.p8.1.m1.1.9.cmml">m</mi><mrow id="S2.SS1.p8.1.m1.1.10" xref="S2.SS1.p8.1.m1.1.10.cmml"><mo stretchy="false" id="S2.SS1.p8.1.m1.1.10.1" xref="S2.SS1.p8.1.m1.1.10.1.cmml">(</mo><mi mathvariant="normal" id="S2.SS1.p8.1.m1.1.10.2" xref="S2.SS1.p8.1.m1.1.10.2.cmml">l</mi><mo stretchy="false" id="S2.SS1.p8.1.m1.1.10.3" xref="S2.SS1.p8.1.m1.1.10.3.cmml">)</mo></mrow></mrow><annotation-xml encoding="MathML-Content" id="S2.SS1.p8.1.m1.1c"><cerror id="S2.SS1.p8.1.m1.1d"><csymbol cd="ambiguous" id="S2.SS1.p8.1.m1.1e">fragments</csymbol><csymbol cd="latexml" id="S2.SS1.p8.1.m1.1.1.cmml" xref="S2.SS1.p8.1.m1.1.1">similar-to</csymbol><cn type="integer" id="S2.SS1.p8.1.m1.1.2.cmml" xref="S2.SS1.p8.1.m1.1.2">200</cn><csymbol cd="unknown" id="S2.SS1.p8.1.m1.1.3.cmml" xref="S2.SS1.p8.1.m1.1.3">m</csymbol><csymbol cd="unknown" id="S2.SS1.p8.1.m1.1.4.cmml" xref="S2.SS1.p8.1.m1.1.4">m</csymbol><ci id="S2.SS1.p8.1.m1.1.5.cmml" xref="S2.SS1.p8.1.m1.1.5">⊘</ci><times id="S2.SS1.p8.1.m1.1.6.cmml" xref="S2.SS1.p8.1.m1.1.6"></times><cn type="integer" id="S2.SS1.p8.1.m1.1.7.cmml" xref="S2.SS1.p8.1.m1.1.7">400</cn><csymbol cd="unknown" id="S2.SS1.p8.1.m1.1.8.cmml" xref="S2.SS1.p8.1.m1.1.8">m</csymbol><csymbol cd="unknown" id="S2.SS1.p8.1.m1.1.9.cmml" xref="S2.SS1.p8.1.m1.1.9">m</csymbol><cerror id="S2.SS1.p8.1.m1.1.10.cmml" xref="S2.SS1.p8.1.m1.1.10"><csymbol cd="ambiguous" id="S2.SS1.p8.1.m1.1.10a.cmml" xref="S2.SS1.p8.1.m1.1.10">fragments</csymbol><ci id="S2.SS1.p8.1.m1.1.10.1.cmml" xref="S2.SS1.p8.1.m1.1.10.1">(</ci><csymbol cd="unknown" id="S2.SS1.p8.1.m1.1.10.2.cmml" xref="S2.SS1.p8.1.m1.1.10.2">l</csymbol><ci id="S2.SS1.p8.1.m1.1.10.3.cmml" xref="S2.SS1.p8.1.m1.1.10.3">)</ci></cerror></cerror></annotation-xml><annotation encoding="application/x-tex" id="S2.SS1.p8.1.m1.1f">\rm\sim 200mm\oslash\times 400mm(l)</annotation><annotation encoding="application/x-llamapun" id="S2.SS1.p8.1.m1.1g">∼ 200 roman_m roman_m ⊘ × 400 roman_m roman_m ( roman_l )</annotation></semantics></math>.
The maximal size of the detector is limited by the inner half-sphere of
the horseshoe, but most CCDs, such as Apogee AP10
(<math id="S2.SS1.p8.2.m2.1" class="ltx_Math" alttext="\rm\sim 200\times 200\times 60mm" display="inline"><semantics id="S2.SS1.p8.2.m2.1a"><mrow id="S2.SS1.p8.2.m2.1.1" xref="S2.SS1.p8.2.m2.1.1.cmml"><mi id="S2.SS1.p8.2.m2.1.1.2" xref="S2.SS1.p8.2.m2.1.1.2.cmml"></mi><mo id="S2.SS1.p8.2.m2.1.1.1" xref="S2.SS1.p8.2.m2.1.1.1.cmml">∼</mo><mrow id="S2.SS1.p8.2.m2.1.1.3" xref="S2.SS1.p8.2.m2.1.1.3.cmml"><mrow id="S2.SS1.p8.2.m2.1.1.3.2" xref="S2.SS1.p8.2.m2.1.1.3.2.cmml"><mn id="S2.SS1.p8.2.m2.1.1.3.2.2" xref="S2.SS1.p8.2.m2.1.1.3.2.2.cmml">200</mn><mo id="S2.SS1.p8.2.m2.1.1.3.2.1" xref="S2.SS1.p8.2.m2.1.1.3.2.1.cmml">×</mo><mn id="S2.SS1.p8.2.m2.1.1.3.2.3" xref="S2.SS1.p8.2.m2.1.1.3.2.3.cmml">200</mn><mo id="S2.SS1.p8.2.m2.1.1.3.2.1a" xref="S2.SS1.p8.2.m2.1.1.3.2.1.cmml">×</mo><mn id="S2.SS1.p8.2.m2.1.1.3.2.4" xref="S2.SS1.p8.2.m2.1.1.3.2.4.cmml">60</mn></mrow><mo id="S2.SS1.p8.2.m2.1.1.3.1" xref="S2.SS1.p8.2.m2.1.1.3.1.cmml">⁢</mo><mi mathvariant="normal" id="S2.SS1.p8.2.m2.1.1.3.3" xref="S2.SS1.p8.2.m2.1.1.3.3.cmml">m</mi><mo id="S2.SS1.p8.2.m2.1.1.3.1a" xref="S2.SS1.p8.2.m2.1.1.3.1.cmml">⁢</mo><mi mathvariant="normal" id="S2.SS1.p8.2.m2.1.1.3.4" xref="S2.SS1.p8.2.m2.1.1.3.4.cmml">m</mi></mrow></mrow><annotation-xml encoding="MathML-Content" id="S2.SS1.p8.2.m2.1b"><apply id="S2.SS1.p8.2.m2.1.1.cmml" xref="S2.SS1.p8.2.m2.1.1"><csymbol cd="latexml" id="S2.SS1.p8.2.m2.1.1.1.cmml" xref="S2.SS1.p8.2.m2.1.1.1">similar-to</csymbol><csymbol cd="latexml" id="S2.SS1.p8.2.m2.1.1.2.cmml" xref="S2.SS1.p8.2.m2.1.1.2">absent</csymbol><apply id="S2.SS1.p8.2.m2.1.1.3.cmml" xref="S2.SS1.p8.2.m2.1.1.3"><times id="S2.SS1.p8.2.m2.1.1.3.1.cmml" xref="S2.SS1.p8.2.m2.1.1.3.1"></times><apply id="S2.SS1.p8.2.m2.1.1.3.2.cmml" xref="S2.SS1.p8.2.m2.1.1.3.2"><times id="S2.SS1.p8.2.m2.1.1.3.2.1.cmml" xref="S2.SS1.p8.2.m2.1.1.3.2.1"></times><cn type="integer" id="S2.SS1.p8.2.m2.1.1.3.2.2.cmml" xref="S2.SS1.p8.2.m2.1.1.3.2.2">200</cn><cn type="integer" id="S2.SS1.p8.2.m2.1.1.3.2.3.cmml" xref="S2.SS1.p8.2.m2.1.1.3.2.3">200</cn><cn type="integer" id="S2.SS1.p8.2.m2.1.1.3.2.4.cmml" xref="S2.SS1.p8.2.m2.1.1.3.2.4">60</cn></apply><ci id="S2.SS1.p8.2.m2.1.1.3.3.cmml" xref="S2.SS1.p8.2.m2.1.1.3.3">m</ci><ci id="S2.SS1.p8.2.m2.1.1.3.4.cmml" xref="S2.SS1.p8.2.m2.1.1.3.4">m</ci></apply></apply></annotation-xml><annotation encoding="application/x-tex" id="S2.SS1.p8.2.m2.1c">\rm\sim 200\times 200\times 60mm</annotation><annotation encoding="application/x-llamapun" id="S2.SS1.p8.2.m2.1d">∼ 200 × 200 × 60 roman_m roman_m</annotation></semantics></math>), fit readily. Balance of axes is
achieved by counterweights on the top of the horseshoe, on the
declination disc and on the instrument-mounting plate. The telescope
(telephoto lens) is fixed to the instrument-mounting plate by a
converter ring, and rigidly held by screws from the upper two lateral
bars connecting the Dec discs, while the CCD is attached to the other
side.</p>
</div>
<div id="S2.SS1.p9" class="ltx_para">
<p id="S2.SS1.p9.1" class="ltx_p">Inductive proximity sensors on both axes detect home and end positions
in order to ensure fail-safe operation. Even though we use an open-loop
control system (no costly encoder employed), and it might happen that
the telescope loses orientation (due to e.g., power-failure), it can
quickly recover by an iterative procedure of finding the home position.
Physical end positions on both the RA and Dec axes guarantee that even
if the proximity sensors fail, we cannot run off the rollers, and the
telescope tube or detector never hits the mount.</p>
</div>
<div id="S2.SS1.p10" class="ltx_para">
<p id="S2.SS1.p10.1" class="ltx_p">Polar setting of the mount is simplified by the possibility of fitting
a polar telescope to the central hole in the RA bearing house, and
crude setting can be achieved in a few minutes. Final setting is done
by standard methods such as described in <cite class="ltx_cite ltx_citemacro_citet">Leung (<a href="#bib.bib27" title="" class="ltx_ref">1962</a>)</cite> and references
therein.</p>
</div>
</section>
<section id="S2.SS2" class="ltx_subsection">
<h3 class="ltx_title ltx_title_subsection">
<span class="ltx_tag ltx_tag_subsection">2.2 </span>The dome</h3>

<div id="S2.SS2.p1" class="ltx_para">
<p id="S2.SS2.p1.1" class="ltx_p">Automated, remote dome operations have to be perfectly fail-safe,
because the detector can be damaged by inclement weather conditions, or
the Sun might edge slowly across the field. A few of the encountered
risks when closing the dome include gusty wind, power outage and
failure of control software. In order to minimize software control and
construction costs, and to enable rapid observation of targets of
opportunity, only mechanisms that completely flip out of the way were
considered. We designed a structure which proved to work in the past
one year, with only a few minor failures. These failures, in turn,
aided improvement of the design.
</p>
</div>
<div id="S2.SS2.p2" class="ltx_para">
<p id="S2.SS2.p2.1" class="ltx_p">We experimented with simple schemes, such as a lid opening towards the
north around the upper edge of the box, but this setup has strong wind
resistance during the opening/closing phase. The standard rollover-roof
structure is not compact enough and blocks a considerable part of the
sky.</p>
</div>
<div id="S2.SS2.p3" class="ltx_para">
<p id="S2.SS2.p3.1" class="ltx_p">HAT mount is enclosed by a
<math id="S2.SS2.p3.1.m1.3" class="ltx_Math" alttext="\rm\sim 0.6(w)\times 0.75(l)\times 0.75m(h)" display="inline"><semantics id="S2.SS2.p3.1.m1.3a"><mrow id="S2.SS2.p3.1.m1.3.4" xref="S2.SS2.p3.1.m1.3.4.cmml"><mi id="S2.SS2.p3.1.m1.3.4.2" xref="S2.SS2.p3.1.m1.3.4.2.cmml"></mi><mo id="S2.SS2.p3.1.m1.3.4.1" xref="S2.SS2.p3.1.m1.3.4.1.cmml">∼</mo><mrow id="S2.SS2.p3.1.m1.3.4.3" xref="S2.SS2.p3.1.m1.3.4.3.cmml"><mrow id="S2.SS2.p3.1.m1.3.4.3.2" xref="S2.SS2.p3.1.m1.3.4.3.2.cmml"><mrow id="S2.SS2.p3.1.m1.3.4.3.2.2" xref="S2.SS2.p3.1.m1.3.4.3.2.2.cmml"><mrow id="S2.SS2.p3.1.m1.3.4.3.2.2.2" xref="S2.SS2.p3.1.m1.3.4.3.2.2.2.cmml"><mrow id="S2.SS2.p3.1.m1.3.4.3.2.2.2.2" xref="S2.SS2.p3.1.m1.3.4.3.2.2.2.2.cmml"><mn id="S2.SS2.p3.1.m1.3.4.3.2.2.2.2.2" xref="S2.SS2.p3.1.m1.3.4.3.2.2.2.2.2.cmml">0.6</mn><mo id="S2.SS2.p3.1.m1.3.4.3.2.2.2.2.1" xref="S2.SS2.p3.1.m1.3.4.3.2.2.2.2.1.cmml">⁢</mo><mrow id="S2.SS2.p3.1.m1.3.4.3.2.2.2.2.3.2" xref="S2.SS2.p3.1.m1.3.4.3.2.2.2.2.cmml"><mo stretchy="false" id="S2.SS2.p3.1.m1.3.4.3.2.2.2.2.3.2.1" xref="S2.SS2.p3.1.m1.3.4.3.2.2.2.2.cmml">(</mo><mi mathvariant="normal" id="S2.SS2.p3.1.m1.1.1" xref="S2.SS2.p3.1.m1.1.1.cmml">w</mi><mo stretchy="false" id="S2.SS2.p3.1.m1.3.4.3.2.2.2.2.3.2.2" xref="S2.SS2.p3.1.m1.3.4.3.2.2.2.2.cmml">)</mo></mrow></mrow><mo id="S2.SS2.p3.1.m1.3.4.3.2.2.2.1" xref="S2.SS2.p3.1.m1.3.4.3.2.2.2.1.cmml">×</mo><mn id="S2.SS2.p3.1.m1.3.4.3.2.2.2.3" xref="S2.SS2.p3.1.m1.3.4.3.2.2.2.3.cmml">0.75</mn></mrow><mo id="S2.SS2.p3.1.m1.3.4.3.2.2.1" xref="S2.SS2.p3.1.m1.3.4.3.2.2.1.cmml">⁢</mo><mrow id="S2.SS2.p3.1.m1.3.4.3.2.2.3.2" xref="S2.SS2.p3.1.m1.3.4.3.2.2.cmml"><mo stretchy="false" id="S2.SS2.p3.1.m1.3.4.3.2.2.3.2.1" xref="S2.SS2.p3.1.m1.3.4.3.2.2.cmml">(</mo><mi mathvariant="normal" id="S2.SS2.p3.1.m1.2.2" xref="S2.SS2.p3.1.m1.2.2.cmml">l</mi><mo stretchy="false" id="S2.SS2.p3.1.m1.3.4.3.2.2.3.2.2" xref="S2.SS2.p3.1.m1.3.4.3.2.2.cmml">)</mo></mrow></mrow><mo id="S2.SS2.p3.1.m1.3.4.3.2.1" xref="S2.SS2.p3.1.m1.3.4.3.2.1.cmml">×</mo><mn id="S2.SS2.p3.1.m1.3.4.3.2.3" xref="S2.SS2.p3.1.m1.3.4.3.2.3.cmml">0.75</mn></mrow><mo id="S2.SS2.p3.1.m1.3.4.3.1" xref="S2.SS2.p3.1.m1.3.4.3.1.cmml">⁢</mo><mi mathvariant="normal" id="S2.SS2.p3.1.m1.3.4.3.3" xref="S2.SS2.p3.1.m1.3.4.3.3.cmml">m</mi><mo id="S2.SS2.p3.1.m1.3.4.3.1a" xref="S2.SS2.p3.1.m1.3.4.3.1.cmml">⁢</mo><mrow id="S2.SS2.p3.1.m1.3.4.3.4.2" xref="S2.SS2.p3.1.m1.3.4.3.cmml"><mo stretchy="false" id="S2.SS2.p3.1.m1.3.4.3.4.2.1" xref="S2.SS2.p3.1.m1.3.4.3.cmml">(</mo><mi mathvariant="normal" id="S2.SS2.p3.1.m1.3.3" xref="S2.SS2.p3.1.m1.3.3.cmml">h</mi><mo stretchy="false" id="S2.SS2.p3.1.m1.3.4.3.4.2.2" xref="S2.SS2.p3.1.m1.3.4.3.cmml">)</mo></mrow></mrow></mrow><annotation-xml encoding="MathML-Content" id="S2.SS2.p3.1.m1.3b"><apply id="S2.SS2.p3.1.m1.3.4.cmml" xref="S2.SS2.p3.1.m1.3.4"><csymbol cd="latexml" id="S2.SS2.p3.1.m1.3.4.1.cmml" xref="S2.SS2.p3.1.m1.3.4.1">similar-to</csymbol><csymbol cd="latexml" id="S2.SS2.p3.1.m1.3.4.2.cmml" xref="S2.SS2.p3.1.m1.3.4.2">absent</csymbol><apply id="S2.SS2.p3.1.m1.3.4.3.cmml" xref="S2.SS2.p3.1.m1.3.4.3"><times id="S2.SS2.p3.1.m1.3.4.3.1.cmml" xref="S2.SS2.p3.1.m1.3.4.3.1"></times><apply id="S2.SS2.p3.1.m1.3.4.3.2.cmml" xref="S2.SS2.p3.1.m1.3.4.3.2"><times id="S2.SS2.p3.1.m1.3.4.3.2.1.cmml" xref="S2.SS2.p3.1.m1.3.4.3.2.1"></times><apply id="S2.SS2.p3.1.m1.3.4.3.2.2.cmml" xref="S2.SS2.p3.1.m1.3.4.3.2.2"><times id="S2.SS2.p3.1.m1.3.4.3.2.2.1.cmml" xref="S2.SS2.p3.1.m1.3.4.3.2.2.1"></times><apply id="S2.SS2.p3.1.m1.3.4.3.2.2.2.cmml" xref="S2.SS2.p3.1.m1.3.4.3.2.2.2"><times id="S2.SS2.p3.1.m1.3.4.3.2.2.2.1.cmml" xref="S2.SS2.p3.1.m1.3.4.3.2.2.2.1"></times><apply id="S2.SS2.p3.1.m1.3.4.3.2.2.2.2.cmml" xref="S2.SS2.p3.1.m1.3.4.3.2.2.2.2"><times id="S2.SS2.p3.1.m1.3.4.3.2.2.2.2.1.cmml" xref="S2.SS2.p3.1.m1.3.4.3.2.2.2.2.1"></times><cn type="float" id="S2.SS2.p3.1.m1.3.4.3.2.2.2.2.2.cmml" xref="S2.SS2.p3.1.m1.3.4.3.2.2.2.2.2">0.6</cn><ci id="S2.SS2.p3.1.m1.1.1.cmml" xref="S2.SS2.p3.1.m1.1.1">w</ci></apply><cn type="float" id="S2.SS2.p3.1.m1.3.4.3.2.2.2.3.cmml" xref="S2.SS2.p3.1.m1.3.4.3.2.2.2.3">0.75</cn></apply><ci id="S2.SS2.p3.1.m1.2.2.cmml" xref="S2.SS2.p3.1.m1.2.2">l</ci></apply><cn type="float" id="S2.SS2.p3.1.m1.3.4.3.2.3.cmml" xref="S2.SS2.p3.1.m1.3.4.3.2.3">0.75</cn></apply><ci id="S2.SS2.p3.1.m1.3.4.3.3.cmml" xref="S2.SS2.p3.1.m1.3.4.3.3">m</ci><ci id="S2.SS2.p3.1.m1.3.3.cmml" xref="S2.SS2.p3.1.m1.3.3">h</ci></apply></apply></annotation-xml><annotation encoding="application/x-tex" id="S2.SS2.p3.1.m1.3c">\rm\sim 0.6(w)\times 0.75(l)\times 0.75m(h)</annotation><annotation encoding="application/x-llamapun" id="S2.SS2.p3.1.m1.3d">∼ 0.6 ( roman_w ) × 0.75 ( roman_l ) × 0.75 roman_m ( roman_h )</annotation></semantics></math> weather-proof box (See
Fig. <a href="#S2.F3" title="Figure 3 ‣ 2.2 The dome ‣ 2 Hardware System ‣ System description and first light-curves of HAT, an autonomous observatory for variability search" class="ltx_ref"><span class="ltx_text ltx_ref_tag">3</span></a> for a schematic drawing). A separate slant roof on
top is connected to the rest of the dome by swivel joints; a long bar
is attached to its south corner and a shorter bar to the north, on both
sides (assuming an installation on the northern hemisphere). The roof
can be opened northwards by rotation around the bottom axes of the two
arms, resembling an asymmetric clamshell. As the lengths of the arms
differ, the roof co-rotates in such a way that resistance against wind
is minimal in every position, and eventually it gets blocked from the
wind behind the bulk of the dome, with its hollow part facing down. The
longer bar has counterweights on the bottom such that the dome closes
even if the driving mechanism is broken, simply through gravity.</p>
</div>
<figure id="S2.F3" class="ltx_figure"><img src="x3.png" id="S2.F3.g1" class="ltx_graphics" width="677" height="588" alt="Half-open dome of HAT. The asymmetric clamshell structure ensures
fail-safe operation and minimal resistance against wind.
">
<figcaption class="ltx_caption"><span class="ltx_tag ltx_tag_figure">Figure 3: </span>Half-open dome of HAT. The asymmetric clamshell structure ensures
fail-safe operation and minimal resistance against wind.
</figcaption>
</figure>
<div id="S2.SS2.p4" class="ltx_para">
<p id="S2.SS2.p4.1" class="ltx_p">The shorter bar is fixed to a shaft inside the dome, which is driven by
a windshield-wiper DC motor through low gear ratio. Two sensors detect
if the dome reached any of its end positions (closed or open). Safety
power supply for the motor is a 12V 7.5Ah battery, continuously
recharged from the main PC power supply of the dome, which in turn is
connected to the UPS in the warmroom. Even without recharging, this
battery would be able to open/close the dome several times a day for a
week, though there is no need for this, as power failures are detected
by the software, and the system is automatically shut down. Although
normally the PC controls dome operations, a light sensor on the bottom
of the dome acts as a “big-brother”, and closes the dome if
illuminance reaches that of a dawn/dusk sky. Similarly, a research
grade Vaisala rain-detector has the capability of emergency closing.</p>
</div>
<div id="S2.SS2.p5" class="ltx_para">
<p id="S2.SS2.p5.1" class="ltx_p">Due to problems with skyflat frames (see later in §<a href="#S6" title="6 Observations from Kitt Peak ‣ System description and first light-curves of HAT, an autonomous observatory for variability search" class="ltx_ref"><span class="ltx_text ltx_ref_tag">6</span></a>) a
domeflat screen is fixed to the inner north roof, and lit by small
bulbs from positions which yield almost even illuminance not affected
by the telescope’s position. When the telescope is pointed to the north
pole, the screen is perpendicular to its aperture. Rotation of the
entire mount about the pole averages out approx. half of the residual
illuminance pattern.
</p>
</div>
<div id="S2.SS2.p6" class="ltx_para">
<p id="S2.SS2.p6.1" class="ltx_p">Dome operations can be also performed manually, overriding other
control. All electronics are installed in a separate compartment on the
south wall, and can be easily dismounted from the outside for servicing
without opening the dome.</p>
</div>
</section>
<section id="S2.SS3" class="ltx_subsection">
<h3 class="ltx_title ltx_title_subsection">
<span class="ltx_tag ltx_tag_subsection">2.3 </span>Electronics</h3>

<div id="S2.SS3.p1" class="ltx_para">
<p id="S2.SS3.p1.1" class="ltx_p">The electronic cards and devices in the dome are connected to the 12V
DC outputs of a simple PC power supply, which in turn gets power from
the warmroom’s UPS. Some devices are also connected to the
re-chargeable battery, and can be operated even during power failure.
One card is responsible for control of the mount, another for all other
devices, including the dome.</p>
</div>
<div id="S2.SS3.p2" class="ltx_para">
<p id="S2.SS3.p2.1" class="ltx_p">Signals coming from the parallel port of the PC are optically isolated,
and micro-controller units convert them for direct control of the
stepper motors. Cards are simply <span id="S2.SS3.p2.1.1" class="ltx_text ltx_font_italic">converters</span>, and the
motordrive/clockdrive is the PC’s central processing unit, controlled
by the RTLinux OS.</p>
</div>
</section>
<section id="S2.SS4" class="ltx_subsection">
<h3 class="ltx_title ltx_title_subsection">
<span class="ltx_tag ltx_tag_subsection">2.4 </span>The CCD</h3>

<div id="S2.SS4.p1" class="ltx_para">
<p id="S2.SS4.p1.1" class="ltx_p">HAT has been operated with two CCDs. The camera during the test period
of 2000/2001 from Budapest, Konkoly Observatory, was an amateur-class
<span id="S2.SS4.p1.1.1" class="ltx_text ltx_font_bold">Meade Pictor 416xt</span> camera with a Kodak KAF-0401E chip of
<math id="S2.SS4.p1.1.m1.2" class="ltx_Math" alttext="\rm 512\times 768,\;9\mu m" display="inline"><semantics id="S2.SS4.p1.1.m1.2a"><mrow id="S2.SS4.p1.1.m1.2.2.2" xref="S2.SS4.p1.1.m1.2.2.3.cmml"><mrow id="S2.SS4.p1.1.m1.1.1.1.1" xref="S2.SS4.p1.1.m1.1.1.1.1.cmml"><mn id="S2.SS4.p1.1.m1.1.1.1.1.2" xref="S2.SS4.p1.1.m1.1.1.1.1.2.cmml">512</mn><mo id="S2.SS4.p1.1.m1.1.1.1.1.1" xref="S2.SS4.p1.1.m1.1.1.1.1.1.cmml">×</mo><mn id="S2.SS4.p1.1.m1.1.1.1.1.3" xref="S2.SS4.p1.1.m1.1.1.1.1.3.cmml">768</mn></mrow><mo id="S2.SS4.p1.1.m1.2.2.2.3" xref="S2.SS4.p1.1.m1.2.2.3.cmml">,</mo><mrow id="S2.SS4.p1.1.m1.2.2.2.2" xref="S2.SS4.p1.1.m1.2.2.2.2.cmml"><mn id="S2.SS4.p1.1.m1.2.2.2.2.2" xref="S2.SS4.p1.1.m1.2.2.2.2.2.cmml"> 9</mn><mo id="S2.SS4.p1.1.m1.2.2.2.2.1" xref="S2.SS4.p1.1.m1.2.2.2.2.1.cmml">⁢</mo><mi id="S2.SS4.p1.1.m1.2.2.2.2.3" xref="S2.SS4.p1.1.m1.2.2.2.2.3.cmml">μ</mi><mo id="S2.SS4.p1.1.m1.2.2.2.2.1a" xref="S2.SS4.p1.1.m1.2.2.2.2.1.cmml">⁢</mo><mi mathvariant="normal" id="S2.SS4.p1.1.m1.2.2.2.2.4" xref="S2.SS4.p1.1.m1.2.2.2.2.4.cmml">m</mi></mrow></mrow><annotation-xml encoding="MathML-Content" id="S2.SS4.p1.1.m1.2b"><list id="S2.SS4.p1.1.m1.2.2.3.cmml" xref="S2.SS4.p1.1.m1.2.2.2"><apply id="S2.SS4.p1.1.m1.1.1.1.1.cmml" xref="S2.SS4.p1.1.m1.1.1.1.1"><times id="S2.SS4.p1.1.m1.1.1.1.1.1.cmml" xref="S2.SS4.p1.1.m1.1.1.1.1.1"></times><cn type="integer" id="S2.SS4.p1.1.m1.1.1.1.1.2.cmml" xref="S2.SS4.p1.1.m1.1.1.1.1.2">512</cn><cn type="integer" id="S2.SS4.p1.1.m1.1.1.1.1.3.cmml" xref="S2.SS4.p1.1.m1.1.1.1.1.3">768</cn></apply><apply id="S2.SS4.p1.1.m1.2.2.2.2.cmml" xref="S2.SS4.p1.1.m1.2.2.2.2"><times id="S2.SS4.p1.1.m1.2.2.2.2.1.cmml" xref="S2.SS4.p1.1.m1.2.2.2.2.1"></times><cn type="integer" id="S2.SS4.p1.1.m1.2.2.2.2.2.cmml" xref="S2.SS4.p1.1.m1.2.2.2.2.2">9</cn><ci id="S2.SS4.p1.1.m1.2.2.2.2.3.cmml" xref="S2.SS4.p1.1.m1.2.2.2.2.3">𝜇</ci><ci id="S2.SS4.p1.1.m1.2.2.2.2.4.cmml" xref="S2.SS4.p1.1.m1.2.2.2.2.4">m</ci></apply></list></annotation-xml><annotation encoding="application/x-tex" id="S2.SS4.p1.1.m1.2c">\rm 512\times 768,\;9\mu m</annotation><annotation encoding="application/x-llamapun" id="S2.SS4.p1.1.m1.2d">512 × 768 , 9 italic_μ roman_m</annotation></semantics></math> pixels. This camera yielded noisy images with bad
electronic interference pattern and functioned erratically, but was suitable
for test-mode operation <cite class="ltx_cite ltx_citemacro_citep">(Pojmański, <a href="#bib.bib39" title="" class="ltx_ref">1997</a>; Bakos, <a href="#bib.bib6" title="" class="ltx_ref">2001</a>, for more details)</cite>. All astrometric
calibration described in §<a href="#S4" title="4 Pointing precision of HAT ‣ System description and first light-curves of HAT, an autonomous observatory for variability search" class="ltx_ref"><span class="ltx_text ltx_ref_tag">4</span></a> was performed with this CCD.</p>
</div>
<div id="S2.SS4.p2" class="ltx_para">
<p id="S2.SS4.p2.7" class="ltx_p">The current Kitt Peak setup uses an <span id="S2.SS4.p2.7.1" class="ltx_text ltx_font_bold">Apogee AP10</span> camera with
Thomson THX 7899M <math id="S2.SS4.p2.1.m1.2" class="ltx_Math" alttext="\rm 2K\times 2K,\;14\mu m" display="inline"><semantics id="S2.SS4.p2.1.m1.2a"><mrow id="S2.SS4.p2.1.m1.2.2.2" xref="S2.SS4.p2.1.m1.2.2.3.cmml"><mrow id="S2.SS4.p2.1.m1.1.1.1.1" xref="S2.SS4.p2.1.m1.1.1.1.1.cmml"><mrow id="S2.SS4.p2.1.m1.1.1.1.1.2" xref="S2.SS4.p2.1.m1.1.1.1.1.2.cmml"><mrow id="S2.SS4.p2.1.m1.1.1.1.1.2.2" xref="S2.SS4.p2.1.m1.1.1.1.1.2.2.cmml"><mn id="S2.SS4.p2.1.m1.1.1.1.1.2.2.2" xref="S2.SS4.p2.1.m1.1.1.1.1.2.2.2.cmml">2</mn><mo id="S2.SS4.p2.1.m1.1.1.1.1.2.2.1" xref="S2.SS4.p2.1.m1.1.1.1.1.2.2.1.cmml">⁢</mo><mi mathvariant="normal" id="S2.SS4.p2.1.m1.1.1.1.1.2.2.3" xref="S2.SS4.p2.1.m1.1.1.1.1.2.2.3.cmml">K</mi></mrow><mo id="S2.SS4.p2.1.m1.1.1.1.1.2.1" xref="S2.SS4.p2.1.m1.1.1.1.1.2.1.cmml">×</mo><mn id="S2.SS4.p2.1.m1.1.1.1.1.2.3" xref="S2.SS4.p2.1.m1.1.1.1.1.2.3.cmml">2</mn></mrow><mo id="S2.SS4.p2.1.m1.1.1.1.1.1" xref="S2.SS4.p2.1.m1.1.1.1.1.1.cmml">⁢</mo><mi mathvariant="normal" id="S2.SS4.p2.1.m1.1.1.1.1.3" xref="S2.SS4.p2.1.m1.1.1.1.1.3.cmml">K</mi></mrow><mo id="S2.SS4.p2.1.m1.2.2.2.3" xref="S2.SS4.p2.1.m1.2.2.3.cmml">,</mo><mrow id="S2.SS4.p2.1.m1.2.2.2.2" xref="S2.SS4.p2.1.m1.2.2.2.2.cmml"><mn id="S2.SS4.p2.1.m1.2.2.2.2.2" xref="S2.SS4.p2.1.m1.2.2.2.2.2.cmml"> 14</mn><mo id="S2.SS4.p2.1.m1.2.2.2.2.1" xref="S2.SS4.p2.1.m1.2.2.2.2.1.cmml">⁢</mo><mi id="S2.SS4.p2.1.m1.2.2.2.2.3" xref="S2.SS4.p2.1.m1.2.2.2.2.3.cmml">μ</mi><mo id="S2.SS4.p2.1.m1.2.2.2.2.1a" xref="S2.SS4.p2.1.m1.2.2.2.2.1.cmml">⁢</mo><mi mathvariant="normal" id="S2.SS4.p2.1.m1.2.2.2.2.4" xref="S2.SS4.p2.1.m1.2.2.2.2.4.cmml">m</mi></mrow></mrow><annotation-xml encoding="MathML-Content" id="S2.SS4.p2.1.m1.2b"><list id="S2.SS4.p2.1.m1.2.2.3.cmml" xref="S2.SS4.p2.1.m1.2.2.2"><apply id="S2.SS4.p2.1.m1.1.1.1.1.cmml" xref="S2.SS4.p2.1.m1.1.1.1.1"><times id="S2.SS4.p2.1.m1.1.1.1.1.1.cmml" xref="S2.SS4.p2.1.m1.1.1.1.1.1"></times><apply id="S2.SS4.p2.1.m1.1.1.1.1.2.cmml" xref="S2.SS4.p2.1.m1.1.1.1.1.2"><times id="S2.SS4.p2.1.m1.1.1.1.1.2.1.cmml" xref="S2.SS4.p2.1.m1.1.1.1.1.2.1"></times><apply id="S2.SS4.p2.1.m1.1.1.1.1.2.2.cmml" xref="S2.SS4.p2.1.m1.1.1.1.1.2.2"><times id="S2.SS4.p2.1.m1.1.1.1.1.2.2.1.cmml" xref="S2.SS4.p2.1.m1.1.1.1.1.2.2.1"></times><cn type="integer" id="S2.SS4.p2.1.m1.1.1.1.1.2.2.2.cmml" xref="S2.SS4.p2.1.m1.1.1.1.1.2.2.2">2</cn><ci id="S2.SS4.p2.1.m1.1.1.1.1.2.2.3.cmml" xref="S2.SS4.p2.1.m1.1.1.1.1.2.2.3">K</ci></apply><cn type="integer" id="S2.SS4.p2.1.m1.1.1.1.1.2.3.cmml" xref="S2.SS4.p2.1.m1.1.1.1.1.2.3">2</cn></apply><ci id="S2.SS4.p2.1.m1.1.1.1.1.3.cmml" xref="S2.SS4.p2.1.m1.1.1.1.1.3">K</ci></apply><apply id="S2.SS4.p2.1.m1.2.2.2.2.cmml" xref="S2.SS4.p2.1.m1.2.2.2.2"><times id="S2.SS4.p2.1.m1.2.2.2.2.1.cmml" xref="S2.SS4.p2.1.m1.2.2.2.2.1"></times><cn type="integer" id="S2.SS4.p2.1.m1.2.2.2.2.2.cmml" xref="S2.SS4.p2.1.m1.2.2.2.2.2">14</cn><ci id="S2.SS4.p2.1.m1.2.2.2.2.3.cmml" xref="S2.SS4.p2.1.m1.2.2.2.2.3">𝜇</ci><ci id="S2.SS4.p2.1.m1.2.2.2.2.4.cmml" xref="S2.SS4.p2.1.m1.2.2.2.2.4">m</ci></apply></list></annotation-xml><annotation encoding="application/x-tex" id="S2.SS4.p2.1.m1.2c">\rm 2K\times 2K,\;14\mu m</annotation><annotation encoding="application/x-llamapun" id="S2.SS4.p2.1.m1.2d">2 roman_K × 2 roman_K , 14 italic_μ roman_m</annotation></semantics></math>, grade 2 chip. The chip
covers <math id="S2.SS4.p2.2.m2.1" class="ltx_Math" alttext="\rm\sim 29\times 29mm" display="inline"><semantics id="S2.SS4.p2.2.m2.1a"><mrow id="S2.SS4.p2.2.m2.1.1" xref="S2.SS4.p2.2.m2.1.1.cmml"><mi id="S2.SS4.p2.2.m2.1.1.2" xref="S2.SS4.p2.2.m2.1.1.2.cmml"></mi><mo id="S2.SS4.p2.2.m2.1.1.1" xref="S2.SS4.p2.2.m2.1.1.1.cmml">∼</mo><mrow id="S2.SS4.p2.2.m2.1.1.3" xref="S2.SS4.p2.2.m2.1.1.3.cmml"><mrow id="S2.SS4.p2.2.m2.1.1.3.2" xref="S2.SS4.p2.2.m2.1.1.3.2.cmml"><mn id="S2.SS4.p2.2.m2.1.1.3.2.2" xref="S2.SS4.p2.2.m2.1.1.3.2.2.cmml">29</mn><mo id="S2.SS4.p2.2.m2.1.1.3.2.1" xref="S2.SS4.p2.2.m2.1.1.3.2.1.cmml">×</mo><mn id="S2.SS4.p2.2.m2.1.1.3.2.3" xref="S2.SS4.p2.2.m2.1.1.3.2.3.cmml">29</mn></mrow><mo id="S2.SS4.p2.2.m2.1.1.3.1" xref="S2.SS4.p2.2.m2.1.1.3.1.cmml">⁢</mo><mi mathvariant="normal" id="S2.SS4.p2.2.m2.1.1.3.3" xref="S2.SS4.p2.2.m2.1.1.3.3.cmml">m</mi><mo id="S2.SS4.p2.2.m2.1.1.3.1a" xref="S2.SS4.p2.2.m2.1.1.3.1.cmml">⁢</mo><mi mathvariant="normal" id="S2.SS4.p2.2.m2.1.1.3.4" xref="S2.SS4.p2.2.m2.1.1.3.4.cmml">m</mi></mrow></mrow><annotation-xml encoding="MathML-Content" id="S2.SS4.p2.2.m2.1b"><apply id="S2.SS4.p2.2.m2.1.1.cmml" xref="S2.SS4.p2.2.m2.1.1"><csymbol cd="latexml" id="S2.SS4.p2.2.m2.1.1.1.cmml" xref="S2.SS4.p2.2.m2.1.1.1">similar-to</csymbol><csymbol cd="latexml" id="S2.SS4.p2.2.m2.1.1.2.cmml" xref="S2.SS4.p2.2.m2.1.1.2">absent</csymbol><apply id="S2.SS4.p2.2.m2.1.1.3.cmml" xref="S2.SS4.p2.2.m2.1.1.3"><times id="S2.SS4.p2.2.m2.1.1.3.1.cmml" xref="S2.SS4.p2.2.m2.1.1.3.1"></times><apply id="S2.SS4.p2.2.m2.1.1.3.2.cmml" xref="S2.SS4.p2.2.m2.1.1.3.2"><times id="S2.SS4.p2.2.m2.1.1.3.2.1.cmml" xref="S2.SS4.p2.2.m2.1.1.3.2.1"></times><cn type="integer" id="S2.SS4.p2.2.m2.1.1.3.2.2.cmml" xref="S2.SS4.p2.2.m2.1.1.3.2.2">29</cn><cn type="integer" id="S2.SS4.p2.2.m2.1.1.3.2.3.cmml" xref="S2.SS4.p2.2.m2.1.1.3.2.3">29</cn></apply><ci id="S2.SS4.p2.2.m2.1.1.3.3.cmml" xref="S2.SS4.p2.2.m2.1.1.3.3">m</ci><ci id="S2.SS4.p2.2.m2.1.1.3.4.cmml" xref="S2.SS4.p2.2.m2.1.1.3.4">m</ci></apply></apply></annotation-xml><annotation encoding="application/x-tex" id="S2.SS4.p2.2.m2.1c">\rm\sim 29\times 29mm</annotation><annotation encoding="application/x-llamapun" id="S2.SS4.p2.2.m2.1d">∼ 29 × 29 roman_m roman_m</annotation></semantics></math>, i.e., comparable to the size of small
format photographic films (<math id="S2.SS4.p2.3.m3.1" class="ltx_Math" alttext="\rm 24\times 36mm" display="inline"><semantics id="S2.SS4.p2.3.m3.1a"><mrow id="S2.SS4.p2.3.m3.1.1" xref="S2.SS4.p2.3.m3.1.1.cmml"><mrow id="S2.SS4.p2.3.m3.1.1.2" xref="S2.SS4.p2.3.m3.1.1.2.cmml"><mn id="S2.SS4.p2.3.m3.1.1.2.2" xref="S2.SS4.p2.3.m3.1.1.2.2.cmml">24</mn><mo id="S2.SS4.p2.3.m3.1.1.2.1" xref="S2.SS4.p2.3.m3.1.1.2.1.cmml">×</mo><mn id="S2.SS4.p2.3.m3.1.1.2.3" xref="S2.SS4.p2.3.m3.1.1.2.3.cmml">36</mn></mrow><mo id="S2.SS4.p2.3.m3.1.1.1" xref="S2.SS4.p2.3.m3.1.1.1.cmml">⁢</mo><mi mathvariant="normal" id="S2.SS4.p2.3.m3.1.1.3" xref="S2.SS4.p2.3.m3.1.1.3.cmml">m</mi><mo id="S2.SS4.p2.3.m3.1.1.1a" xref="S2.SS4.p2.3.m3.1.1.1.cmml">⁢</mo><mi mathvariant="normal" id="S2.SS4.p2.3.m3.1.1.4" xref="S2.SS4.p2.3.m3.1.1.4.cmml">m</mi></mrow><annotation-xml encoding="MathML-Content" id="S2.SS4.p2.3.m3.1b"><apply id="S2.SS4.p2.3.m3.1.1.cmml" xref="S2.SS4.p2.3.m3.1.1"><times id="S2.SS4.p2.3.m3.1.1.1.cmml" xref="S2.SS4.p2.3.m3.1.1.1"></times><apply id="S2.SS4.p2.3.m3.1.1.2.cmml" xref="S2.SS4.p2.3.m3.1.1.2"><times id="S2.SS4.p2.3.m3.1.1.2.1.cmml" xref="S2.SS4.p2.3.m3.1.1.2.1"></times><cn type="integer" id="S2.SS4.p2.3.m3.1.1.2.2.cmml" xref="S2.SS4.p2.3.m3.1.1.2.2">24</cn><cn type="integer" id="S2.SS4.p2.3.m3.1.1.2.3.cmml" xref="S2.SS4.p2.3.m3.1.1.2.3">36</cn></apply><ci id="S2.SS4.p2.3.m3.1.1.3.cmml" xref="S2.SS4.p2.3.m3.1.1.3">m</ci><ci id="S2.SS4.p2.3.m3.1.1.4.cmml" xref="S2.SS4.p2.3.m3.1.1.4">m</ci></apply></annotation-xml><annotation encoding="application/x-tex" id="S2.SS4.p2.3.m3.1c">\rm 24\times 36mm</annotation><annotation encoding="application/x-llamapun" id="S2.SS4.p2.3.m3.1d">24 × 36 roman_m roman_m</annotation></semantics></math>). Full-well capacity is
<math id="S2.SS4.p2.4.m4.2" class="ltx_Math" alttext="200,000" display="inline"><semantics id="S2.SS4.p2.4.m4.2a"><mrow id="S2.SS4.p2.4.m4.2.3.2" xref="S2.SS4.p2.4.m4.2.3.1.cmml"><mn id="S2.SS4.p2.4.m4.1.1" xref="S2.SS4.p2.4.m4.1.1.cmml">200</mn><mo id="S2.SS4.p2.4.m4.2.3.2.1" xref="S2.SS4.p2.4.m4.2.3.1.cmml">,</mo><mn id="S2.SS4.p2.4.m4.2.2" xref="S2.SS4.p2.4.m4.2.2.cmml">000</mn></mrow><annotation-xml encoding="MathML-Content" id="S2.SS4.p2.4.m4.2b"><list id="S2.SS4.p2.4.m4.2.3.1.cmml" xref="S2.SS4.p2.4.m4.2.3.2"><cn type="integer" id="S2.SS4.p2.4.m4.1.1.cmml" xref="S2.SS4.p2.4.m4.1.1">200</cn><cn type="integer" id="S2.SS4.p2.4.m4.2.2.cmml" xref="S2.SS4.p2.4.m4.2.2">000</cn></list></annotation-xml><annotation encoding="application/x-tex" id="S2.SS4.p2.4.m4.2c">200,000</annotation><annotation encoding="application/x-llamapun" id="S2.SS4.p2.4.m4.2d">200 , 000</annotation></semantics></math><math id="S2.SS4.p2.5.m5.1" class="ltx_Math" alttext="e^{-}" display="inline"><semantics id="S2.SS4.p2.5.m5.1a"><msup id="S2.SS4.p2.5.m5.1.1" xref="S2.SS4.p2.5.m5.1.1.cmml"><mi id="S2.SS4.p2.5.m5.1.1.2" xref="S2.SS4.p2.5.m5.1.1.2.cmml">e</mi><mo id="S2.SS4.p2.5.m5.1.1.3" xref="S2.SS4.p2.5.m5.1.1.3.cmml">-</mo></msup><annotation-xml encoding="MathML-Content" id="S2.SS4.p2.5.m5.1b"><apply id="S2.SS4.p2.5.m5.1.1.cmml" xref="S2.SS4.p2.5.m5.1.1"><csymbol cd="ambiguous" id="S2.SS4.p2.5.m5.1.1.1.cmml" xref="S2.SS4.p2.5.m5.1.1">superscript</csymbol><ci id="S2.SS4.p2.5.m5.1.1.2.cmml" xref="S2.SS4.p2.5.m5.1.1.2">𝑒</ci><minus id="S2.SS4.p2.5.m5.1.1.3.cmml" xref="S2.SS4.p2.5.m5.1.1.3"></minus></apply></annotation-xml><annotation encoding="application/x-tex" id="S2.SS4.p2.5.m5.1c">e^{-}</annotation><annotation encoding="application/x-llamapun" id="S2.SS4.p2.5.m5.1d">italic_e start_POSTSUPERSCRIPT - end_POSTSUPERSCRIPT</annotation></semantics></math>, factory gain setting is <math id="S2.SS4.p2.6.m6.1" class="ltx_Math" alttext="\rm 10e^{-}/ADU" display="inline"><semantics id="S2.SS4.p2.6.m6.1a"><mrow id="S2.SS4.p2.6.m6.1.1" xref="S2.SS4.p2.6.m6.1.1.cmml"><mrow id="S2.SS4.p2.6.m6.1.1.2" xref="S2.SS4.p2.6.m6.1.1.2.cmml"><mn id="S2.SS4.p2.6.m6.1.1.2.2" xref="S2.SS4.p2.6.m6.1.1.2.2.cmml">10</mn><mo id="S2.SS4.p2.6.m6.1.1.2.1" xref="S2.SS4.p2.6.m6.1.1.2.1.cmml">⁢</mo><msup id="S2.SS4.p2.6.m6.1.1.2.3" xref="S2.SS4.p2.6.m6.1.1.2.3.cmml"><mi mathvariant="normal" id="S2.SS4.p2.6.m6.1.1.2.3.2" xref="S2.SS4.p2.6.m6.1.1.2.3.2.cmml">e</mi><mo id="S2.SS4.p2.6.m6.1.1.2.3.3" xref="S2.SS4.p2.6.m6.1.1.2.3.3.cmml">-</mo></msup></mrow><mo id="S2.SS4.p2.6.m6.1.1.1" xref="S2.SS4.p2.6.m6.1.1.1.cmml">/</mo><mi id="S2.SS4.p2.6.m6.1.1.3" xref="S2.SS4.p2.6.m6.1.1.3.cmml">ADU</mi></mrow><annotation-xml encoding="MathML-Content" id="S2.SS4.p2.6.m6.1b"><apply id="S2.SS4.p2.6.m6.1.1.cmml" xref="S2.SS4.p2.6.m6.1.1"><divide id="S2.SS4.p2.6.m6.1.1.1.cmml" xref="S2.SS4.p2.6.m6.1.1.1"></divide><apply id="S2.SS4.p2.6.m6.1.1.2.cmml" xref="S2.SS4.p2.6.m6.1.1.2"><times id="S2.SS4.p2.6.m6.1.1.2.1.cmml" xref="S2.SS4.p2.6.m6.1.1.2.1"></times><cn type="integer" id="S2.SS4.p2.6.m6.1.1.2.2.cmml" xref="S2.SS4.p2.6.m6.1.1.2.2">10</cn><apply id="S2.SS4.p2.6.m6.1.1.2.3.cmml" xref="S2.SS4.p2.6.m6.1.1.2.3"><csymbol cd="ambiguous" id="S2.SS4.p2.6.m6.1.1.2.3.1.cmml" xref="S2.SS4.p2.6.m6.1.1.2.3">superscript</csymbol><ci id="S2.SS4.p2.6.m6.1.1.2.3.2.cmml" xref="S2.SS4.p2.6.m6.1.1.2.3.2">e</ci><minus id="S2.SS4.p2.6.m6.1.1.2.3.3.cmml" xref="S2.SS4.p2.6.m6.1.1.2.3.3"></minus></apply></apply><ci id="S2.SS4.p2.6.m6.1.1.3.cmml" xref="S2.SS4.p2.6.m6.1.1.3">ADU</ci></apply></annotation-xml><annotation encoding="application/x-tex" id="S2.SS4.p2.6.m6.1c">\rm 10e^{-}/ADU</annotation><annotation encoding="application/x-llamapun" id="S2.SS4.p2.6.m6.1d">10 roman_e start_POSTSUPERSCRIPT - end_POSTSUPERSCRIPT / roman_ADU</annotation></semantics></math>, thus given the
14-bit (16000 ADU) dynamic range, saturation is slightly limited by the
A/D conversion. Quantum efficiency of the Thomson chip peaks at 40%
between <math id="S2.SS4.p2.7.m7.1" class="ltx_Math" alttext="\rm 650nm\--800nm" display="inline"><semantics id="S2.SS4.p2.7.m7.1a"><mrow id="S2.SS4.p2.7.m7.1.1" xref="S2.SS4.p2.7.m7.1.1.cmml"><mrow id="S2.SS4.p2.7.m7.1.1.2" xref="S2.SS4.p2.7.m7.1.1.2.cmml"><mn id="S2.SS4.p2.7.m7.1.1.2.2" xref="S2.SS4.p2.7.m7.1.1.2.2.cmml">650</mn><mo id="S2.SS4.p2.7.m7.1.1.2.1" xref="S2.SS4.p2.7.m7.1.1.2.1.cmml">⁢</mo><mi mathvariant="normal" id="S2.SS4.p2.7.m7.1.1.2.3" xref="S2.SS4.p2.7.m7.1.1.2.3.cmml">n</mi><mo id="S2.SS4.p2.7.m7.1.1.2.1a" xref="S2.SS4.p2.7.m7.1.1.2.1.cmml">⁢</mo><mi mathvariant="normal" id="S2.SS4.p2.7.m7.1.1.2.4" xref="S2.SS4.p2.7.m7.1.1.2.4.cmml">m</mi></mrow><mo id="S2.SS4.p2.7.m7.1.1.1" xref="S2.SS4.p2.7.m7.1.1.1.cmml">-</mo><mrow id="S2.SS4.p2.7.m7.1.1.3" xref="S2.SS4.p2.7.m7.1.1.3.cmml"><mn id="S2.SS4.p2.7.m7.1.1.3.2" xref="S2.SS4.p2.7.m7.1.1.3.2.cmml">800</mn><mo id="S2.SS4.p2.7.m7.1.1.3.1" xref="S2.SS4.p2.7.m7.1.1.3.1.cmml">⁢</mo><mi mathvariant="normal" id="S2.SS4.p2.7.m7.1.1.3.3" xref="S2.SS4.p2.7.m7.1.1.3.3.cmml">n</mi><mo id="S2.SS4.p2.7.m7.1.1.3.1a" xref="S2.SS4.p2.7.m7.1.1.3.1.cmml">⁢</mo><mi mathvariant="normal" id="S2.SS4.p2.7.m7.1.1.3.4" xref="S2.SS4.p2.7.m7.1.1.3.4.cmml">m</mi></mrow></mrow><annotation-xml encoding="MathML-Content" id="S2.SS4.p2.7.m7.1b"><apply id="S2.SS4.p2.7.m7.1.1.cmml" xref="S2.SS4.p2.7.m7.1.1"><minus id="S2.SS4.p2.7.m7.1.1.1.cmml" xref="S2.SS4.p2.7.m7.1.1.1"></minus><apply id="S2.SS4.p2.7.m7.1.1.2.cmml" xref="S2.SS4.p2.7.m7.1.1.2"><times id="S2.SS4.p2.7.m7.1.1.2.1.cmml" xref="S2.SS4.p2.7.m7.1.1.2.1"></times><cn type="integer" id="S2.SS4.p2.7.m7.1.1.2.2.cmml" xref="S2.SS4.p2.7.m7.1.1.2.2">650</cn><ci id="S2.SS4.p2.7.m7.1.1.2.3.cmml" xref="S2.SS4.p2.7.m7.1.1.2.3">n</ci><ci id="S2.SS4.p2.7.m7.1.1.2.4.cmml" xref="S2.SS4.p2.7.m7.1.1.2.4">m</ci></apply><apply id="S2.SS4.p2.7.m7.1.1.3.cmml" xref="S2.SS4.p2.7.m7.1.1.3"><times id="S2.SS4.p2.7.m7.1.1.3.1.cmml" xref="S2.SS4.p2.7.m7.1.1.3.1"></times><cn type="integer" id="S2.SS4.p2.7.m7.1.1.3.2.cmml" xref="S2.SS4.p2.7.m7.1.1.3.2">800</cn><ci id="S2.SS4.p2.7.m7.1.1.3.3.cmml" xref="S2.SS4.p2.7.m7.1.1.3.3">n</ci><ci id="S2.SS4.p2.7.m7.1.1.3.4.cmml" xref="S2.SS4.p2.7.m7.1.1.3.4">m</ci></apply></apply></annotation-xml><annotation encoding="application/x-tex" id="S2.SS4.p2.7.m7.1c">\rm 650nm\--800nm</annotation><annotation encoding="application/x-llamapun" id="S2.SS4.p2.7.m7.1d">650 roman_n roman_m - 800 roman_n roman_m</annotation></semantics></math>.</p>
</div>
<div id="S2.SS4.p3" class="ltx_para">
<p id="S2.SS4.p3.10" class="ltx_p">Readout noise is variable, being <math id="S2.SS4.p3.1.m1.1" class="ltx_Math" alttext="\gtrsim 20" display="inline"><semantics id="S2.SS4.p3.1.m1.1a"><mrow id="S2.SS4.p3.1.m1.1.1" xref="S2.SS4.p3.1.m1.1.1.cmml"><mi id="S2.SS4.p3.1.m1.1.1.2" xref="S2.SS4.p3.1.m1.1.1.2.cmml"></mi><mo id="S2.SS4.p3.1.m1.1.1.1" xref="S2.SS4.p3.1.m1.1.1.1.cmml">≳</mo><mn id="S2.SS4.p3.1.m1.1.1.3" xref="S2.SS4.p3.1.m1.1.1.3.cmml">20</mn></mrow><annotation-xml encoding="MathML-Content" id="S2.SS4.p3.1.m1.1b"><apply id="S2.SS4.p3.1.m1.1.1.cmml" xref="S2.SS4.p3.1.m1.1.1"><csymbol cd="latexml" id="S2.SS4.p3.1.m1.1.1.1.cmml" xref="S2.SS4.p3.1.m1.1.1.1">greater-than-or-equivalent-to</csymbol><csymbol cd="latexml" id="S2.SS4.p3.1.m1.1.1.2.cmml" xref="S2.SS4.p3.1.m1.1.1.2">absent</csymbol><cn type="integer" id="S2.SS4.p3.1.m1.1.1.3.cmml" xref="S2.SS4.p3.1.m1.1.1.3">20</cn></apply></annotation-xml><annotation encoding="application/x-tex" id="S2.SS4.p3.1.m1.1c">\gtrsim 20</annotation><annotation encoding="application/x-llamapun" id="S2.SS4.p3.1.m1.1d">≳ 20</annotation></semantics></math><math id="S2.SS4.p3.2.m2.1" class="ltx_Math" alttext="e^{-}" display="inline"><semantics id="S2.SS4.p3.2.m2.1a"><msup id="S2.SS4.p3.2.m2.1.1" xref="S2.SS4.p3.2.m2.1.1.cmml"><mi id="S2.SS4.p3.2.m2.1.1.2" xref="S2.SS4.p3.2.m2.1.1.2.cmml">e</mi><mo id="S2.SS4.p3.2.m2.1.1.3" xref="S2.SS4.p3.2.m2.1.1.3.cmml">-</mo></msup><annotation-xml encoding="MathML-Content" id="S2.SS4.p3.2.m2.1b"><apply id="S2.SS4.p3.2.m2.1.1.cmml" xref="S2.SS4.p3.2.m2.1.1"><csymbol cd="ambiguous" id="S2.SS4.p3.2.m2.1.1.1.cmml" xref="S2.SS4.p3.2.m2.1.1">superscript</csymbol><ci id="S2.SS4.p3.2.m2.1.1.2.cmml" xref="S2.SS4.p3.2.m2.1.1.2">𝑒</ci><minus id="S2.SS4.p3.2.m2.1.1.3.cmml" xref="S2.SS4.p3.2.m2.1.1.3"></minus></apply></annotation-xml><annotation encoding="application/x-tex" id="S2.SS4.p3.2.m2.1c">e^{-}</annotation><annotation encoding="application/x-llamapun" id="S2.SS4.p3.2.m2.1d">italic_e start_POSTSUPERSCRIPT - end_POSTSUPERSCRIPT</annotation></semantics></math>. Bias level is <math id="S2.SS4.p3.3.m3.1" class="ltx_Math" alttext="\rm 270\pm 5ADU" display="inline"><semantics id="S2.SS4.p3.3.m3.1a"><mrow id="S2.SS4.p3.3.m3.1.1" xref="S2.SS4.p3.3.m3.1.1.cmml"><mn id="S2.SS4.p3.3.m3.1.1.2" xref="S2.SS4.p3.3.m3.1.1.2.cmml">270</mn><mo id="S2.SS4.p3.3.m3.1.1.1" xref="S2.SS4.p3.3.m3.1.1.1.cmml">±</mo><mrow id="S2.SS4.p3.3.m3.1.1.3" xref="S2.SS4.p3.3.m3.1.1.3.cmml"><mn id="S2.SS4.p3.3.m3.1.1.3.2" xref="S2.SS4.p3.3.m3.1.1.3.2.cmml">5</mn><mo id="S2.SS4.p3.3.m3.1.1.3.1" xref="S2.SS4.p3.3.m3.1.1.3.1.cmml">⁢</mo><mi mathvariant="normal" id="S2.SS4.p3.3.m3.1.1.3.3" xref="S2.SS4.p3.3.m3.1.1.3.3.cmml">A</mi><mo id="S2.SS4.p3.3.m3.1.1.3.1a" xref="S2.SS4.p3.3.m3.1.1.3.1.cmml">⁢</mo><mi mathvariant="normal" id="S2.SS4.p3.3.m3.1.1.3.4" xref="S2.SS4.p3.3.m3.1.1.3.4.cmml">D</mi><mo id="S2.SS4.p3.3.m3.1.1.3.1b" xref="S2.SS4.p3.3.m3.1.1.3.1.cmml">⁢</mo><mi mathvariant="normal" id="S2.SS4.p3.3.m3.1.1.3.5" xref="S2.SS4.p3.3.m3.1.1.3.5.cmml">U</mi></mrow></mrow><annotation-xml encoding="MathML-Content" id="S2.SS4.p3.3.m3.1b"><apply id="S2.SS4.p3.3.m3.1.1.cmml" xref="S2.SS4.p3.3.m3.1.1"><csymbol cd="latexml" id="S2.SS4.p3.3.m3.1.1.1.cmml" xref="S2.SS4.p3.3.m3.1.1.1">plus-or-minus</csymbol><cn type="integer" id="S2.SS4.p3.3.m3.1.1.2.cmml" xref="S2.SS4.p3.3.m3.1.1.2">270</cn><apply id="S2.SS4.p3.3.m3.1.1.3.cmml" xref="S2.SS4.p3.3.m3.1.1.3"><times id="S2.SS4.p3.3.m3.1.1.3.1.cmml" xref="S2.SS4.p3.3.m3.1.1.3.1"></times><cn type="integer" id="S2.SS4.p3.3.m3.1.1.3.2.cmml" xref="S2.SS4.p3.3.m3.1.1.3.2">5</cn><ci id="S2.SS4.p3.3.m3.1.1.3.3.cmml" xref="S2.SS4.p3.3.m3.1.1.3.3">A</ci><ci id="S2.SS4.p3.3.m3.1.1.3.4.cmml" xref="S2.SS4.p3.3.m3.1.1.3.4">D</ci><ci id="S2.SS4.p3.3.m3.1.1.3.5.cmml" xref="S2.SS4.p3.3.m3.1.1.3.5">U</ci></apply></apply></annotation-xml><annotation encoding="application/x-tex" id="S2.SS4.p3.3.m3.1c">\rm 270\pm 5ADU</annotation><annotation encoding="application/x-llamapun" id="S2.SS4.p3.3.m3.1d">270 ± 5 roman_A roman_D roman_U</annotation></semantics></math> (nightly variation), with considerable long-term
(day-to-day) instability of <math id="S2.SS4.p3.4.m4.1" class="ltx_Math" alttext="\rm\pm 15ADU" display="inline"><semantics id="S2.SS4.p3.4.m4.1a"><mrow id="S2.SS4.p3.4.m4.1.1" xref="S2.SS4.p3.4.m4.1.1.cmml"><mo id="S2.SS4.p3.4.m4.1.1.1" xref="S2.SS4.p3.4.m4.1.1.1.cmml">±</mo><mrow id="S2.SS4.p3.4.m4.1.1.2" xref="S2.SS4.p3.4.m4.1.1.2.cmml"><mn id="S2.SS4.p3.4.m4.1.1.2.2" xref="S2.SS4.p3.4.m4.1.1.2.2.cmml">15</mn><mo id="S2.SS4.p3.4.m4.1.1.2.1" xref="S2.SS4.p3.4.m4.1.1.2.1.cmml">⁢</mo><mi mathvariant="normal" id="S2.SS4.p3.4.m4.1.1.2.3" xref="S2.SS4.p3.4.m4.1.1.2.3.cmml">A</mi><mo id="S2.SS4.p3.4.m4.1.1.2.1a" xref="S2.SS4.p3.4.m4.1.1.2.1.cmml">⁢</mo><mi mathvariant="normal" id="S2.SS4.p3.4.m4.1.1.2.4" xref="S2.SS4.p3.4.m4.1.1.2.4.cmml">D</mi><mo id="S2.SS4.p3.4.m4.1.1.2.1b" xref="S2.SS4.p3.4.m4.1.1.2.1.cmml">⁢</mo><mi mathvariant="normal" id="S2.SS4.p3.4.m4.1.1.2.5" xref="S2.SS4.p3.4.m4.1.1.2.5.cmml">U</mi></mrow></mrow><annotation-xml encoding="MathML-Content" id="S2.SS4.p3.4.m4.1b"><apply id="S2.SS4.p3.4.m4.1.1.cmml" xref="S2.SS4.p3.4.m4.1.1"><csymbol cd="latexml" id="S2.SS4.p3.4.m4.1.1.1.cmml" xref="S2.SS4.p3.4.m4.1.1.1">plus-or-minus</csymbol><apply id="S2.SS4.p3.4.m4.1.1.2.cmml" xref="S2.SS4.p3.4.m4.1.1.2"><times id="S2.SS4.p3.4.m4.1.1.2.1.cmml" xref="S2.SS4.p3.4.m4.1.1.2.1"></times><cn type="integer" id="S2.SS4.p3.4.m4.1.1.2.2.cmml" xref="S2.SS4.p3.4.m4.1.1.2.2">15</cn><ci id="S2.SS4.p3.4.m4.1.1.2.3.cmml" xref="S2.SS4.p3.4.m4.1.1.2.3">A</ci><ci id="S2.SS4.p3.4.m4.1.1.2.4.cmml" xref="S2.SS4.p3.4.m4.1.1.2.4">D</ci><ci id="S2.SS4.p3.4.m4.1.1.2.5.cmml" xref="S2.SS4.p3.4.m4.1.1.2.5">U</ci></apply></apply></annotation-xml><annotation encoding="application/x-tex" id="S2.SS4.p3.4.m4.1c">\rm\pm 15ADU</annotation><annotation encoding="application/x-llamapun" id="S2.SS4.p3.4.m4.1d">± 15 roman_A roman_D roman_U</annotation></semantics></math>. Dark current at
<math id="S2.SS4.p3.5.m5.1" class="ltx_Math" alttext="-15" display="inline"><semantics id="S2.SS4.p3.5.m5.1a"><mrow id="S2.SS4.p3.5.m5.1.1" xref="S2.SS4.p3.5.m5.1.1.cmml"><mo id="S2.SS4.p3.5.m5.1.1.1" xref="S2.SS4.p3.5.m5.1.1.1.cmml">-</mo><mn id="S2.SS4.p3.5.m5.1.1.2" xref="S2.SS4.p3.5.m5.1.1.2.cmml">15</mn></mrow><annotation-xml encoding="MathML-Content" id="S2.SS4.p3.5.m5.1b"><apply id="S2.SS4.p3.5.m5.1.1.cmml" xref="S2.SS4.p3.5.m5.1.1"><minus id="S2.SS4.p3.5.m5.1.1.1.cmml" xref="S2.SS4.p3.5.m5.1.1.1"></minus><cn type="integer" id="S2.SS4.p3.5.m5.1.1.2.cmml" xref="S2.SS4.p3.5.m5.1.1.2">15</cn></apply></annotation-xml><annotation encoding="application/x-tex" id="S2.SS4.p3.5.m5.1c">-15</annotation><annotation encoding="application/x-llamapun" id="S2.SS4.p3.5.m5.1d">- 15</annotation></semantics></math>°C is <math id="S2.SS4.p3.6.m6.1" class="ltx_Math" alttext="\rm\sim 0.05ADU/sec" display="inline"><semantics id="S2.SS4.p3.6.m6.1a"><mrow id="S2.SS4.p3.6.m6.1.1" xref="S2.SS4.p3.6.m6.1.1.cmml"><mi id="S2.SS4.p3.6.m6.1.1.2" xref="S2.SS4.p3.6.m6.1.1.2.cmml"></mi><mo id="S2.SS4.p3.6.m6.1.1.1" xref="S2.SS4.p3.6.m6.1.1.1.cmml">∼</mo><mrow id="S2.SS4.p3.6.m6.1.1.3" xref="S2.SS4.p3.6.m6.1.1.3.cmml"><mrow id="S2.SS4.p3.6.m6.1.1.3.2" xref="S2.SS4.p3.6.m6.1.1.3.2.cmml"><mn id="S2.SS4.p3.6.m6.1.1.3.2.2" xref="S2.SS4.p3.6.m6.1.1.3.2.2.cmml">0.05</mn><mo id="S2.SS4.p3.6.m6.1.1.3.2.1" xref="S2.SS4.p3.6.m6.1.1.3.2.1.cmml">⁢</mo><mi id="S2.SS4.p3.6.m6.1.1.3.2.3" xref="S2.SS4.p3.6.m6.1.1.3.2.3.cmml">ADU</mi></mrow><mo id="S2.SS4.p3.6.m6.1.1.3.1" xref="S2.SS4.p3.6.m6.1.1.3.1.cmml">/</mo><mi id="S2.SS4.p3.6.m6.1.1.3.3" xref="S2.SS4.p3.6.m6.1.1.3.3.cmml">sec</mi></mrow></mrow><annotation-xml encoding="MathML-Content" id="S2.SS4.p3.6.m6.1b"><apply id="S2.SS4.p3.6.m6.1.1.cmml" xref="S2.SS4.p3.6.m6.1.1"><csymbol cd="latexml" id="S2.SS4.p3.6.m6.1.1.1.cmml" xref="S2.SS4.p3.6.m6.1.1.1">similar-to</csymbol><csymbol cd="latexml" id="S2.SS4.p3.6.m6.1.1.2.cmml" xref="S2.SS4.p3.6.m6.1.1.2">absent</csymbol><apply id="S2.SS4.p3.6.m6.1.1.3.cmml" xref="S2.SS4.p3.6.m6.1.1.3"><divide id="S2.SS4.p3.6.m6.1.1.3.1.cmml" xref="S2.SS4.p3.6.m6.1.1.3.1"></divide><apply id="S2.SS4.p3.6.m6.1.1.3.2.cmml" xref="S2.SS4.p3.6.m6.1.1.3.2"><times id="S2.SS4.p3.6.m6.1.1.3.2.1.cmml" xref="S2.SS4.p3.6.m6.1.1.3.2.1"></times><cn type="float" id="S2.SS4.p3.6.m6.1.1.3.2.2.cmml" xref="S2.SS4.p3.6.m6.1.1.3.2.2">0.05</cn><ci id="S2.SS4.p3.6.m6.1.1.3.2.3.cmml" xref="S2.SS4.p3.6.m6.1.1.3.2.3">ADU</ci></apply><ci id="S2.SS4.p3.6.m6.1.1.3.3.cmml" xref="S2.SS4.p3.6.m6.1.1.3.3">sec</ci></apply></apply></annotation-xml><annotation encoding="application/x-tex" id="S2.SS4.p3.6.m6.1c">\rm\sim 0.05ADU/sec</annotation><annotation encoding="application/x-llamapun" id="S2.SS4.p3.6.m6.1d">∼ 0.05 roman_ADU / roman_sec</annotation></semantics></math> and highly dependent on temperature
setting (<math id="S2.SS4.p3.7.m7.1" class="ltx_Math" alttext="\rm\sim 0.02ADU/sec" display="inline"><semantics id="S2.SS4.p3.7.m7.1a"><mrow id="S2.SS4.p3.7.m7.1.1" xref="S2.SS4.p3.7.m7.1.1.cmml"><mi id="S2.SS4.p3.7.m7.1.1.2" xref="S2.SS4.p3.7.m7.1.1.2.cmml"></mi><mo id="S2.SS4.p3.7.m7.1.1.1" xref="S2.SS4.p3.7.m7.1.1.1.cmml">∼</mo><mrow id="S2.SS4.p3.7.m7.1.1.3" xref="S2.SS4.p3.7.m7.1.1.3.cmml"><mrow id="S2.SS4.p3.7.m7.1.1.3.2" xref="S2.SS4.p3.7.m7.1.1.3.2.cmml"><mn id="S2.SS4.p3.7.m7.1.1.3.2.2" xref="S2.SS4.p3.7.m7.1.1.3.2.2.cmml">0.02</mn><mo id="S2.SS4.p3.7.m7.1.1.3.2.1" xref="S2.SS4.p3.7.m7.1.1.3.2.1.cmml">⁢</mo><mi id="S2.SS4.p3.7.m7.1.1.3.2.3" xref="S2.SS4.p3.7.m7.1.1.3.2.3.cmml">ADU</mi></mrow><mo id="S2.SS4.p3.7.m7.1.1.3.1" xref="S2.SS4.p3.7.m7.1.1.3.1.cmml">/</mo><mi id="S2.SS4.p3.7.m7.1.1.3.3" xref="S2.SS4.p3.7.m7.1.1.3.3.cmml">sec</mi></mrow></mrow><annotation-xml encoding="MathML-Content" id="S2.SS4.p3.7.m7.1b"><apply id="S2.SS4.p3.7.m7.1.1.cmml" xref="S2.SS4.p3.7.m7.1.1"><csymbol cd="latexml" id="S2.SS4.p3.7.m7.1.1.1.cmml" xref="S2.SS4.p3.7.m7.1.1.1">similar-to</csymbol><csymbol cd="latexml" id="S2.SS4.p3.7.m7.1.1.2.cmml" xref="S2.SS4.p3.7.m7.1.1.2">absent</csymbol><apply id="S2.SS4.p3.7.m7.1.1.3.cmml" xref="S2.SS4.p3.7.m7.1.1.3"><divide id="S2.SS4.p3.7.m7.1.1.3.1.cmml" xref="S2.SS4.p3.7.m7.1.1.3.1"></divide><apply id="S2.SS4.p3.7.m7.1.1.3.2.cmml" xref="S2.SS4.p3.7.m7.1.1.3.2"><times id="S2.SS4.p3.7.m7.1.1.3.2.1.cmml" xref="S2.SS4.p3.7.m7.1.1.3.2.1"></times><cn type="float" id="S2.SS4.p3.7.m7.1.1.3.2.2.cmml" xref="S2.SS4.p3.7.m7.1.1.3.2.2">0.02</cn><ci id="S2.SS4.p3.7.m7.1.1.3.2.3.cmml" xref="S2.SS4.p3.7.m7.1.1.3.2.3">ADU</ci></apply><ci id="S2.SS4.p3.7.m7.1.1.3.3.cmml" xref="S2.SS4.p3.7.m7.1.1.3.3">sec</ci></apply></apply></annotation-xml><annotation encoding="application/x-tex" id="S2.SS4.p3.7.m7.1c">\rm\sim 0.02ADU/sec</annotation><annotation encoding="application/x-llamapun" id="S2.SS4.p3.7.m7.1d">∼ 0.02 roman_ADU / roman_sec</annotation></semantics></math> at <math id="S2.SS4.p3.8.m8.1" class="ltx_Math" alttext="-25" display="inline"><semantics id="S2.SS4.p3.8.m8.1a"><mrow id="S2.SS4.p3.8.m8.1.1" xref="S2.SS4.p3.8.m8.1.1.cmml"><mo id="S2.SS4.p3.8.m8.1.1.1" xref="S2.SS4.p3.8.m8.1.1.1.cmml">-</mo><mn id="S2.SS4.p3.8.m8.1.1.2" xref="S2.SS4.p3.8.m8.1.1.2.cmml">25</mn></mrow><annotation-xml encoding="MathML-Content" id="S2.SS4.p3.8.m8.1b"><apply id="S2.SS4.p3.8.m8.1.1.cmml" xref="S2.SS4.p3.8.m8.1.1"><minus id="S2.SS4.p3.8.m8.1.1.1.cmml" xref="S2.SS4.p3.8.m8.1.1.1"></minus><cn type="integer" id="S2.SS4.p3.8.m8.1.1.2.cmml" xref="S2.SS4.p3.8.m8.1.1.2">25</cn></apply></annotation-xml><annotation encoding="application/x-tex" id="S2.SS4.p3.8.m8.1c">-25</annotation><annotation encoding="application/x-llamapun" id="S2.SS4.p3.8.m8.1d">- 25</annotation></semantics></math>°C). We would like to keep the
temperature of the system at a constant level throughout the year for the
uniformity of the data. Unfortunately the two-stage Peltier cooling is
capable of only <math id="S2.SS4.p3.9.m9.1" class="ltx_Math" alttext="\rm\Delta T\approx 30" display="inline"><semantics id="S2.SS4.p3.9.m9.1a"><mrow id="S2.SS4.p3.9.m9.1.1" xref="S2.SS4.p3.9.m9.1.1.cmml"><mrow id="S2.SS4.p3.9.m9.1.1.2" xref="S2.SS4.p3.9.m9.1.1.2.cmml"><mi mathvariant="normal" id="S2.SS4.p3.9.m9.1.1.2.2" xref="S2.SS4.p3.9.m9.1.1.2.2.cmml">Δ</mi><mo id="S2.SS4.p3.9.m9.1.1.2.1" xref="S2.SS4.p3.9.m9.1.1.2.1.cmml">⁢</mo><mi mathvariant="normal" id="S2.SS4.p3.9.m9.1.1.2.3" xref="S2.SS4.p3.9.m9.1.1.2.3.cmml">T</mi></mrow><mo id="S2.SS4.p3.9.m9.1.1.1" xref="S2.SS4.p3.9.m9.1.1.1.cmml">≈</mo><mn id="S2.SS4.p3.9.m9.1.1.3" xref="S2.SS4.p3.9.m9.1.1.3.cmml">30</mn></mrow><annotation-xml encoding="MathML-Content" id="S2.SS4.p3.9.m9.1b"><apply id="S2.SS4.p3.9.m9.1.1.cmml" xref="S2.SS4.p3.9.m9.1.1"><approx id="S2.SS4.p3.9.m9.1.1.1.cmml" xref="S2.SS4.p3.9.m9.1.1.1"></approx><apply id="S2.SS4.p3.9.m9.1.1.2.cmml" xref="S2.SS4.p3.9.m9.1.1.2"><times id="S2.SS4.p3.9.m9.1.1.2.1.cmml" xref="S2.SS4.p3.9.m9.1.1.2.1"></times><ci id="S2.SS4.p3.9.m9.1.1.2.2.cmml" xref="S2.SS4.p3.9.m9.1.1.2.2">Δ</ci><ci id="S2.SS4.p3.9.m9.1.1.2.3.cmml" xref="S2.SS4.p3.9.m9.1.1.2.3">T</ci></apply><cn type="integer" id="S2.SS4.p3.9.m9.1.1.3.cmml" xref="S2.SS4.p3.9.m9.1.1.3">30</cn></apply></annotation-xml><annotation encoding="application/x-tex" id="S2.SS4.p3.9.m9.1c">\rm\Delta T\approx 30</annotation><annotation encoding="application/x-llamapun" id="S2.SS4.p3.9.m9.1d">roman_Δ roman_T ≈ 30</annotation></semantics></math>°C, and the lowest value we can
achieve is <math id="S2.SS4.p3.10.m10.1" class="ltx_Math" alttext="-15\arcdeg C" display="inline"><semantics id="S2.SS4.p3.10.m10.1a"><mrow id="S2.SS4.p3.10.m10.1.1" xref="S2.SS4.p3.10.m10.1.1.cmml"><mo id="S2.SS4.p3.10.m10.1.1.1" xref="S2.SS4.p3.10.m10.1.1.1.cmml">-</mo><mrow id="S2.SS4.p3.10.m10.1.1.2" xref="S2.SS4.p3.10.m10.1.1.2.cmml"><mn id="S2.SS4.p3.10.m10.1.1.2.2" xref="S2.SS4.p3.10.m10.1.1.2.2.cmml">15</mn><mo id="S2.SS4.p3.10.m10.1.1.2.1" xref="S2.SS4.p3.10.m10.1.1.2.1.cmml">⁢</mo><mi mathvariant="normal" id="S2.SS4.p3.10.m10.1.1.2.3" xref="S2.SS4.p3.10.m10.1.1.2.3.cmml">°</mi><mo id="S2.SS4.p3.10.m10.1.1.2.1a" xref="S2.SS4.p3.10.m10.1.1.2.1.cmml">⁢</mo><mi id="S2.SS4.p3.10.m10.1.1.2.4" xref="S2.SS4.p3.10.m10.1.1.2.4.cmml">C</mi></mrow></mrow><annotation-xml encoding="MathML-Content" id="S2.SS4.p3.10.m10.1b"><apply id="S2.SS4.p3.10.m10.1.1.cmml" xref="S2.SS4.p3.10.m10.1.1"><minus id="S2.SS4.p3.10.m10.1.1.1.cmml" xref="S2.SS4.p3.10.m10.1.1.1"></minus><apply id="S2.SS4.p3.10.m10.1.1.2.cmml" xref="S2.SS4.p3.10.m10.1.1.2"><times id="S2.SS4.p3.10.m10.1.1.2.1.cmml" xref="S2.SS4.p3.10.m10.1.1.2.1"></times><cn type="integer" id="S2.SS4.p3.10.m10.1.1.2.2.cmml" xref="S2.SS4.p3.10.m10.1.1.2.2">15</cn><ci id="S2.SS4.p3.10.m10.1.1.2.3.cmml" xref="S2.SS4.p3.10.m10.1.1.2.3">°</ci><ci id="S2.SS4.p3.10.m10.1.1.2.4.cmml" xref="S2.SS4.p3.10.m10.1.1.2.4">𝐶</ci></apply></apply></annotation-xml><annotation encoding="application/x-tex" id="S2.SS4.p3.10.m10.1c">-15\arcdeg C</annotation><annotation encoding="application/x-llamapun" id="S2.SS4.p3.10.m10.1d">- 15 ° italic_C</annotation></semantics></math>.</p>
</div>
<div id="S2.SS4.p4" class="ltx_para">
<p id="S2.SS4.p4.1" class="ltx_p">The camera is connected to the PC by a data and a control cable plugged
into an ISA-card. The cable length in our setup is 18m, much longer
than the factory default (8m), although well below the company-claimed
maximum limit. This caused problems during the installation, and
hindered us from using the camera for the first 3 months, as the images
contained only a few bits. Typical readout time of a frame is <math id="S2.SS4.p4.1.m1.1" class="ltx_Math" alttext="\rm\lesssim 10sec" display="inline"><semantics id="S2.SS4.p4.1.m1.1a"><mrow id="S2.SS4.p4.1.m1.1.1" xref="S2.SS4.p4.1.m1.1.1.cmml"><mi id="S2.SS4.p4.1.m1.1.1.2" xref="S2.SS4.p4.1.m1.1.1.2.cmml"></mi><mo id="S2.SS4.p4.1.m1.1.1.1" xref="S2.SS4.p4.1.m1.1.1.1.cmml">≲</mo><mrow id="S2.SS4.p4.1.m1.1.1.3" xref="S2.SS4.p4.1.m1.1.1.3.cmml"><mn id="S2.SS4.p4.1.m1.1.1.3.2" xref="S2.SS4.p4.1.m1.1.1.3.2.cmml">10</mn><mo id="S2.SS4.p4.1.m1.1.1.3.1" xref="S2.SS4.p4.1.m1.1.1.3.1.cmml">⁢</mo><mi mathvariant="normal" id="S2.SS4.p4.1.m1.1.1.3.3" xref="S2.SS4.p4.1.m1.1.1.3.3.cmml">s</mi><mo id="S2.SS4.p4.1.m1.1.1.3.1a" xref="S2.SS4.p4.1.m1.1.1.3.1.cmml">⁢</mo><mi mathvariant="normal" id="S2.SS4.p4.1.m1.1.1.3.4" xref="S2.SS4.p4.1.m1.1.1.3.4.cmml">e</mi><mo id="S2.SS4.p4.1.m1.1.1.3.1b" xref="S2.SS4.p4.1.m1.1.1.3.1.cmml">⁢</mo><mi mathvariant="normal" id="S2.SS4.p4.1.m1.1.1.3.5" xref="S2.SS4.p4.1.m1.1.1.3.5.cmml">c</mi></mrow></mrow><annotation-xml encoding="MathML-Content" id="S2.SS4.p4.1.m1.1b"><apply id="S2.SS4.p4.1.m1.1.1.cmml" xref="S2.SS4.p4.1.m1.1.1"><csymbol cd="latexml" id="S2.SS4.p4.1.m1.1.1.1.cmml" xref="S2.SS4.p4.1.m1.1.1.1">less-than-or-similar-to</csymbol><csymbol cd="latexml" id="S2.SS4.p4.1.m1.1.1.2.cmml" xref="S2.SS4.p4.1.m1.1.1.2">absent</csymbol><apply id="S2.SS4.p4.1.m1.1.1.3.cmml" xref="S2.SS4.p4.1.m1.1.1.3"><times id="S2.SS4.p4.1.m1.1.1.3.1.cmml" xref="S2.SS4.p4.1.m1.1.1.3.1"></times><cn type="integer" id="S2.SS4.p4.1.m1.1.1.3.2.cmml" xref="S2.SS4.p4.1.m1.1.1.3.2">10</cn><ci id="S2.SS4.p4.1.m1.1.1.3.3.cmml" xref="S2.SS4.p4.1.m1.1.1.3.3">s</ci><ci id="S2.SS4.p4.1.m1.1.1.3.4.cmml" xref="S2.SS4.p4.1.m1.1.1.3.4">e</ci><ci id="S2.SS4.p4.1.m1.1.1.3.5.cmml" xref="S2.SS4.p4.1.m1.1.1.3.5">c</ci></apply></apply></annotation-xml><annotation encoding="application/x-tex" id="S2.SS4.p4.1.m1.1c">\rm\lesssim 10sec</annotation><annotation encoding="application/x-llamapun" id="S2.SS4.p4.1.m1.1d">≲ 10 roman_s roman_e roman_c</annotation></semantics></math> at 1.3MHz speed, so there is negligible dead-time due
to readout.</p>
</div>
<div id="S2.SS4.p5" class="ltx_para">
<p id="S2.SS4.p5.1" class="ltx_p">Bias frames have a distinct, relatively constant pattern; few bad
columns, many warm pixels, horizontal streaks and clusters of warm
pixels. Dark frames have similar structure, which does not completely
disappear after bias correction. About 10% of the frames have
anomalous noise and background, the latter is <math id="S2.SS4.p5.1.m1.1" class="ltx_Math" alttext="\sim 10\--20" display="inline"><semantics id="S2.SS4.p5.1.m1.1a"><mrow id="S2.SS4.p5.1.m1.1.1" xref="S2.SS4.p5.1.m1.1.1.cmml"><mi id="S2.SS4.p5.1.m1.1.1.2" xref="S2.SS4.p5.1.m1.1.1.2.cmml"></mi><mo id="S2.SS4.p5.1.m1.1.1.1" xref="S2.SS4.p5.1.m1.1.1.1.cmml">∼</mo><mrow id="S2.SS4.p5.1.m1.1.1.3" xref="S2.SS4.p5.1.m1.1.1.3.cmml"><mn id="S2.SS4.p5.1.m1.1.1.3.2" xref="S2.SS4.p5.1.m1.1.1.3.2.cmml">10</mn><mo id="S2.SS4.p5.1.m1.1.1.3.1" xref="S2.SS4.p5.1.m1.1.1.3.1.cmml">-</mo><mn id="S2.SS4.p5.1.m1.1.1.3.3" xref="S2.SS4.p5.1.m1.1.1.3.3.cmml">20</mn></mrow></mrow><annotation-xml encoding="MathML-Content" id="S2.SS4.p5.1.m1.1b"><apply id="S2.SS4.p5.1.m1.1.1.cmml" xref="S2.SS4.p5.1.m1.1.1"><csymbol cd="latexml" id="S2.SS4.p5.1.m1.1.1.1.cmml" xref="S2.SS4.p5.1.m1.1.1.1">similar-to</csymbol><csymbol cd="latexml" id="S2.SS4.p5.1.m1.1.1.2.cmml" xref="S2.SS4.p5.1.m1.1.1.2">absent</csymbol><apply id="S2.SS4.p5.1.m1.1.1.3.cmml" xref="S2.SS4.p5.1.m1.1.1.3"><minus id="S2.SS4.p5.1.m1.1.1.3.1.cmml" xref="S2.SS4.p5.1.m1.1.1.3.1"></minus><cn type="integer" id="S2.SS4.p5.1.m1.1.1.3.2.cmml" xref="S2.SS4.p5.1.m1.1.1.3.2">10</cn><cn type="integer" id="S2.SS4.p5.1.m1.1.1.3.3.cmml" xref="S2.SS4.p5.1.m1.1.1.3.3">20</cn></apply></apply></annotation-xml><annotation encoding="application/x-tex" id="S2.SS4.p5.1.m1.1c">\sim 10\--20</annotation><annotation encoding="application/x-llamapun" id="S2.SS4.p5.1.m1.1d">∼ 10 - 20</annotation></semantics></math> times
higher than the normal. Most likely these frames are due to a bug in
the readout electronics.</p>
</div>
</section>
<section id="S2.SS5" class="ltx_subsection">
<h3 class="ltx_title ltx_title_subsection">
<span class="ltx_tag ltx_tag_subsection">2.5 </span>The Telephoto Lens</h3>

<div id="S2.SS5.p1" class="ltx_para">
<p id="S2.SS5.p1.1" class="ltx_p">In principle any kind of telephoto lens or small telescope can be
attached to HAT’s mounting plate, which has dimensions smaller than as
described in §<a href="#S2.SS1" title="2.1 The Robotic Mount ‣ 2 Hardware System ‣ System description and first light-curves of HAT, an autonomous observatory for variability search" class="ltx_ref"><span class="ltx_text ltx_ref_tag">2.1</span></a>. Our choice of a Nikon 180mm f/2.8
manual focus telephoto lens (<math id="S2.SS5.p1.1.m1.1" class="ltx_Math" alttext="\rm 64mm\oslash" display="inline"><semantics id="S2.SS5.p1.1.m1.1a"><mrow id="S2.SS5.p1.1.m1.1b"><mn id="S2.SS5.p1.1.m1.1.1" xref="S2.SS5.p1.1.m1.1.1.cmml">64</mn><mi mathvariant="normal" id="S2.SS5.p1.1.m1.1.2" xref="S2.SS5.p1.1.m1.1.2.cmml">m</mi><mi mathvariant="normal" id="S2.SS5.p1.1.m1.1.3" xref="S2.SS5.p1.1.m1.1.3.cmml">m</mi><mo id="S2.SS5.p1.1.m1.1.4" xref="S2.SS5.p1.1.m1.1.4.cmml">⊘</mo></mrow><annotation-xml encoding="MathML-Content" id="S2.SS5.p1.1.m1.1c"><cerror id="S2.SS5.p1.1.m1.1d"><csymbol cd="ambiguous" id="S2.SS5.p1.1.m1.1e">fragments</csymbol><cn type="integer" id="S2.SS5.p1.1.m1.1.1.cmml" xref="S2.SS5.p1.1.m1.1.1">64</cn><csymbol cd="unknown" id="S2.SS5.p1.1.m1.1.2.cmml" xref="S2.SS5.p1.1.m1.1.2">m</csymbol><csymbol cd="unknown" id="S2.SS5.p1.1.m1.1.3.cmml" xref="S2.SS5.p1.1.m1.1.3">m</csymbol><ci id="S2.SS5.p1.1.m1.1.4.cmml" xref="S2.SS5.p1.1.m1.1.4">⊘</ci></cerror></annotation-xml><annotation encoding="application/x-tex" id="S2.SS5.p1.1.m1.1f">\rm 64mm\oslash</annotation><annotation encoding="application/x-llamapun" id="S2.SS5.p1.1.m1.1g">64 roman_m roman_m ⊘</annotation></semantics></math>) for the Kitt Peak setup
was motivated by several factors, such as our limited budget, the
acceptable quality of Nikon lenses, and approximate compatibility with
the ASAS-2 project, which uses 200mm focal length with the same Apogee
CCD <cite class="ltx_cite ltx_citemacro_citep">(Pojmański, <a href="#bib.bib42" title="" class="ltx_ref">2002</a>)</cite>.</p>
</div>
<div id="S2.SS5.p2" class="ltx_para">
<p id="S2.SS5.p2.4" class="ltx_p">The resulting FOV is <math id="S2.SS5.p2.1.m1.1" class="ltx_Math" alttext="9\arcdeg\times 9\arcdeg" display="inline"><semantics id="S2.SS5.p2.1.m1.1a"><mrow id="S2.SS5.p2.1.m1.1.1" xref="S2.SS5.p2.1.m1.1.1.cmml"><mrow id="S2.SS5.p2.1.m1.1.1.2" xref="S2.SS5.p2.1.m1.1.1.2.cmml"><mrow id="S2.SS5.p2.1.m1.1.1.2.2" xref="S2.SS5.p2.1.m1.1.1.2.2.cmml"><mn id="S2.SS5.p2.1.m1.1.1.2.2.2" xref="S2.SS5.p2.1.m1.1.1.2.2.2.cmml">9</mn><mo id="S2.SS5.p2.1.m1.1.1.2.2.1" xref="S2.SS5.p2.1.m1.1.1.2.2.1.cmml">⁢</mo><mi mathvariant="normal" id="S2.SS5.p2.1.m1.1.1.2.2.3" xref="S2.SS5.p2.1.m1.1.1.2.2.3.cmml">°</mi></mrow><mo id="S2.SS5.p2.1.m1.1.1.2.1" xref="S2.SS5.p2.1.m1.1.1.2.1.cmml">×</mo><mn id="S2.SS5.p2.1.m1.1.1.2.3" xref="S2.SS5.p2.1.m1.1.1.2.3.cmml">9</mn></mrow><mo id="S2.SS5.p2.1.m1.1.1.1" xref="S2.SS5.p2.1.m1.1.1.1.cmml">⁢</mo><mi mathvariant="normal" id="S2.SS5.p2.1.m1.1.1.3" xref="S2.SS5.p2.1.m1.1.1.3.cmml">°</mi></mrow><annotation-xml encoding="MathML-Content" id="S2.SS5.p2.1.m1.1b"><apply id="S2.SS5.p2.1.m1.1.1.cmml" xref="S2.SS5.p2.1.m1.1.1"><times id="S2.SS5.p2.1.m1.1.1.1.cmml" xref="S2.SS5.p2.1.m1.1.1.1"></times><apply id="S2.SS5.p2.1.m1.1.1.2.cmml" xref="S2.SS5.p2.1.m1.1.1.2"><times id="S2.SS5.p2.1.m1.1.1.2.1.cmml" xref="S2.SS5.p2.1.m1.1.1.2.1"></times><apply id="S2.SS5.p2.1.m1.1.1.2.2.cmml" xref="S2.SS5.p2.1.m1.1.1.2.2"><times id="S2.SS5.p2.1.m1.1.1.2.2.1.cmml" xref="S2.SS5.p2.1.m1.1.1.2.2.1"></times><cn type="integer" id="S2.SS5.p2.1.m1.1.1.2.2.2.cmml" xref="S2.SS5.p2.1.m1.1.1.2.2.2">9</cn><ci id="S2.SS5.p2.1.m1.1.1.2.2.3.cmml" xref="S2.SS5.p2.1.m1.1.1.2.2.3">°</ci></apply><cn type="integer" id="S2.SS5.p2.1.m1.1.1.2.3.cmml" xref="S2.SS5.p2.1.m1.1.1.2.3">9</cn></apply><ci id="S2.SS5.p2.1.m1.1.1.3.cmml" xref="S2.SS5.p2.1.m1.1.1.3">°</ci></apply></annotation-xml><annotation encoding="application/x-tex" id="S2.SS5.p2.1.m1.1c">9\arcdeg\times 9\arcdeg</annotation><annotation encoding="application/x-llamapun" id="S2.SS5.p2.1.m1.1d">9 ° × 9 °</annotation></semantics></math>, where one pixel corresponds to
<math id="S2.SS5.p2.2.m2.1" class="ltx_Math" alttext="\sim 16\arcsec" display="inline"><semantics id="S2.SS5.p2.2.m2.1a"><mrow id="S2.SS5.p2.2.m2.1.1" xref="S2.SS5.p2.2.m2.1.1.cmml"><mi id="S2.SS5.p2.2.m2.1.1.2" xref="S2.SS5.p2.2.m2.1.1.2.cmml"></mi><mo id="S2.SS5.p2.2.m2.1.1.1" xref="S2.SS5.p2.2.m2.1.1.1.cmml">∼</mo><mrow id="S2.SS5.p2.2.m2.1.1.3" xref="S2.SS5.p2.2.m2.1.1.3.cmml"><mn id="S2.SS5.p2.2.m2.1.1.3.2" xref="S2.SS5.p2.2.m2.1.1.3.2.cmml">16</mn><mo id="S2.SS5.p2.2.m2.1.1.3.1" xref="S2.SS5.p2.2.m2.1.1.3.1.cmml">⁢</mo><mi mathvariant="normal" id="S2.SS5.p2.2.m2.1.1.3.3" xref="S2.SS5.p2.2.m2.1.1.3.3.cmml">″</mi></mrow></mrow><annotation-xml encoding="MathML-Content" id="S2.SS5.p2.2.m2.1b"><apply id="S2.SS5.p2.2.m2.1.1.cmml" xref="S2.SS5.p2.2.m2.1.1"><csymbol cd="latexml" id="S2.SS5.p2.2.m2.1.1.1.cmml" xref="S2.SS5.p2.2.m2.1.1.1">similar-to</csymbol><csymbol cd="latexml" id="S2.SS5.p2.2.m2.1.1.2.cmml" xref="S2.SS5.p2.2.m2.1.1.2">absent</csymbol><apply id="S2.SS5.p2.2.m2.1.1.3.cmml" xref="S2.SS5.p2.2.m2.1.1.3"><times id="S2.SS5.p2.2.m2.1.1.3.1.cmml" xref="S2.SS5.p2.2.m2.1.1.3.1"></times><cn type="integer" id="S2.SS5.p2.2.m2.1.1.3.2.cmml" xref="S2.SS5.p2.2.m2.1.1.3.2">16</cn><ci id="S2.SS5.p2.2.m2.1.1.3.3.cmml" xref="S2.SS5.p2.2.m2.1.1.3.3">″</ci></apply></apply></annotation-xml><annotation encoding="application/x-tex" id="S2.SS5.p2.2.m2.1c">\sim 16\arcsec</annotation><annotation encoding="application/x-llamapun" id="S2.SS5.p2.2.m2.1d">∼ 16 ″</annotation></semantics></math>. The lens gives moderately sharp profiles throughout the
entire field, with half-width of the psf <math id="S2.SS5.p2.3.m3.1" class="ltx_Math" alttext="\sim 1.6\--2.0" display="inline"><semantics id="S2.SS5.p2.3.m3.1a"><mrow id="S2.SS5.p2.3.m3.1.1" xref="S2.SS5.p2.3.m3.1.1.cmml"><mi id="S2.SS5.p2.3.m3.1.1.2" xref="S2.SS5.p2.3.m3.1.1.2.cmml"></mi><mo id="S2.SS5.p2.3.m3.1.1.1" xref="S2.SS5.p2.3.m3.1.1.1.cmml">∼</mo><mrow id="S2.SS5.p2.3.m3.1.1.3" xref="S2.SS5.p2.3.m3.1.1.3.cmml"><mn id="S2.SS5.p2.3.m3.1.1.3.2" xref="S2.SS5.p2.3.m3.1.1.3.2.cmml">1.6</mn><mo id="S2.SS5.p2.3.m3.1.1.3.1" xref="S2.SS5.p2.3.m3.1.1.3.1.cmml">-</mo><mn id="S2.SS5.p2.3.m3.1.1.3.3" xref="S2.SS5.p2.3.m3.1.1.3.3.cmml">2.0</mn></mrow></mrow><annotation-xml encoding="MathML-Content" id="S2.SS5.p2.3.m3.1b"><apply id="S2.SS5.p2.3.m3.1.1.cmml" xref="S2.SS5.p2.3.m3.1.1"><csymbol cd="latexml" id="S2.SS5.p2.3.m3.1.1.1.cmml" xref="S2.SS5.p2.3.m3.1.1.1">similar-to</csymbol><csymbol cd="latexml" id="S2.SS5.p2.3.m3.1.1.2.cmml" xref="S2.SS5.p2.3.m3.1.1.2">absent</csymbol><apply id="S2.SS5.p2.3.m3.1.1.3.cmml" xref="S2.SS5.p2.3.m3.1.1.3"><minus id="S2.SS5.p2.3.m3.1.1.3.1.cmml" xref="S2.SS5.p2.3.m3.1.1.3.1"></minus><cn type="float" id="S2.SS5.p2.3.m3.1.1.3.2.cmml" xref="S2.SS5.p2.3.m3.1.1.3.2">1.6</cn><cn type="float" id="S2.SS5.p2.3.m3.1.1.3.3.cmml" xref="S2.SS5.p2.3.m3.1.1.3.3">2.0</cn></apply></apply></annotation-xml><annotation encoding="application/x-tex" id="S2.SS5.p2.3.m3.1c">\sim 1.6\--2.0</annotation><annotation encoding="application/x-llamapun" id="S2.SS5.p2.3.m3.1d">∼ 1.6 - 2.0</annotation></semantics></math> pixel
(<math id="S2.SS5.p2.4.m4.1" class="ltx_Math" alttext="25\arcsec\--32\arcsec" display="inline"><semantics id="S2.SS5.p2.4.m4.1a"><mrow id="S2.SS5.p2.4.m4.1.1" xref="S2.SS5.p2.4.m4.1.1.cmml"><mrow id="S2.SS5.p2.4.m4.1.1.2" xref="S2.SS5.p2.4.m4.1.1.2.cmml"><mn id="S2.SS5.p2.4.m4.1.1.2.2" xref="S2.SS5.p2.4.m4.1.1.2.2.cmml">25</mn><mo id="S2.SS5.p2.4.m4.1.1.2.1" xref="S2.SS5.p2.4.m4.1.1.2.1.cmml">⁢</mo><mi mathvariant="normal" id="S2.SS5.p2.4.m4.1.1.2.3" xref="S2.SS5.p2.4.m4.1.1.2.3.cmml">″</mi></mrow><mo id="S2.SS5.p2.4.m4.1.1.1" xref="S2.SS5.p2.4.m4.1.1.1.cmml">-</mo><mrow id="S2.SS5.p2.4.m4.1.1.3" xref="S2.SS5.p2.4.m4.1.1.3.cmml"><mn id="S2.SS5.p2.4.m4.1.1.3.2" xref="S2.SS5.p2.4.m4.1.1.3.2.cmml">32</mn><mo id="S2.SS5.p2.4.m4.1.1.3.1" xref="S2.SS5.p2.4.m4.1.1.3.1.cmml">⁢</mo><mi mathvariant="normal" id="S2.SS5.p2.4.m4.1.1.3.3" xref="S2.SS5.p2.4.m4.1.1.3.3.cmml">″</mi></mrow></mrow><annotation-xml encoding="MathML-Content" id="S2.SS5.p2.4.m4.1b"><apply id="S2.SS5.p2.4.m4.1.1.cmml" xref="S2.SS5.p2.4.m4.1.1"><minus id="S2.SS5.p2.4.m4.1.1.1.cmml" xref="S2.SS5.p2.4.m4.1.1.1"></minus><apply id="S2.SS5.p2.4.m4.1.1.2.cmml" xref="S2.SS5.p2.4.m4.1.1.2"><times id="S2.SS5.p2.4.m4.1.1.2.1.cmml" xref="S2.SS5.p2.4.m4.1.1.2.1"></times><cn type="integer" id="S2.SS5.p2.4.m4.1.1.2.2.cmml" xref="S2.SS5.p2.4.m4.1.1.2.2">25</cn><ci id="S2.SS5.p2.4.m4.1.1.2.3.cmml" xref="S2.SS5.p2.4.m4.1.1.2.3">″</ci></apply><apply id="S2.SS5.p2.4.m4.1.1.3.cmml" xref="S2.SS5.p2.4.m4.1.1.3"><times id="S2.SS5.p2.4.m4.1.1.3.1.cmml" xref="S2.SS5.p2.4.m4.1.1.3.1"></times><cn type="integer" id="S2.SS5.p2.4.m4.1.1.3.2.cmml" xref="S2.SS5.p2.4.m4.1.1.3.2">32</cn><ci id="S2.SS5.p2.4.m4.1.1.3.3.cmml" xref="S2.SS5.p2.4.m4.1.1.3.3">″</ci></apply></apply></annotation-xml><annotation encoding="application/x-tex" id="S2.SS5.p2.4.m4.1c">25\arcsec\--32\arcsec</annotation><annotation encoding="application/x-llamapun" id="S2.SS5.p2.4.m4.1d">25 ″ - 32 ″</annotation></semantics></math>). The corners show some coma and astigmatism. There
is substantial field-dependent vignetting, reaching a 40% intensity loss
near the edges, which is not surprising at lenses designed for small-format
photography.</p>
</div>
<div id="S2.SS5.p3" class="ltx_para">
<p id="S2.SS5.p3.1" class="ltx_p">A strong resistive heating of the lens is installed in the light-baffle
(dew cap). This not only prevents formation of dew on the front lens,
but also creates turbulence to blur the stellar profiles to some
extent. As our psf is undersampled, we have deliberately blurred the
image in this way in an attempt to improve photometry <cite class="ltx_cite ltx_citemacro_citep">(Pojmański, <a href="#bib.bib42" title="" class="ltx_ref">2002</a>, and
§<a href="#S7" title="7 Photometric precision of HAT ‣ System description and first light-curves of HAT, an autonomous observatory for variability search" class="ltx_ref"><span class="ltx_text ltx_ref_tag">7</span></a>)</cite>. Slight defocusing of the lens is not
possible, as it introduces strong, spatially dependent distortion of
the profiles.
</p>
</div>
<div id="S2.SS5.p4" class="ltx_para">
<p id="S2.SS5.p4.3" class="ltx_p">We observe through a single Cousins I-band filter (Bessel-made), which
has favorable combined sensitivity with the Thomson chip, yielding
considerably higher signal-to-noise ratio for typical stars and
integration times than e.g. a V filter would. The brightness of the
night sky is more stable in I-band at different lunar phases
<cite class="ltx_cite ltx_citemacro_citep">(Walker, <a href="#bib.bib49" title="" class="ltx_ref">1987</a>)</cite>,
<math id="S2.SS5.p4.1.m1.1" class="ltx_Math" alttext="I\approx 19.9^{m}/{\sq\arcsec}" display="inline"><semantics id="S2.SS5.p4.1.m1.1a"><mrow id="S2.SS5.p4.1.m1.1.1" xref="S2.SS5.p4.1.m1.1.1.cmml"><mi id="S2.SS5.p4.1.m1.1.1.2" xref="S2.SS5.p4.1.m1.1.1.2.cmml">I</mi><mo id="S2.SS5.p4.1.m1.1.1.1" xref="S2.SS5.p4.1.m1.1.1.1.cmml">≈</mo><mrow id="S2.SS5.p4.1.m1.1.1.3" xref="S2.SS5.p4.1.m1.1.1.3.cmml"><mrow id="S2.SS5.p4.1.m1.1.1.3.2" xref="S2.SS5.p4.1.m1.1.1.3.2.cmml"><msup id="S2.SS5.p4.1.m1.1.1.3.2.2" xref="S2.SS5.p4.1.m1.1.1.3.2.2.cmml"><mn id="S2.SS5.p4.1.m1.1.1.3.2.2.2" xref="S2.SS5.p4.1.m1.1.1.3.2.2.2.cmml">19.9</mn><mi id="S2.SS5.p4.1.m1.1.1.3.2.2.3" xref="S2.SS5.p4.1.m1.1.1.3.2.2.3.cmml">m</mi></msup><mo id="S2.SS5.p4.1.m1.1.1.3.2.1" xref="S2.SS5.p4.1.m1.1.1.3.2.1.cmml">/</mo><mi mathvariant="normal" id="S2.SS5.p4.1.m1.1.1.3.2.3" xref="S2.SS5.p4.1.m1.1.1.3.2.3.cmml">□</mi></mrow><mo id="S2.SS5.p4.1.m1.1.1.3.1" xref="S2.SS5.p4.1.m1.1.1.3.1.cmml">⁢</mo><mi mathvariant="normal" id="S2.SS5.p4.1.m1.1.1.3.3" xref="S2.SS5.p4.1.m1.1.1.3.3.cmml">″</mi></mrow></mrow><annotation-xml encoding="MathML-Content" id="S2.SS5.p4.1.m1.1b"><apply id="S2.SS5.p4.1.m1.1.1.cmml" xref="S2.SS5.p4.1.m1.1.1"><approx id="S2.SS5.p4.1.m1.1.1.1.cmml" xref="S2.SS5.p4.1.m1.1.1.1"></approx><ci id="S2.SS5.p4.1.m1.1.1.2.cmml" xref="S2.SS5.p4.1.m1.1.1.2">𝐼</ci><apply id="S2.SS5.p4.1.m1.1.1.3.cmml" xref="S2.SS5.p4.1.m1.1.1.3"><times id="S2.SS5.p4.1.m1.1.1.3.1.cmml" xref="S2.SS5.p4.1.m1.1.1.3.1"></times><apply id="S2.SS5.p4.1.m1.1.1.3.2.cmml" xref="S2.SS5.p4.1.m1.1.1.3.2"><divide id="S2.SS5.p4.1.m1.1.1.3.2.1.cmml" xref="S2.SS5.p4.1.m1.1.1.3.2.1"></divide><apply id="S2.SS5.p4.1.m1.1.1.3.2.2.cmml" xref="S2.SS5.p4.1.m1.1.1.3.2.2"><csymbol cd="ambiguous" id="S2.SS5.p4.1.m1.1.1.3.2.2.1.cmml" xref="S2.SS5.p4.1.m1.1.1.3.2.2">superscript</csymbol><cn type="float" id="S2.SS5.p4.1.m1.1.1.3.2.2.2.cmml" xref="S2.SS5.p4.1.m1.1.1.3.2.2.2">19.9</cn><ci id="S2.SS5.p4.1.m1.1.1.3.2.2.3.cmml" xref="S2.SS5.p4.1.m1.1.1.3.2.2.3">𝑚</ci></apply><ci id="S2.SS5.p4.1.m1.1.1.3.2.3.cmml" xref="S2.SS5.p4.1.m1.1.1.3.2.3">□</ci></apply><ci id="S2.SS5.p4.1.m1.1.1.3.3.cmml" xref="S2.SS5.p4.1.m1.1.1.3.3">″</ci></apply></apply></annotation-xml><annotation encoding="application/x-tex" id="S2.SS5.p4.1.m1.1c">I\approx 19.9^{m}/{\sq\arcsec}</annotation><annotation encoding="application/x-llamapun" id="S2.SS5.p4.1.m1.1d">italic_I ≈ 19.9 start_POSTSUPERSCRIPT italic_m end_POSTSUPERSCRIPT / □ ″</annotation></semantics></math> (new moon) to
<math id="S2.SS5.p4.2.m2.1" class="ltx_Math" alttext="I\approx 19.2^{m}/{\sq\arcsec}" display="inline"><semantics id="S2.SS5.p4.2.m2.1a"><mrow id="S2.SS5.p4.2.m2.1.1" xref="S2.SS5.p4.2.m2.1.1.cmml"><mi id="S2.SS5.p4.2.m2.1.1.2" xref="S2.SS5.p4.2.m2.1.1.2.cmml">I</mi><mo id="S2.SS5.p4.2.m2.1.1.1" xref="S2.SS5.p4.2.m2.1.1.1.cmml">≈</mo><mrow id="S2.SS5.p4.2.m2.1.1.3" xref="S2.SS5.p4.2.m2.1.1.3.cmml"><mrow id="S2.SS5.p4.2.m2.1.1.3.2" xref="S2.SS5.p4.2.m2.1.1.3.2.cmml"><msup id="S2.SS5.p4.2.m2.1.1.3.2.2" xref="S2.SS5.p4.2.m2.1.1.3.2.2.cmml"><mn id="S2.SS5.p4.2.m2.1.1.3.2.2.2" xref="S2.SS5.p4.2.m2.1.1.3.2.2.2.cmml">19.2</mn><mi id="S2.SS5.p4.2.m2.1.1.3.2.2.3" xref="S2.SS5.p4.2.m2.1.1.3.2.2.3.cmml">m</mi></msup><mo id="S2.SS5.p4.2.m2.1.1.3.2.1" xref="S2.SS5.p4.2.m2.1.1.3.2.1.cmml">/</mo><mi mathvariant="normal" id="S2.SS5.p4.2.m2.1.1.3.2.3" xref="S2.SS5.p4.2.m2.1.1.3.2.3.cmml">□</mi></mrow><mo id="S2.SS5.p4.2.m2.1.1.3.1" xref="S2.SS5.p4.2.m2.1.1.3.1.cmml">⁢</mo><mi mathvariant="normal" id="S2.SS5.p4.2.m2.1.1.3.3" xref="S2.SS5.p4.2.m2.1.1.3.3.cmml">″</mi></mrow></mrow><annotation-xml encoding="MathML-Content" id="S2.SS5.p4.2.m2.1b"><apply id="S2.SS5.p4.2.m2.1.1.cmml" xref="S2.SS5.p4.2.m2.1.1"><approx id="S2.SS5.p4.2.m2.1.1.1.cmml" xref="S2.SS5.p4.2.m2.1.1.1"></approx><ci id="S2.SS5.p4.2.m2.1.1.2.cmml" xref="S2.SS5.p4.2.m2.1.1.2">𝐼</ci><apply id="S2.SS5.p4.2.m2.1.1.3.cmml" xref="S2.SS5.p4.2.m2.1.1.3"><times id="S2.SS5.p4.2.m2.1.1.3.1.cmml" xref="S2.SS5.p4.2.m2.1.1.3.1"></times><apply id="S2.SS5.p4.2.m2.1.1.3.2.cmml" xref="S2.SS5.p4.2.m2.1.1.3.2"><divide id="S2.SS5.p4.2.m2.1.1.3.2.1.cmml" xref="S2.SS5.p4.2.m2.1.1.3.2.1"></divide><apply id="S2.SS5.p4.2.m2.1.1.3.2.2.cmml" xref="S2.SS5.p4.2.m2.1.1.3.2.2"><csymbol cd="ambiguous" id="S2.SS5.p4.2.m2.1.1.3.2.2.1.cmml" xref="S2.SS5.p4.2.m2.1.1.3.2.2">superscript</csymbol><cn type="float" id="S2.SS5.p4.2.m2.1.1.3.2.2.2.cmml" xref="S2.SS5.p4.2.m2.1.1.3.2.2.2">19.2</cn><ci id="S2.SS5.p4.2.m2.1.1.3.2.2.3.cmml" xref="S2.SS5.p4.2.m2.1.1.3.2.2.3">𝑚</ci></apply><ci id="S2.SS5.p4.2.m2.1.1.3.2.3.cmml" xref="S2.SS5.p4.2.m2.1.1.3.2.3">□</ci></apply><ci id="S2.SS5.p4.2.m2.1.1.3.3.cmml" xref="S2.SS5.p4.2.m2.1.1.3.3">″</ci></apply></apply></annotation-xml><annotation encoding="application/x-tex" id="S2.SS5.p4.2.m2.1c">I\approx 19.2^{m}/{\sq\arcsec}</annotation><annotation encoding="application/x-llamapun" id="S2.SS5.p4.2.m2.1d">italic_I ≈ 19.2 start_POSTSUPERSCRIPT italic_m end_POSTSUPERSCRIPT / □ ″</annotation></semantics></math> (full-moon), as compared to
e.g. <math id="S2.SS5.p4.3.m3.2" class="ltx_Math" alttext="V\approx 21.8^{m},\;20.0^{m}" display="inline"><semantics id="S2.SS5.p4.3.m3.2a"><mrow id="S2.SS5.p4.3.m3.2.2" xref="S2.SS5.p4.3.m3.2.2.cmml"><mi id="S2.SS5.p4.3.m3.2.2.4" xref="S2.SS5.p4.3.m3.2.2.4.cmml">V</mi><mo id="S2.SS5.p4.3.m3.2.2.3" xref="S2.SS5.p4.3.m3.2.2.3.cmml">≈</mo><mrow id="S2.SS5.p4.3.m3.2.2.2.2" xref="S2.SS5.p4.3.m3.2.2.2.3.cmml"><msup id="S2.SS5.p4.3.m3.1.1.1.1.1" xref="S2.SS5.p4.3.m3.1.1.1.1.1.cmml"><mn id="S2.SS5.p4.3.m3.1.1.1.1.1.2" xref="S2.SS5.p4.3.m3.1.1.1.1.1.2.cmml">21.8</mn><mi id="S2.SS5.p4.3.m3.1.1.1.1.1.3" xref="S2.SS5.p4.3.m3.1.1.1.1.1.3.cmml">m</mi></msup><mo id="S2.SS5.p4.3.m3.2.2.2.2.3" xref="S2.SS5.p4.3.m3.2.2.2.3.cmml">,</mo><msup id="S2.SS5.p4.3.m3.2.2.2.2.2" xref="S2.SS5.p4.3.m3.2.2.2.2.2.cmml"><mn id="S2.SS5.p4.3.m3.2.2.2.2.2.2" xref="S2.SS5.p4.3.m3.2.2.2.2.2.2.cmml"> 20.0</mn><mi id="S2.SS5.p4.3.m3.2.2.2.2.2.3" xref="S2.SS5.p4.3.m3.2.2.2.2.2.3.cmml">m</mi></msup></mrow></mrow><annotation-xml encoding="MathML-Content" id="S2.SS5.p4.3.m3.2b"><apply id="S2.SS5.p4.3.m3.2.2.cmml" xref="S2.SS5.p4.3.m3.2.2"><approx id="S2.SS5.p4.3.m3.2.2.3.cmml" xref="S2.SS5.p4.3.m3.2.2.3"></approx><ci id="S2.SS5.p4.3.m3.2.2.4.cmml" xref="S2.SS5.p4.3.m3.2.2.4">𝑉</ci><list id="S2.SS5.p4.3.m3.2.2.2.3.cmml" xref="S2.SS5.p4.3.m3.2.2.2.2"><apply id="S2.SS5.p4.3.m3.1.1.1.1.1.cmml" xref="S2.SS5.p4.3.m3.1.1.1.1.1"><csymbol cd="ambiguous" id="S2.SS5.p4.3.m3.1.1.1.1.1.1.cmml" xref="S2.SS5.p4.3.m3.1.1.1.1.1">superscript</csymbol><cn type="float" id="S2.SS5.p4.3.m3.1.1.1.1.1.2.cmml" xref="S2.SS5.p4.3.m3.1.1.1.1.1.2">21.8</cn><ci id="S2.SS5.p4.3.m3.1.1.1.1.1.3.cmml" xref="S2.SS5.p4.3.m3.1.1.1.1.1.3">𝑚</ci></apply><apply id="S2.SS5.p4.3.m3.2.2.2.2.2.cmml" xref="S2.SS5.p4.3.m3.2.2.2.2.2"><csymbol cd="ambiguous" id="S2.SS5.p4.3.m3.2.2.2.2.2.1.cmml" xref="S2.SS5.p4.3.m3.2.2.2.2.2">superscript</csymbol><cn type="float" id="S2.SS5.p4.3.m3.2.2.2.2.2.2.cmml" xref="S2.SS5.p4.3.m3.2.2.2.2.2.2">20.0</cn><ci id="S2.SS5.p4.3.m3.2.2.2.2.2.3.cmml" xref="S2.SS5.p4.3.m3.2.2.2.2.2.3">𝑚</ci></apply></list></apply></annotation-xml><annotation encoding="application/x-tex" id="S2.SS5.p4.3.m3.2c">V\approx 21.8^{m},\;20.0^{m}</annotation><annotation encoding="application/x-llamapun" id="S2.SS5.p4.3.m3.2d">italic_V ≈ 21.8 start_POSTSUPERSCRIPT italic_m end_POSTSUPERSCRIPT , 20.0 start_POSTSUPERSCRIPT italic_m end_POSTSUPERSCRIPT</annotation></semantics></math>, which ensures more uniform photometry.
New filters can be inserted only manually.</p>
</div>
</section>
<section id="S2.SS6" class="ltx_subsection">
<h3 class="ltx_title ltx_title_subsection">
<span class="ltx_tag ltx_tag_subsection">2.6 </span>The PC</h3>

<div id="S2.SS6.p1" class="ltx_para">
<p id="S2.SS6.p1.1" class="ltx_p">Any reasonable (<math id="S2.SS6.p1.1.m1.1" class="ltx_Math" alttext="\rm\gtrsim 300MHz" display="inline"><semantics id="S2.SS6.p1.1.m1.1a"><mrow id="S2.SS6.p1.1.m1.1.1" xref="S2.SS6.p1.1.m1.1.1.cmml"><mi id="S2.SS6.p1.1.m1.1.1.2" xref="S2.SS6.p1.1.m1.1.1.2.cmml"></mi><mo id="S2.SS6.p1.1.m1.1.1.1" xref="S2.SS6.p1.1.m1.1.1.1.cmml">≳</mo><mrow id="S2.SS6.p1.1.m1.1.1.3" xref="S2.SS6.p1.1.m1.1.1.3.cmml"><mn id="S2.SS6.p1.1.m1.1.1.3.2" xref="S2.SS6.p1.1.m1.1.1.3.2.cmml">300</mn><mo id="S2.SS6.p1.1.m1.1.1.3.1" xref="S2.SS6.p1.1.m1.1.1.3.1.cmml">⁢</mo><mi mathvariant="normal" id="S2.SS6.p1.1.m1.1.1.3.3" xref="S2.SS6.p1.1.m1.1.1.3.3.cmml">M</mi><mo id="S2.SS6.p1.1.m1.1.1.3.1a" xref="S2.SS6.p1.1.m1.1.1.3.1.cmml">⁢</mo><mi mathvariant="normal" id="S2.SS6.p1.1.m1.1.1.3.4" xref="S2.SS6.p1.1.m1.1.1.3.4.cmml">H</mi><mo id="S2.SS6.p1.1.m1.1.1.3.1b" xref="S2.SS6.p1.1.m1.1.1.3.1.cmml">⁢</mo><mi mathvariant="normal" id="S2.SS6.p1.1.m1.1.1.3.5" xref="S2.SS6.p1.1.m1.1.1.3.5.cmml">z</mi></mrow></mrow><annotation-xml encoding="MathML-Content" id="S2.SS6.p1.1.m1.1b"><apply id="S2.SS6.p1.1.m1.1.1.cmml" xref="S2.SS6.p1.1.m1.1.1"><csymbol cd="latexml" id="S2.SS6.p1.1.m1.1.1.1.cmml" xref="S2.SS6.p1.1.m1.1.1.1">greater-than-or-equivalent-to</csymbol><csymbol cd="latexml" id="S2.SS6.p1.1.m1.1.1.2.cmml" xref="S2.SS6.p1.1.m1.1.1.2">absent</csymbol><apply id="S2.SS6.p1.1.m1.1.1.3.cmml" xref="S2.SS6.p1.1.m1.1.1.3"><times id="S2.SS6.p1.1.m1.1.1.3.1.cmml" xref="S2.SS6.p1.1.m1.1.1.3.1"></times><cn type="integer" id="S2.SS6.p1.1.m1.1.1.3.2.cmml" xref="S2.SS6.p1.1.m1.1.1.3.2">300</cn><ci id="S2.SS6.p1.1.m1.1.1.3.3.cmml" xref="S2.SS6.p1.1.m1.1.1.3.3">M</ci><ci id="S2.SS6.p1.1.m1.1.1.3.4.cmml" xref="S2.SS6.p1.1.m1.1.1.3.4">H</ci><ci id="S2.SS6.p1.1.m1.1.1.3.5.cmml" xref="S2.SS6.p1.1.m1.1.1.3.5">z</ci></apply></apply></annotation-xml><annotation encoding="application/x-tex" id="S2.SS6.p1.1.m1.1c">\rm\gtrsim 300MHz</annotation><annotation encoding="application/x-llamapun" id="S2.SS6.p1.1.m1.1d">≳ 300 roman_M roman_H roman_z</annotation></semantics></math>) personal computer which can run a
Linux Operating System and real-time kernels<span id="footnote4" class="ltx_note ltx_role_footnote"><sup class="ltx_note_mark">4</sup><span class="ltx_note_outer"><span class="ltx_note_content"><sup class="ltx_note_mark">4</sup><span class="ltx_tag ltx_tag_note">4</span>As of writing,
Real-Time Linux 3.1, kernels (core of the OS) 2.2.19 or 2.4.<math id="footnote4.m1.1" class="ltx_Math" alttext="\ast" display="inline"><semantics id="footnote4.m1.1b"><mo id="footnote4.m1.1.1" xref="footnote4.m1.1.1.cmml">∗</mo><annotation-xml encoding="MathML-Content" id="footnote4.m1.1c"><ci id="footnote4.m1.1.1.cmml" xref="footnote4.m1.1.1">∗</ci></annotation-xml><annotation encoding="application/x-tex" id="footnote4.m1.1d">\ast</annotation><annotation encoding="application/x-llamapun" id="footnote4.m1.1e">∗</annotation></semantics></math>.</span></span></span>,
and has two parallel ports, is suitable for controlling the HAT mount
and dome. An ISA-slot is requisite for controlling the Apogee AP10 CCD,
and serial line is needed for the Meade Pictor CCD.</p>
</div>
<div id="S2.SS6.p2" class="ltx_para">
<p id="S2.SS6.p2.1" class="ltx_p">We use a custom-built PC with 900MHz Athlon AMD processor, 256Mb RAM,
DAT-DDS3 tape archiving facility, and a single 80Mb disc for temporary
storage. Outgoing cables are lightning-protected. Power is from a UPS
plugged in the same circuit as the UPS for the dome power supply. A
watchdog-card automatically hard-resets the computer if that freezes
(does not respond via a special software) for more than 10 minutes.</p>
</div>
</section>
</section>
<section id="S3" class="ltx_section">
<h2 class="ltx_title ltx_title_section">
<span class="ltx_tag ltx_tag_section">3 </span>Software System</h2>

<div id="S3.p1" class="ltx_para">
<p id="S3.p1.1" class="ltx_p">Our control PC runs an ordinary Linux OS with a Real-Time Linux
kernel<span id="footnote5" class="ltx_note ltx_role_footnote"><sup class="ltx_note_mark">5</sup><span class="ltx_note_outer"><span class="ltx_note_content"><sup class="ltx_note_mark">5</sup><span class="ltx_tag ltx_tag_note">5</span>www.fsmlabs.com</span></span></span>(denoted RTLinux). Accurate positioning
(e.g. sidereal speed tracking) needs precisely scheduled stepping
instructions from the control electronics, with the possibility of
frequency tuning (adjusting the tracking speed) and switching between
speeds (e.g. ramping). These are either directly emitted by a complex
external hardware with built-in frequency standard, <span id="S3.p1.1.1" class="ltx_text ltx_font_italic">or by the
computer</span> and its central processing unit (in fact a very complex
hardware, but off-the-shelf). Both approaches have pros and cons, but
we found the latter more flexible, because software can be adjusted
remotely, and because it requires less hardware development. The CPU’s
capability of emitting signals with a tight schedule depends on the OS.
With “single task” operating systems, such as DOS, proper scheduling
of a <span id="S3.p1.1.2" class="ltx_text ltx_font_italic">single</span> task was possible. However, a further demand is that
simultaneously with signal-generation, one should be able to execute
other tasks, such as CCD operation, disc IO, etc., each of them
requiring some CPU time, thus distorting the frequency of the high
priority processes requesting periodic operations.</p>
</div>
<div id="S3.p2" class="ltx_para">
<p id="S3.p2.1" class="ltx_p">RTLinux is such a <span id="S3.p2.1.1" class="ltx_text ltx_font_italic">real-time, multitask</span> operating system, where
the kernel treats the real-time processes as independent “threads”
with response times better than <math id="S3.p2.1.m1.1" class="ltx_Math" alttext="\rm 15\mu s" display="inline"><semantics id="S3.p2.1.m1.1a"><mrow id="S3.p2.1.m1.1.1" xref="S3.p2.1.m1.1.1.cmml"><mn id="S3.p2.1.m1.1.1.2" xref="S3.p2.1.m1.1.1.2.cmml">15</mn><mo id="S3.p2.1.m1.1.1.1" xref="S3.p2.1.m1.1.1.1.cmml">⁢</mo><mi id="S3.p2.1.m1.1.1.3" xref="S3.p2.1.m1.1.1.3.cmml">μ</mi><mo id="S3.p2.1.m1.1.1.1a" xref="S3.p2.1.m1.1.1.1.cmml">⁢</mo><mi mathvariant="normal" id="S3.p2.1.m1.1.1.4" xref="S3.p2.1.m1.1.1.4.cmml">s</mi></mrow><annotation-xml encoding="MathML-Content" id="S3.p2.1.m1.1b"><apply id="S3.p2.1.m1.1.1.cmml" xref="S3.p2.1.m1.1.1"><times id="S3.p2.1.m1.1.1.1.cmml" xref="S3.p2.1.m1.1.1.1"></times><cn type="integer" id="S3.p2.1.m1.1.1.2.cmml" xref="S3.p2.1.m1.1.1.2">15</cn><ci id="S3.p2.1.m1.1.1.3.cmml" xref="S3.p2.1.m1.1.1.3">𝜇</ci><ci id="S3.p2.1.m1.1.1.4.cmml" xref="S3.p2.1.m1.1.1.4">s</ci></apply></annotation-xml><annotation encoding="application/x-tex" id="S3.p2.1.m1.1c">\rm 15\mu s</annotation><annotation encoding="application/x-llamapun" id="S3.p2.1.m1.1d">15 italic_μ roman_s</annotation></semantics></math>. The Linux environment,
the user’s interface, is run at the lowest priority, but on relatively
fast PCs and assuming only a few real-time processes with no extreme
rescheduling frequencies, there is no noticeable difference to the
user.</p>
</div>
<div id="S3.p3" class="ltx_para">
<p id="S3.p3.1" class="ltx_p">The software environment consists of low level programs: <span id="S3.p3.1.1" class="ltx_text ltx_font_italic">scope and dome
drivers, cameraserver, HAT access module</span> (HAM), and high level software:
<span id="S3.p3.1.2" class="ltx_text ltx_font_italic">central database</span> (DB) and <span id="S3.p3.1.3" class="ltx_text ltx_font_italic">virtual observer</span> (“Observer”).
See Fig. <a href="#S3.F4" title="Figure 4 ‣ 3 Software System ‣ System description and first light-curves of HAT, an autonomous observatory for variability search" class="ltx_ref"><span class="ltx_text ltx_ref_tag">4</span></a> for a schematic flowchart.</p>
</div>
<figure id="S3.F4" class="ltx_figure"><img src="x4.png" id="S3.F4.g1" class="ltx_graphics" width="677" height="518" alt="Overview of the software system: low and high level control,
central database, remote control and monitoring via Internet. The
“PC” inscriptions show a possible setup; the computers on which the
given software run, although it is possible to run everything on a
single PC. The programming language is indicated by the backgrounds.
">
<figcaption class="ltx_caption"><span class="ltx_tag ltx_tag_figure">Figure 4: </span>Overview of the software system: low and high level control,
central database, remote control and monitoring via Internet. The
“PC” inscriptions show a possible setup; the computers on which the
given software run, although it is possible to run everything on a
single PC. The programming language is indicated by the backgrounds.
</figcaption>
</figure>
<section id="S3.SS1" class="ltx_subsection">
<h3 class="ltx_title ltx_title_subsection">
<span class="ltx_tag ltx_tag_subsection">3.1 </span>Low level programs</h3>

<div id="S3.SS1.p1" class="ltx_para">
<p id="S3.SS1.p1.1" class="ltx_p">The telescope mount’s kernel driver is mainly responsible for
positioning the mount. It is a character device driver, i.e., complex
commands can be issued to it as ascii strings, which are executed after
being parsed. Status information is readable from a special file,
showing relative positions in steps, RA and Dec, and so on. Both axes
can independently track (with any sidereal rate), fine-slew or slew
(maximal speed).</p>
</div>
<div id="S3.SS1.p2" class="ltx_para">
<p id="S3.SS1.p2.1" class="ltx_p">As mentioned in §<a href="#S2.SS1" title="2.1 The Robotic Mount ‣ 2 Hardware System ‣ System description and first light-curves of HAT, an autonomous observatory for variability search" class="ltx_ref"><span class="ltx_text ltx_ref_tag">2.1</span></a>, acceleration and deceleration of the
axes is done in finite angular-velocity, or in other words frequency
steps, so as to minimize slip of the friction drive. The “infinite”
accelerations between the finite velocity levels are in fact smoothed
to continuity by the elasticity of the components. The driver also
reads interrupts coming from the proximity sensors indicating the home
or end positions. It can “home” the mount (move marked points on the
horseshoe and declination discs precisely above the proximity sensors)
from any unknown position, which is handy for resetting the
continuously accumulating slips, and for orientation if the telescope
is “lost”. The home position’s hour angle and declination is
calibrated at the installation of the telescope. After nulling values
at the home position, the number of steps taken in any axis and any
direction is always stored, and the residuals after a subsequent homing
indicate the amount of slip. The tracking speed can be fine-tuned with
a precision of <math id="S3.SS1.p2.1.m1.1" class="ltx_Math" alttext="10^{-7}" display="inline"><semantics id="S3.SS1.p2.1.m1.1a"><msup id="S3.SS1.p2.1.m1.1.1" xref="S3.SS1.p2.1.m1.1.1.cmml"><mn id="S3.SS1.p2.1.m1.1.1.2" xref="S3.SS1.p2.1.m1.1.1.2.cmml">10</mn><mrow id="S3.SS1.p2.1.m1.1.1.3" xref="S3.SS1.p2.1.m1.1.1.3.cmml"><mo id="S3.SS1.p2.1.m1.1.1.3.1" xref="S3.SS1.p2.1.m1.1.1.3.1.cmml">-</mo><mn id="S3.SS1.p2.1.m1.1.1.3.2" xref="S3.SS1.p2.1.m1.1.1.3.2.cmml">7</mn></mrow></msup><annotation-xml encoding="MathML-Content" id="S3.SS1.p2.1.m1.1b"><apply id="S3.SS1.p2.1.m1.1.1.cmml" xref="S3.SS1.p2.1.m1.1.1"><csymbol cd="ambiguous" id="S3.SS1.p2.1.m1.1.1.1.cmml" xref="S3.SS1.p2.1.m1.1.1">superscript</csymbol><cn type="integer" id="S3.SS1.p2.1.m1.1.1.2.cmml" xref="S3.SS1.p2.1.m1.1.1.2">10</cn><apply id="S3.SS1.p2.1.m1.1.1.3.cmml" xref="S3.SS1.p2.1.m1.1.1.3"><minus id="S3.SS1.p2.1.m1.1.1.3.1.cmml" xref="S3.SS1.p2.1.m1.1.1.3.1"></minus><cn type="integer" id="S3.SS1.p2.1.m1.1.1.3.2.cmml" xref="S3.SS1.p2.1.m1.1.1.3.2">7</cn></apply></apply></annotation-xml><annotation encoding="application/x-tex" id="S3.SS1.p2.1.m1.1c">10^{-7}</annotation><annotation encoding="application/x-llamapun" id="S3.SS1.p2.1.m1.1d">10 start_POSTSUPERSCRIPT - 7 end_POSTSUPERSCRIPT</annotation></semantics></math>, which means that the hardware element’s
absolute dimensions (e.g., the ratio of the driving roller to the
diameter of the horseshoe) are not that restricted (but only important
in absolute pointing). Finally, RA and Dec values can be assigned to
any position.</p>
</div>
<div id="S3.SS1.p3" class="ltx_para">
<p id="S3.SS1.p3.1" class="ltx_p">The dome driver is a similar RTLinux kernel driver responsible for i)
turning the main power on/off, ii) closing or opening the dome, iii)
activating lens heating, iv) control of domeflat lights, v) detection
of rain. Status of the dome can be queried from the driver.</p>
</div>
<div id="S3.SS1.p4" class="ltx_para">
<p id="S3.SS1.p4.1" class="ltx_p">CCD acquisition software of the Apogee AP10 CCD is based on the
application programming interface from <cite class="ltx_cite ltx_citemacro_citet">Pojmański (<a href="#bib.bib42" title="" class="ltx_ref">2002</a>)</cite>, modified to
fulfill our needs. The camera server can run as a standalone
application, or as a “daemon”, accepting commands from other software
via Internet Protocol (TCP/IP).</p>
</div>
<div id="S3.SS1.p5" class="ltx_para">
<p id="S3.SS1.p5.1" class="ltx_p">The so-called HAT Access Module (HAM) is the main low-level program
joining all the aforementioned codes. It accepts input via TCP/IP from
higher level software (e.g., the virtual Observer), parses the
commands, and distributes them to the relevant resources. That is, all
telescope, dome and CCD operations are piped through this program.
Another aspect of HAM is allocation of the drivers, i.e. no other
program can issue conflicting commands to the hardware.</p>
</div>
</section>
<section id="S3.SS2" class="ltx_subsection">
<h3 class="ltx_title ltx_title_subsection">
<span class="ltx_tag ltx_tag_subsection">3.2 </span>High level programs</h3>

<div id="S3.SS2.p1" class="ltx_para">
<p id="S3.SS2.p1.1" class="ltx_p">We use the fast, reliable and open source relational Structured Query
Language (MySQL<span id="footnote6" class="ltx_note ltx_role_footnote"><sup class="ltx_note_mark">6</sup><span class="ltx_note_outer"><span class="ltx_note_content"><sup class="ltx_note_mark">6</sup><span class="ltx_tag ltx_tag_note">6</span>Information on MySQL is available at
www.mysql.org</span></span></span>) database for storing associated parameters of the
station (setup of mount, telescope, CCD, scheduled tasks, etc.), and
for keeping continuously accumulating logs of operations (parameters of
archived images, etc.), all of them arranged into tables. This has
several advantages over simply keeping information in files without a
database wrapper: i) faster operation on data, ii) TCP/IP access (for
instance through the Internet), iii) use of existing interfaces to
MySQL from various programming languages. The smooth TCP/IP
communication enables installation of multiple HAT observatories in any
preferred topology, with central database management, and with the
possibility that the sites interact with each other, for example,
request the last object observed or instruct another telescope.</p>
</div>
<div id="S3.SS2.p2" class="ltx_para">
<p id="S3.SS2.p2.1" class="ltx_p">The observatory as a whole is managed by the <span id="S3.SS2.p2.1.1" class="ltx_text ltx_font_italic">virtual</span> Observer,
which communicates with the access module, central database and
executes (observer) programs. This is a Tcl<span id="footnote7" class="ltx_note ltx_role_footnote"><sup class="ltx_note_mark">7</sup><span class="ltx_note_outer"><span class="ltx_note_content"><sup class="ltx_note_mark">7</sup><span class="ltx_tag ltx_tag_note">7</span>Information on Tcl
is available at www.scriptics.com</span></span></span> language-based simulation of
near-perfect observer, modeled as a finite-state machine, which makes
decisions/transitions depending on the circumstances.</p>
</div>
<div id="S3.SS2.p3" class="ltx_para">
<p id="S3.SS2.p3.1" class="ltx_p">The Observer programs (such as taking flatfield frames and all-sky
monitoring) are run as independent threads, so the Observer never hangs
by waiting for a time-consuming operation to finish; it can be aborted
any time. Tasks have start times (relative to sunset), allowed
durations and priorities – both taken from the DB, whose properties
are used by a task-manager (part of the Observer) to launch or stop
them and to suspend or interrupt other tasks. There is a template,
which makes writing new tasks very simple for the user, using a command
library of a few hundred commands, such as “ObserveObject”.</p>
</div>
<div id="S3.SS2.p4" class="ltx_para">
<p id="S3.SS2.p4.1" class="ltx_p">During normal operation, the Observer is in “run” state. If the
weather is unfavorable, the Observer passes to a “weather sleep”
state, where the CCD is kept cooled, the dome is closed, and the system
waits for clearing. In the daytime the system is in “daysleep”, power
is turned off, CCD is warmed back, and we are waiting for the first
scheduled task to appear within the characteristic time needed for
starting up. If the Internet connection with the monitoring stations is
lost, system jumps to “suspend” state, turns off power, and waits a
long time before finally exiting to “service” state. Recovery from
service is possible only by manually restarting the system.</p>
</div>
<div id="S3.SS2.p5" class="ltx_para">
<p id="S3.SS2.p5.1" class="ltx_p">Upon entering the “run” state, Observer checks the tasks to be
executed during the given session, checks weather information, and if
needed, starts up the system (turns power on, starts cooling the CCD).
Following this, it enters an endless cycle, and breaks out only in case
of transition to another state. During the cycle it responds to direct
TCP/IP messages from outside, and checks if connection with the access
module and outside-connection monitor is alive. The ephemerides are
continuously updated, such as apparent position of the Sun and Moon.
The task manager launches, checks or stops tasks, administrates changes
in the DB. If a task exits abnormally, perhaps due to some kind of
exception (programming error), it is flagged, and not launched again
any more. Proper scheduling of tasks is one of most complex part in the
code.</p>
</div>
<div id="S3.SS2.p6" class="ltx_para">
<p id="S3.SS2.p6.1" class="ltx_p">The weather status is checked every <math id="S3.SS2.p6.1.m1.1" class="ltx_Math" alttext="\rm\sim 10sec" display="inline"><semantics id="S3.SS2.p6.1.m1.1a"><mrow id="S3.SS2.p6.1.m1.1.1" xref="S3.SS2.p6.1.m1.1.1.cmml"><mi id="S3.SS2.p6.1.m1.1.1.2" xref="S3.SS2.p6.1.m1.1.1.2.cmml"></mi><mo id="S3.SS2.p6.1.m1.1.1.1" xref="S3.SS2.p6.1.m1.1.1.1.cmml">∼</mo><mrow id="S3.SS2.p6.1.m1.1.1.3" xref="S3.SS2.p6.1.m1.1.1.3.cmml"><mn id="S3.SS2.p6.1.m1.1.1.3.2" xref="S3.SS2.p6.1.m1.1.1.3.2.cmml">10</mn><mo id="S3.SS2.p6.1.m1.1.1.3.1" xref="S3.SS2.p6.1.m1.1.1.3.1.cmml">⁢</mo><mi mathvariant="normal" id="S3.SS2.p6.1.m1.1.1.3.3" xref="S3.SS2.p6.1.m1.1.1.3.3.cmml">s</mi><mo id="S3.SS2.p6.1.m1.1.1.3.1a" xref="S3.SS2.p6.1.m1.1.1.3.1.cmml">⁢</mo><mi mathvariant="normal" id="S3.SS2.p6.1.m1.1.1.3.4" xref="S3.SS2.p6.1.m1.1.1.3.4.cmml">e</mi><mo id="S3.SS2.p6.1.m1.1.1.3.1b" xref="S3.SS2.p6.1.m1.1.1.3.1.cmml">⁢</mo><mi mathvariant="normal" id="S3.SS2.p6.1.m1.1.1.3.5" xref="S3.SS2.p6.1.m1.1.1.3.5.cmml">c</mi></mrow></mrow><annotation-xml encoding="MathML-Content" id="S3.SS2.p6.1.m1.1b"><apply id="S3.SS2.p6.1.m1.1.1.cmml" xref="S3.SS2.p6.1.m1.1.1"><csymbol cd="latexml" id="S3.SS2.p6.1.m1.1.1.1.cmml" xref="S3.SS2.p6.1.m1.1.1.1">similar-to</csymbol><csymbol cd="latexml" id="S3.SS2.p6.1.m1.1.1.2.cmml" xref="S3.SS2.p6.1.m1.1.1.2">absent</csymbol><apply id="S3.SS2.p6.1.m1.1.1.3.cmml" xref="S3.SS2.p6.1.m1.1.1.3"><times id="S3.SS2.p6.1.m1.1.1.3.1.cmml" xref="S3.SS2.p6.1.m1.1.1.3.1"></times><cn type="integer" id="S3.SS2.p6.1.m1.1.1.3.2.cmml" xref="S3.SS2.p6.1.m1.1.1.3.2">10</cn><ci id="S3.SS2.p6.1.m1.1.1.3.3.cmml" xref="S3.SS2.p6.1.m1.1.1.3.3">s</ci><ci id="S3.SS2.p6.1.m1.1.1.3.4.cmml" xref="S3.SS2.p6.1.m1.1.1.3.4">e</ci><ci id="S3.SS2.p6.1.m1.1.1.3.5.cmml" xref="S3.SS2.p6.1.m1.1.1.3.5">c</ci></apply></apply></annotation-xml><annotation encoding="application/x-tex" id="S3.SS2.p6.1.m1.1c">\rm\sim 10sec</annotation><annotation encoding="application/x-llamapun" id="S3.SS2.p6.1.m1.1d">∼ 10 roman_s roman_e roman_c</annotation></semantics></math>, and if
needed, the current task is aborted, and the observatory transits to
another state. There is a possibility of looking for targets of
opportunity by parsing remote TCP/IP packets, such as the GRB
Coordinates Network (GCN)<span id="footnote8" class="ltx_note ltx_role_footnote"><sup class="ltx_note_mark">8</sup><span class="ltx_note_outer"><span class="ltx_note_content"><sup class="ltx_note_mark">8</sup><span class="ltx_tag ltx_tag_note">8</span>http://gcn.gsfc.nasa.gov/</span></span></span>. The
Observer checks and can take action upon incoming email communication,
which enables the scientist to control or monitor the observatory even
from a cellphone.</p>
</div>
<div id="S3.SS2.p7" class="ltx_para">
<p id="S3.SS2.p7.1" class="ltx_p">Eventually, with a given time resolution (<math id="S3.SS2.p7.1.m1.1" class="ltx_Math" alttext="\rm\sim 30min" display="inline"><semantics id="S3.SS2.p7.1.m1.1a"><mrow id="S3.SS2.p7.1.m1.1.1" xref="S3.SS2.p7.1.m1.1.1.cmml"><mi id="S3.SS2.p7.1.m1.1.1.2" xref="S3.SS2.p7.1.m1.1.1.2.cmml"></mi><mo id="S3.SS2.p7.1.m1.1.1.1" xref="S3.SS2.p7.1.m1.1.1.1.cmml">∼</mo><mrow id="S3.SS2.p7.1.m1.1.1.3" xref="S3.SS2.p7.1.m1.1.1.3.cmml"><mn id="S3.SS2.p7.1.m1.1.1.3.2" xref="S3.SS2.p7.1.m1.1.1.3.2.cmml">30</mn><mo id="S3.SS2.p7.1.m1.1.1.3.1" xref="S3.SS2.p7.1.m1.1.1.3.1.cmml">⁢</mo><mi mathvariant="normal" id="S3.SS2.p7.1.m1.1.1.3.3" xref="S3.SS2.p7.1.m1.1.1.3.3.cmml">m</mi><mo id="S3.SS2.p7.1.m1.1.1.3.1a" xref="S3.SS2.p7.1.m1.1.1.3.1.cmml">⁢</mo><mi mathvariant="normal" id="S3.SS2.p7.1.m1.1.1.3.4" xref="S3.SS2.p7.1.m1.1.1.3.4.cmml">i</mi><mo id="S3.SS2.p7.1.m1.1.1.3.1b" xref="S3.SS2.p7.1.m1.1.1.3.1.cmml">⁢</mo><mi mathvariant="normal" id="S3.SS2.p7.1.m1.1.1.3.5" xref="S3.SS2.p7.1.m1.1.1.3.5.cmml">n</mi></mrow></mrow><annotation-xml encoding="MathML-Content" id="S3.SS2.p7.1.m1.1b"><apply id="S3.SS2.p7.1.m1.1.1.cmml" xref="S3.SS2.p7.1.m1.1.1"><csymbol cd="latexml" id="S3.SS2.p7.1.m1.1.1.1.cmml" xref="S3.SS2.p7.1.m1.1.1.1">similar-to</csymbol><csymbol cd="latexml" id="S3.SS2.p7.1.m1.1.1.2.cmml" xref="S3.SS2.p7.1.m1.1.1.2">absent</csymbol><apply id="S3.SS2.p7.1.m1.1.1.3.cmml" xref="S3.SS2.p7.1.m1.1.1.3"><times id="S3.SS2.p7.1.m1.1.1.3.1.cmml" xref="S3.SS2.p7.1.m1.1.1.3.1"></times><cn type="integer" id="S3.SS2.p7.1.m1.1.1.3.2.cmml" xref="S3.SS2.p7.1.m1.1.1.3.2">30</cn><ci id="S3.SS2.p7.1.m1.1.1.3.3.cmml" xref="S3.SS2.p7.1.m1.1.1.3.3">m</ci><ci id="S3.SS2.p7.1.m1.1.1.3.4.cmml" xref="S3.SS2.p7.1.m1.1.1.3.4">i</ci><ci id="S3.SS2.p7.1.m1.1.1.3.5.cmml" xref="S3.SS2.p7.1.m1.1.1.3.5">n</ci></apply></apply></annotation-xml><annotation encoding="application/x-tex" id="S3.SS2.p7.1.m1.1c">\rm\sim 30min</annotation><annotation encoding="application/x-llamapun" id="S3.SS2.p7.1.m1.1d">∼ 30 roman_m roman_i roman_n</annotation></semantics></math>),
ConCam<span id="footnote9" class="ltx_note ltx_role_footnote"><sup class="ltx_note_mark">9</sup><span class="ltx_note_outer"><span class="ltx_note_content"><sup class="ltx_note_mark">9</sup><span class="ltx_tag ltx_tag_note">9</span>For more information: www.concam.net</span></span></span> <cite class="ltx_cite ltx_citemacro_citep">(Pereira et al., <a href="#bib.bib35" title="" class="ltx_ref">2000</a>)</cite>
all-sky night-time images and infrared satellite images<span id="footnote10" class="ltx_note ltx_role_footnote"><sup class="ltx_note_mark">10</sup><span class="ltx_note_outer"><span class="ltx_note_content"><sup class="ltx_note_mark">10</sup><span class="ltx_tag ltx_tag_note">10</span>
National Weather Service – www.wrh.noaa.gov</span></span></span>
are downloaded and stored. These proved to be very useful later in
flagging the photometric quality of a night. The Observer also has the
ability to recover from various emergency situations (e.g., disk is
full, device missing), and notify recipients via email.</p>
</div>
<div id="S3.SS2.p8" class="ltx_para">
<p id="S3.SS2.p8.1" class="ltx_p">Automatic start up of HAT is included in the booting procedure of the
computer, similarly, closing the observatory in the shutdown phase.
This way the service staff can easily start up/bring down the system.
Also, HAT operations resume as the power outage ends.</p>
</div>
</section>
</section>
<section id="S4" class="ltx_section">
<h2 class="ltx_title ltx_title_section">
<span class="ltx_tag ltx_tag_section">4 </span>Pointing precision of HAT</h2>

<div id="S4.p1" class="ltx_para">
<p id="S4.p1.1" class="ltx_p">While highly accurate pointing of a telescope covering a 9 degree field
might appear to be merely a luxury, we tried to have a general purpose
design with the flexibility of upgrading the focal length, thus
decreasing the field of view, which requires more precise pointing.
Moreover, sufficient pointing capability of even a wide field
instrument can be relevant to its photometric precision. There are
indications that precise aperture photometry depends on positioning the
same star over the same pixels of the CCD, due to inter and intra-pixel
variability <cite class="ltx_cite ltx_citemacro_citep">(Buffington, Booth, &amp; Hudson, <a href="#bib.bib11" title="" class="ltx_ref">1991</a>; Robinson et al., <a href="#bib.bib45" title="" class="ltx_ref">1995</a>)</cite>, although minute
displacements of the telescope also do have advantages in filtering out
internal reflections due to the fast focal ratio of the instrument. Due
to our open-loop system, deviation of absolute pointing accumulates
proportionally to the individual pointing accuracy, which has to be
minimized.</p>
</div>
<div id="S4.p2" class="ltx_para">
<p id="S4.p2.2" class="ltx_p">Astrometric calibration was carried out at Konkoly Observatory,
Budapest during the summer of 2001. We used a replica (HAT-2) of the
mount at Kitt Peak, the Meade Pictor 416xt CCD, and a Soviet “MTO”
<math id="S4.p2.1.m1.1" class="ltx_Math" alttext="\rm 100mm\oslash" display="inline"><semantics id="S4.p2.1.m1.1a"><mrow id="S4.p2.1.m1.1b"><mn id="S4.p2.1.m1.1.1" xref="S4.p2.1.m1.1.1.cmml">100</mn><mi mathvariant="normal" id="S4.p2.1.m1.1.2" xref="S4.p2.1.m1.1.2.cmml">m</mi><mi mathvariant="normal" id="S4.p2.1.m1.1.3" xref="S4.p2.1.m1.1.3.cmml">m</mi><mo id="S4.p2.1.m1.1.4" xref="S4.p2.1.m1.1.4.cmml">⊘</mo></mrow><annotation-xml encoding="MathML-Content" id="S4.p2.1.m1.1c"><cerror id="S4.p2.1.m1.1d"><csymbol cd="ambiguous" id="S4.p2.1.m1.1e">fragments</csymbol><cn type="integer" id="S4.p2.1.m1.1.1.cmml" xref="S4.p2.1.m1.1.1">100</cn><csymbol cd="unknown" id="S4.p2.1.m1.1.2.cmml" xref="S4.p2.1.m1.1.2">m</csymbol><csymbol cd="unknown" id="S4.p2.1.m1.1.3.cmml" xref="S4.p2.1.m1.1.3">m</csymbol><ci id="S4.p2.1.m1.1.4.cmml" xref="S4.p2.1.m1.1.4">⊘</ci></cerror></annotation-xml><annotation encoding="application/x-tex" id="S4.p2.1.m1.1f">\rm 100mm\oslash</annotation><annotation encoding="application/x-llamapun" id="S4.p2.1.m1.1g">100 roman_m roman_m ⊘</annotation></semantics></math>, f/10 Maksutov telephoto lens<span id="footnote11" class="ltx_note ltx_role_footnote"><sup class="ltx_note_mark">11</sup><span class="ltx_note_outer"><span class="ltx_note_content"><sup class="ltx_note_mark">11</sup><span class="ltx_tag ltx_tag_note">11</span>Special
thanks to A. Holl for lending us the lens.</span></span></span>. The 1m focal length and
the fine, <math id="S4.p2.2.m2.1" class="ltx_Math" alttext="\rm 1.86\arcsec/pixel" display="inline"><semantics id="S4.p2.2.m2.1a"><mrow id="S4.p2.2.m2.1.1" xref="S4.p2.2.m2.1.1.cmml"><mrow id="S4.p2.2.m2.1.1.2" xref="S4.p2.2.m2.1.1.2.cmml"><mn id="S4.p2.2.m2.1.1.2.2" xref="S4.p2.2.m2.1.1.2.2.cmml">1.86</mn><mo id="S4.p2.2.m2.1.1.2.1" xref="S4.p2.2.m2.1.1.2.1.cmml">⁢</mo><mi mathvariant="normal" id="S4.p2.2.m2.1.1.2.3" xref="S4.p2.2.m2.1.1.2.3.cmml">″</mi></mrow><mo id="S4.p2.2.m2.1.1.1" xref="S4.p2.2.m2.1.1.1.cmml">/</mo><mi id="S4.p2.2.m2.1.1.3" xref="S4.p2.2.m2.1.1.3.cmml">pixel</mi></mrow><annotation-xml encoding="MathML-Content" id="S4.p2.2.m2.1b"><apply id="S4.p2.2.m2.1.1.cmml" xref="S4.p2.2.m2.1.1"><divide id="S4.p2.2.m2.1.1.1.cmml" xref="S4.p2.2.m2.1.1.1"></divide><apply id="S4.p2.2.m2.1.1.2.cmml" xref="S4.p2.2.m2.1.1.2"><times id="S4.p2.2.m2.1.1.2.1.cmml" xref="S4.p2.2.m2.1.1.2.1"></times><cn type="float" id="S4.p2.2.m2.1.1.2.2.cmml" xref="S4.p2.2.m2.1.1.2.2">1.86</cn><ci id="S4.p2.2.m2.1.1.2.3.cmml" xref="S4.p2.2.m2.1.1.2.3">″</ci></apply><ci id="S4.p2.2.m2.1.1.3.cmml" xref="S4.p2.2.m2.1.1.3">pixel</ci></apply></annotation-xml><annotation encoding="application/x-tex" id="S4.p2.2.m2.1c">\rm 1.86\arcsec/pixel</annotation><annotation encoding="application/x-llamapun" id="S4.p2.2.m2.1d">1.86 ″ / roman_pixel</annotation></semantics></math> resolution were useful for
calibration, which used the starry sky as reference grid. Whenever
needed, we made approximate correction for atmospheric refraction, and
used Guide Star Catalogue <cite class="ltx_cite ltx_citemacro_citep">(GSC; Lasker et al., <a href="#bib.bib25" title="" class="ltx_ref">1990</a>)</cite> for finding the
astrometric solution of the frames, and coordinates of the central
pixel.</p>
</div>
<div id="S4.p3" class="ltx_para">
<p id="S4.p3.1" class="ltx_p">We distinguish between the following attributes of pointing: <span id="S4.p3.1.1" class="ltx_text ltx_font_italic">homing, tracking, absolute-pointing</span> precisions and <span id="S4.p3.1.2" class="ltx_text ltx_font_italic">pointing
repeatability</span>. In general, depending on the kind of precision, crucial
factors can be the balance of the axes, ramping-up distance, maximal
slewing speed, surface conditions and shape of the horseshoe, polar
setting and orthogonality of the axes (RA, Dec, optical axis).</p>
</div>
<div id="S4.p4" class="ltx_para">
<p id="S4.p4.2" class="ltx_p">Homing precision measures how precisely can we set the mount to the
home-position defined by the proximity sensors. The scatter of
individual homings turned out to be <math id="S4.p4.1.m1.1" class="ltx_Math" alttext="\rm\sigma_{RA}=0.5sec" display="inline"><semantics id="S4.p4.1.m1.1a"><mrow id="S4.p4.1.m1.1.1" xref="S4.p4.1.m1.1.1.cmml"><msub id="S4.p4.1.m1.1.1.2" xref="S4.p4.1.m1.1.1.2.cmml"><mi id="S4.p4.1.m1.1.1.2.2" xref="S4.p4.1.m1.1.1.2.2.cmml">σ</mi><mi id="S4.p4.1.m1.1.1.2.3" xref="S4.p4.1.m1.1.1.2.3.cmml">RA</mi></msub><mo id="S4.p4.1.m1.1.1.1" xref="S4.p4.1.m1.1.1.1.cmml">=</mo><mrow id="S4.p4.1.m1.1.1.3" xref="S4.p4.1.m1.1.1.3.cmml"><mn id="S4.p4.1.m1.1.1.3.2" xref="S4.p4.1.m1.1.1.3.2.cmml">0.5</mn><mo id="S4.p4.1.m1.1.1.3.1" xref="S4.p4.1.m1.1.1.3.1.cmml">⁢</mo><mi id="S4.p4.1.m1.1.1.3.3" xref="S4.p4.1.m1.1.1.3.3.cmml">sec</mi></mrow></mrow><annotation-xml encoding="MathML-Content" id="S4.p4.1.m1.1b"><apply id="S4.p4.1.m1.1.1.cmml" xref="S4.p4.1.m1.1.1"><eq id="S4.p4.1.m1.1.1.1.cmml" xref="S4.p4.1.m1.1.1.1"></eq><apply id="S4.p4.1.m1.1.1.2.cmml" xref="S4.p4.1.m1.1.1.2"><csymbol cd="ambiguous" id="S4.p4.1.m1.1.1.2.1.cmml" xref="S4.p4.1.m1.1.1.2">subscript</csymbol><ci id="S4.p4.1.m1.1.1.2.2.cmml" xref="S4.p4.1.m1.1.1.2.2">𝜎</ci><ci id="S4.p4.1.m1.1.1.2.3.cmml" xref="S4.p4.1.m1.1.1.2.3">RA</ci></apply><apply id="S4.p4.1.m1.1.1.3.cmml" xref="S4.p4.1.m1.1.1.3"><times id="S4.p4.1.m1.1.1.3.1.cmml" xref="S4.p4.1.m1.1.1.3.1"></times><cn type="float" id="S4.p4.1.m1.1.1.3.2.cmml" xref="S4.p4.1.m1.1.1.3.2">0.5</cn><ci id="S4.p4.1.m1.1.1.3.3.cmml" xref="S4.p4.1.m1.1.1.3.3">sec</ci></apply></apply></annotation-xml><annotation encoding="application/x-tex" id="S4.p4.1.m1.1c">\rm\sigma_{RA}=0.5sec</annotation><annotation encoding="application/x-llamapun" id="S4.p4.1.m1.1d">italic_σ start_POSTSUBSCRIPT roman_RA end_POSTSUBSCRIPT = 0.5 roman_sec</annotation></semantics></math>, <math id="S4.p4.2.m2.1" class="ltx_Math" alttext="\rm\sigma_{Dec}=25\arcsec" display="inline"><semantics id="S4.p4.2.m2.1a"><mrow id="S4.p4.2.m2.1.1" xref="S4.p4.2.m2.1.1.cmml"><msub id="S4.p4.2.m2.1.1.2" xref="S4.p4.2.m2.1.1.2.cmml"><mi id="S4.p4.2.m2.1.1.2.2" xref="S4.p4.2.m2.1.1.2.2.cmml">σ</mi><mi id="S4.p4.2.m2.1.1.2.3" xref="S4.p4.2.m2.1.1.2.3.cmml">Dec</mi></msub><mo id="S4.p4.2.m2.1.1.1" xref="S4.p4.2.m2.1.1.1.cmml">=</mo><mrow id="S4.p4.2.m2.1.1.3" xref="S4.p4.2.m2.1.1.3.cmml"><mn id="S4.p4.2.m2.1.1.3.2" xref="S4.p4.2.m2.1.1.3.2.cmml">25</mn><mo id="S4.p4.2.m2.1.1.3.1" xref="S4.p4.2.m2.1.1.3.1.cmml">⁢</mo><mi mathvariant="normal" id="S4.p4.2.m2.1.1.3.3" xref="S4.p4.2.m2.1.1.3.3.cmml">″</mi></mrow></mrow><annotation-xml encoding="MathML-Content" id="S4.p4.2.m2.1b"><apply id="S4.p4.2.m2.1.1.cmml" xref="S4.p4.2.m2.1.1"><eq id="S4.p4.2.m2.1.1.1.cmml" xref="S4.p4.2.m2.1.1.1"></eq><apply id="S4.p4.2.m2.1.1.2.cmml" xref="S4.p4.2.m2.1.1.2"><csymbol cd="ambiguous" id="S4.p4.2.m2.1.1.2.1.cmml" xref="S4.p4.2.m2.1.1.2">subscript</csymbol><ci id="S4.p4.2.m2.1.1.2.2.cmml" xref="S4.p4.2.m2.1.1.2.2">𝜎</ci><ci id="S4.p4.2.m2.1.1.2.3.cmml" xref="S4.p4.2.m2.1.1.2.3">Dec</ci></apply><apply id="S4.p4.2.m2.1.1.3.cmml" xref="S4.p4.2.m2.1.1.3"><times id="S4.p4.2.m2.1.1.3.1.cmml" xref="S4.p4.2.m2.1.1.3.1"></times><cn type="integer" id="S4.p4.2.m2.1.1.3.2.cmml" xref="S4.p4.2.m2.1.1.3.2">25</cn><ci id="S4.p4.2.m2.1.1.3.3.cmml" xref="S4.p4.2.m2.1.1.3.3">″</ci></apply></apply></annotation-xml><annotation encoding="application/x-tex" id="S4.p4.2.m2.1c">\rm\sigma_{Dec}=25\arcsec</annotation><annotation encoding="application/x-llamapun" id="S4.p4.2.m2.1d">italic_σ start_POSTSUBSCRIPT roman_Dec end_POSTSUBSCRIPT = 25 ″</annotation></semantics></math>.</p>
</div>
<div id="S4.p5" class="ltx_para">
<p id="S4.p5.2" class="ltx_p">The tracking of the system depends on adjustment of the polar axis,
tracking speed (set by the mount driver), and the effect of
environmental changes (e.g., strong wind) on the axes. The projected
position of the mount’s polar axis on the sky was determined from the
arcs of stars from exposures with <math id="S4.p5.1.m1.1" class="ltx_Math" alttext="\approx 20\times" display="inline"><semantics id="S4.p5.1.m1.1a"><mrow id="S4.p5.1.m1.1b"><mo id="S4.p5.1.m1.1.1" xref="S4.p5.1.m1.1.1.cmml">≈</mo><mn id="S4.p5.1.m1.1.2" xref="S4.p5.1.m1.1.2.cmml">20</mn><mo id="S4.p5.1.m1.1.3" xref="S4.p5.1.m1.1.3.cmml">×</mo></mrow><annotation-xml encoding="MathML-Content" id="S4.p5.1.m1.1c"><cerror id="S4.p5.1.m1.1d"><csymbol cd="ambiguous" id="S4.p5.1.m1.1e">fragments</csymbol><approx id="S4.p5.1.m1.1.1.cmml" xref="S4.p5.1.m1.1.1"></approx><cn type="integer" id="S4.p5.1.m1.1.2.cmml" xref="S4.p5.1.m1.1.2">20</cn><times id="S4.p5.1.m1.1.3.cmml" xref="S4.p5.1.m1.1.3"></times></cerror></annotation-xml><annotation encoding="application/x-tex" id="S4.p5.1.m1.1f">\approx 20\times</annotation><annotation encoding="application/x-llamapun" id="S4.p5.1.m1.1g">≈ 20 ×</annotation></semantics></math> sidereal tracking
around the pole, during which the diurnal motion of the stars was
negligible. As actual position of the celestial pole is known from
astrometry databases, we could adjust the polar axis with <math id="S4.p5.2.m2.1" class="ltx_Math" alttext="\rm\pm 20pix\approx 0.5\arcmin" display="inline"><semantics id="S4.p5.2.m2.1a"><mrow id="S4.p5.2.m2.1.1" xref="S4.p5.2.m2.1.1.cmml"><mrow id="S4.p5.2.m2.1.1.2" xref="S4.p5.2.m2.1.1.2.cmml"><mo id="S4.p5.2.m2.1.1.2.1" xref="S4.p5.2.m2.1.1.2.1.cmml">±</mo><mrow id="S4.p5.2.m2.1.1.2.2" xref="S4.p5.2.m2.1.1.2.2.cmml"><mn id="S4.p5.2.m2.1.1.2.2.2" xref="S4.p5.2.m2.1.1.2.2.2.cmml">20</mn><mo id="S4.p5.2.m2.1.1.2.2.1" xref="S4.p5.2.m2.1.1.2.2.1.cmml">⁢</mo><mi mathvariant="normal" id="S4.p5.2.m2.1.1.2.2.3" xref="S4.p5.2.m2.1.1.2.2.3.cmml">p</mi><mo id="S4.p5.2.m2.1.1.2.2.1a" xref="S4.p5.2.m2.1.1.2.2.1.cmml">⁢</mo><mi mathvariant="normal" id="S4.p5.2.m2.1.1.2.2.4" xref="S4.p5.2.m2.1.1.2.2.4.cmml">i</mi><mo id="S4.p5.2.m2.1.1.2.2.1b" xref="S4.p5.2.m2.1.1.2.2.1.cmml">⁢</mo><mi mathvariant="normal" id="S4.p5.2.m2.1.1.2.2.5" xref="S4.p5.2.m2.1.1.2.2.5.cmml">x</mi></mrow></mrow><mo id="S4.p5.2.m2.1.1.1" xref="S4.p5.2.m2.1.1.1.cmml">≈</mo><mrow id="S4.p5.2.m2.1.1.3" xref="S4.p5.2.m2.1.1.3.cmml"><mn id="S4.p5.2.m2.1.1.3.2" xref="S4.p5.2.m2.1.1.3.2.cmml">0.5</mn><mo id="S4.p5.2.m2.1.1.3.1" xref="S4.p5.2.m2.1.1.3.1.cmml">⁢</mo><mi mathvariant="normal" id="S4.p5.2.m2.1.1.3.3" xref="S4.p5.2.m2.1.1.3.3.cmml">′</mi></mrow></mrow><annotation-xml encoding="MathML-Content" id="S4.p5.2.m2.1b"><apply id="S4.p5.2.m2.1.1.cmml" xref="S4.p5.2.m2.1.1"><approx id="S4.p5.2.m2.1.1.1.cmml" xref="S4.p5.2.m2.1.1.1"></approx><apply id="S4.p5.2.m2.1.1.2.cmml" xref="S4.p5.2.m2.1.1.2"><csymbol cd="latexml" id="S4.p5.2.m2.1.1.2.1.cmml" xref="S4.p5.2.m2.1.1.2.1">plus-or-minus</csymbol><apply id="S4.p5.2.m2.1.1.2.2.cmml" xref="S4.p5.2.m2.1.1.2.2"><times id="S4.p5.2.m2.1.1.2.2.1.cmml" xref="S4.p5.2.m2.1.1.2.2.1"></times><cn type="integer" id="S4.p5.2.m2.1.1.2.2.2.cmml" xref="S4.p5.2.m2.1.1.2.2.2">20</cn><ci id="S4.p5.2.m2.1.1.2.2.3.cmml" xref="S4.p5.2.m2.1.1.2.2.3">p</ci><ci id="S4.p5.2.m2.1.1.2.2.4.cmml" xref="S4.p5.2.m2.1.1.2.2.4">i</ci><ci id="S4.p5.2.m2.1.1.2.2.5.cmml" xref="S4.p5.2.m2.1.1.2.2.5">x</ci></apply></apply><apply id="S4.p5.2.m2.1.1.3.cmml" xref="S4.p5.2.m2.1.1.3"><times id="S4.p5.2.m2.1.1.3.1.cmml" xref="S4.p5.2.m2.1.1.3.1"></times><cn type="float" id="S4.p5.2.m2.1.1.3.2.cmml" xref="S4.p5.2.m2.1.1.3.2">0.5</cn><ci id="S4.p5.2.m2.1.1.3.3.cmml" xref="S4.p5.2.m2.1.1.3.3">′</ci></apply></apply></annotation-xml><annotation encoding="application/x-tex" id="S4.p5.2.m2.1c">\rm\pm 20pix\approx 0.5\arcmin</annotation><annotation encoding="application/x-llamapun" id="S4.p5.2.m2.1d">± 20 roman_p roman_i roman_x ≈ 0.5 ′</annotation></semantics></math> precision.</p>
</div>
<div id="S4.p6" class="ltx_para">
<p id="S4.p6.5" class="ltx_p">The tracking was measured in calm weather conditions on fields
culminating near zenith, from hour angle <math id="S4.p6.1.m1.1" class="ltx_Math" alttext="\rm HA\approx-1h" display="inline"><semantics id="S4.p6.1.m1.1a"><mrow id="S4.p6.1.m1.1.1" xref="S4.p6.1.m1.1.1.cmml"><mi id="S4.p6.1.m1.1.1.2" xref="S4.p6.1.m1.1.1.2.cmml">HA</mi><mo id="S4.p6.1.m1.1.1.1" xref="S4.p6.1.m1.1.1.1.cmml">≈</mo><mrow id="S4.p6.1.m1.1.1.3" xref="S4.p6.1.m1.1.1.3.cmml"><mo id="S4.p6.1.m1.1.1.3.1" xref="S4.p6.1.m1.1.1.3.1.cmml">-</mo><mrow id="S4.p6.1.m1.1.1.3.2" xref="S4.p6.1.m1.1.1.3.2.cmml"><mn id="S4.p6.1.m1.1.1.3.2.2" xref="S4.p6.1.m1.1.1.3.2.2.cmml">1</mn><mo id="S4.p6.1.m1.1.1.3.2.1" xref="S4.p6.1.m1.1.1.3.2.1.cmml">⁢</mo><mi mathvariant="normal" id="S4.p6.1.m1.1.1.3.2.3" xref="S4.p6.1.m1.1.1.3.2.3.cmml">h</mi></mrow></mrow></mrow><annotation-xml encoding="MathML-Content" id="S4.p6.1.m1.1b"><apply id="S4.p6.1.m1.1.1.cmml" xref="S4.p6.1.m1.1.1"><approx id="S4.p6.1.m1.1.1.1.cmml" xref="S4.p6.1.m1.1.1.1"></approx><ci id="S4.p6.1.m1.1.1.2.cmml" xref="S4.p6.1.m1.1.1.2">HA</ci><apply id="S4.p6.1.m1.1.1.3.cmml" xref="S4.p6.1.m1.1.1.3"><minus id="S4.p6.1.m1.1.1.3.1.cmml" xref="S4.p6.1.m1.1.1.3.1"></minus><apply id="S4.p6.1.m1.1.1.3.2.cmml" xref="S4.p6.1.m1.1.1.3.2"><times id="S4.p6.1.m1.1.1.3.2.1.cmml" xref="S4.p6.1.m1.1.1.3.2.1"></times><cn type="integer" id="S4.p6.1.m1.1.1.3.2.2.cmml" xref="S4.p6.1.m1.1.1.3.2.2">1</cn><ci id="S4.p6.1.m1.1.1.3.2.3.cmml" xref="S4.p6.1.m1.1.1.3.2.3">h</ci></apply></apply></apply></annotation-xml><annotation encoding="application/x-tex" id="S4.p6.1.m1.1c">\rm HA\approx-1h</annotation><annotation encoding="application/x-llamapun" id="S4.p6.1.m1.1d">roman_HA ≈ - 1 roman_h</annotation></semantics></math> to <math id="S4.p6.2.m2.1" class="ltx_Math" alttext="\rm HA\approx 1h" display="inline"><semantics id="S4.p6.2.m2.1a"><mrow id="S4.p6.2.m2.1.1" xref="S4.p6.2.m2.1.1.cmml"><mi id="S4.p6.2.m2.1.1.2" xref="S4.p6.2.m2.1.1.2.cmml">HA</mi><mo id="S4.p6.2.m2.1.1.1" xref="S4.p6.2.m2.1.1.1.cmml">≈</mo><mrow id="S4.p6.2.m2.1.1.3" xref="S4.p6.2.m2.1.1.3.cmml"><mn id="S4.p6.2.m2.1.1.3.2" xref="S4.p6.2.m2.1.1.3.2.cmml">1</mn><mo id="S4.p6.2.m2.1.1.3.1" xref="S4.p6.2.m2.1.1.3.1.cmml">⁢</mo><mi mathvariant="normal" id="S4.p6.2.m2.1.1.3.3" xref="S4.p6.2.m2.1.1.3.3.cmml">h</mi></mrow></mrow><annotation-xml encoding="MathML-Content" id="S4.p6.2.m2.1b"><apply id="S4.p6.2.m2.1.1.cmml" xref="S4.p6.2.m2.1.1"><approx id="S4.p6.2.m2.1.1.1.cmml" xref="S4.p6.2.m2.1.1.1"></approx><ci id="S4.p6.2.m2.1.1.2.cmml" xref="S4.p6.2.m2.1.1.2">HA</ci><apply id="S4.p6.2.m2.1.1.3.cmml" xref="S4.p6.2.m2.1.1.3"><times id="S4.p6.2.m2.1.1.3.1.cmml" xref="S4.p6.2.m2.1.1.3.1"></times><cn type="integer" id="S4.p6.2.m2.1.1.3.2.cmml" xref="S4.p6.2.m2.1.1.3.2">1</cn><ci id="S4.p6.2.m2.1.1.3.3.cmml" xref="S4.p6.2.m2.1.1.3.3">h</ci></apply></apply></annotation-xml><annotation encoding="application/x-tex" id="S4.p6.2.m2.1c">\rm HA\approx 1h</annotation><annotation encoding="application/x-llamapun" id="S4.p6.2.m2.1d">roman_HA ≈ 1 roman_h</annotation></semantics></math>. Tracking speed was adjusted and then kept fixed in such a
way that tracking residuals were minimal. The overall tracking error
during 2 hours was less than <math id="S4.p6.3.m3.1" class="ltx_Math" alttext="\rm 0.5sec" display="inline"><semantics id="S4.p6.3.m3.1a"><mrow id="S4.p6.3.m3.1.1" xref="S4.p6.3.m3.1.1.cmml"><mn id="S4.p6.3.m3.1.1.2" xref="S4.p6.3.m3.1.1.2.cmml">0.5</mn><mo id="S4.p6.3.m3.1.1.1" xref="S4.p6.3.m3.1.1.1.cmml">⁢</mo><mi id="S4.p6.3.m3.1.1.3" xref="S4.p6.3.m3.1.1.3.cmml">sec</mi></mrow><annotation-xml encoding="MathML-Content" id="S4.p6.3.m3.1b"><apply id="S4.p6.3.m3.1.1.cmml" xref="S4.p6.3.m3.1.1"><times id="S4.p6.3.m3.1.1.1.cmml" xref="S4.p6.3.m3.1.1.1"></times><cn type="float" id="S4.p6.3.m3.1.1.2.cmml" xref="S4.p6.3.m3.1.1.2">0.5</cn><ci id="S4.p6.3.m3.1.1.3.cmml" xref="S4.p6.3.m3.1.1.3">sec</ci></apply></annotation-xml><annotation encoding="application/x-tex" id="S4.p6.3.m3.1c">\rm 0.5sec</annotation><annotation encoding="application/x-llamapun" id="S4.p6.3.m3.1d">0.5 roman_sec</annotation></semantics></math> in RA. We also found
quasi-periodic errors with <math id="S4.p6.4.m4.1" class="ltx_Math" alttext="\rm\sim 20min" display="inline"><semantics id="S4.p6.4.m4.1a"><mrow id="S4.p6.4.m4.1.1" xref="S4.p6.4.m4.1.1.cmml"><mi id="S4.p6.4.m4.1.1.2" xref="S4.p6.4.m4.1.1.2.cmml"></mi><mo id="S4.p6.4.m4.1.1.1" xref="S4.p6.4.m4.1.1.1.cmml">∼</mo><mrow id="S4.p6.4.m4.1.1.3" xref="S4.p6.4.m4.1.1.3.cmml"><mn id="S4.p6.4.m4.1.1.3.2" xref="S4.p6.4.m4.1.1.3.2.cmml">20</mn><mo id="S4.p6.4.m4.1.1.3.1" xref="S4.p6.4.m4.1.1.3.1.cmml">⁢</mo><mi mathvariant="normal" id="S4.p6.4.m4.1.1.3.3" xref="S4.p6.4.m4.1.1.3.3.cmml">m</mi><mo id="S4.p6.4.m4.1.1.3.1a" xref="S4.p6.4.m4.1.1.3.1.cmml">⁢</mo><mi mathvariant="normal" id="S4.p6.4.m4.1.1.3.4" xref="S4.p6.4.m4.1.1.3.4.cmml">i</mi><mo id="S4.p6.4.m4.1.1.3.1b" xref="S4.p6.4.m4.1.1.3.1.cmml">⁢</mo><mi mathvariant="normal" id="S4.p6.4.m4.1.1.3.5" xref="S4.p6.4.m4.1.1.3.5.cmml">n</mi></mrow></mrow><annotation-xml encoding="MathML-Content" id="S4.p6.4.m4.1b"><apply id="S4.p6.4.m4.1.1.cmml" xref="S4.p6.4.m4.1.1"><csymbol cd="latexml" id="S4.p6.4.m4.1.1.1.cmml" xref="S4.p6.4.m4.1.1.1">similar-to</csymbol><csymbol cd="latexml" id="S4.p6.4.m4.1.1.2.cmml" xref="S4.p6.4.m4.1.1.2">absent</csymbol><apply id="S4.p6.4.m4.1.1.3.cmml" xref="S4.p6.4.m4.1.1.3"><times id="S4.p6.4.m4.1.1.3.1.cmml" xref="S4.p6.4.m4.1.1.3.1"></times><cn type="integer" id="S4.p6.4.m4.1.1.3.2.cmml" xref="S4.p6.4.m4.1.1.3.2">20</cn><ci id="S4.p6.4.m4.1.1.3.3.cmml" xref="S4.p6.4.m4.1.1.3.3">m</ci><ci id="S4.p6.4.m4.1.1.3.4.cmml" xref="S4.p6.4.m4.1.1.3.4">i</ci><ci id="S4.p6.4.m4.1.1.3.5.cmml" xref="S4.p6.4.m4.1.1.3.5">n</ci></apply></apply></annotation-xml><annotation encoding="application/x-tex" id="S4.p6.4.m4.1c">\rm\sim 20min</annotation><annotation encoding="application/x-llamapun" id="S4.p6.4.m4.1d">∼ 20 roman_m roman_i roman_n</annotation></semantics></math> characteristic timescale and
<math id="S4.p6.5.m5.1" class="ltx_Math" alttext="\rm 0.5sec" display="inline"><semantics id="S4.p6.5.m5.1a"><mrow id="S4.p6.5.m5.1.1" xref="S4.p6.5.m5.1.1.cmml"><mn id="S4.p6.5.m5.1.1.2" xref="S4.p6.5.m5.1.1.2.cmml">0.5</mn><mo id="S4.p6.5.m5.1.1.1" xref="S4.p6.5.m5.1.1.1.cmml">⁢</mo><mi id="S4.p6.5.m5.1.1.3" xref="S4.p6.5.m5.1.1.3.cmml">sec</mi></mrow><annotation-xml encoding="MathML-Content" id="S4.p6.5.m5.1b"><apply id="S4.p6.5.m5.1.1.cmml" xref="S4.p6.5.m5.1.1"><times id="S4.p6.5.m5.1.1.1.cmml" xref="S4.p6.5.m5.1.1.1"></times><cn type="float" id="S4.p6.5.m5.1.1.2.cmml" xref="S4.p6.5.m5.1.1.2">0.5</cn><ci id="S4.p6.5.m5.1.1.3.cmml" xref="S4.p6.5.m5.1.1.3">sec</ci></apply></annotation-xml><annotation encoding="application/x-tex" id="S4.p6.5.m5.1c">\rm 0.5sec</annotation><annotation encoding="application/x-llamapun" id="S4.p6.5.m5.1d">0.5 roman_sec</annotation></semantics></math> amplitude, which were probably due to irregularities on
the small RA roller (see Fig. <a href="#S4.F5" title="Figure 5 ‣ 4 Pointing precision of HAT ‣ System description and first light-curves of HAT, an autonomous observatory for variability search" class="ltx_ref"><span class="ltx_text ltx_ref_tag">5</span></a>).</p>
</div>
<figure id="S4.F5" class="ltx_figure"><img src="x5.png" id="S4.F5.g1" class="ltx_graphics" width="676" height="676" alt="Tracking errors of HAT and MTO ">
<figcaption class="ltx_caption"><span class="ltx_tag ltx_tag_figure">Figure 5: </span>Tracking errors of HAT and MTO <math id="S4.F5.3.m1.1" class="ltx_Math" alttext="\rm 100mm\oslash" display="inline"><semantics id="S4.F5.3.m1.1b"><mrow id="S4.F5.3.m1.1c"><mn id="S4.F5.3.m1.1.1" xref="S4.F5.3.m1.1.1.cmml">100</mn><mi mathvariant="normal" id="S4.F5.3.m1.1.2" xref="S4.F5.3.m1.1.2.cmml">m</mi><mi mathvariant="normal" id="S4.F5.3.m1.1.3" xref="S4.F5.3.m1.1.3.cmml">m</mi><mo id="S4.F5.3.m1.1.4" xref="S4.F5.3.m1.1.4.cmml">⊘</mo></mrow><annotation-xml encoding="MathML-Content" id="S4.F5.3.m1.1d"><cerror id="S4.F5.3.m1.1e"><csymbol cd="ambiguous" id="S4.F5.3.m1.1f">fragments</csymbol><cn type="integer" id="S4.F5.3.m1.1.1.cmml" xref="S4.F5.3.m1.1.1">100</cn><csymbol cd="unknown" id="S4.F5.3.m1.1.2.cmml" xref="S4.F5.3.m1.1.2">m</csymbol><csymbol cd="unknown" id="S4.F5.3.m1.1.3.cmml" xref="S4.F5.3.m1.1.3">m</csymbol><ci id="S4.F5.3.m1.1.4.cmml" xref="S4.F5.3.m1.1.4">⊘</ci></cerror></annotation-xml><annotation encoding="application/x-tex" id="S4.F5.3.m1.1g">\rm 100mm\oslash</annotation><annotation encoding="application/x-llamapun" id="S4.F5.3.m1.1h">100 roman_m roman_m ⊘</annotation></semantics></math> lens in a 2 hour
run. The upper line and empty circles show improperly tuned tracking
speed. The lower line represents calibrated tracking. Deviations with
characteristic time <math id="S4.F5.4.m2.1" class="ltx_Math" alttext="\rm\sim 10^{3}sec" display="inline"><semantics id="S4.F5.4.m2.1b"><mrow id="S4.F5.4.m2.1.1" xref="S4.F5.4.m2.1.1.cmml"><mi id="S4.F5.4.m2.1.1.2" xref="S4.F5.4.m2.1.1.2.cmml"></mi><mo id="S4.F5.4.m2.1.1.1" xref="S4.F5.4.m2.1.1.1.cmml">∼</mo><mrow id="S4.F5.4.m2.1.1.3" xref="S4.F5.4.m2.1.1.3.cmml"><msup id="S4.F5.4.m2.1.1.3.2" xref="S4.F5.4.m2.1.1.3.2.cmml"><mn id="S4.F5.4.m2.1.1.3.2.2" xref="S4.F5.4.m2.1.1.3.2.2.cmml">10</mn><mn id="S4.F5.4.m2.1.1.3.2.3" xref="S4.F5.4.m2.1.1.3.2.3.cmml">3</mn></msup><mo id="S4.F5.4.m2.1.1.3.1" xref="S4.F5.4.m2.1.1.3.1.cmml">⁢</mo><mi id="S4.F5.4.m2.1.1.3.3" xref="S4.F5.4.m2.1.1.3.3.cmml">sec</mi></mrow></mrow><annotation-xml encoding="MathML-Content" id="S4.F5.4.m2.1c"><apply id="S4.F5.4.m2.1.1.cmml" xref="S4.F5.4.m2.1.1"><csymbol cd="latexml" id="S4.F5.4.m2.1.1.1.cmml" xref="S4.F5.4.m2.1.1.1">similar-to</csymbol><csymbol cd="latexml" id="S4.F5.4.m2.1.1.2.cmml" xref="S4.F5.4.m2.1.1.2">absent</csymbol><apply id="S4.F5.4.m2.1.1.3.cmml" xref="S4.F5.4.m2.1.1.3"><times id="S4.F5.4.m2.1.1.3.1.cmml" xref="S4.F5.4.m2.1.1.3.1"></times><apply id="S4.F5.4.m2.1.1.3.2.cmml" xref="S4.F5.4.m2.1.1.3.2"><csymbol cd="ambiguous" id="S4.F5.4.m2.1.1.3.2.1.cmml" xref="S4.F5.4.m2.1.1.3.2">superscript</csymbol><cn type="integer" id="S4.F5.4.m2.1.1.3.2.2.cmml" xref="S4.F5.4.m2.1.1.3.2.2">10</cn><cn type="integer" id="S4.F5.4.m2.1.1.3.2.3.cmml" xref="S4.F5.4.m2.1.1.3.2.3">3</cn></apply><ci id="S4.F5.4.m2.1.1.3.3.cmml" xref="S4.F5.4.m2.1.1.3.3">sec</ci></apply></apply></annotation-xml><annotation encoding="application/x-tex" id="S4.F5.4.m2.1d">\rm\sim 10^{3}sec</annotation><annotation encoding="application/x-llamapun" id="S4.F5.4.m2.1e">∼ 10 start_POSTSUPERSCRIPT 3 end_POSTSUPERSCRIPT roman_sec</annotation></semantics></math> arise from the
irregularities on the RA roller. </figcaption>
</figure>
<div id="S4.p7" class="ltx_para">
<p id="S4.p7.4" class="ltx_p">We defined absolute pointing precision as the precision of moving the
telescope to a celestial object (grid star) with respect to the home
position or another star (reference star). Pointing errors scaled with
the arc of movement, but overall scatter <span id="S4.p7.4.1" class="ltx_text ltx_font_italic">without</span> removing any
systematic error with a correction map were in the order of
<math id="S4.p7.1.m1.1" class="ltx_Math" alttext="\rm\sigma_{RA}\approx 20sec" display="inline"><semantics id="S4.p7.1.m1.1a"><mrow id="S4.p7.1.m1.1.1" xref="S4.p7.1.m1.1.1.cmml"><msub id="S4.p7.1.m1.1.1.2" xref="S4.p7.1.m1.1.1.2.cmml"><mi id="S4.p7.1.m1.1.1.2.2" xref="S4.p7.1.m1.1.1.2.2.cmml">σ</mi><mi id="S4.p7.1.m1.1.1.2.3" xref="S4.p7.1.m1.1.1.2.3.cmml">RA</mi></msub><mo id="S4.p7.1.m1.1.1.1" xref="S4.p7.1.m1.1.1.1.cmml">≈</mo><mrow id="S4.p7.1.m1.1.1.3" xref="S4.p7.1.m1.1.1.3.cmml"><mn id="S4.p7.1.m1.1.1.3.2" xref="S4.p7.1.m1.1.1.3.2.cmml">20</mn><mo id="S4.p7.1.m1.1.1.3.1" xref="S4.p7.1.m1.1.1.3.1.cmml">⁢</mo><mi mathvariant="normal" id="S4.p7.1.m1.1.1.3.3" xref="S4.p7.1.m1.1.1.3.3.cmml">s</mi><mo id="S4.p7.1.m1.1.1.3.1a" xref="S4.p7.1.m1.1.1.3.1.cmml">⁢</mo><mi mathvariant="normal" id="S4.p7.1.m1.1.1.3.4" xref="S4.p7.1.m1.1.1.3.4.cmml">e</mi><mo id="S4.p7.1.m1.1.1.3.1b" xref="S4.p7.1.m1.1.1.3.1.cmml">⁢</mo><mi mathvariant="normal" id="S4.p7.1.m1.1.1.3.5" xref="S4.p7.1.m1.1.1.3.5.cmml">c</mi></mrow></mrow><annotation-xml encoding="MathML-Content" id="S4.p7.1.m1.1b"><apply id="S4.p7.1.m1.1.1.cmml" xref="S4.p7.1.m1.1.1"><approx id="S4.p7.1.m1.1.1.1.cmml" xref="S4.p7.1.m1.1.1.1"></approx><apply id="S4.p7.1.m1.1.1.2.cmml" xref="S4.p7.1.m1.1.1.2"><csymbol cd="ambiguous" id="S4.p7.1.m1.1.1.2.1.cmml" xref="S4.p7.1.m1.1.1.2">subscript</csymbol><ci id="S4.p7.1.m1.1.1.2.2.cmml" xref="S4.p7.1.m1.1.1.2.2">𝜎</ci><ci id="S4.p7.1.m1.1.1.2.3.cmml" xref="S4.p7.1.m1.1.1.2.3">RA</ci></apply><apply id="S4.p7.1.m1.1.1.3.cmml" xref="S4.p7.1.m1.1.1.3"><times id="S4.p7.1.m1.1.1.3.1.cmml" xref="S4.p7.1.m1.1.1.3.1"></times><cn type="integer" id="S4.p7.1.m1.1.1.3.2.cmml" xref="S4.p7.1.m1.1.1.3.2">20</cn><ci id="S4.p7.1.m1.1.1.3.3.cmml" xref="S4.p7.1.m1.1.1.3.3">s</ci><ci id="S4.p7.1.m1.1.1.3.4.cmml" xref="S4.p7.1.m1.1.1.3.4">e</ci><ci id="S4.p7.1.m1.1.1.3.5.cmml" xref="S4.p7.1.m1.1.1.3.5">c</ci></apply></apply></annotation-xml><annotation encoding="application/x-tex" id="S4.p7.1.m1.1c">\rm\sigma_{RA}\approx 20sec</annotation><annotation encoding="application/x-llamapun" id="S4.p7.1.m1.1d">italic_σ start_POSTSUBSCRIPT roman_RA end_POSTSUBSCRIPT ≈ 20 roman_s roman_e roman_c</annotation></semantics></math>,
<math id="S4.p7.2.m2.1" class="ltx_Math" alttext="\rm\sigma_{Dec}\approx 70\arcsec" display="inline"><semantics id="S4.p7.2.m2.1a"><mrow id="S4.p7.2.m2.1.1" xref="S4.p7.2.m2.1.1.cmml"><msub id="S4.p7.2.m2.1.1.2" xref="S4.p7.2.m2.1.1.2.cmml"><mi id="S4.p7.2.m2.1.1.2.2" xref="S4.p7.2.m2.1.1.2.2.cmml">σ</mi><mi id="S4.p7.2.m2.1.1.2.3" xref="S4.p7.2.m2.1.1.2.3.cmml">Dec</mi></msub><mo id="S4.p7.2.m2.1.1.1" xref="S4.p7.2.m2.1.1.1.cmml">≈</mo><mrow id="S4.p7.2.m2.1.1.3" xref="S4.p7.2.m2.1.1.3.cmml"><mn id="S4.p7.2.m2.1.1.3.2" xref="S4.p7.2.m2.1.1.3.2.cmml">70</mn><mo id="S4.p7.2.m2.1.1.3.1" xref="S4.p7.2.m2.1.1.3.1.cmml">⁢</mo><mi mathvariant="normal" id="S4.p7.2.m2.1.1.3.3" xref="S4.p7.2.m2.1.1.3.3.cmml">″</mi></mrow></mrow><annotation-xml encoding="MathML-Content" id="S4.p7.2.m2.1b"><apply id="S4.p7.2.m2.1.1.cmml" xref="S4.p7.2.m2.1.1"><approx id="S4.p7.2.m2.1.1.1.cmml" xref="S4.p7.2.m2.1.1.1"></approx><apply id="S4.p7.2.m2.1.1.2.cmml" xref="S4.p7.2.m2.1.1.2"><csymbol cd="ambiguous" id="S4.p7.2.m2.1.1.2.1.cmml" xref="S4.p7.2.m2.1.1.2">subscript</csymbol><ci id="S4.p7.2.m2.1.1.2.2.cmml" xref="S4.p7.2.m2.1.1.2.2">𝜎</ci><ci id="S4.p7.2.m2.1.1.2.3.cmml" xref="S4.p7.2.m2.1.1.2.3">Dec</ci></apply><apply id="S4.p7.2.m2.1.1.3.cmml" xref="S4.p7.2.m2.1.1.3"><times id="S4.p7.2.m2.1.1.3.1.cmml" xref="S4.p7.2.m2.1.1.3.1"></times><cn type="integer" id="S4.p7.2.m2.1.1.3.2.cmml" xref="S4.p7.2.m2.1.1.3.2">70</cn><ci id="S4.p7.2.m2.1.1.3.3.cmml" xref="S4.p7.2.m2.1.1.3.3">″</ci></apply></apply></annotation-xml><annotation encoding="application/x-tex" id="S4.p7.2.m2.1c">\rm\sigma_{Dec}\approx 70\arcsec</annotation><annotation encoding="application/x-llamapun" id="S4.p7.2.m2.1d">italic_σ start_POSTSUBSCRIPT roman_Dec end_POSTSUBSCRIPT ≈ 70 ″</annotation></semantics></math>.
Repeatability was measured by moving the telescope back to the
reference point, usually a bright star close to zenith. Repeatability
is better than absolute pointing:
<math id="S4.p7.3.m3.1" class="ltx_Math" alttext="\rm\sigma_{RA}\approx 7sec" display="inline"><semantics id="S4.p7.3.m3.1a"><mrow id="S4.p7.3.m3.1.1" xref="S4.p7.3.m3.1.1.cmml"><msub id="S4.p7.3.m3.1.1.2" xref="S4.p7.3.m3.1.1.2.cmml"><mi id="S4.p7.3.m3.1.1.2.2" xref="S4.p7.3.m3.1.1.2.2.cmml">σ</mi><mi id="S4.p7.3.m3.1.1.2.3" xref="S4.p7.3.m3.1.1.2.3.cmml">RA</mi></msub><mo id="S4.p7.3.m3.1.1.1" xref="S4.p7.3.m3.1.1.1.cmml">≈</mo><mrow id="S4.p7.3.m3.1.1.3" xref="S4.p7.3.m3.1.1.3.cmml"><mn id="S4.p7.3.m3.1.1.3.2" xref="S4.p7.3.m3.1.1.3.2.cmml">7</mn><mo id="S4.p7.3.m3.1.1.3.1" xref="S4.p7.3.m3.1.1.3.1.cmml">⁢</mo><mi mathvariant="normal" id="S4.p7.3.m3.1.1.3.3" xref="S4.p7.3.m3.1.1.3.3.cmml">s</mi><mo id="S4.p7.3.m3.1.1.3.1a" xref="S4.p7.3.m3.1.1.3.1.cmml">⁢</mo><mi mathvariant="normal" id="S4.p7.3.m3.1.1.3.4" xref="S4.p7.3.m3.1.1.3.4.cmml">e</mi><mo id="S4.p7.3.m3.1.1.3.1b" xref="S4.p7.3.m3.1.1.3.1.cmml">⁢</mo><mi mathvariant="normal" id="S4.p7.3.m3.1.1.3.5" xref="S4.p7.3.m3.1.1.3.5.cmml">c</mi></mrow></mrow><annotation-xml encoding="MathML-Content" id="S4.p7.3.m3.1b"><apply id="S4.p7.3.m3.1.1.cmml" xref="S4.p7.3.m3.1.1"><approx id="S4.p7.3.m3.1.1.1.cmml" xref="S4.p7.3.m3.1.1.1"></approx><apply id="S4.p7.3.m3.1.1.2.cmml" xref="S4.p7.3.m3.1.1.2"><csymbol cd="ambiguous" id="S4.p7.3.m3.1.1.2.1.cmml" xref="S4.p7.3.m3.1.1.2">subscript</csymbol><ci id="S4.p7.3.m3.1.1.2.2.cmml" xref="S4.p7.3.m3.1.1.2.2">𝜎</ci><ci id="S4.p7.3.m3.1.1.2.3.cmml" xref="S4.p7.3.m3.1.1.2.3">RA</ci></apply><apply id="S4.p7.3.m3.1.1.3.cmml" xref="S4.p7.3.m3.1.1.3"><times id="S4.p7.3.m3.1.1.3.1.cmml" xref="S4.p7.3.m3.1.1.3.1"></times><cn type="integer" id="S4.p7.3.m3.1.1.3.2.cmml" xref="S4.p7.3.m3.1.1.3.2">7</cn><ci id="S4.p7.3.m3.1.1.3.3.cmml" xref="S4.p7.3.m3.1.1.3.3">s</ci><ci id="S4.p7.3.m3.1.1.3.4.cmml" xref="S4.p7.3.m3.1.1.3.4">e</ci><ci id="S4.p7.3.m3.1.1.3.5.cmml" xref="S4.p7.3.m3.1.1.3.5">c</ci></apply></apply></annotation-xml><annotation encoding="application/x-tex" id="S4.p7.3.m3.1c">\rm\sigma_{RA}\approx 7sec</annotation><annotation encoding="application/x-llamapun" id="S4.p7.3.m3.1d">italic_σ start_POSTSUBSCRIPT roman_RA end_POSTSUBSCRIPT ≈ 7 roman_s roman_e roman_c</annotation></semantics></math>,
<math id="S4.p7.4.m4.1" class="ltx_Math" alttext="\rm\sigma_{Dec}\approx 50\arcsec" display="inline"><semantics id="S4.p7.4.m4.1a"><mrow id="S4.p7.4.m4.1.1" xref="S4.p7.4.m4.1.1.cmml"><msub id="S4.p7.4.m4.1.1.2" xref="S4.p7.4.m4.1.1.2.cmml"><mi id="S4.p7.4.m4.1.1.2.2" xref="S4.p7.4.m4.1.1.2.2.cmml">σ</mi><mi id="S4.p7.4.m4.1.1.2.3" xref="S4.p7.4.m4.1.1.2.3.cmml">Dec</mi></msub><mo id="S4.p7.4.m4.1.1.1" xref="S4.p7.4.m4.1.1.1.cmml">≈</mo><mrow id="S4.p7.4.m4.1.1.3" xref="S4.p7.4.m4.1.1.3.cmml"><mn id="S4.p7.4.m4.1.1.3.2" xref="S4.p7.4.m4.1.1.3.2.cmml">50</mn><mo id="S4.p7.4.m4.1.1.3.1" xref="S4.p7.4.m4.1.1.3.1.cmml">⁢</mo><mi mathvariant="normal" id="S4.p7.4.m4.1.1.3.3" xref="S4.p7.4.m4.1.1.3.3.cmml">″</mi></mrow></mrow><annotation-xml encoding="MathML-Content" id="S4.p7.4.m4.1b"><apply id="S4.p7.4.m4.1.1.cmml" xref="S4.p7.4.m4.1.1"><approx id="S4.p7.4.m4.1.1.1.cmml" xref="S4.p7.4.m4.1.1.1"></approx><apply id="S4.p7.4.m4.1.1.2.cmml" xref="S4.p7.4.m4.1.1.2"><csymbol cd="ambiguous" id="S4.p7.4.m4.1.1.2.1.cmml" xref="S4.p7.4.m4.1.1.2">subscript</csymbol><ci id="S4.p7.4.m4.1.1.2.2.cmml" xref="S4.p7.4.m4.1.1.2.2">𝜎</ci><ci id="S4.p7.4.m4.1.1.2.3.cmml" xref="S4.p7.4.m4.1.1.2.3">Dec</ci></apply><apply id="S4.p7.4.m4.1.1.3.cmml" xref="S4.p7.4.m4.1.1.3"><times id="S4.p7.4.m4.1.1.3.1.cmml" xref="S4.p7.4.m4.1.1.3.1"></times><cn type="integer" id="S4.p7.4.m4.1.1.3.2.cmml" xref="S4.p7.4.m4.1.1.3.2">50</cn><ci id="S4.p7.4.m4.1.1.3.3.cmml" xref="S4.p7.4.m4.1.1.3.3">″</ci></apply></apply></annotation-xml><annotation encoding="application/x-tex" id="S4.p7.4.m4.1c">\rm\sigma_{Dec}\approx 50\arcsec</annotation><annotation encoding="application/x-llamapun" id="S4.p7.4.m4.1d">italic_σ start_POSTSUBSCRIPT roman_Dec end_POSTSUBSCRIPT ≈ 50 ″</annotation></semantics></math>.
This indicates that our absolute pointing errors – at least partly
– originate from non-perpendicularity of the axes and imperfect
polar axis setting.</p>
</div>
</section>
<section id="S5" class="ltx_section">
<h2 class="ltx_title ltx_title_section">
<span class="ltx_tag ltx_tag_section">5 </span>Installation of HAT-1 on Kitt Peak</h2>

<figure id="S5.F6" class="ltx_figure"><img src="x6.png" id="S5.F6.g1" class="ltx_graphics" width="676" height="507" alt="
HAT-1 at Kitt Peak: asymmetric clamshell dome and
horseshoe-mount. The white tube is the light baffle on the Nikon lens.
">
<figcaption class="ltx_caption"><span class="ltx_tag ltx_tag_figure">Figure 6: </span>
HAT-1 at Kitt Peak: asymmetric clamshell dome and
horseshoe-mount. The white tube is the light baffle on the Nikon lens.
</figcaption>
</figure>
<div id="S5.p1" class="ltx_para">
<p id="S5.p1.1" class="ltx_p">HAT-1 was transported (airmail) within its dome from Budapest to
Steward Observatory, Tucson in January, 2001. The telescope was badly
damaged during the trip, but thanks to Steward Observatory’s
hospitality, it was repaired in a few weeks. The telescope was
installed to Steward Observatory, Kitt Peak in the following month, and
was ready for operations by March 2001.</p>
</div>
<div id="S5.p2" class="ltx_para">
<p id="S5.p2.1" class="ltx_p">Our PC is hosted by the SuperLotis building’s warmroom, and cables go
out to the <math id="S5.p2.1.m1.1" class="ltx_Math" alttext="\rm\sim 2m" display="inline"><semantics id="S5.p2.1.m1.1a"><mrow id="S5.p2.1.m1.1.1" xref="S5.p2.1.m1.1.1.cmml"><mi id="S5.p2.1.m1.1.1.2" xref="S5.p2.1.m1.1.1.2.cmml"></mi><mo id="S5.p2.1.m1.1.1.1" xref="S5.p2.1.m1.1.1.1.cmml">∼</mo><mrow id="S5.p2.1.m1.1.1.3" xref="S5.p2.1.m1.1.1.3.cmml"><mn id="S5.p2.1.m1.1.1.3.2" xref="S5.p2.1.m1.1.1.3.2.cmml">2</mn><mo id="S5.p2.1.m1.1.1.3.1" xref="S5.p2.1.m1.1.1.3.1.cmml">⁢</mo><mi mathvariant="normal" id="S5.p2.1.m1.1.1.3.3" xref="S5.p2.1.m1.1.1.3.3.cmml">m</mi></mrow></mrow><annotation-xml encoding="MathML-Content" id="S5.p2.1.m1.1b"><apply id="S5.p2.1.m1.1.1.cmml" xref="S5.p2.1.m1.1.1"><csymbol cd="latexml" id="S5.p2.1.m1.1.1.1.cmml" xref="S5.p2.1.m1.1.1.1">similar-to</csymbol><csymbol cd="latexml" id="S5.p2.1.m1.1.1.2.cmml" xref="S5.p2.1.m1.1.1.2">absent</csymbol><apply id="S5.p2.1.m1.1.1.3.cmml" xref="S5.p2.1.m1.1.1.3"><times id="S5.p2.1.m1.1.1.3.1.cmml" xref="S5.p2.1.m1.1.1.3.1"></times><cn type="integer" id="S5.p2.1.m1.1.1.3.2.cmml" xref="S5.p2.1.m1.1.1.3.2">2</cn><ci id="S5.p2.1.m1.1.1.3.3.cmml" xref="S5.p2.1.m1.1.1.3.3">m</ci></apply></apply></annotation-xml><annotation encoding="application/x-tex" id="S5.p2.1.m1.1c">\rm\sim 2m</annotation><annotation encoding="application/x-llamapun" id="S5.p2.1.m1.1d">∼ 2 roman_m</annotation></semantics></math> high, massive concrete pillar on the western
edge of the ridge, holding the HAT dome and telescope. The bug (§<a href="#S2.SS4" title="2.4 The CCD ‣ 2 Hardware System ‣ System description and first light-curves of HAT, an autonomous observatory for variability search" class="ltx_ref"><span class="ltx_text ltx_ref_tag">2.4</span></a>) with the AP10 CCD delayed operation of HAT until
mid-May, but finally Apogee serviced the camera.</p>
</div>
<div id="S5.p3" class="ltx_para">
<p id="S5.p3.1" class="ltx_p">The first few month period was devoted to debugging the system and
finalizing observing programs. Although HAT worked in robotic mode, its
operation was monitored from Hungary. As reliable dome status
(open/close) information is not available for any of the domes at Kitt
Peak, we had to set aside the idea of slaving our dome opening to that
of a manually operated telescope, and open/close jointly with the other
dome. Thus, weather status was assessed at early Kitt Peak evening (5am
Central European Time) from the Internet using the daytime webcamera of
Kitt Peak<span id="footnote12" class="ltx_note ltx_role_footnote"><sup class="ltx_note_mark">12</sup><span class="ltx_note_outer"><span class="ltx_note_content"><sup class="ltx_note_mark">12</sup><span class="ltx_tag ltx_tag_note">12</span>http://www.noao.edu/cgi-bin/kpno/axim.cgi</span></span></span>, the
National Weather Service forecast<span id="footnote13" class="ltx_note ltx_role_footnote"><sup class="ltx_note_mark">13</sup><span class="ltx_note_outer"><span class="ltx_note_content"><sup class="ltx_note_mark">13</sup><span class="ltx_tag ltx_tag_note">13</span>http://www.wrh.noaa.gov</span></span></span> and
ConCam images during the night. Some information is automatically
downloaded and parsed by HAT, e.g. we switch to “weather-sleep” state
if the wind sensors of the 4m Mayall telescope show gusts in excess of
45km/h.</p>
</div>
<div id="S5.p4" class="ltx_para">
<p id="S5.p4.1" class="ltx_p">During the debugging period we started regular observations of selected
fields. The logfiles were examined next day, and updated software was
transferred to the site by the next evening. This way, in a few months,
the software became very reliable, and no major modifications have been
performed since.</p>
</div>
<div id="S5.p5" class="ltx_para">
<p id="S5.p5.1" class="ltx_p">Since September, 2001, HAT has been operated from the
Harvard-Smithsonian Center for Astrophysics, Cambridge, MA.</p>
</div>
</section>
<section id="S6" class="ltx_section">
<h2 class="ltx_title ltx_title_section">
<span class="ltx_tag ltx_tag_section">6 </span>Observations from Kitt Peak</h2>

<div id="S6.p1" class="ltx_para">
<p id="S6.p1.3" class="ltx_p">A typical observing session of HAT is described in the following.
Roughly 1.5 hours before sunset, camera cooling is started, and after
the service temperature is reached, 20 bias frames are taken. Following
this, twenty 4-minute dark frames are exposed. If weather is clear, the
“skyflat” task starts up 10 minutes after sunset. After opening the
dome, HAT selects an optimal flatfield region from the database, which
is close to the preferred point of minimal sky gradient
<cite class="ltx_cite ltx_citemacro_citep">(Chromey &amp; Hasselbacher, <a href="#bib.bib14" title="" class="ltx_ref">1996</a>)</cite>. This point is usually opposite to the Sun in
azimuth, and <math id="S6.p1.1.m1.1" class="ltx_Math" alttext="20\arcdeg" display="inline"><semantics id="S6.p1.1.m1.1a"><mrow id="S6.p1.1.m1.1.1" xref="S6.p1.1.m1.1.1.cmml"><mn id="S6.p1.1.m1.1.1.2" xref="S6.p1.1.m1.1.1.2.cmml">20</mn><mo id="S6.p1.1.m1.1.1.1" xref="S6.p1.1.m1.1.1.1.cmml">⁢</mo><mi mathvariant="normal" id="S6.p1.1.m1.1.1.3" xref="S6.p1.1.m1.1.1.3.cmml">°</mi></mrow><annotation-xml encoding="MathML-Content" id="S6.p1.1.m1.1b"><apply id="S6.p1.1.m1.1.1.cmml" xref="S6.p1.1.m1.1.1"><times id="S6.p1.1.m1.1.1.1.cmml" xref="S6.p1.1.m1.1.1.1"></times><cn type="integer" id="S6.p1.1.m1.1.1.2.cmml" xref="S6.p1.1.m1.1.1.2">20</cn><ci id="S6.p1.1.m1.1.1.3.cmml" xref="S6.p1.1.m1.1.1.3">°</ci></apply></annotation-xml><annotation encoding="application/x-tex" id="S6.p1.1.m1.1c">20\arcdeg</annotation><annotation encoding="application/x-llamapun" id="S6.p1.1.m1.1d">20 °</annotation></semantics></math> from zenith. The telescope is randomly moved
between skyflat frames so as to permit median averaging out brighter
stars, and exposure time is recursively tuned to keep the central
intensity at <math id="S6.p1.2.m2.1" class="ltx_Math" alttext="\sim 50" display="inline"><semantics id="S6.p1.2.m2.1a"><mrow id="S6.p1.2.m2.1.1" xref="S6.p1.2.m2.1.1.cmml"><mi id="S6.p1.2.m2.1.1.2" xref="S6.p1.2.m2.1.1.2.cmml"></mi><mo id="S6.p1.2.m2.1.1.1" xref="S6.p1.2.m2.1.1.1.cmml">∼</mo><mn id="S6.p1.2.m2.1.1.3" xref="S6.p1.2.m2.1.1.3.cmml">50</mn></mrow><annotation-xml encoding="MathML-Content" id="S6.p1.2.m2.1b"><apply id="S6.p1.2.m2.1.1.cmml" xref="S6.p1.2.m2.1.1"><csymbol cd="latexml" id="S6.p1.2.m2.1.1.1.cmml" xref="S6.p1.2.m2.1.1.1">similar-to</csymbol><csymbol cd="latexml" id="S6.p1.2.m2.1.1.2.cmml" xref="S6.p1.2.m2.1.1.2">absent</csymbol><cn type="integer" id="S6.p1.2.m2.1.1.3.cmml" xref="S6.p1.2.m2.1.1.3">50</cn></apply></annotation-xml><annotation encoding="application/x-tex" id="S6.p1.2.m2.1c">\sim 50</annotation><annotation encoding="application/x-llamapun" id="S6.p1.2.m2.1d">∼ 50</annotation></semantics></math>% of saturation. Even with these considerations,
the twilight sky on a <math id="S6.p1.3.m3.1" class="ltx_Math" alttext="9\arcdeg\times 9\arcdeg" display="inline"><semantics id="S6.p1.3.m3.1a"><mrow id="S6.p1.3.m3.1.1" xref="S6.p1.3.m3.1.1.cmml"><mrow id="S6.p1.3.m3.1.1.2" xref="S6.p1.3.m3.1.1.2.cmml"><mrow id="S6.p1.3.m3.1.1.2.2" xref="S6.p1.3.m3.1.1.2.2.cmml"><mn id="S6.p1.3.m3.1.1.2.2.2" xref="S6.p1.3.m3.1.1.2.2.2.cmml">9</mn><mo id="S6.p1.3.m3.1.1.2.2.1" xref="S6.p1.3.m3.1.1.2.2.1.cmml">⁢</mo><mi mathvariant="normal" id="S6.p1.3.m3.1.1.2.2.3" xref="S6.p1.3.m3.1.1.2.2.3.cmml">°</mi></mrow><mo id="S6.p1.3.m3.1.1.2.1" xref="S6.p1.3.m3.1.1.2.1.cmml">×</mo><mn id="S6.p1.3.m3.1.1.2.3" xref="S6.p1.3.m3.1.1.2.3.cmml">9</mn></mrow><mo id="S6.p1.3.m3.1.1.1" xref="S6.p1.3.m3.1.1.1.cmml">⁢</mo><mi mathvariant="normal" id="S6.p1.3.m3.1.1.3" xref="S6.p1.3.m3.1.1.3.cmml">°</mi></mrow><annotation-xml encoding="MathML-Content" id="S6.p1.3.m3.1b"><apply id="S6.p1.3.m3.1.1.cmml" xref="S6.p1.3.m3.1.1"><times id="S6.p1.3.m3.1.1.1.cmml" xref="S6.p1.3.m3.1.1.1"></times><apply id="S6.p1.3.m3.1.1.2.cmml" xref="S6.p1.3.m3.1.1.2"><times id="S6.p1.3.m3.1.1.2.1.cmml" xref="S6.p1.3.m3.1.1.2.1"></times><apply id="S6.p1.3.m3.1.1.2.2.cmml" xref="S6.p1.3.m3.1.1.2.2"><times id="S6.p1.3.m3.1.1.2.2.1.cmml" xref="S6.p1.3.m3.1.1.2.2.1"></times><cn type="integer" id="S6.p1.3.m3.1.1.2.2.2.cmml" xref="S6.p1.3.m3.1.1.2.2.2">9</cn><ci id="S6.p1.3.m3.1.1.2.2.3.cmml" xref="S6.p1.3.m3.1.1.2.2.3">°</ci></apply><cn type="integer" id="S6.p1.3.m3.1.1.2.3.cmml" xref="S6.p1.3.m3.1.1.2.3">9</cn></apply><ci id="S6.p1.3.m3.1.1.3.cmml" xref="S6.p1.3.m3.1.1.3">°</ci></apply></annotation-xml><annotation encoding="application/x-tex" id="S6.p1.3.m3.1c">9\arcdeg\times 9\arcdeg</annotation><annotation encoding="application/x-llamapun" id="S6.p1.3.m3.1d">9 ° × 9 °</annotation></semantics></math> area has gradients up to
10%, and therefore we installed a domeflat screen. Unfortunately this
brought up further complications, and dome-flat observations are still
under development. At the end of the night session, skyflats, darks and
biases are taken again.</p>
</div>
<div id="S6.p2" class="ltx_para">
<p id="S6.p2.3" class="ltx_p">The monitoring of selected fields (“monfield”) starts when the Sun
sinks <math id="S6.p2.1.m1.1" class="ltx_Math" alttext="11\arcdeg" display="inline"><semantics id="S6.p2.1.m1.1a"><mrow id="S6.p2.1.m1.1.1" xref="S6.p2.1.m1.1.1.cmml"><mn id="S6.p2.1.m1.1.1.2" xref="S6.p2.1.m1.1.1.2.cmml">11</mn><mo id="S6.p2.1.m1.1.1.1" xref="S6.p2.1.m1.1.1.1.cmml">⁢</mo><mi mathvariant="normal" id="S6.p2.1.m1.1.1.3" xref="S6.p2.1.m1.1.1.3.cmml">°</mi></mrow><annotation-xml encoding="MathML-Content" id="S6.p2.1.m1.1b"><apply id="S6.p2.1.m1.1.1.cmml" xref="S6.p2.1.m1.1.1"><times id="S6.p2.1.m1.1.1.1.cmml" xref="S6.p2.1.m1.1.1.1"></times><cn type="integer" id="S6.p2.1.m1.1.1.2.cmml" xref="S6.p2.1.m1.1.1.2">11</cn><ci id="S6.p2.1.m1.1.1.3.cmml" xref="S6.p2.1.m1.1.1.3">°</ci></apply></annotation-xml><annotation encoding="application/x-tex" id="S6.p2.1.m1.1c">11\arcdeg</annotation><annotation encoding="application/x-llamapun" id="S6.p2.1.m1.1d">11 °</annotation></semantics></math> below the horizon. The entire sky was split up into
696 slightly overlapping fields, each <math id="S6.p2.2.m2.1" class="ltx_Math" alttext="8\arcdeg\times 8\arcdeg" display="inline"><semantics id="S6.p2.2.m2.1a"><mrow id="S6.p2.2.m2.1.1" xref="S6.p2.2.m2.1.1.cmml"><mrow id="S6.p2.2.m2.1.1.2" xref="S6.p2.2.m2.1.1.2.cmml"><mrow id="S6.p2.2.m2.1.1.2.2" xref="S6.p2.2.m2.1.1.2.2.cmml"><mn id="S6.p2.2.m2.1.1.2.2.2" xref="S6.p2.2.m2.1.1.2.2.2.cmml">8</mn><mo id="S6.p2.2.m2.1.1.2.2.1" xref="S6.p2.2.m2.1.1.2.2.1.cmml">⁢</mo><mi mathvariant="normal" id="S6.p2.2.m2.1.1.2.2.3" xref="S6.p2.2.m2.1.1.2.2.3.cmml">°</mi></mrow><mo id="S6.p2.2.m2.1.1.2.1" xref="S6.p2.2.m2.1.1.2.1.cmml">×</mo><mn id="S6.p2.2.m2.1.1.2.3" xref="S6.p2.2.m2.1.1.2.3.cmml">8</mn></mrow><mo id="S6.p2.2.m2.1.1.1" xref="S6.p2.2.m2.1.1.1.cmml">⁢</mo><mi mathvariant="normal" id="S6.p2.2.m2.1.1.3" xref="S6.p2.2.m2.1.1.3.cmml">°</mi></mrow><annotation-xml encoding="MathML-Content" id="S6.p2.2.m2.1b"><apply id="S6.p2.2.m2.1.1.cmml" xref="S6.p2.2.m2.1.1"><times id="S6.p2.2.m2.1.1.1.cmml" xref="S6.p2.2.m2.1.1.1"></times><apply id="S6.p2.2.m2.1.1.2.cmml" xref="S6.p2.2.m2.1.1.2"><times id="S6.p2.2.m2.1.1.2.1.cmml" xref="S6.p2.2.m2.1.1.2.1"></times><apply id="S6.p2.2.m2.1.1.2.2.cmml" xref="S6.p2.2.m2.1.1.2.2"><times id="S6.p2.2.m2.1.1.2.2.1.cmml" xref="S6.p2.2.m2.1.1.2.2.1"></times><cn type="integer" id="S6.p2.2.m2.1.1.2.2.2.cmml" xref="S6.p2.2.m2.1.1.2.2.2">8</cn><ci id="S6.p2.2.m2.1.1.2.2.3.cmml" xref="S6.p2.2.m2.1.1.2.2.3">°</ci></apply><cn type="integer" id="S6.p2.2.m2.1.1.2.3.cmml" xref="S6.p2.2.m2.1.1.2.3">8</cn></apply><ci id="S6.p2.2.m2.1.1.3.cmml" xref="S6.p2.2.m2.1.1.3">°</ci></apply></annotation-xml><annotation encoding="application/x-tex" id="S6.p2.2.m2.1c">8\arcdeg\times 8\arcdeg</annotation><annotation encoding="application/x-llamapun" id="S6.p2.2.m2.1d">8 ° × 8 °</annotation></semantics></math> wide.
Choosing the next field to be observed is done by a sophisticated
algorithm. A list of enabled fields is loaded from the database, and
visible ones with high enough elevation are selected. Visibility means
that the object is within the limits of the mount, above the artificial
horizon-grid, and far enough from the Sun and Moon (<math id="S6.p2.3.m3.1" class="ltx_Math" alttext="\gtrsim 45\arcdeg" display="inline"><semantics id="S6.p2.3.m3.1a"><mrow id="S6.p2.3.m3.1.1" xref="S6.p2.3.m3.1.1.cmml"><mi id="S6.p2.3.m3.1.1.2" xref="S6.p2.3.m3.1.1.2.cmml"></mi><mo id="S6.p2.3.m3.1.1.1" xref="S6.p2.3.m3.1.1.1.cmml">≳</mo><mrow id="S6.p2.3.m3.1.1.3" xref="S6.p2.3.m3.1.1.3.cmml"><mn id="S6.p2.3.m3.1.1.3.2" xref="S6.p2.3.m3.1.1.3.2.cmml">45</mn><mo id="S6.p2.3.m3.1.1.3.1" xref="S6.p2.3.m3.1.1.3.1.cmml">⁢</mo><mi mathvariant="normal" id="S6.p2.3.m3.1.1.3.3" xref="S6.p2.3.m3.1.1.3.3.cmml">°</mi></mrow></mrow><annotation-xml encoding="MathML-Content" id="S6.p2.3.m3.1b"><apply id="S6.p2.3.m3.1.1.cmml" xref="S6.p2.3.m3.1.1"><csymbol cd="latexml" id="S6.p2.3.m3.1.1.1.cmml" xref="S6.p2.3.m3.1.1.1">greater-than-or-equivalent-to</csymbol><csymbol cd="latexml" id="S6.p2.3.m3.1.1.2.cmml" xref="S6.p2.3.m3.1.1.2">absent</csymbol><apply id="S6.p2.3.m3.1.1.3.cmml" xref="S6.p2.3.m3.1.1.3"><times id="S6.p2.3.m3.1.1.3.1.cmml" xref="S6.p2.3.m3.1.1.3.1"></times><cn type="integer" id="S6.p2.3.m3.1.1.3.2.cmml" xref="S6.p2.3.m3.1.1.3.2">45</cn><ci id="S6.p2.3.m3.1.1.3.3.cmml" xref="S6.p2.3.m3.1.1.3.3">°</ci></apply></apply></annotation-xml><annotation encoding="application/x-tex" id="S6.p2.3.m3.1c">\gtrsim 45\arcdeg</annotation><annotation encoding="application/x-llamapun" id="S6.p2.3.m3.1d">≳ 45 °</annotation></semantics></math>).
Ranking of the fields is done by combining their previous observation
times (the more recent, the lower rank), their proximity to the
meridian or the western horizon (depending on the ranking method) and
manually set priorities from the database. After any observation,
parameters (e.g. time of last observation) are updated and synchronized
with the DB. This ensures that none of the regions are observed too
frequently until there are other, favorably situated fields. Fields
close to culmination, thus having the smallest possible airmass and
differential refraction, are not missed. Priorities of fields can be
such that given their visibility, they are observed with higher
frequency, or even exclusively. Finally, a small random factor is
appended to the ranks in order to avoid the same selection on two
consecutive nights, thus daily aliases in frequency analysis. The
telescope is homed every hour to keep pointing accuracy.</p>
</div>
<div id="S6.p3" class="ltx_para">
<p id="S6.p3.2" class="ltx_p">Our present observing tactic is to observe most of the visible fields
sparsely, for information on long-time variation, and only concentrate
on a few fields for short-term variability. The high priority is
re-assigned to different fields on a weekly - monthly timescale. With
two consecutive, randomly displaced 240s exposures per field, we can
carry out photometry in the range of <math id="S6.p3.1.m1.1" class="ltx_Math" alttext="\rm I_{c}\approx 6-13^{m}" display="inline"><semantics id="S6.p3.1.m1.1a"><mrow id="S6.p3.1.m1.1.1" xref="S6.p3.1.m1.1.1.cmml"><msub id="S6.p3.1.m1.1.1.2" xref="S6.p3.1.m1.1.1.2.cmml"><mi mathvariant="normal" id="S6.p3.1.m1.1.1.2.2" xref="S6.p3.1.m1.1.1.2.2.cmml">I</mi><mi mathvariant="normal" id="S6.p3.1.m1.1.1.2.3" xref="S6.p3.1.m1.1.1.2.3.cmml">c</mi></msub><mo id="S6.p3.1.m1.1.1.1" xref="S6.p3.1.m1.1.1.1.cmml">≈</mo><mrow id="S6.p3.1.m1.1.1.3" xref="S6.p3.1.m1.1.1.3.cmml"><mn id="S6.p3.1.m1.1.1.3.2" xref="S6.p3.1.m1.1.1.3.2.cmml">6</mn><mo id="S6.p3.1.m1.1.1.3.1" xref="S6.p3.1.m1.1.1.3.1.cmml">-</mo><msup id="S6.p3.1.m1.1.1.3.3" xref="S6.p3.1.m1.1.1.3.3.cmml"><mn id="S6.p3.1.m1.1.1.3.3.2" xref="S6.p3.1.m1.1.1.3.3.2.cmml">13</mn><mi mathvariant="normal" id="S6.p3.1.m1.1.1.3.3.3" xref="S6.p3.1.m1.1.1.3.3.3.cmml">m</mi></msup></mrow></mrow><annotation-xml encoding="MathML-Content" id="S6.p3.1.m1.1b"><apply id="S6.p3.1.m1.1.1.cmml" xref="S6.p3.1.m1.1.1"><approx id="S6.p3.1.m1.1.1.1.cmml" xref="S6.p3.1.m1.1.1.1"></approx><apply id="S6.p3.1.m1.1.1.2.cmml" xref="S6.p3.1.m1.1.1.2"><csymbol cd="ambiguous" id="S6.p3.1.m1.1.1.2.1.cmml" xref="S6.p3.1.m1.1.1.2">subscript</csymbol><ci id="S6.p3.1.m1.1.1.2.2.cmml" xref="S6.p3.1.m1.1.1.2.2">I</ci><ci id="S6.p3.1.m1.1.1.2.3.cmml" xref="S6.p3.1.m1.1.1.2.3">c</ci></apply><apply id="S6.p3.1.m1.1.1.3.cmml" xref="S6.p3.1.m1.1.1.3"><minus id="S6.p3.1.m1.1.1.3.1.cmml" xref="S6.p3.1.m1.1.1.3.1"></minus><cn type="integer" id="S6.p3.1.m1.1.1.3.2.cmml" xref="S6.p3.1.m1.1.1.3.2">6</cn><apply id="S6.p3.1.m1.1.1.3.3.cmml" xref="S6.p3.1.m1.1.1.3.3"><csymbol cd="ambiguous" id="S6.p3.1.m1.1.1.3.3.1.cmml" xref="S6.p3.1.m1.1.1.3.3">superscript</csymbol><cn type="integer" id="S6.p3.1.m1.1.1.3.3.2.cmml" xref="S6.p3.1.m1.1.1.3.3.2">13</cn><ci id="S6.p3.1.m1.1.1.3.3.3.cmml" xref="S6.p3.1.m1.1.1.3.3.3">m</ci></apply></apply></apply></annotation-xml><annotation encoding="application/x-tex" id="S6.p3.1.m1.1c">\rm I_{c}\approx 6-13^{m}</annotation><annotation encoding="application/x-llamapun" id="S6.p3.1.m1.1d">roman_I start_POSTSUBSCRIPT roman_c end_POSTSUBSCRIPT ≈ 6 - 13 start_POSTSUPERSCRIPT roman_m end_POSTSUPERSCRIPT</annotation></semantics></math>. We also
experimented with 30s exposures so as not to lose information on very
bright, saturated stars, but this increased the data flow without
considerable amount of extra photometric information. With this
approach we collect about 80 (winter) to 50 (summer) frame-pairs per
clear night, which means approx. 30 photometric points per night and
star for high priority fields (<math id="S6.p3.2.m2.1" class="ltx_Math" alttext="10^{6}" display="inline"><semantics id="S6.p3.2.m2.1a"><msup id="S6.p3.2.m2.1.1" xref="S6.p3.2.m2.1.1.cmml"><mn id="S6.p3.2.m2.1.1.2" xref="S6.p3.2.m2.1.1.2.cmml">10</mn><mn id="S6.p3.2.m2.1.1.3" xref="S6.p3.2.m2.1.1.3.cmml">6</mn></msup><annotation-xml encoding="MathML-Content" id="S6.p3.2.m2.1b"><apply id="S6.p3.2.m2.1.1.cmml" xref="S6.p3.2.m2.1.1"><csymbol cd="ambiguous" id="S6.p3.2.m2.1.1.1.cmml" xref="S6.p3.2.m2.1.1">superscript</csymbol><cn type="integer" id="S6.p3.2.m2.1.1.2.cmml" xref="S6.p3.2.m2.1.1.2">10</cn><cn type="integer" id="S6.p3.2.m2.1.1.3.cmml" xref="S6.p3.2.m2.1.1.3">6</cn></apply></annotation-xml><annotation encoding="application/x-tex" id="S6.p3.2.m2.1c">10^{6}</annotation><annotation encoding="application/x-llamapun" id="S6.p3.2.m2.1d">10 start_POSTSUPERSCRIPT 6 end_POSTSUPERSCRIPT</annotation></semantics></math> photometric measurement
altogether).</p>
</div>
<div id="S6.p4" class="ltx_para">
<p id="S6.p4.2" class="ltx_p">Calibration of the data to standard <math id="S6.p4.1.m1.1" class="ltx_Math" alttext="\rm I_{c}" display="inline"><semantics id="S6.p4.1.m1.1a"><msub id="S6.p4.1.m1.1.1" xref="S6.p4.1.m1.1.1.cmml"><mi mathvariant="normal" id="S6.p4.1.m1.1.1.2" xref="S6.p4.1.m1.1.1.2.cmml">I</mi><mi mathvariant="normal" id="S6.p4.1.m1.1.1.3" xref="S6.p4.1.m1.1.1.3.cmml">c</mi></msub><annotation-xml encoding="MathML-Content" id="S6.p4.1.m1.1b"><apply id="S6.p4.1.m1.1.1.cmml" xref="S6.p4.1.m1.1.1"><csymbol cd="ambiguous" id="S6.p4.1.m1.1.1.1.cmml" xref="S6.p4.1.m1.1.1">subscript</csymbol><ci id="S6.p4.1.m1.1.1.2.cmml" xref="S6.p4.1.m1.1.1.2">I</ci><ci id="S6.p4.1.m1.1.1.3.cmml" xref="S6.p4.1.m1.1.1.3">c</ci></apply></annotation-xml><annotation encoding="application/x-tex" id="S6.p4.1.m1.1c">\rm I_{c}</annotation><annotation encoding="application/x-llamapun" id="S6.p4.1.m1.1d">roman_I start_POSTSUBSCRIPT roman_c end_POSTSUBSCRIPT</annotation></semantics></math>-band is not
straightforward, as we currently observe only with a single <math id="S6.p4.2.m2.1" class="ltx_Math" alttext="\rm I_{c}" display="inline"><semantics id="S6.p4.2.m2.1a"><msub id="S6.p4.2.m2.1.1" xref="S6.p4.2.m2.1.1.cmml"><mi mathvariant="normal" id="S6.p4.2.m2.1.1.2" xref="S6.p4.2.m2.1.1.2.cmml">I</mi><mi mathvariant="normal" id="S6.p4.2.m2.1.1.3" xref="S6.p4.2.m2.1.1.3.cmml">c</mi></msub><annotation-xml encoding="MathML-Content" id="S6.p4.2.m2.1b"><apply id="S6.p4.2.m2.1.1.cmml" xref="S6.p4.2.m2.1.1"><csymbol cd="ambiguous" id="S6.p4.2.m2.1.1.1.cmml" xref="S6.p4.2.m2.1.1">subscript</csymbol><ci id="S6.p4.2.m2.1.1.2.cmml" xref="S6.p4.2.m2.1.1.2">I</ci><ci id="S6.p4.2.m2.1.1.3.cmml" xref="S6.p4.2.m2.1.1.3">c</ci></apply></annotation-xml><annotation encoding="application/x-tex" id="S6.p4.2.m2.1c">\rm I_{c}</annotation><annotation encoding="application/x-llamapun" id="S6.p4.2.m2.1d">roman_I start_POSTSUBSCRIPT roman_c end_POSTSUBSCRIPT</annotation></semantics></math>
filter. Either Hipparcos Catalogue stars <cite class="ltx_cite ltx_citemacro_citep">(van Leeuwen et al., <a href="#bib.bib26" title="" class="ltx_ref">1997</a>)</cite> in our
fields are used for direct calibration, or we observe Landolt standards
<cite class="ltx_cite ltx_citemacro_citep">(Landolt, <a href="#bib.bib23" title="" class="ltx_ref">1983</a>, <a href="#bib.bib24" title="" class="ltx_ref">1992</a>)</cite> throughout the night. In both cases we have
to restrict ourselves to color-independent terms:</p>
<table id="S6.Ex1" class="ltx_equation ltx_eqn_table">

<tr class="ltx_equation ltx_eqn_row ltx_align_baseline">
<td class="ltx_eqn_cell ltx_eqn_center_padleft"></td>
<td class="ltx_eqn_cell ltx_align_center"><math id="S6.Ex1.m1.1" class="ltx_Math" alttext="i_{inst}=I_{c}+\xi_{i}+k_{i}\cdot X-\dot{\xi}\cdot(t-t_{0})," display="block"><semantics id="S6.Ex1.m1.1a"><mrow id="S6.Ex1.m1.1.1.1" xref="S6.Ex1.m1.1.1.1.1.cmml"><mrow id="S6.Ex1.m1.1.1.1.1" xref="S6.Ex1.m1.1.1.1.1.cmml"><msub id="S6.Ex1.m1.1.1.1.1.3" xref="S6.Ex1.m1.1.1.1.1.3.cmml"><mi id="S6.Ex1.m1.1.1.1.1.3.2" xref="S6.Ex1.m1.1.1.1.1.3.2.cmml">i</mi><mrow id="S6.Ex1.m1.1.1.1.1.3.3" xref="S6.Ex1.m1.1.1.1.1.3.3.cmml"><mi id="S6.Ex1.m1.1.1.1.1.3.3.2" xref="S6.Ex1.m1.1.1.1.1.3.3.2.cmml">i</mi><mo id="S6.Ex1.m1.1.1.1.1.3.3.1" xref="S6.Ex1.m1.1.1.1.1.3.3.1.cmml">⁢</mo><mi id="S6.Ex1.m1.1.1.1.1.3.3.3" xref="S6.Ex1.m1.1.1.1.1.3.3.3.cmml">n</mi><mo id="S6.Ex1.m1.1.1.1.1.3.3.1a" xref="S6.Ex1.m1.1.1.1.1.3.3.1.cmml">⁢</mo><mi id="S6.Ex1.m1.1.1.1.1.3.3.4" xref="S6.Ex1.m1.1.1.1.1.3.3.4.cmml">s</mi><mo id="S6.Ex1.m1.1.1.1.1.3.3.1b" xref="S6.Ex1.m1.1.1.1.1.3.3.1.cmml">⁢</mo><mi id="S6.Ex1.m1.1.1.1.1.3.3.5" xref="S6.Ex1.m1.1.1.1.1.3.3.5.cmml">t</mi></mrow></msub><mo id="S6.Ex1.m1.1.1.1.1.2" xref="S6.Ex1.m1.1.1.1.1.2.cmml">=</mo><mrow id="S6.Ex1.m1.1.1.1.1.1" xref="S6.Ex1.m1.1.1.1.1.1.cmml"><mrow id="S6.Ex1.m1.1.1.1.1.1.3" xref="S6.Ex1.m1.1.1.1.1.1.3.cmml"><msub id="S6.Ex1.m1.1.1.1.1.1.3.2" xref="S6.Ex1.m1.1.1.1.1.1.3.2.cmml"><mi id="S6.Ex1.m1.1.1.1.1.1.3.2.2" xref="S6.Ex1.m1.1.1.1.1.1.3.2.2.cmml">I</mi><mi id="S6.Ex1.m1.1.1.1.1.1.3.2.3" xref="S6.Ex1.m1.1.1.1.1.1.3.2.3.cmml">c</mi></msub><mo id="S6.Ex1.m1.1.1.1.1.1.3.1" xref="S6.Ex1.m1.1.1.1.1.1.3.1.cmml">+</mo><msub id="S6.Ex1.m1.1.1.1.1.1.3.3" xref="S6.Ex1.m1.1.1.1.1.1.3.3.cmml"><mi id="S6.Ex1.m1.1.1.1.1.1.3.3.2" xref="S6.Ex1.m1.1.1.1.1.1.3.3.2.cmml">ξ</mi><mi id="S6.Ex1.m1.1.1.1.1.1.3.3.3" xref="S6.Ex1.m1.1.1.1.1.1.3.3.3.cmml">i</mi></msub><mo id="S6.Ex1.m1.1.1.1.1.1.3.1a" xref="S6.Ex1.m1.1.1.1.1.1.3.1.cmml">+</mo><mrow id="S6.Ex1.m1.1.1.1.1.1.3.4" xref="S6.Ex1.m1.1.1.1.1.1.3.4.cmml"><msub id="S6.Ex1.m1.1.1.1.1.1.3.4.2" xref="S6.Ex1.m1.1.1.1.1.1.3.4.2.cmml"><mi id="S6.Ex1.m1.1.1.1.1.1.3.4.2.2" xref="S6.Ex1.m1.1.1.1.1.1.3.4.2.2.cmml">k</mi><mi id="S6.Ex1.m1.1.1.1.1.1.3.4.2.3" xref="S6.Ex1.m1.1.1.1.1.1.3.4.2.3.cmml">i</mi></msub><mo id="S6.Ex1.m1.1.1.1.1.1.3.4.1" xref="S6.Ex1.m1.1.1.1.1.1.3.4.1.cmml">⋅</mo><mi id="S6.Ex1.m1.1.1.1.1.1.3.4.3" xref="S6.Ex1.m1.1.1.1.1.1.3.4.3.cmml">X</mi></mrow></mrow><mo id="S6.Ex1.m1.1.1.1.1.1.2" xref="S6.Ex1.m1.1.1.1.1.1.2.cmml">-</mo><mrow id="S6.Ex1.m1.1.1.1.1.1.1" xref="S6.Ex1.m1.1.1.1.1.1.1.cmml"><mover accent="true" id="S6.Ex1.m1.1.1.1.1.1.1.3" xref="S6.Ex1.m1.1.1.1.1.1.1.3.cmml"><mi id="S6.Ex1.m1.1.1.1.1.1.1.3.2" xref="S6.Ex1.m1.1.1.1.1.1.1.3.2.cmml">ξ</mi><mo id="S6.Ex1.m1.1.1.1.1.1.1.3.1" xref="S6.Ex1.m1.1.1.1.1.1.1.3.1.cmml">˙</mo></mover><mo id="S6.Ex1.m1.1.1.1.1.1.1.2" xref="S6.Ex1.m1.1.1.1.1.1.1.2.cmml">⋅</mo><mrow id="S6.Ex1.m1.1.1.1.1.1.1.1.1" xref="S6.Ex1.m1.1.1.1.1.1.1.1.1.1.cmml"><mo stretchy="false" id="S6.Ex1.m1.1.1.1.1.1.1.1.1.2" xref="S6.Ex1.m1.1.1.1.1.1.1.1.1.1.cmml">(</mo><mrow id="S6.Ex1.m1.1.1.1.1.1.1.1.1.1" xref="S6.Ex1.m1.1.1.1.1.1.1.1.1.1.cmml"><mi id="S6.Ex1.m1.1.1.1.1.1.1.1.1.1.2" xref="S6.Ex1.m1.1.1.1.1.1.1.1.1.1.2.cmml">t</mi><mo id="S6.Ex1.m1.1.1.1.1.1.1.1.1.1.1" xref="S6.Ex1.m1.1.1.1.1.1.1.1.1.1.1.cmml">-</mo><msub id="S6.Ex1.m1.1.1.1.1.1.1.1.1.1.3" xref="S6.Ex1.m1.1.1.1.1.1.1.1.1.1.3.cmml"><mi id="S6.Ex1.m1.1.1.1.1.1.1.1.1.1.3.2" xref="S6.Ex1.m1.1.1.1.1.1.1.1.1.1.3.2.cmml">t</mi><mn id="S6.Ex1.m1.1.1.1.1.1.1.1.1.1.3.3" xref="S6.Ex1.m1.1.1.1.1.1.1.1.1.1.3.3.cmml">0</mn></msub></mrow><mo stretchy="false" id="S6.Ex1.m1.1.1.1.1.1.1.1.1.3" xref="S6.Ex1.m1.1.1.1.1.1.1.1.1.1.cmml">)</mo></mrow></mrow></mrow></mrow><mo id="S6.Ex1.m1.1.1.1.2" xref="S6.Ex1.m1.1.1.1.1.cmml">,</mo></mrow><annotation-xml encoding="MathML-Content" id="S6.Ex1.m1.1b"><apply id="S6.Ex1.m1.1.1.1.1.cmml" xref="S6.Ex1.m1.1.1.1"><eq id="S6.Ex1.m1.1.1.1.1.2.cmml" xref="S6.Ex1.m1.1.1.1.1.2"></eq><apply id="S6.Ex1.m1.1.1.1.1.3.cmml" xref="S6.Ex1.m1.1.1.1.1.3"><csymbol cd="ambiguous" id="S6.Ex1.m1.1.1.1.1.3.1.cmml" xref="S6.Ex1.m1.1.1.1.1.3">subscript</csymbol><ci id="S6.Ex1.m1.1.1.1.1.3.2.cmml" xref="S6.Ex1.m1.1.1.1.1.3.2">𝑖</ci><apply id="S6.Ex1.m1.1.1.1.1.3.3.cmml" xref="S6.Ex1.m1.1.1.1.1.3.3"><times id="S6.Ex1.m1.1.1.1.1.3.3.1.cmml" xref="S6.Ex1.m1.1.1.1.1.3.3.1"></times><ci id="S6.Ex1.m1.1.1.1.1.3.3.2.cmml" xref="S6.Ex1.m1.1.1.1.1.3.3.2">𝑖</ci><ci id="S6.Ex1.m1.1.1.1.1.3.3.3.cmml" xref="S6.Ex1.m1.1.1.1.1.3.3.3">𝑛</ci><ci id="S6.Ex1.m1.1.1.1.1.3.3.4.cmml" xref="S6.Ex1.m1.1.1.1.1.3.3.4">𝑠</ci><ci id="S6.Ex1.m1.1.1.1.1.3.3.5.cmml" xref="S6.Ex1.m1.1.1.1.1.3.3.5">𝑡</ci></apply></apply><apply id="S6.Ex1.m1.1.1.1.1.1.cmml" xref="S6.Ex1.m1.1.1.1.1.1"><minus id="S6.Ex1.m1.1.1.1.1.1.2.cmml" xref="S6.Ex1.m1.1.1.1.1.1.2"></minus><apply id="S6.Ex1.m1.1.1.1.1.1.3.cmml" xref="S6.Ex1.m1.1.1.1.1.1.3"><plus id="S6.Ex1.m1.1.1.1.1.1.3.1.cmml" xref="S6.Ex1.m1.1.1.1.1.1.3.1"></plus><apply id="S6.Ex1.m1.1.1.1.1.1.3.2.cmml" xref="S6.Ex1.m1.1.1.1.1.1.3.2"><csymbol cd="ambiguous" id="S6.Ex1.m1.1.1.1.1.1.3.2.1.cmml" xref="S6.Ex1.m1.1.1.1.1.1.3.2">subscript</csymbol><ci id="S6.Ex1.m1.1.1.1.1.1.3.2.2.cmml" xref="S6.Ex1.m1.1.1.1.1.1.3.2.2">𝐼</ci><ci id="S6.Ex1.m1.1.1.1.1.1.3.2.3.cmml" xref="S6.Ex1.m1.1.1.1.1.1.3.2.3">𝑐</ci></apply><apply id="S6.Ex1.m1.1.1.1.1.1.3.3.cmml" xref="S6.Ex1.m1.1.1.1.1.1.3.3"><csymbol cd="ambiguous" id="S6.Ex1.m1.1.1.1.1.1.3.3.1.cmml" xref="S6.Ex1.m1.1.1.1.1.1.3.3">subscript</csymbol><ci id="S6.Ex1.m1.1.1.1.1.1.3.3.2.cmml" xref="S6.Ex1.m1.1.1.1.1.1.3.3.2">𝜉</ci><ci id="S6.Ex1.m1.1.1.1.1.1.3.3.3.cmml" xref="S6.Ex1.m1.1.1.1.1.1.3.3.3">𝑖</ci></apply><apply id="S6.Ex1.m1.1.1.1.1.1.3.4.cmml" xref="S6.Ex1.m1.1.1.1.1.1.3.4"><ci id="S6.Ex1.m1.1.1.1.1.1.3.4.1.cmml" xref="S6.Ex1.m1.1.1.1.1.1.3.4.1">⋅</ci><apply id="S6.Ex1.m1.1.1.1.1.1.3.4.2.cmml" xref="S6.Ex1.m1.1.1.1.1.1.3.4.2"><csymbol cd="ambiguous" id="S6.Ex1.m1.1.1.1.1.1.3.4.2.1.cmml" xref="S6.Ex1.m1.1.1.1.1.1.3.4.2">subscript</csymbol><ci id="S6.Ex1.m1.1.1.1.1.1.3.4.2.2.cmml" xref="S6.Ex1.m1.1.1.1.1.1.3.4.2.2">𝑘</ci><ci id="S6.Ex1.m1.1.1.1.1.1.3.4.2.3.cmml" xref="S6.Ex1.m1.1.1.1.1.1.3.4.2.3">𝑖</ci></apply><ci id="S6.Ex1.m1.1.1.1.1.1.3.4.3.cmml" xref="S6.Ex1.m1.1.1.1.1.1.3.4.3">𝑋</ci></apply></apply><apply id="S6.Ex1.m1.1.1.1.1.1.1.cmml" xref="S6.Ex1.m1.1.1.1.1.1.1"><ci id="S6.Ex1.m1.1.1.1.1.1.1.2.cmml" xref="S6.Ex1.m1.1.1.1.1.1.1.2">⋅</ci><apply id="S6.Ex1.m1.1.1.1.1.1.1.3.cmml" xref="S6.Ex1.m1.1.1.1.1.1.1.3"><ci id="S6.Ex1.m1.1.1.1.1.1.1.3.1.cmml" xref="S6.Ex1.m1.1.1.1.1.1.1.3.1">˙</ci><ci id="S6.Ex1.m1.1.1.1.1.1.1.3.2.cmml" xref="S6.Ex1.m1.1.1.1.1.1.1.3.2">𝜉</ci></apply><apply id="S6.Ex1.m1.1.1.1.1.1.1.1.1.1.cmml" xref="S6.Ex1.m1.1.1.1.1.1.1.1.1"><minus id="S6.Ex1.m1.1.1.1.1.1.1.1.1.1.1.cmml" xref="S6.Ex1.m1.1.1.1.1.1.1.1.1.1.1"></minus><ci id="S6.Ex1.m1.1.1.1.1.1.1.1.1.1.2.cmml" xref="S6.Ex1.m1.1.1.1.1.1.1.1.1.1.2">𝑡</ci><apply id="S6.Ex1.m1.1.1.1.1.1.1.1.1.1.3.cmml" xref="S6.Ex1.m1.1.1.1.1.1.1.1.1.1.3"><csymbol cd="ambiguous" id="S6.Ex1.m1.1.1.1.1.1.1.1.1.1.3.1.cmml" xref="S6.Ex1.m1.1.1.1.1.1.1.1.1.1.3">subscript</csymbol><ci id="S6.Ex1.m1.1.1.1.1.1.1.1.1.1.3.2.cmml" xref="S6.Ex1.m1.1.1.1.1.1.1.1.1.1.3.2">𝑡</ci><cn type="integer" id="S6.Ex1.m1.1.1.1.1.1.1.1.1.1.3.3.cmml" xref="S6.Ex1.m1.1.1.1.1.1.1.1.1.1.3.3">0</cn></apply></apply></apply></apply></apply></annotation-xml><annotation encoding="application/x-tex" id="S6.Ex1.m1.1c">i_{inst}=I_{c}+\xi_{i}+k_{i}\cdot X-\dot{\xi}\cdot(t-t_{0}),</annotation><annotation encoding="application/x-llamapun" id="S6.Ex1.m1.1d">italic_i start_POSTSUBSCRIPT italic_i italic_n italic_s italic_t end_POSTSUBSCRIPT = italic_I start_POSTSUBSCRIPT italic_c end_POSTSUBSCRIPT + italic_ξ start_POSTSUBSCRIPT italic_i end_POSTSUBSCRIPT + italic_k start_POSTSUBSCRIPT italic_i end_POSTSUBSCRIPT ⋅ italic_X - ˙ start_ARG italic_ξ end_ARG ⋅ ( italic_t - italic_t start_POSTSUBSCRIPT 0 end_POSTSUBSCRIPT ) ,</annotation></semantics></math></td>
<td class="ltx_eqn_cell ltx_eqn_center_padright"></td>
</tr>
</table>
</div>
<div id="S6.p5" class="ltx_para">
<p id="S6.p5.8" class="ltx_p">where <math id="S6.p5.1.m1.1" class="ltx_Math" alttext="i_{inst}" display="inline"><semantics id="S6.p5.1.m1.1a"><msub id="S6.p5.1.m1.1.1" xref="S6.p5.1.m1.1.1.cmml"><mi id="S6.p5.1.m1.1.1.2" xref="S6.p5.1.m1.1.1.2.cmml">i</mi><mrow id="S6.p5.1.m1.1.1.3" xref="S6.p5.1.m1.1.1.3.cmml"><mi id="S6.p5.1.m1.1.1.3.2" xref="S6.p5.1.m1.1.1.3.2.cmml">i</mi><mo id="S6.p5.1.m1.1.1.3.1" xref="S6.p5.1.m1.1.1.3.1.cmml">⁢</mo><mi id="S6.p5.1.m1.1.1.3.3" xref="S6.p5.1.m1.1.1.3.3.cmml">n</mi><mo id="S6.p5.1.m1.1.1.3.1a" xref="S6.p5.1.m1.1.1.3.1.cmml">⁢</mo><mi id="S6.p5.1.m1.1.1.3.4" xref="S6.p5.1.m1.1.1.3.4.cmml">s</mi><mo id="S6.p5.1.m1.1.1.3.1b" xref="S6.p5.1.m1.1.1.3.1.cmml">⁢</mo><mi id="S6.p5.1.m1.1.1.3.5" xref="S6.p5.1.m1.1.1.3.5.cmml">t</mi></mrow></msub><annotation-xml encoding="MathML-Content" id="S6.p5.1.m1.1b"><apply id="S6.p5.1.m1.1.1.cmml" xref="S6.p5.1.m1.1.1"><csymbol cd="ambiguous" id="S6.p5.1.m1.1.1.1.cmml" xref="S6.p5.1.m1.1.1">subscript</csymbol><ci id="S6.p5.1.m1.1.1.2.cmml" xref="S6.p5.1.m1.1.1.2">𝑖</ci><apply id="S6.p5.1.m1.1.1.3.cmml" xref="S6.p5.1.m1.1.1.3"><times id="S6.p5.1.m1.1.1.3.1.cmml" xref="S6.p5.1.m1.1.1.3.1"></times><ci id="S6.p5.1.m1.1.1.3.2.cmml" xref="S6.p5.1.m1.1.1.3.2">𝑖</ci><ci id="S6.p5.1.m1.1.1.3.3.cmml" xref="S6.p5.1.m1.1.1.3.3">𝑛</ci><ci id="S6.p5.1.m1.1.1.3.4.cmml" xref="S6.p5.1.m1.1.1.3.4">𝑠</ci><ci id="S6.p5.1.m1.1.1.3.5.cmml" xref="S6.p5.1.m1.1.1.3.5">𝑡</ci></apply></apply></annotation-xml><annotation encoding="application/x-tex" id="S6.p5.1.m1.1c">i_{inst}</annotation><annotation encoding="application/x-llamapun" id="S6.p5.1.m1.1d">italic_i start_POSTSUBSCRIPT italic_i italic_n italic_s italic_t end_POSTSUBSCRIPT</annotation></semantics></math> and <math id="S6.p5.2.m2.1" class="ltx_Math" alttext="I_{c}" display="inline"><semantics id="S6.p5.2.m2.1a"><msub id="S6.p5.2.m2.1.1" xref="S6.p5.2.m2.1.1.cmml"><mi id="S6.p5.2.m2.1.1.2" xref="S6.p5.2.m2.1.1.2.cmml">I</mi><mi id="S6.p5.2.m2.1.1.3" xref="S6.p5.2.m2.1.1.3.cmml">c</mi></msub><annotation-xml encoding="MathML-Content" id="S6.p5.2.m2.1b"><apply id="S6.p5.2.m2.1.1.cmml" xref="S6.p5.2.m2.1.1"><csymbol cd="ambiguous" id="S6.p5.2.m2.1.1.1.cmml" xref="S6.p5.2.m2.1.1">subscript</csymbol><ci id="S6.p5.2.m2.1.1.2.cmml" xref="S6.p5.2.m2.1.1.2">𝐼</ci><ci id="S6.p5.2.m2.1.1.3.cmml" xref="S6.p5.2.m2.1.1.3">𝑐</ci></apply></annotation-xml><annotation encoding="application/x-tex" id="S6.p5.2.m2.1c">I_{c}</annotation><annotation encoding="application/x-llamapun" id="S6.p5.2.m2.1d">italic_I start_POSTSUBSCRIPT italic_c end_POSTSUBSCRIPT</annotation></semantics></math> are the instrumental and (close to) standard
magnitudes, <math id="S6.p5.3.m3.1" class="ltx_Math" alttext="X" display="inline"><semantics id="S6.p5.3.m3.1a"><mi id="S6.p5.3.m3.1.1" xref="S6.p5.3.m3.1.1.cmml">X</mi><annotation-xml encoding="MathML-Content" id="S6.p5.3.m3.1b"><ci id="S6.p5.3.m3.1.1.cmml" xref="S6.p5.3.m3.1.1">𝑋</ci></annotation-xml><annotation encoding="application/x-tex" id="S6.p5.3.m3.1c">X</annotation><annotation encoding="application/x-llamapun" id="S6.p5.3.m3.1d">italic_X</annotation></semantics></math> is the airmass, <math id="S6.p5.4.m4.1" class="ltx_Math" alttext="t" display="inline"><semantics id="S6.p5.4.m4.1a"><mi id="S6.p5.4.m4.1.1" xref="S6.p5.4.m4.1.1.cmml">t</mi><annotation-xml encoding="MathML-Content" id="S6.p5.4.m4.1b"><ci id="S6.p5.4.m4.1.1.cmml" xref="S6.p5.4.m4.1.1">𝑡</ci></annotation-xml><annotation encoding="application/x-tex" id="S6.p5.4.m4.1c">t</annotation><annotation encoding="application/x-llamapun" id="S6.p5.4.m4.1d">italic_t</annotation></semantics></math> is the time from an arbitrary epoch
<math id="S6.p5.5.m5.1" class="ltx_Math" alttext="t_{0}" display="inline"><semantics id="S6.p5.5.m5.1a"><msub id="S6.p5.5.m5.1.1" xref="S6.p5.5.m5.1.1.cmml"><mi id="S6.p5.5.m5.1.1.2" xref="S6.p5.5.m5.1.1.2.cmml">t</mi><mn id="S6.p5.5.m5.1.1.3" xref="S6.p5.5.m5.1.1.3.cmml">0</mn></msub><annotation-xml encoding="MathML-Content" id="S6.p5.5.m5.1b"><apply id="S6.p5.5.m5.1.1.cmml" xref="S6.p5.5.m5.1.1"><csymbol cd="ambiguous" id="S6.p5.5.m5.1.1.1.cmml" xref="S6.p5.5.m5.1.1">subscript</csymbol><ci id="S6.p5.5.m5.1.1.2.cmml" xref="S6.p5.5.m5.1.1.2">𝑡</ci><cn type="integer" id="S6.p5.5.m5.1.1.3.cmml" xref="S6.p5.5.m5.1.1.3">0</cn></apply></annotation-xml><annotation encoding="application/x-tex" id="S6.p5.5.m5.1c">t_{0}</annotation><annotation encoding="application/x-llamapun" id="S6.p5.5.m5.1d">italic_t start_POSTSUBSCRIPT 0 end_POSTSUBSCRIPT</annotation></semantics></math>, and <math id="S6.p5.6.m6.1" class="ltx_Math" alttext="\xi_{i}" display="inline"><semantics id="S6.p5.6.m6.1a"><msub id="S6.p5.6.m6.1.1" xref="S6.p5.6.m6.1.1.cmml"><mi id="S6.p5.6.m6.1.1.2" xref="S6.p5.6.m6.1.1.2.cmml">ξ</mi><mi id="S6.p5.6.m6.1.1.3" xref="S6.p5.6.m6.1.1.3.cmml">i</mi></msub><annotation-xml encoding="MathML-Content" id="S6.p5.6.m6.1b"><apply id="S6.p5.6.m6.1.1.cmml" xref="S6.p5.6.m6.1.1"><csymbol cd="ambiguous" id="S6.p5.6.m6.1.1.1.cmml" xref="S6.p5.6.m6.1.1">subscript</csymbol><ci id="S6.p5.6.m6.1.1.2.cmml" xref="S6.p5.6.m6.1.1.2">𝜉</ci><ci id="S6.p5.6.m6.1.1.3.cmml" xref="S6.p5.6.m6.1.1.3">𝑖</ci></apply></annotation-xml><annotation encoding="application/x-tex" id="S6.p5.6.m6.1c">\xi_{i}</annotation><annotation encoding="application/x-llamapun" id="S6.p5.6.m6.1d">italic_ξ start_POSTSUBSCRIPT italic_i end_POSTSUBSCRIPT</annotation></semantics></math>, <math id="S6.p5.7.m7.1" class="ltx_Math" alttext="k_{i}" display="inline"><semantics id="S6.p5.7.m7.1a"><msub id="S6.p5.7.m7.1.1" xref="S6.p5.7.m7.1.1.cmml"><mi id="S6.p5.7.m7.1.1.2" xref="S6.p5.7.m7.1.1.2.cmml">k</mi><mi id="S6.p5.7.m7.1.1.3" xref="S6.p5.7.m7.1.1.3.cmml">i</mi></msub><annotation-xml encoding="MathML-Content" id="S6.p5.7.m7.1b"><apply id="S6.p5.7.m7.1.1.cmml" xref="S6.p5.7.m7.1.1"><csymbol cd="ambiguous" id="S6.p5.7.m7.1.1.1.cmml" xref="S6.p5.7.m7.1.1">subscript</csymbol><ci id="S6.p5.7.m7.1.1.2.cmml" xref="S6.p5.7.m7.1.1.2">𝑘</ci><ci id="S6.p5.7.m7.1.1.3.cmml" xref="S6.p5.7.m7.1.1.3">𝑖</ci></apply></annotation-xml><annotation encoding="application/x-tex" id="S6.p5.7.m7.1c">k_{i}</annotation><annotation encoding="application/x-llamapun" id="S6.p5.7.m7.1d">italic_k start_POSTSUBSCRIPT italic_i end_POSTSUBSCRIPT</annotation></semantics></math>, <math id="S6.p5.8.m8.1" class="ltx_Math" alttext="\dot{\xi}" display="inline"><semantics id="S6.p5.8.m8.1a"><mover accent="true" id="S6.p5.8.m8.1.1" xref="S6.p5.8.m8.1.1.cmml"><mi id="S6.p5.8.m8.1.1.2" xref="S6.p5.8.m8.1.1.2.cmml">ξ</mi><mo id="S6.p5.8.m8.1.1.1" xref="S6.p5.8.m8.1.1.1.cmml">˙</mo></mover><annotation-xml encoding="MathML-Content" id="S6.p5.8.m8.1b"><apply id="S6.p5.8.m8.1.1.cmml" xref="S6.p5.8.m8.1.1"><ci id="S6.p5.8.m8.1.1.1.cmml" xref="S6.p5.8.m8.1.1.1">˙</ci><ci id="S6.p5.8.m8.1.1.2.cmml" xref="S6.p5.8.m8.1.1.2">𝜉</ci></apply></annotation-xml><annotation encoding="application/x-tex" id="S6.p5.8.m8.1c">\dot{\xi}</annotation><annotation encoding="application/x-llamapun" id="S6.p5.8.m8.1d">˙ start_ARG italic_ξ end_ARG</annotation></semantics></math> are coefficients to be
determined by the regression.</p>
</div>
<div id="S6.p6" class="ltx_para">
<p id="S6.p6.1" class="ltx_p">Observation of Landolt standards is performed by the “standard” task,
which selects suitable standard stars (bright enough, not extreme
colors, have large airmass and hour-angle span), and observes the star
with fine time and airmass resolution plus the culminating fields. This
task is launched maximum few times a month, during
absolute-photometric, new-moon nights.</p>
</div>
<div id="S6.p7" class="ltx_para">
<p id="S6.p7.2" class="ltx_p">Current data flow with lossless compression is <math id="S6.p7.1.m1.1" class="ltx_Math" alttext="\rm\sim 1.6Gb/day" display="inline"><semantics id="S6.p7.1.m1.1a"><mrow id="S6.p7.1.m1.1.1" xref="S6.p7.1.m1.1.1.cmml"><mi id="S6.p7.1.m1.1.1.2" xref="S6.p7.1.m1.1.1.2.cmml"></mi><mo id="S6.p7.1.m1.1.1.1" xref="S6.p7.1.m1.1.1.1.cmml">∼</mo><mrow id="S6.p7.1.m1.1.1.3" xref="S6.p7.1.m1.1.1.3.cmml"><mrow id="S6.p7.1.m1.1.1.3.2" xref="S6.p7.1.m1.1.1.3.2.cmml"><mn id="S6.p7.1.m1.1.1.3.2.2" xref="S6.p7.1.m1.1.1.3.2.2.cmml">1.6</mn><mo id="S6.p7.1.m1.1.1.3.2.1" xref="S6.p7.1.m1.1.1.3.2.1.cmml">⁢</mo><mi id="S6.p7.1.m1.1.1.3.2.3" xref="S6.p7.1.m1.1.1.3.2.3.cmml">Gb</mi></mrow><mo id="S6.p7.1.m1.1.1.3.1" xref="S6.p7.1.m1.1.1.3.1.cmml">/</mo><mi id="S6.p7.1.m1.1.1.3.3" xref="S6.p7.1.m1.1.1.3.3.cmml">day</mi></mrow></mrow><annotation-xml encoding="MathML-Content" id="S6.p7.1.m1.1b"><apply id="S6.p7.1.m1.1.1.cmml" xref="S6.p7.1.m1.1.1"><csymbol cd="latexml" id="S6.p7.1.m1.1.1.1.cmml" xref="S6.p7.1.m1.1.1.1">similar-to</csymbol><csymbol cd="latexml" id="S6.p7.1.m1.1.1.2.cmml" xref="S6.p7.1.m1.1.1.2">absent</csymbol><apply id="S6.p7.1.m1.1.1.3.cmml" xref="S6.p7.1.m1.1.1.3"><divide id="S6.p7.1.m1.1.1.3.1.cmml" xref="S6.p7.1.m1.1.1.3.1"></divide><apply id="S6.p7.1.m1.1.1.3.2.cmml" xref="S6.p7.1.m1.1.1.3.2"><times id="S6.p7.1.m1.1.1.3.2.1.cmml" xref="S6.p7.1.m1.1.1.3.2.1"></times><cn type="float" id="S6.p7.1.m1.1.1.3.2.2.cmml" xref="S6.p7.1.m1.1.1.3.2.2">1.6</cn><ci id="S6.p7.1.m1.1.1.3.2.3.cmml" xref="S6.p7.1.m1.1.1.3.2.3">Gb</ci></apply><ci id="S6.p7.1.m1.1.1.3.3.cmml" xref="S6.p7.1.m1.1.1.3.3">day</ci></apply></apply></annotation-xml><annotation encoding="application/x-tex" id="S6.p7.1.m1.1c">\rm\sim 1.6Gb/day</annotation><annotation encoding="application/x-llamapun" id="S6.p7.1.m1.1d">∼ 1.6 roman_Gb / roman_day</annotation></semantics></math>
(winter) to <math id="S6.p7.2.m2.1" class="ltx_Math" alttext="\rm 1Gb/day" display="inline"><semantics id="S6.p7.2.m2.1a"><mrow id="S6.p7.2.m2.1.1" xref="S6.p7.2.m2.1.1.cmml"><mrow id="S6.p7.2.m2.1.1.2" xref="S6.p7.2.m2.1.1.2.cmml"><mn id="S6.p7.2.m2.1.1.2.2" xref="S6.p7.2.m2.1.1.2.2.cmml">1</mn><mo id="S6.p7.2.m2.1.1.2.1" xref="S6.p7.2.m2.1.1.2.1.cmml">⁢</mo><mi mathvariant="normal" id="S6.p7.2.m2.1.1.2.3" xref="S6.p7.2.m2.1.1.2.3.cmml">G</mi><mo id="S6.p7.2.m2.1.1.2.1a" xref="S6.p7.2.m2.1.1.2.1.cmml">⁢</mo><mi mathvariant="normal" id="S6.p7.2.m2.1.1.2.4" xref="S6.p7.2.m2.1.1.2.4.cmml">b</mi></mrow><mo id="S6.p7.2.m2.1.1.1" xref="S6.p7.2.m2.1.1.1.cmml">/</mo><mi id="S6.p7.2.m2.1.1.3" xref="S6.p7.2.m2.1.1.3.cmml">day</mi></mrow><annotation-xml encoding="MathML-Content" id="S6.p7.2.m2.1b"><apply id="S6.p7.2.m2.1.1.cmml" xref="S6.p7.2.m2.1.1"><divide id="S6.p7.2.m2.1.1.1.cmml" xref="S6.p7.2.m2.1.1.1"></divide><apply id="S6.p7.2.m2.1.1.2.cmml" xref="S6.p7.2.m2.1.1.2"><times id="S6.p7.2.m2.1.1.2.1.cmml" xref="S6.p7.2.m2.1.1.2.1"></times><cn type="integer" id="S6.p7.2.m2.1.1.2.2.cmml" xref="S6.p7.2.m2.1.1.2.2">1</cn><ci id="S6.p7.2.m2.1.1.2.3.cmml" xref="S6.p7.2.m2.1.1.2.3">G</ci><ci id="S6.p7.2.m2.1.1.2.4.cmml" xref="S6.p7.2.m2.1.1.2.4">b</ci></apply><ci id="S6.p7.2.m2.1.1.3.cmml" xref="S6.p7.2.m2.1.1.3">day</ci></apply></annotation-xml><annotation encoding="application/x-tex" id="S6.p7.2.m2.1c">\rm 1Gb/day</annotation><annotation encoding="application/x-llamapun" id="S6.p7.2.m2.1d">1 roman_G roman_b / roman_day</annotation></semantics></math> (summer), thus new DAT DDS-3 tapes have to be
inserted every week.</p>
</div>
<div id="S6.p8" class="ltx_para">
<p id="S6.p8.1" class="ltx_p">The only unautomated procedures in HAT operation are checking weather
status at evening (rain is detected automatically, but clouds not) and
tape changing.</p>
</div>
<div id="S6.p9" class="ltx_para">
<p id="S6.p9.2" class="ltx_p">During a year’s operation HAT has completed approx. <math id="S6.p9.1.m1.1" class="ltx_Math" alttext="140" display="inline"><semantics id="S6.p9.1.m1.1a"><mn id="S6.p9.1.m1.1.1" xref="S6.p9.1.m1.1.1.cmml">140</mn><annotation-xml encoding="MathML-Content" id="S6.p9.1.m1.1b"><cn type="integer" id="S6.p9.1.m1.1.1.cmml" xref="S6.p9.1.m1.1.1">140</cn></annotation-xml><annotation encoding="application/x-tex" id="S6.p9.1.m1.1c">140</annotation><annotation encoding="application/x-llamapun" id="S6.p9.1.m1.1d">140</annotation></semantics></math> observing
sessions, with a total number of <math id="S6.p9.2.m2.1" class="ltx_Math" alttext="\sim 35000" display="inline"><semantics id="S6.p9.2.m2.1a"><mrow id="S6.p9.2.m2.1.1" xref="S6.p9.2.m2.1.1.cmml"><mi id="S6.p9.2.m2.1.1.2" xref="S6.p9.2.m2.1.1.2.cmml"></mi><mo id="S6.p9.2.m2.1.1.1" xref="S6.p9.2.m2.1.1.1.cmml">∼</mo><mn id="S6.p9.2.m2.1.1.3" xref="S6.p9.2.m2.1.1.3.cmml">35000</mn></mrow><annotation-xml encoding="MathML-Content" id="S6.p9.2.m2.1b"><apply id="S6.p9.2.m2.1.1.cmml" xref="S6.p9.2.m2.1.1"><csymbol cd="latexml" id="S6.p9.2.m2.1.1.1.cmml" xref="S6.p9.2.m2.1.1.1">similar-to</csymbol><csymbol cd="latexml" id="S6.p9.2.m2.1.1.2.cmml" xref="S6.p9.2.m2.1.1.2">absent</csymbol><cn type="integer" id="S6.p9.2.m2.1.1.3.cmml" xref="S6.p9.2.m2.1.1.3">35000</cn></apply></annotation-xml><annotation encoding="application/x-tex" id="S6.p9.2.m2.1c">\sim 35000</annotation><annotation encoding="application/x-llamapun" id="S6.p9.2.m2.1d">∼ 35000</annotation></semantics></math> exposures of which 19000
were field observations, the rest were calibration frames.</p>
</div>
</section>
<section id="S7" class="ltx_section">
<h2 class="ltx_title ltx_title_section">
<span class="ltx_tag ltx_tag_section">7 </span>Photometric precision of HAT</h2>

<div id="S7.p1" class="ltx_para">
<p id="S7.p1.5" class="ltx_p">Reduction of the current 160Gb of data is under way. Photometric
precision of HAT-1 at Kitt Peak was tested using a 12-night subset of
images from September/October 2001 for the moderately crowded field
“F077” (<math id="S7.p1.1.m1.1" class="ltx_Math" alttext="\rm\alpha=23^{h}15^{m}00^{s}" display="inline"><semantics id="S7.p1.1.m1.1a"><mrow id="S7.p1.1.m1.1.1" xref="S7.p1.1.m1.1.1.cmml"><mi id="S7.p1.1.m1.1.1.2" xref="S7.p1.1.m1.1.1.2.cmml">α</mi><mo id="S7.p1.1.m1.1.1.1" xref="S7.p1.1.m1.1.1.1.cmml">=</mo><mrow id="S7.p1.1.m1.1.1.3" xref="S7.p1.1.m1.1.1.3.cmml"><msup id="S7.p1.1.m1.1.1.3.2" xref="S7.p1.1.m1.1.1.3.2.cmml"><mn id="S7.p1.1.m1.1.1.3.2.2" xref="S7.p1.1.m1.1.1.3.2.2.cmml">23</mn><mi mathvariant="normal" id="S7.p1.1.m1.1.1.3.2.3" xref="S7.p1.1.m1.1.1.3.2.3.cmml">h</mi></msup><mo id="S7.p1.1.m1.1.1.3.1" xref="S7.p1.1.m1.1.1.3.1.cmml">⁢</mo><msup id="S7.p1.1.m1.1.1.3.3" xref="S7.p1.1.m1.1.1.3.3.cmml"><mn id="S7.p1.1.m1.1.1.3.3.2" xref="S7.p1.1.m1.1.1.3.3.2.cmml">15</mn><mi mathvariant="normal" id="S7.p1.1.m1.1.1.3.3.3" xref="S7.p1.1.m1.1.1.3.3.3.cmml">m</mi></msup><mo id="S7.p1.1.m1.1.1.3.1a" xref="S7.p1.1.m1.1.1.3.1.cmml">⁢</mo><msup id="S7.p1.1.m1.1.1.3.4" xref="S7.p1.1.m1.1.1.3.4.cmml"><mn id="S7.p1.1.m1.1.1.3.4.2" xref="S7.p1.1.m1.1.1.3.4.2.cmml">00</mn><mi mathvariant="normal" id="S7.p1.1.m1.1.1.3.4.3" xref="S7.p1.1.m1.1.1.3.4.3.cmml">s</mi></msup></mrow></mrow><annotation-xml encoding="MathML-Content" id="S7.p1.1.m1.1b"><apply id="S7.p1.1.m1.1.1.cmml" xref="S7.p1.1.m1.1.1"><eq id="S7.p1.1.m1.1.1.1.cmml" xref="S7.p1.1.m1.1.1.1"></eq><ci id="S7.p1.1.m1.1.1.2.cmml" xref="S7.p1.1.m1.1.1.2">𝛼</ci><apply id="S7.p1.1.m1.1.1.3.cmml" xref="S7.p1.1.m1.1.1.3"><times id="S7.p1.1.m1.1.1.3.1.cmml" xref="S7.p1.1.m1.1.1.3.1"></times><apply id="S7.p1.1.m1.1.1.3.2.cmml" xref="S7.p1.1.m1.1.1.3.2"><csymbol cd="ambiguous" id="S7.p1.1.m1.1.1.3.2.1.cmml" xref="S7.p1.1.m1.1.1.3.2">superscript</csymbol><cn type="integer" id="S7.p1.1.m1.1.1.3.2.2.cmml" xref="S7.p1.1.m1.1.1.3.2.2">23</cn><ci id="S7.p1.1.m1.1.1.3.2.3.cmml" xref="S7.p1.1.m1.1.1.3.2.3">h</ci></apply><apply id="S7.p1.1.m1.1.1.3.3.cmml" xref="S7.p1.1.m1.1.1.3.3"><csymbol cd="ambiguous" id="S7.p1.1.m1.1.1.3.3.1.cmml" xref="S7.p1.1.m1.1.1.3.3">superscript</csymbol><cn type="integer" id="S7.p1.1.m1.1.1.3.3.2.cmml" xref="S7.p1.1.m1.1.1.3.3.2">15</cn><ci id="S7.p1.1.m1.1.1.3.3.3.cmml" xref="S7.p1.1.m1.1.1.3.3.3">m</ci></apply><apply id="S7.p1.1.m1.1.1.3.4.cmml" xref="S7.p1.1.m1.1.1.3.4"><csymbol cd="ambiguous" id="S7.p1.1.m1.1.1.3.4.1.cmml" xref="S7.p1.1.m1.1.1.3.4">superscript</csymbol><cn type="integer" id="S7.p1.1.m1.1.1.3.4.2.cmml" xref="S7.p1.1.m1.1.1.3.4.2">00</cn><ci id="S7.p1.1.m1.1.1.3.4.3.cmml" xref="S7.p1.1.m1.1.1.3.4.3">s</ci></apply></apply></apply></annotation-xml><annotation encoding="application/x-tex" id="S7.p1.1.m1.1c">\rm\alpha=23^{h}15^{m}00^{s}</annotation><annotation encoding="application/x-llamapun" id="S7.p1.1.m1.1d">italic_α = 23 start_POSTSUPERSCRIPT roman_h end_POSTSUPERSCRIPT 15 start_POSTSUPERSCRIPT roman_m end_POSTSUPERSCRIPT 00 start_POSTSUPERSCRIPT roman_s end_POSTSUPERSCRIPT</annotation></semantics></math>,<math id="S7.p1.2.m2.1" class="ltx_Math" alttext="\rm\delta=48\arcdeg 00\arcmin 00\arcsec" display="inline"><semantics id="S7.p1.2.m2.1a"><mrow id="S7.p1.2.m2.1.1" xref="S7.p1.2.m2.1.1.cmml"><mi id="S7.p1.2.m2.1.1.2" xref="S7.p1.2.m2.1.1.2.cmml">δ</mi><mo id="S7.p1.2.m2.1.1.1" xref="S7.p1.2.m2.1.1.1.cmml">=</mo><mrow id="S7.p1.2.m2.1.1.3" xref="S7.p1.2.m2.1.1.3.cmml"><mn id="S7.p1.2.m2.1.1.3.2" xref="S7.p1.2.m2.1.1.3.2.cmml">48</mn><mo id="S7.p1.2.m2.1.1.3.1" xref="S7.p1.2.m2.1.1.3.1.cmml">⁢</mo><mi mathvariant="normal" id="S7.p1.2.m2.1.1.3.3" xref="S7.p1.2.m2.1.1.3.3.cmml">°</mi><mo id="S7.p1.2.m2.1.1.3.1a" xref="S7.p1.2.m2.1.1.3.1.cmml">⁢</mo><mn id="S7.p1.2.m2.1.1.3.4" xref="S7.p1.2.m2.1.1.3.4.cmml">00</mn><mo id="S7.p1.2.m2.1.1.3.1b" xref="S7.p1.2.m2.1.1.3.1.cmml">⁢</mo><mi mathvariant="normal" id="S7.p1.2.m2.1.1.3.5" xref="S7.p1.2.m2.1.1.3.5.cmml">′</mi><mo id="S7.p1.2.m2.1.1.3.1c" xref="S7.p1.2.m2.1.1.3.1.cmml">⁢</mo><mn id="S7.p1.2.m2.1.1.3.6" xref="S7.p1.2.m2.1.1.3.6.cmml">00</mn><mo id="S7.p1.2.m2.1.1.3.1d" xref="S7.p1.2.m2.1.1.3.1.cmml">⁢</mo><mi mathvariant="normal" id="S7.p1.2.m2.1.1.3.7" xref="S7.p1.2.m2.1.1.3.7.cmml">″</mi></mrow></mrow><annotation-xml encoding="MathML-Content" id="S7.p1.2.m2.1b"><apply id="S7.p1.2.m2.1.1.cmml" xref="S7.p1.2.m2.1.1"><eq id="S7.p1.2.m2.1.1.1.cmml" xref="S7.p1.2.m2.1.1.1"></eq><ci id="S7.p1.2.m2.1.1.2.cmml" xref="S7.p1.2.m2.1.1.2">𝛿</ci><apply id="S7.p1.2.m2.1.1.3.cmml" xref="S7.p1.2.m2.1.1.3"><times id="S7.p1.2.m2.1.1.3.1.cmml" xref="S7.p1.2.m2.1.1.3.1"></times><cn type="integer" id="S7.p1.2.m2.1.1.3.2.cmml" xref="S7.p1.2.m2.1.1.3.2">48</cn><ci id="S7.p1.2.m2.1.1.3.3.cmml" xref="S7.p1.2.m2.1.1.3.3">°</ci><cn type="integer" id="S7.p1.2.m2.1.1.3.4.cmml" xref="S7.p1.2.m2.1.1.3.4">00</cn><ci id="S7.p1.2.m2.1.1.3.5.cmml" xref="S7.p1.2.m2.1.1.3.5">′</ci><cn type="integer" id="S7.p1.2.m2.1.1.3.6.cmml" xref="S7.p1.2.m2.1.1.3.6">00</cn><ci id="S7.p1.2.m2.1.1.3.7.cmml" xref="S7.p1.2.m2.1.1.3.7">″</ci></apply></apply></annotation-xml><annotation encoding="application/x-tex" id="S7.p1.2.m2.1c">\rm\delta=48\arcdeg 00\arcmin 00\arcsec</annotation><annotation encoding="application/x-llamapun" id="S7.p1.2.m2.1d">italic_δ = 48 ° 00 ′ 00 ″</annotation></semantics></math>). This contains only about 5% of
our current data set. We primarily concentrate on the repeatability of
the measurements, i.e., the precision for generating light-curves (in a
system as close to <math id="S7.p1.3.m3.1" class="ltx_Math" alttext="\rm I_{c}" display="inline"><semantics id="S7.p1.3.m3.1a"><msub id="S7.p1.3.m3.1.1" xref="S7.p1.3.m3.1.1.cmml"><mi mathvariant="normal" id="S7.p1.3.m3.1.1.2" xref="S7.p1.3.m3.1.1.2.cmml">I</mi><mi mathvariant="normal" id="S7.p1.3.m3.1.1.3" xref="S7.p1.3.m3.1.1.3.cmml">c</mi></msub><annotation-xml encoding="MathML-Content" id="S7.p1.3.m3.1b"><apply id="S7.p1.3.m3.1.1.cmml" xref="S7.p1.3.m3.1.1"><csymbol cd="ambiguous" id="S7.p1.3.m3.1.1.1.cmml" xref="S7.p1.3.m3.1.1">subscript</csymbol><ci id="S7.p1.3.m3.1.1.2.cmml" xref="S7.p1.3.m3.1.1.2">I</ci><ci id="S7.p1.3.m3.1.1.3.cmml" xref="S7.p1.3.m3.1.1.3">c</ci></apply></annotation-xml><annotation encoding="application/x-tex" id="S7.p1.3.m3.1c">\rm I_{c}</annotation><annotation encoding="application/x-llamapun" id="S7.p1.3.m3.1d">roman_I start_POSTSUBSCRIPT roman_c end_POSTSUBSCRIPT</annotation></semantics></math> as possible) as compared to the <span id="S7.p1.5.1" class="ltx_text ltx_font_italic">accuracy</span>, which is relative to absolute standards. Our crude estimate
for the latter from Hipparcos I-band stars, standard star observations
and the problematic flatfielding is that absolute calibration errors
can be as high <math id="S7.p1.4.m4.1" class="ltx_Math" alttext="\rm 0.1^{m}" display="inline"><semantics id="S7.p1.4.m4.1a"><msup id="S7.p1.4.m4.1.1" xref="S7.p1.4.m4.1.1.cmml"><mn id="S7.p1.4.m4.1.1.2" xref="S7.p1.4.m4.1.1.2.cmml">0.1</mn><mi mathvariant="normal" id="S7.p1.4.m4.1.1.3" xref="S7.p1.4.m4.1.1.3.cmml">m</mi></msup><annotation-xml encoding="MathML-Content" id="S7.p1.4.m4.1b"><apply id="S7.p1.4.m4.1.1.cmml" xref="S7.p1.4.m4.1.1"><csymbol cd="ambiguous" id="S7.p1.4.m4.1.1.1.cmml" xref="S7.p1.4.m4.1.1">superscript</csymbol><cn type="float" id="S7.p1.4.m4.1.1.2.cmml" xref="S7.p1.4.m4.1.1.2">0.1</cn><ci id="S7.p1.4.m4.1.1.3.cmml" xref="S7.p1.4.m4.1.1.3">m</ci></apply></annotation-xml><annotation encoding="application/x-tex" id="S7.p1.4.m4.1c">\rm 0.1^{m}</annotation><annotation encoding="application/x-llamapun" id="S7.p1.4.m4.1d">0.1 start_POSTSUPERSCRIPT roman_m end_POSTSUPERSCRIPT</annotation></semantics></math> in the field corners, but less than <math id="S7.p1.5.m5.1" class="ltx_Math" alttext="\rm 0.05^{m}" display="inline"><semantics id="S7.p1.5.m5.1a"><msup id="S7.p1.5.m5.1.1" xref="S7.p1.5.m5.1.1.cmml"><mn id="S7.p1.5.m5.1.1.2" xref="S7.p1.5.m5.1.1.2.cmml">0.05</mn><mi mathvariant="normal" id="S7.p1.5.m5.1.1.3" xref="S7.p1.5.m5.1.1.3.cmml">m</mi></msup><annotation-xml encoding="MathML-Content" id="S7.p1.5.m5.1b"><apply id="S7.p1.5.m5.1.1.cmml" xref="S7.p1.5.m5.1.1"><csymbol cd="ambiguous" id="S7.p1.5.m5.1.1.1.cmml" xref="S7.p1.5.m5.1.1">superscript</csymbol><cn type="float" id="S7.p1.5.m5.1.1.2.cmml" xref="S7.p1.5.m5.1.1.2">0.05</cn><ci id="S7.p1.5.m5.1.1.3.cmml" xref="S7.p1.5.m5.1.1.3">m</ci></apply></annotation-xml><annotation encoding="application/x-tex" id="S7.p1.5.m5.1c">\rm 0.05^{m}</annotation><annotation encoding="application/x-llamapun" id="S7.p1.5.m5.1d">0.05 start_POSTSUPERSCRIPT roman_m end_POSTSUPERSCRIPT</annotation></semantics></math> in the center.</p>
</div>
<div id="S7.p2" class="ltx_para">
<p id="S7.p2.1" class="ltx_p">A summary of the sessions is listed in Table <a href="#S7.T2" title="Table 2 ‣ 7 Photometric precision of HAT ‣ System description and first light-curves of HAT, an autonomous observatory for variability search" class="ltx_ref"><span class="ltx_text ltx_ref_tag">2</span></a>.
Observations were taken during a servicing mission, thus calibration
frames are not available for all nights, and the system was not
providing its maximal performance. Baffle on the lens was installed
only in January 2002, and overscan region of the chip was not read out
properly. The field was observed to zenith angles as high as
<math id="S7.p2.1.m1.1" class="ltx_Math" alttext="60\arcdeg" display="inline"><semantics id="S7.p2.1.m1.1a"><mrow id="S7.p2.1.m1.1.1" xref="S7.p2.1.m1.1.1.cmml"><mn id="S7.p2.1.m1.1.1.2" xref="S7.p2.1.m1.1.1.2.cmml">60</mn><mo id="S7.p2.1.m1.1.1.1" xref="S7.p2.1.m1.1.1.1.cmml">⁢</mo><mi mathvariant="normal" id="S7.p2.1.m1.1.1.3" xref="S7.p2.1.m1.1.1.3.cmml">°</mi></mrow><annotation-xml encoding="MathML-Content" id="S7.p2.1.m1.1b"><apply id="S7.p2.1.m1.1.1.cmml" xref="S7.p2.1.m1.1.1"><times id="S7.p2.1.m1.1.1.1.cmml" xref="S7.p2.1.m1.1.1.1"></times><cn type="integer" id="S7.p2.1.m1.1.1.2.cmml" xref="S7.p2.1.m1.1.1.2">60</cn><ci id="S7.p2.1.m1.1.1.3.cmml" xref="S7.p2.1.m1.1.1.3">°</ci></apply></annotation-xml><annotation encoding="application/x-tex" id="S7.p2.1.m1.1c">60\arcdeg</annotation><annotation encoding="application/x-llamapun" id="S7.p2.1.m1.1d">60 °</annotation></semantics></math>.</p>
</div>
<figure id="S7.T1" class="ltx_table">
<figcaption class="ltx_caption"><span class="ltx_tag ltx_tag_table">Table 1: </span>Summary of HAT-1 test observations</figcaption>
<table id="S7.T1.2" class="ltx_tabular ltx_guessed_headers ltx_align_middle">
<thead class="ltx_thead">
<tr id="S7.T1.2.1.1" class="ltx_tr">
<th id="S7.T1.2.1.1.1" class="ltx_td ltx_align_left ltx_th ltx_th_column ltx_border_r ltx_border_t">Date</th>
<th id="S7.T1.2.1.1.2" class="ltx_td ltx_align_center ltx_th ltx_th_column ltx_border_r ltx_border_t">N</th>
<th id="S7.T1.2.1.1.3" class="ltx_td ltx_align_center ltx_th ltx_th_column ltx_border_r ltx_border_t">Bias</th>
<th id="S7.T1.2.1.1.4" class="ltx_td ltx_align_center ltx_th ltx_th_column ltx_border_r ltx_border_t">Dark</th>
<th id="S7.T1.2.1.1.5" class="ltx_td ltx_align_center ltx_th ltx_th_column ltx_border_r ltx_border_t">Flat</th>
<th id="S7.T1.2.1.1.6" class="ltx_td ltx_align_center ltx_th ltx_th_column ltx_border_r ltx_border_t">Moon</th>
<th id="S7.T1.2.1.1.7" class="ltx_td ltx_align_left ltx_th ltx_th_column ltx_border_t">Com</th>
</tr>
</thead>
<tbody class="ltx_tbody">
<tr id="S7.T1.2.2.1" class="ltx_tr">
<td id="S7.T1.2.2.1.1" class="ltx_td ltx_align_left ltx_border_r ltx_border_tt">21/09</td>
<td id="S7.T1.2.2.1.2" class="ltx_td ltx_align_center ltx_border_r ltx_border_tt">43</td>
<td id="S7.T1.2.2.1.3" class="ltx_td ltx_align_center ltx_border_r ltx_border_tt">0</td>
<td id="S7.T1.2.2.1.4" class="ltx_td ltx_align_center ltx_border_r ltx_border_tt">0</td>
<td id="S7.T1.2.2.1.5" class="ltx_td ltx_align_center ltx_border_r ltx_border_tt">0</td>
<td id="S7.T1.2.2.1.6" class="ltx_td ltx_align_center ltx_border_r ltx_border_tt">20%</td>
<td id="S7.T1.2.2.1.7" class="ltx_td ltx_align_left ltx_border_tt">Clear</td>
</tr>
<tr id="S7.T1.2.3.2" class="ltx_tr">
<td id="S7.T1.2.3.2.1" class="ltx_td ltx_align_left ltx_border_r">22/09</td>
<td id="S7.T1.2.3.2.2" class="ltx_td ltx_align_center ltx_border_r">62</td>
<td id="S7.T1.2.3.2.3" class="ltx_td ltx_align_center ltx_border_r">0</td>
<td id="S7.T1.2.3.2.4" class="ltx_td ltx_align_center ltx_border_r">0</td>
<td id="S7.T1.2.3.2.5" class="ltx_td ltx_align_center ltx_border_r">0</td>
<td id="S7.T1.2.3.2.6" class="ltx_td ltx_align_center ltx_border_r">30%</td>
<td id="S7.T1.2.3.2.7" class="ltx_td ltx_align_left">Clear</td>
</tr>
<tr id="S7.T1.2.4.3" class="ltx_tr">
<td id="S7.T1.2.4.3.1" class="ltx_td ltx_align_left ltx_border_r">24/09</td>
<td id="S7.T1.2.4.3.2" class="ltx_td ltx_align_center ltx_border_r">42</td>
<td id="S7.T1.2.4.3.3" class="ltx_td ltx_align_center ltx_border_r">0</td>
<td id="S7.T1.2.4.3.4" class="ltx_td ltx_align_center ltx_border_r">0</td>
<td id="S7.T1.2.4.3.5" class="ltx_td ltx_align_center ltx_border_r">0</td>
<td id="S7.T1.2.4.3.6" class="ltx_td ltx_align_center ltx_border_r">50%</td>
<td id="S7.T1.2.4.3.7" class="ltx_td ltx_align_left">Clear</td>
</tr>
<tr id="S7.T1.2.5.4" class="ltx_tr">
<td id="S7.T1.2.5.4.1" class="ltx_td ltx_align_left ltx_border_r">25/09</td>
<td id="S7.T1.2.5.4.2" class="ltx_td ltx_align_center ltx_border_r">52</td>
<td id="S7.T1.2.5.4.3" class="ltx_td ltx_align_center ltx_border_r">0</td>
<td id="S7.T1.2.5.4.4" class="ltx_td ltx_align_center ltx_border_r">14</td>
<td id="S7.T1.2.5.4.5" class="ltx_td ltx_align_center ltx_border_r">53</td>
<td id="S7.T1.2.5.4.6" class="ltx_td ltx_align_center ltx_border_r">60%</td>
<td id="S7.T1.2.5.4.7" class="ltx_td ltx_align_left">Clear</td>
</tr>
<tr id="S7.T1.2.6.5" class="ltx_tr">
<td id="S7.T1.2.6.5.1" class="ltx_td ltx_align_left ltx_border_r">26/09</td>
<td id="S7.T1.2.6.5.2" class="ltx_td ltx_align_center ltx_border_r">38</td>
<td id="S7.T1.2.6.5.3" class="ltx_td ltx_align_center ltx_border_r">16</td>
<td id="S7.T1.2.6.5.4" class="ltx_td ltx_align_center ltx_border_r">18</td>
<td id="S7.T1.2.6.5.5" class="ltx_td ltx_align_center ltx_border_r">0</td>
<td id="S7.T1.2.6.5.6" class="ltx_td ltx_align_center ltx_border_r">70%</td>
<td id="S7.T1.2.6.5.7" class="ltx_td ltx_align_left">Clear</td>
</tr>
<tr id="S7.T1.2.7.6" class="ltx_tr">
<td id="S7.T1.2.7.6.1" class="ltx_td ltx_align_left ltx_border_r">27/09</td>
<td id="S7.T1.2.7.6.2" class="ltx_td ltx_align_center ltx_border_r">32</td>
<td id="S7.T1.2.7.6.3" class="ltx_td ltx_align_center ltx_border_r">4</td>
<td id="S7.T1.2.7.6.4" class="ltx_td ltx_align_center ltx_border_r">18</td>
<td id="S7.T1.2.7.6.5" class="ltx_td ltx_align_center ltx_border_r">0</td>
<td id="S7.T1.2.7.6.6" class="ltx_td ltx_align_center ltx_border_r">80%</td>
<td id="S7.T1.2.7.6.7" class="ltx_td"></td>
</tr>
<tr id="S7.T1.2.8.7" class="ltx_tr">
<td id="S7.T1.2.8.7.1" class="ltx_td ltx_align_left ltx_border_r">04/10</td>
<td id="S7.T1.2.8.7.2" class="ltx_td ltx_align_center ltx_border_r">58</td>
<td id="S7.T1.2.8.7.3" class="ltx_td ltx_align_center ltx_border_r">20</td>
<td id="S7.T1.2.8.7.4" class="ltx_td ltx_align_center ltx_border_r">18</td>
<td id="S7.T1.2.8.7.5" class="ltx_td ltx_align_center ltx_border_r">0</td>
<td id="S7.T1.2.8.7.6" class="ltx_td ltx_align_center ltx_border_r">99%</td>
<td id="S7.T1.2.8.7.7" class="ltx_td"></td>
</tr>
<tr id="S7.T1.2.9.8" class="ltx_tr">
<td id="S7.T1.2.9.8.1" class="ltx_td ltx_align_left ltx_border_r">05/10</td>
<td id="S7.T1.2.9.8.2" class="ltx_td ltx_align_center ltx_border_r">62</td>
<td id="S7.T1.2.9.8.3" class="ltx_td ltx_align_center ltx_border_r">26</td>
<td id="S7.T1.2.9.8.4" class="ltx_td ltx_align_center ltx_border_r">18</td>
<td id="S7.T1.2.9.8.5" class="ltx_td ltx_align_center ltx_border_r">34</td>
<td id="S7.T1.2.9.8.6" class="ltx_td ltx_align_center ltx_border_r">90%</td>
<td id="S7.T1.2.9.8.7" class="ltx_td ltx_align_left">Cirrus</td>
</tr>
<tr id="S7.T1.2.10.9" class="ltx_tr">
<td id="S7.T1.2.10.9.1" class="ltx_td ltx_align_left ltx_border_r">10/10</td>
<td id="S7.T1.2.10.9.2" class="ltx_td ltx_align_center ltx_border_r">36</td>
<td id="S7.T1.2.10.9.3" class="ltx_td ltx_align_center ltx_border_r">25</td>
<td id="S7.T1.2.10.9.4" class="ltx_td ltx_align_center ltx_border_r">18</td>
<td id="S7.T1.2.10.9.5" class="ltx_td ltx_align_center ltx_border_r">35</td>
<td id="S7.T1.2.10.9.6" class="ltx_td ltx_align_center ltx_border_r">50%</td>
<td id="S7.T1.2.10.9.7" class="ltx_td ltx_align_left">Cirrus</td>
</tr>
<tr id="S7.T1.2.11.10" class="ltx_tr">
<td id="S7.T1.2.11.10.1" class="ltx_td ltx_align_left ltx_border_r">12/10</td>
<td id="S7.T1.2.11.10.2" class="ltx_td ltx_align_center ltx_border_r">26</td>
<td id="S7.T1.2.11.10.3" class="ltx_td ltx_align_center ltx_border_r">19</td>
<td id="S7.T1.2.11.10.4" class="ltx_td ltx_align_center ltx_border_r">18</td>
<td id="S7.T1.2.11.10.5" class="ltx_td ltx_align_center ltx_border_r">3</td>
<td id="S7.T1.2.11.10.6" class="ltx_td ltx_align_center ltx_border_r">20%</td>
<td id="S7.T1.2.11.10.7" class="ltx_td ltx_align_left">Cirrus</td>
</tr>
<tr id="S7.T1.2.12.11" class="ltx_tr">
<td id="S7.T1.2.12.11.1" class="ltx_td ltx_align_left ltx_border_r">13/10</td>
<td id="S7.T1.2.12.11.2" class="ltx_td ltx_align_center ltx_border_r">26</td>
<td id="S7.T1.2.12.11.3" class="ltx_td ltx_align_center ltx_border_r">24</td>
<td id="S7.T1.2.12.11.4" class="ltx_td ltx_align_center ltx_border_r">18</td>
<td id="S7.T1.2.12.11.5" class="ltx_td ltx_align_center ltx_border_r">34</td>
<td id="S7.T1.2.12.11.6" class="ltx_td ltx_align_center ltx_border_r">15%</td>
<td id="S7.T1.2.12.11.7" class="ltx_td ltx_align_left">Cirrus</td>
</tr>
<tr id="S7.T1.2.13.12" class="ltx_tr">
<td id="S7.T1.2.13.12.1" class="ltx_td ltx_align_left ltx_border_bb ltx_border_r">14/10</td>
<td id="S7.T1.2.13.12.2" class="ltx_td ltx_align_center ltx_border_bb ltx_border_r">27</td>
<td id="S7.T1.2.13.12.3" class="ltx_td ltx_align_center ltx_border_bb ltx_border_r">24</td>
<td id="S7.T1.2.13.12.4" class="ltx_td ltx_align_center ltx_border_bb ltx_border_r">18</td>
<td id="S7.T1.2.13.12.5" class="ltx_td ltx_align_center ltx_border_bb ltx_border_r">29</td>
<td id="S7.T1.2.13.12.6" class="ltx_td ltx_align_center ltx_border_bb ltx_border_r">10%</td>
<td id="S7.T1.2.13.12.7" class="ltx_td ltx_align_left ltx_border_bb">Cirrus</td>
</tr>
</tbody>
</table>
<p id="S7.T1.1" class="ltx_p"><span id="S7.T1.1.1" class="ltx_text ltx_font_smallcaps">Note.–</span> The table only shows data used for finding out
the photometric precision of HAT-1, and comprises <math id="S7.T1.1.m1.1" class="ltx_Math" alttext="\sim 5\%" display="inline"><semantics id="S7.T1.1.m1.1a"><mrow id="S7.T1.1.m1.1.1" xref="S7.T1.1.m1.1.1.cmml"><mi id="S7.T1.1.m1.1.1.2" xref="S7.T1.1.m1.1.1.2.cmml"></mi><mo id="S7.T1.1.m1.1.1.1" xref="S7.T1.1.m1.1.1.1.cmml">∼</mo><mrow id="S7.T1.1.m1.1.1.3" xref="S7.T1.1.m1.1.1.3.cmml"><mn id="S7.T1.1.m1.1.1.3.2" xref="S7.T1.1.m1.1.1.3.2.cmml">5</mn><mo id="S7.T1.1.m1.1.1.3.1" xref="S7.T1.1.m1.1.1.3.1.cmml">%</mo></mrow></mrow><annotation-xml encoding="MathML-Content" id="S7.T1.1.m1.1b"><apply id="S7.T1.1.m1.1.1.cmml" xref="S7.T1.1.m1.1.1"><csymbol cd="latexml" id="S7.T1.1.m1.1.1.1.cmml" xref="S7.T1.1.m1.1.1.1">similar-to</csymbol><csymbol cd="latexml" id="S7.T1.1.m1.1.1.2.cmml" xref="S7.T1.1.m1.1.1.2">absent</csymbol><apply id="S7.T1.1.m1.1.1.3.cmml" xref="S7.T1.1.m1.1.1.3"><csymbol cd="latexml" id="S7.T1.1.m1.1.1.3.1.cmml" xref="S7.T1.1.m1.1.1.3.1">percent</csymbol><cn type="integer" id="S7.T1.1.m1.1.1.3.2.cmml" xref="S7.T1.1.m1.1.1.3.2">5</cn></apply></apply></annotation-xml><annotation encoding="application/x-tex" id="S7.T1.1.m1.1c">\sim 5\%</annotation><annotation encoding="application/x-llamapun" id="S7.T1.1.m1.1d">∼ 5 %</annotation></semantics></math> of the complete
data set. “Date” format is day/month, year 2001. “N” denotes the number
of frames for field F077. “Moon” is the phase of the Moon.</p>
</figure><span id="S7.1" class="ltx_ERROR undefined">{comment}</span>
<figure id="S7.T2" class="ltx_table">
<figcaption class="ltx_caption"><span class="ltx_tag ltx_tag_table">Table 2: </span>Summary of HAT-1 test observations</figcaption>
<table id="S7.T2.2" class="ltx_tabular">
<thead class="ltx_thead">
<tr id="S7.T2.2.1.1" class="ltx_tr">
<th id="S7.T2.2.1.1.1" class="ltx_td ltx_align_center ltx_th ltx_th_column ltx_border_tt">
Date<span id="S7.T2.2.1.1.1.1" class="ltx_note ltx_role_footnote"><sup class="ltx_note_mark">a</sup><span class="ltx_note_outer"><span class="ltx_note_content"><sup class="ltx_note_mark">a</sup>Year 2001, day/month.</span></span></span></th>
<th id="S7.T2.2.1.1.2" class="ltx_td ltx_align_center ltx_th ltx_th_column ltx_border_tt">N<span id="S7.T2.2.1.1.2.1" class="ltx_note ltx_role_footnote"><sup class="ltx_note_mark">b</sup><span class="ltx_note_outer"><span class="ltx_note_content"><sup class="ltx_note_mark">b</sup>Denotes the number of frames for field F077.</span></span></span>
</th>
<th id="S7.T2.2.1.1.3" class="ltx_td ltx_align_center ltx_th ltx_th_column ltx_border_tt">Bias</th>
<th id="S7.T2.2.1.1.4" class="ltx_td ltx_align_center ltx_th ltx_th_column ltx_border_tt">Dark</th>
<th id="S7.T2.2.1.1.5" class="ltx_td ltx_align_center ltx_th ltx_th_column ltx_border_tt">Flat</th>
<th id="S7.T2.2.1.1.6" class="ltx_td ltx_align_center ltx_th ltx_th_column ltx_border_tt">Moon<span id="S7.T2.2.1.1.6.1" class="ltx_note ltx_role_footnote"><sup class="ltx_note_mark">c</sup><span class="ltx_note_outer"><span class="ltx_note_content"><sup class="ltx_note_mark">c</sup>Phase of the Moon.</span></span></span>
</th>
<th id="S7.T2.2.1.1.7" class="ltx_td ltx_align_center ltx_th ltx_th_column ltx_border_tt">Comment</th>
</tr>
</thead>
<tbody class="ltx_tbody">
<tr id="S7.T2.2.2.1" class="ltx_tr">
<td id="S7.T2.2.2.1.1" class="ltx_td ltx_align_left ltx_border_t">21/09</td>
<td id="S7.T2.2.2.1.2" class="ltx_td ltx_align_center ltx_border_t">43</td>
<td id="S7.T2.2.2.1.3" class="ltx_td ltx_align_center ltx_border_t">0</td>
<td id="S7.T2.2.2.1.4" class="ltx_td ltx_align_center ltx_border_t">0</td>
<td id="S7.T2.2.2.1.5" class="ltx_td ltx_align_center ltx_border_t">0</td>
<td id="S7.T2.2.2.1.6" class="ltx_td ltx_align_center ltx_border_t">20%</td>
<td id="S7.T2.2.2.1.7" class="ltx_td ltx_align_left ltx_border_t">Clear</td>
</tr>
<tr id="S7.T2.2.3.2" class="ltx_tr">
<td id="S7.T2.2.3.2.1" class="ltx_td ltx_align_left">22/09</td>
<td id="S7.T2.2.3.2.2" class="ltx_td ltx_align_center">62</td>
<td id="S7.T2.2.3.2.3" class="ltx_td ltx_align_center">0</td>
<td id="S7.T2.2.3.2.4" class="ltx_td ltx_align_center">0</td>
<td id="S7.T2.2.3.2.5" class="ltx_td ltx_align_center">0</td>
<td id="S7.T2.2.3.2.6" class="ltx_td ltx_align_center">30%</td>
<td id="S7.T2.2.3.2.7" class="ltx_td ltx_align_left">Clear</td>
</tr>
<tr id="S7.T2.2.4.3" class="ltx_tr">
<td id="S7.T2.2.4.3.1" class="ltx_td ltx_align_left">24/09</td>
<td id="S7.T2.2.4.3.2" class="ltx_td ltx_align_center">42</td>
<td id="S7.T2.2.4.3.3" class="ltx_td ltx_align_center">0</td>
<td id="S7.T2.2.4.3.4" class="ltx_td ltx_align_center">0</td>
<td id="S7.T2.2.4.3.5" class="ltx_td ltx_align_center">0</td>
<td id="S7.T2.2.4.3.6" class="ltx_td ltx_align_center">50%</td>
<td id="S7.T2.2.4.3.7" class="ltx_td ltx_align_left">Clear</td>
</tr>
<tr id="S7.T2.2.5.4" class="ltx_tr">
<td id="S7.T2.2.5.4.1" class="ltx_td ltx_align_left">25/09</td>
<td id="S7.T2.2.5.4.2" class="ltx_td ltx_align_center">52</td>
<td id="S7.T2.2.5.4.3" class="ltx_td ltx_align_center">0</td>
<td id="S7.T2.2.5.4.4" class="ltx_td ltx_align_center">14</td>
<td id="S7.T2.2.5.4.5" class="ltx_td ltx_align_center">53</td>
<td id="S7.T2.2.5.4.6" class="ltx_td ltx_align_center">60%</td>
<td id="S7.T2.2.5.4.7" class="ltx_td ltx_align_left">Clear</td>
</tr>
<tr id="S7.T2.2.6.5" class="ltx_tr">
<td id="S7.T2.2.6.5.1" class="ltx_td ltx_align_left">26/09</td>
<td id="S7.T2.2.6.5.2" class="ltx_td ltx_align_center">38</td>
<td id="S7.T2.2.6.5.3" class="ltx_td ltx_align_center">16</td>
<td id="S7.T2.2.6.5.4" class="ltx_td ltx_align_center">18</td>
<td id="S7.T2.2.6.5.5" class="ltx_td ltx_align_center">0</td>
<td id="S7.T2.2.6.5.6" class="ltx_td ltx_align_center">70%</td>
<td id="S7.T2.2.6.5.7" class="ltx_td ltx_align_left">Clear</td>
</tr>
<tr id="S7.T2.2.7.6" class="ltx_tr">
<td id="S7.T2.2.7.6.1" class="ltx_td ltx_align_left">27/09</td>
<td id="S7.T2.2.7.6.2" class="ltx_td ltx_align_center">32</td>
<td id="S7.T2.2.7.6.3" class="ltx_td ltx_align_center">4</td>
<td id="S7.T2.2.7.6.4" class="ltx_td ltx_align_center">18</td>
<td id="S7.T2.2.7.6.5" class="ltx_td ltx_align_center">0</td>
<td id="S7.T2.2.7.6.6" class="ltx_td ltx_align_center">80%</td>
<td id="S7.T2.2.7.6.7" class="ltx_td"></td>
</tr>
<tr id="S7.T2.2.8.7" class="ltx_tr">
<td id="S7.T2.2.8.7.1" class="ltx_td ltx_align_left">04/10</td>
<td id="S7.T2.2.8.7.2" class="ltx_td ltx_align_center">58</td>
<td id="S7.T2.2.8.7.3" class="ltx_td ltx_align_center">20</td>
<td id="S7.T2.2.8.7.4" class="ltx_td ltx_align_center">18</td>
<td id="S7.T2.2.8.7.5" class="ltx_td ltx_align_center">0</td>
<td id="S7.T2.2.8.7.6" class="ltx_td ltx_align_center">99%</td>
<td id="S7.T2.2.8.7.7" class="ltx_td"></td>
</tr>
<tr id="S7.T2.2.9.8" class="ltx_tr">
<td id="S7.T2.2.9.8.1" class="ltx_td ltx_align_left">05/10</td>
<td id="S7.T2.2.9.8.2" class="ltx_td ltx_align_center">62</td>
<td id="S7.T2.2.9.8.3" class="ltx_td ltx_align_center">26</td>
<td id="S7.T2.2.9.8.4" class="ltx_td ltx_align_center">18</td>
<td id="S7.T2.2.9.8.5" class="ltx_td ltx_align_center">34</td>
<td id="S7.T2.2.9.8.6" class="ltx_td ltx_align_center">90%</td>
<td id="S7.T2.2.9.8.7" class="ltx_td ltx_align_left">Cirrus</td>
</tr>
<tr id="S7.T2.2.10.9" class="ltx_tr">
<td id="S7.T2.2.10.9.1" class="ltx_td ltx_align_left">10/10</td>
<td id="S7.T2.2.10.9.2" class="ltx_td ltx_align_center">36</td>
<td id="S7.T2.2.10.9.3" class="ltx_td ltx_align_center">25</td>
<td id="S7.T2.2.10.9.4" class="ltx_td ltx_align_center">18</td>
<td id="S7.T2.2.10.9.5" class="ltx_td ltx_align_center">35</td>
<td id="S7.T2.2.10.9.6" class="ltx_td ltx_align_center">50%</td>
<td id="S7.T2.2.10.9.7" class="ltx_td ltx_align_left">Cirrus</td>
</tr>
<tr id="S7.T2.2.11.10" class="ltx_tr">
<td id="S7.T2.2.11.10.1" class="ltx_td ltx_align_left">12/10</td>
<td id="S7.T2.2.11.10.2" class="ltx_td ltx_align_center">26</td>
<td id="S7.T2.2.11.10.3" class="ltx_td ltx_align_center">19</td>
<td id="S7.T2.2.11.10.4" class="ltx_td ltx_align_center">18</td>
<td id="S7.T2.2.11.10.5" class="ltx_td ltx_align_center">3</td>
<td id="S7.T2.2.11.10.6" class="ltx_td ltx_align_center">20%</td>
<td id="S7.T2.2.11.10.7" class="ltx_td ltx_align_left">Cirrus</td>
</tr>
<tr id="S7.T2.2.12.11" class="ltx_tr">
<td id="S7.T2.2.12.11.1" class="ltx_td ltx_align_left">13/10</td>
<td id="S7.T2.2.12.11.2" class="ltx_td ltx_align_center">26</td>
<td id="S7.T2.2.12.11.3" class="ltx_td ltx_align_center">24</td>
<td id="S7.T2.2.12.11.4" class="ltx_td ltx_align_center">18</td>
<td id="S7.T2.2.12.11.5" class="ltx_td ltx_align_center">34</td>
<td id="S7.T2.2.12.11.6" class="ltx_td ltx_align_center">15%</td>
<td id="S7.T2.2.12.11.7" class="ltx_td ltx_align_left">Cirrus</td>
</tr>
<tr id="S7.T2.2.13.12" class="ltx_tr">
<td id="S7.T2.2.13.12.1" class="ltx_td ltx_align_left ltx_border_b">14/10</td>
<td id="S7.T2.2.13.12.2" class="ltx_td ltx_align_center ltx_border_b">27</td>
<td id="S7.T2.2.13.12.3" class="ltx_td ltx_align_center ltx_border_b">24</td>
<td id="S7.T2.2.13.12.4" class="ltx_td ltx_align_center ltx_border_b">18</td>
<td id="S7.T2.2.13.12.5" class="ltx_td ltx_align_center ltx_border_b">29</td>
<td id="S7.T2.2.13.12.6" class="ltx_td ltx_align_center ltx_border_b">10%</td>
<td id="S7.T2.2.13.12.7" class="ltx_td ltx_align_left ltx_border_b">Cirrus</td>
</tr>
</tbody>
</table>
<p id="S7.T2.1" class="ltx_p">Note. – The table only shows data used for finding out the
photometric precision of HAT-1, and comprises <math id="S7.T2.1.m1.1" class="ltx_Math" alttext="\sim 5\%" display="inline"><semantics id="S7.T2.1.m1.1a"><mrow id="S7.T2.1.m1.1.1" xref="S7.T2.1.m1.1.1.cmml"><mi id="S7.T2.1.m1.1.1.2" xref="S7.T2.1.m1.1.1.2.cmml"></mi><mo id="S7.T2.1.m1.1.1.1" xref="S7.T2.1.m1.1.1.1.cmml">∼</mo><mrow id="S7.T2.1.m1.1.1.3" xref="S7.T2.1.m1.1.1.3.cmml"><mn id="S7.T2.1.m1.1.1.3.2" xref="S7.T2.1.m1.1.1.3.2.cmml">5</mn><mo id="S7.T2.1.m1.1.1.3.1" xref="S7.T2.1.m1.1.1.3.1.cmml">%</mo></mrow></mrow><annotation-xml encoding="MathML-Content" id="S7.T2.1.m1.1b"><apply id="S7.T2.1.m1.1.1.cmml" xref="S7.T2.1.m1.1.1"><csymbol cd="latexml" id="S7.T2.1.m1.1.1.1.cmml" xref="S7.T2.1.m1.1.1.1">similar-to</csymbol><csymbol cd="latexml" id="S7.T2.1.m1.1.1.2.cmml" xref="S7.T2.1.m1.1.1.2">absent</csymbol><apply id="S7.T2.1.m1.1.1.3.cmml" xref="S7.T2.1.m1.1.1.3"><csymbol cd="latexml" id="S7.T2.1.m1.1.1.3.1.cmml" xref="S7.T2.1.m1.1.1.3.1">percent</csymbol><cn type="integer" id="S7.T2.1.m1.1.1.3.2.cmml" xref="S7.T2.1.m1.1.1.3.2">5</cn></apply></apply></annotation-xml><annotation encoding="application/x-tex" id="S7.T2.1.m1.1c">\sim 5\%</annotation><annotation encoding="application/x-llamapun" id="S7.T2.1.m1.1d">∼ 5 %</annotation></semantics></math> of the complete
data set


</p>
</figure>
<div id="S7.p3" class="ltx_para">
<p id="S7.p3.1" class="ltx_p">Calibration of frames was done in a standard way, and consisted of bias
and dark subtraction, and flatfield correction. Time-dependence of bias
level was not dealt with. For some of the nights dark frames were not
available, however the dark pattern seems to change in time.
Calibration was done by our Tcl interface to <span id="S7.p3.1.1" class="ltx_text ltx_font_smallcaps">iraf</span>
<span id="footnote14" class="ltx_note ltx_role_footnote"><sup class="ltx_note_mark">14</sup><span class="ltx_note_outer"><span class="ltx_note_content"><sup class="ltx_note_mark">14</sup><span class="ltx_tag ltx_tag_note">14</span>
IRAF is distributed by the National Optical Astronomy Observatories,
which are operated by the Association of Universities for Research
in Astronomy, Inc., under cooperative agreement with the National
Science Foundation.</span></span></span> <cite class="ltx_cite ltx_citemacro_citep">(Tody, <a href="#bib.bib48" title="" class="ltx_ref">1993</a>)</cite>.</p>
</div>
<div id="S7.p4" class="ltx_para">
<p id="S7.p4.1" class="ltx_p">As mentioned in §<a href="#S6" title="6 Observations from Kitt Peak ‣ System description and first light-curves of HAT, an autonomous observatory for variability search" class="ltx_ref"><span class="ltx_text ltx_ref_tag">6</span></a>, flatfielding with skyflat frames of a
<math id="S7.p4.1.m1.1" class="ltx_Math" alttext="9\arcdeg" display="inline"><semantics id="S7.p4.1.m1.1a"><mrow id="S7.p4.1.m1.1.1" xref="S7.p4.1.m1.1.1.cmml"><mn id="S7.p4.1.m1.1.1.2" xref="S7.p4.1.m1.1.1.2.cmml">9</mn><mo id="S7.p4.1.m1.1.1.1" xref="S7.p4.1.m1.1.1.1.cmml">⁢</mo><mi mathvariant="normal" id="S7.p4.1.m1.1.1.3" xref="S7.p4.1.m1.1.1.3.cmml">°</mi></mrow><annotation-xml encoding="MathML-Content" id="S7.p4.1.m1.1b"><apply id="S7.p4.1.m1.1.1.cmml" xref="S7.p4.1.m1.1.1"><times id="S7.p4.1.m1.1.1.1.cmml" xref="S7.p4.1.m1.1.1.1"></times><cn type="integer" id="S7.p4.1.m1.1.1.2.cmml" xref="S7.p4.1.m1.1.1.2">9</cn><ci id="S7.p4.1.m1.1.1.3.cmml" xref="S7.p4.1.m1.1.1.3">°</ci></apply></annotation-xml><annotation encoding="application/x-tex" id="S7.p4.1.m1.1c">9\arcdeg</annotation><annotation encoding="application/x-llamapun" id="S7.p4.1.m1.1d">9 °</annotation></semantics></math> wide field is problematic. We tried to remove the large scale
pattern (partly due to division with flatfields not truly representing
the response function of the system) by blank-sky correction, using
median average of frames taken at moonless, clear nights, close to
zenith, and far from the Milky Way. Unfortunately even these frames
should be treated with caution because in some cases gradients from
zodiacal light or the rippled structure of the night airglow were
easily visible, and confirmed from ConCam images.</p>
</div>
<div id="S7.p5" class="ltx_para">
<p id="S7.p5.1" class="ltx_p">Astrometry of the master frame was performed by
<span id="S7.p5.1.1" class="ltx_text ltx_font_smallcaps">iraf/geomap,geoxytran</span>, and the residuals (offsets in
arcseconds) of the second order polynomial transformation between the
celestial reference frame and the image were in the order of
<math id="S7.p5.1.m1.1" class="ltx_Math" alttext="\rm\sigma\approx 0.8\arcsec" display="inline"><semantics id="S7.p5.1.m1.1a"><mrow id="S7.p5.1.m1.1.1" xref="S7.p5.1.m1.1.1.cmml"><mi id="S7.p5.1.m1.1.1.2" xref="S7.p5.1.m1.1.1.2.cmml">σ</mi><mo id="S7.p5.1.m1.1.1.1" xref="S7.p5.1.m1.1.1.1.cmml">≈</mo><mrow id="S7.p5.1.m1.1.1.3" xref="S7.p5.1.m1.1.1.3.cmml"><mn id="S7.p5.1.m1.1.1.3.2" xref="S7.p5.1.m1.1.1.3.2.cmml">0.8</mn><mo id="S7.p5.1.m1.1.1.3.1" xref="S7.p5.1.m1.1.1.3.1.cmml">⁢</mo><mi mathvariant="normal" id="S7.p5.1.m1.1.1.3.3" xref="S7.p5.1.m1.1.1.3.3.cmml">″</mi></mrow></mrow><annotation-xml encoding="MathML-Content" id="S7.p5.1.m1.1b"><apply id="S7.p5.1.m1.1.1.cmml" xref="S7.p5.1.m1.1.1"><approx id="S7.p5.1.m1.1.1.1.cmml" xref="S7.p5.1.m1.1.1.1"></approx><ci id="S7.p5.1.m1.1.1.2.cmml" xref="S7.p5.1.m1.1.1.2">𝜎</ci><apply id="S7.p5.1.m1.1.1.3.cmml" xref="S7.p5.1.m1.1.1.3"><times id="S7.p5.1.m1.1.1.3.1.cmml" xref="S7.p5.1.m1.1.1.3.1"></times><cn type="float" id="S7.p5.1.m1.1.1.3.2.cmml" xref="S7.p5.1.m1.1.1.3.2">0.8</cn><ci id="S7.p5.1.m1.1.1.3.3.cmml" xref="S7.p5.1.m1.1.1.3.3">″</ci></apply></apply></annotation-xml><annotation encoding="application/x-tex" id="S7.p5.1.m1.1c">\rm\sigma\approx 0.8\arcsec</annotation><annotation encoding="application/x-llamapun" id="S7.p5.1.m1.1d">italic_σ ≈ 0.8 ″</annotation></semantics></math>.</p>
</div>
<div id="S7.p6" class="ltx_para">
<p id="S7.p6.1" class="ltx_p">Photometry of images was finally done by the aperture photometry
routine in the <span id="S7.p6.1.1" class="ltx_text ltx_font_smallcaps">iraf/daophot</span> package <cite class="ltx_cite ltx_citemacro_citep">(Stetson, <a href="#bib.bib46" title="" class="ltx_ref">1987</a>)</cite> after
experimenting with <span id="S7.p6.1.2" class="ltx_text ltx_font_smallcaps">isis-2.1</span> Image Subtraction Method
<cite class="ltx_cite ltx_citemacro_citep">(Alard, <a href="#bib.bib1" title="" class="ltx_ref">2000</a>)</cite> without positive results due to our undersampling
(fwhm=<math id="S7.p6.1.m1.1" class="ltx_Math" alttext="1.6\--2.0" display="inline"><semantics id="S7.p6.1.m1.1a"><mrow id="S7.p6.1.m1.1.1" xref="S7.p6.1.m1.1.1.cmml"><mn id="S7.p6.1.m1.1.1.2" xref="S7.p6.1.m1.1.1.2.cmml">1.6</mn><mo id="S7.p6.1.m1.1.1.1" xref="S7.p6.1.m1.1.1.1.cmml">-</mo><mn id="S7.p6.1.m1.1.1.3" xref="S7.p6.1.m1.1.1.3.cmml">2.0</mn></mrow><annotation-xml encoding="MathML-Content" id="S7.p6.1.m1.1b"><apply id="S7.p6.1.m1.1.1.cmml" xref="S7.p6.1.m1.1.1"><minus id="S7.p6.1.m1.1.1.1.cmml" xref="S7.p6.1.m1.1.1.1"></minus><cn type="float" id="S7.p6.1.m1.1.1.2.cmml" xref="S7.p6.1.m1.1.1.2">1.6</cn><cn type="float" id="S7.p6.1.m1.1.1.3.cmml" xref="S7.p6.1.m1.1.1.3">2.0</cn></apply></annotation-xml><annotation encoding="application/x-tex" id="S7.p6.1.m1.1c">1.6\--2.0</annotation><annotation encoding="application/x-llamapun" id="S7.p6.1.m1.1d">1.6 - 2.0</annotation></semantics></math> pixel). The bottleneck in the latter method seemed to
be spatial interpolation of the images to the reference grid, and
re-sampling the narrow psfs. Experiments with <span id="S7.p6.1.3" class="ltx_text ltx_font_smallcaps">iraf/geotran</span> to
get around this problem did not significantly improve the results.</p>
</div>
<div id="S7.p7" class="ltx_para">
<p id="S7.p7.1" class="ltx_p">An astrometric reference image was montaged from  20 individual frames
by transforming them to the same reference grid. A starlist was
established by <span id="S7.p7.1.1" class="ltx_text ltx_font_smallcaps">daofind</span> (60K stars), and later the same IDs
were assigned to the same stars on all frames for easy
cross-identification. This astrometric reference was useful <span id="S7.p7.1.2" class="ltx_text ltx_font_italic">only</span>
for finding sources with good efficiency, but due to the resampling of
narrow psfs during the transformation of individual images, it was not
used in photometry at all.</p>
</div>
<div id="S7.p8" class="ltx_para">
<p id="S7.p8.4" class="ltx_p">We found roughly 30K stars per individual frame down to a threshold of
5 sigma of the background with <span id="S7.p8.4.1" class="ltx_text ltx_font_smallcaps">daofind</span>. Background scatter
(<math id="S7.p8.1.m1.1" class="ltx_Math" alttext="\rm\sigma_{bg}\approx 10\--20ADU" display="inline"><semantics id="S7.p8.1.m1.1a"><mrow id="S7.p8.1.m1.1.1" xref="S7.p8.1.m1.1.1.cmml"><msub id="S7.p8.1.m1.1.1.2" xref="S7.p8.1.m1.1.1.2.cmml"><mi id="S7.p8.1.m1.1.1.2.2" xref="S7.p8.1.m1.1.1.2.2.cmml">σ</mi><mi id="S7.p8.1.m1.1.1.2.3" xref="S7.p8.1.m1.1.1.2.3.cmml">bg</mi></msub><mo id="S7.p8.1.m1.1.1.1" xref="S7.p8.1.m1.1.1.1.cmml">≈</mo><mrow id="S7.p8.1.m1.1.1.3" xref="S7.p8.1.m1.1.1.3.cmml"><mn id="S7.p8.1.m1.1.1.3.2" xref="S7.p8.1.m1.1.1.3.2.cmml">10</mn><mo id="S7.p8.1.m1.1.1.3.1" xref="S7.p8.1.m1.1.1.3.1.cmml">-</mo><mrow id="S7.p8.1.m1.1.1.3.3" xref="S7.p8.1.m1.1.1.3.3.cmml"><mn id="S7.p8.1.m1.1.1.3.3.2" xref="S7.p8.1.m1.1.1.3.3.2.cmml">20</mn><mo id="S7.p8.1.m1.1.1.3.3.1" xref="S7.p8.1.m1.1.1.3.3.1.cmml">⁢</mo><mi mathvariant="normal" id="S7.p8.1.m1.1.1.3.3.3" xref="S7.p8.1.m1.1.1.3.3.3.cmml">A</mi><mo id="S7.p8.1.m1.1.1.3.3.1a" xref="S7.p8.1.m1.1.1.3.3.1.cmml">⁢</mo><mi mathvariant="normal" id="S7.p8.1.m1.1.1.3.3.4" xref="S7.p8.1.m1.1.1.3.3.4.cmml">D</mi><mo id="S7.p8.1.m1.1.1.3.3.1b" xref="S7.p8.1.m1.1.1.3.3.1.cmml">⁢</mo><mi mathvariant="normal" id="S7.p8.1.m1.1.1.3.3.5" xref="S7.p8.1.m1.1.1.3.3.5.cmml">U</mi></mrow></mrow></mrow><annotation-xml encoding="MathML-Content" id="S7.p8.1.m1.1b"><apply id="S7.p8.1.m1.1.1.cmml" xref="S7.p8.1.m1.1.1"><approx id="S7.p8.1.m1.1.1.1.cmml" xref="S7.p8.1.m1.1.1.1"></approx><apply id="S7.p8.1.m1.1.1.2.cmml" xref="S7.p8.1.m1.1.1.2"><csymbol cd="ambiguous" id="S7.p8.1.m1.1.1.2.1.cmml" xref="S7.p8.1.m1.1.1.2">subscript</csymbol><ci id="S7.p8.1.m1.1.1.2.2.cmml" xref="S7.p8.1.m1.1.1.2.2">𝜎</ci><ci id="S7.p8.1.m1.1.1.2.3.cmml" xref="S7.p8.1.m1.1.1.2.3">bg</ci></apply><apply id="S7.p8.1.m1.1.1.3.cmml" xref="S7.p8.1.m1.1.1.3"><minus id="S7.p8.1.m1.1.1.3.1.cmml" xref="S7.p8.1.m1.1.1.3.1"></minus><cn type="integer" id="S7.p8.1.m1.1.1.3.2.cmml" xref="S7.p8.1.m1.1.1.3.2">10</cn><apply id="S7.p8.1.m1.1.1.3.3.cmml" xref="S7.p8.1.m1.1.1.3.3"><times id="S7.p8.1.m1.1.1.3.3.1.cmml" xref="S7.p8.1.m1.1.1.3.3.1"></times><cn type="integer" id="S7.p8.1.m1.1.1.3.3.2.cmml" xref="S7.p8.1.m1.1.1.3.3.2">20</cn><ci id="S7.p8.1.m1.1.1.3.3.3.cmml" xref="S7.p8.1.m1.1.1.3.3.3">A</ci><ci id="S7.p8.1.m1.1.1.3.3.4.cmml" xref="S7.p8.1.m1.1.1.3.3.4">D</ci><ci id="S7.p8.1.m1.1.1.3.3.5.cmml" xref="S7.p8.1.m1.1.1.3.3.5">U</ci></apply></apply></apply></annotation-xml><annotation encoding="application/x-tex" id="S7.p8.1.m1.1c">\rm\sigma_{bg}\approx 10\--20ADU</annotation><annotation encoding="application/x-llamapun" id="S7.p8.1.m1.1d">italic_σ start_POSTSUBSCRIPT roman_bg end_POSTSUBSCRIPT ≈ 10 - 20 roman_A roman_D roman_U</annotation></semantics></math>) almost entirely comes from the sky
(mean level <math id="S7.p8.2.m2.1" class="ltx_Math" alttext="\rm\approx 300ADU" display="inline"><semantics id="S7.p8.2.m2.1a"><mrow id="S7.p8.2.m2.1.1" xref="S7.p8.2.m2.1.1.cmml"><mi id="S7.p8.2.m2.1.1.2" xref="S7.p8.2.m2.1.1.2.cmml"></mi><mo id="S7.p8.2.m2.1.1.1" xref="S7.p8.2.m2.1.1.1.cmml">≈</mo><mrow id="S7.p8.2.m2.1.1.3" xref="S7.p8.2.m2.1.1.3.cmml"><mn id="S7.p8.2.m2.1.1.3.2" xref="S7.p8.2.m2.1.1.3.2.cmml">300</mn><mo id="S7.p8.2.m2.1.1.3.1" xref="S7.p8.2.m2.1.1.3.1.cmml">⁢</mo><mi mathvariant="normal" id="S7.p8.2.m2.1.1.3.3" xref="S7.p8.2.m2.1.1.3.3.cmml">A</mi><mo id="S7.p8.2.m2.1.1.3.1a" xref="S7.p8.2.m2.1.1.3.1.cmml">⁢</mo><mi mathvariant="normal" id="S7.p8.2.m2.1.1.3.4" xref="S7.p8.2.m2.1.1.3.4.cmml">D</mi><mo id="S7.p8.2.m2.1.1.3.1b" xref="S7.p8.2.m2.1.1.3.1.cmml">⁢</mo><mi mathvariant="normal" id="S7.p8.2.m2.1.1.3.5" xref="S7.p8.2.m2.1.1.3.5.cmml">U</mi></mrow></mrow><annotation-xml encoding="MathML-Content" id="S7.p8.2.m2.1b"><apply id="S7.p8.2.m2.1.1.cmml" xref="S7.p8.2.m2.1.1"><approx id="S7.p8.2.m2.1.1.1.cmml" xref="S7.p8.2.m2.1.1.1"></approx><csymbol cd="latexml" id="S7.p8.2.m2.1.1.2.cmml" xref="S7.p8.2.m2.1.1.2">absent</csymbol><apply id="S7.p8.2.m2.1.1.3.cmml" xref="S7.p8.2.m2.1.1.3"><times id="S7.p8.2.m2.1.1.3.1.cmml" xref="S7.p8.2.m2.1.1.3.1"></times><cn type="integer" id="S7.p8.2.m2.1.1.3.2.cmml" xref="S7.p8.2.m2.1.1.3.2">300</cn><ci id="S7.p8.2.m2.1.1.3.3.cmml" xref="S7.p8.2.m2.1.1.3.3">A</ci><ci id="S7.p8.2.m2.1.1.3.4.cmml" xref="S7.p8.2.m2.1.1.3.4">D</ci><ci id="S7.p8.2.m2.1.1.3.5.cmml" xref="S7.p8.2.m2.1.1.3.5">U</ci></apply></apply></annotation-xml><annotation encoding="application/x-tex" id="S7.p8.2.m2.1c">\rm\approx 300ADU</annotation><annotation encoding="application/x-llamapun" id="S7.p8.2.m2.1d">≈ 300 roman_A roman_D roman_U</annotation></semantics></math> at new moon) and processing noise
<cite class="ltx_cite ltx_citemacro_citep">(Newberry, <a href="#bib.bib30" title="" class="ltx_ref">1991</a>)</cite>, while readout noise (<math id="S7.p8.3.m3.1" class="ltx_Math" alttext="\rm R\approx 2ADU" display="inline"><semantics id="S7.p8.3.m3.1a"><mrow id="S7.p8.3.m3.1.1" xref="S7.p8.3.m3.1.1.cmml"><mi mathvariant="normal" id="S7.p8.3.m3.1.1.2" xref="S7.p8.3.m3.1.1.2.cmml">R</mi><mo id="S7.p8.3.m3.1.1.1" xref="S7.p8.3.m3.1.1.1.cmml">≈</mo><mrow id="S7.p8.3.m3.1.1.3" xref="S7.p8.3.m3.1.1.3.cmml"><mn id="S7.p8.3.m3.1.1.3.2" xref="S7.p8.3.m3.1.1.3.2.cmml">2</mn><mo id="S7.p8.3.m3.1.1.3.1" xref="S7.p8.3.m3.1.1.3.1.cmml">⁢</mo><mi mathvariant="normal" id="S7.p8.3.m3.1.1.3.3" xref="S7.p8.3.m3.1.1.3.3.cmml">A</mi><mo id="S7.p8.3.m3.1.1.3.1a" xref="S7.p8.3.m3.1.1.3.1.cmml">⁢</mo><mi mathvariant="normal" id="S7.p8.3.m3.1.1.3.4" xref="S7.p8.3.m3.1.1.3.4.cmml">D</mi><mo id="S7.p8.3.m3.1.1.3.1b" xref="S7.p8.3.m3.1.1.3.1.cmml">⁢</mo><mi mathvariant="normal" id="S7.p8.3.m3.1.1.3.5" xref="S7.p8.3.m3.1.1.3.5.cmml">U</mi></mrow></mrow><annotation-xml encoding="MathML-Content" id="S7.p8.3.m3.1b"><apply id="S7.p8.3.m3.1.1.cmml" xref="S7.p8.3.m3.1.1"><approx id="S7.p8.3.m3.1.1.1.cmml" xref="S7.p8.3.m3.1.1.1"></approx><ci id="S7.p8.3.m3.1.1.2.cmml" xref="S7.p8.3.m3.1.1.2">R</ci><apply id="S7.p8.3.m3.1.1.3.cmml" xref="S7.p8.3.m3.1.1.3"><times id="S7.p8.3.m3.1.1.3.1.cmml" xref="S7.p8.3.m3.1.1.3.1"></times><cn type="integer" id="S7.p8.3.m3.1.1.3.2.cmml" xref="S7.p8.3.m3.1.1.3.2">2</cn><ci id="S7.p8.3.m3.1.1.3.3.cmml" xref="S7.p8.3.m3.1.1.3.3">A</ci><ci id="S7.p8.3.m3.1.1.3.4.cmml" xref="S7.p8.3.m3.1.1.3.4">D</ci><ci id="S7.p8.3.m3.1.1.3.5.cmml" xref="S7.p8.3.m3.1.1.3.5">U</ci></apply></apply></annotation-xml><annotation encoding="application/x-tex" id="S7.p8.3.m3.1c">\rm R\approx 2ADU</annotation><annotation encoding="application/x-llamapun" id="S7.p8.3.m3.1d">roman_R ≈ 2 roman_A roman_D roman_U</annotation></semantics></math>),
digitization noise (<math id="S7.p8.4.m4.1" class="ltx_Math" alttext="\rm T\approx 0.29ADU" display="inline"><semantics id="S7.p8.4.m4.1a"><mrow id="S7.p8.4.m4.1.1" xref="S7.p8.4.m4.1.1.cmml"><mi mathvariant="normal" id="S7.p8.4.m4.1.1.2" xref="S7.p8.4.m4.1.1.2.cmml">T</mi><mo id="S7.p8.4.m4.1.1.1" xref="S7.p8.4.m4.1.1.1.cmml">≈</mo><mrow id="S7.p8.4.m4.1.1.3" xref="S7.p8.4.m4.1.1.3.cmml"><mn id="S7.p8.4.m4.1.1.3.2" xref="S7.p8.4.m4.1.1.3.2.cmml">0.29</mn><mo id="S7.p8.4.m4.1.1.3.1" xref="S7.p8.4.m4.1.1.3.1.cmml">⁢</mo><mi id="S7.p8.4.m4.1.1.3.3" xref="S7.p8.4.m4.1.1.3.3.cmml">ADU</mi></mrow></mrow><annotation-xml encoding="MathML-Content" id="S7.p8.4.m4.1b"><apply id="S7.p8.4.m4.1.1.cmml" xref="S7.p8.4.m4.1.1"><approx id="S7.p8.4.m4.1.1.1.cmml" xref="S7.p8.4.m4.1.1.1"></approx><ci id="S7.p8.4.m4.1.1.2.cmml" xref="S7.p8.4.m4.1.1.2">T</ci><apply id="S7.p8.4.m4.1.1.3.cmml" xref="S7.p8.4.m4.1.1.3"><times id="S7.p8.4.m4.1.1.3.1.cmml" xref="S7.p8.4.m4.1.1.3.1"></times><cn type="float" id="S7.p8.4.m4.1.1.3.2.cmml" xref="S7.p8.4.m4.1.1.3.2">0.29</cn><ci id="S7.p8.4.m4.1.1.3.3.cmml" xref="S7.p8.4.m4.1.1.3.3">ADU</ci></apply></apply></annotation-xml><annotation encoding="application/x-tex" id="S7.p8.4.m4.1c">\rm T\approx 0.29ADU</annotation><annotation encoding="application/x-llamapun" id="S7.p8.4.m4.1d">roman_T ≈ 0.29 roman_ADU</annotation></semantics></math>), etc., are negligible. The
raw photometric data was first transformed to the master frame in the
XY-plane, stars were cross-identifed, IDs were re-assigned.</p>
</div>
<div id="S7.p9" class="ltx_para">
<p id="S7.p9.2" class="ltx_p">A reference magnitude file was created by averaging individual raw
photometry measurements for a dozen good quality images. Magnitudes for
all frames were shifted to this reference, where the amount of the
shift was spatially dependent, typically done on a <math id="S7.p9.1.m1.1" class="ltx_Math" alttext="6\times 4" display="inline"><semantics id="S7.p9.1.m1.1a"><mrow id="S7.p9.1.m1.1.1" xref="S7.p9.1.m1.1.1.cmml"><mn id="S7.p9.1.m1.1.1.2" xref="S7.p9.1.m1.1.1.2.cmml">6</mn><mo id="S7.p9.1.m1.1.1.1" xref="S7.p9.1.m1.1.1.1.cmml">×</mo><mn id="S7.p9.1.m1.1.1.3" xref="S7.p9.1.m1.1.1.3.cmml">4</mn></mrow><annotation-xml encoding="MathML-Content" id="S7.p9.1.m1.1b"><apply id="S7.p9.1.m1.1.1.cmml" xref="S7.p9.1.m1.1.1"><times id="S7.p9.1.m1.1.1.1.cmml" xref="S7.p9.1.m1.1.1.1"></times><cn type="integer" id="S7.p9.1.m1.1.1.2.cmml" xref="S7.p9.1.m1.1.1.2">6</cn><cn type="integer" id="S7.p9.1.m1.1.1.3.cmml" xref="S7.p9.1.m1.1.1.3">4</cn></apply></annotation-xml><annotation encoding="application/x-tex" id="S7.p9.1.m1.1c">6\times 4</annotation><annotation encoding="application/x-llamapun" id="S7.p9.1.m1.1d">6 × 4</annotation></semantics></math> XY-grid,
and few hundred stars for each block. The residuals of the
transformation indicated our photometric precision, and were in the
order of <math id="S7.p9.2.m2.1" class="ltx_Math" alttext="\rm\sigma\approx 0.01^{m}" display="inline"><semantics id="S7.p9.2.m2.1a"><mrow id="S7.p9.2.m2.1.1" xref="S7.p9.2.m2.1.1.cmml"><mi id="S7.p9.2.m2.1.1.2" xref="S7.p9.2.m2.1.1.2.cmml">σ</mi><mo id="S7.p9.2.m2.1.1.1" xref="S7.p9.2.m2.1.1.1.cmml">≈</mo><msup id="S7.p9.2.m2.1.1.3" xref="S7.p9.2.m2.1.1.3.cmml"><mn id="S7.p9.2.m2.1.1.3.2" xref="S7.p9.2.m2.1.1.3.2.cmml">0.01</mn><mi mathvariant="normal" id="S7.p9.2.m2.1.1.3.3" xref="S7.p9.2.m2.1.1.3.3.cmml">m</mi></msup></mrow><annotation-xml encoding="MathML-Content" id="S7.p9.2.m2.1b"><apply id="S7.p9.2.m2.1.1.cmml" xref="S7.p9.2.m2.1.1"><approx id="S7.p9.2.m2.1.1.1.cmml" xref="S7.p9.2.m2.1.1.1"></approx><ci id="S7.p9.2.m2.1.1.2.cmml" xref="S7.p9.2.m2.1.1.2">𝜎</ci><apply id="S7.p9.2.m2.1.1.3.cmml" xref="S7.p9.2.m2.1.1.3"><csymbol cd="ambiguous" id="S7.p9.2.m2.1.1.3.1.cmml" xref="S7.p9.2.m2.1.1.3">superscript</csymbol><cn type="float" id="S7.p9.2.m2.1.1.3.2.cmml" xref="S7.p9.2.m2.1.1.3.2">0.01</cn><ci id="S7.p9.2.m2.1.1.3.3.cmml" xref="S7.p9.2.m2.1.1.3.3">m</ci></apply></apply></annotation-xml><annotation encoding="application/x-tex" id="S7.p9.2.m2.1c">\rm\sigma\approx 0.01^{m}</annotation><annotation encoding="application/x-llamapun" id="S7.p9.2.m2.1d">italic_σ ≈ 0.01 start_POSTSUPERSCRIPT roman_m end_POSTSUPERSCRIPT</annotation></semantics></math>. Spatial dependence is a reasonable
assumption due to the wide field-of-view and differential extinction.
Some fraction of the scatter is due to color-dependence of extinction,
which we cannot take into account. Standard calibration of the field
was tied to non-saturated Hipparcos stars.</p>
</div>
<div id="S7.p10" class="ltx_para">
<p id="S7.p10.3" class="ltx_p">After construction of the light-curves, we could estimate the
photometric precision of the system by assuming that most of the stars
are constant, and by deriving the rms of the light-curves around their
mean values. Most light-curves have 250 data points, and each point is
the result of <math id="S7.p10.1.m1.1" class="ltx_Math" alttext="\rm 2\times 240s" display="inline"><semantics id="S7.p10.1.m1.1a"><mrow id="S7.p10.1.m1.1.1" xref="S7.p10.1.m1.1.1.cmml"><mrow id="S7.p10.1.m1.1.1.2" xref="S7.p10.1.m1.1.1.2.cmml"><mn id="S7.p10.1.m1.1.1.2.2" xref="S7.p10.1.m1.1.1.2.2.cmml">2</mn><mo id="S7.p10.1.m1.1.1.2.1" xref="S7.p10.1.m1.1.1.2.1.cmml">×</mo><mn id="S7.p10.1.m1.1.1.2.3" xref="S7.p10.1.m1.1.1.2.3.cmml">240</mn></mrow><mo id="S7.p10.1.m1.1.1.1" xref="S7.p10.1.m1.1.1.1.cmml">⁢</mo><mi mathvariant="normal" id="S7.p10.1.m1.1.1.3" xref="S7.p10.1.m1.1.1.3.cmml">s</mi></mrow><annotation-xml encoding="MathML-Content" id="S7.p10.1.m1.1b"><apply id="S7.p10.1.m1.1.1.cmml" xref="S7.p10.1.m1.1.1"><times id="S7.p10.1.m1.1.1.1.cmml" xref="S7.p10.1.m1.1.1.1"></times><apply id="S7.p10.1.m1.1.1.2.cmml" xref="S7.p10.1.m1.1.1.2"><times id="S7.p10.1.m1.1.1.2.1.cmml" xref="S7.p10.1.m1.1.1.2.1"></times><cn type="integer" id="S7.p10.1.m1.1.1.2.2.cmml" xref="S7.p10.1.m1.1.1.2.2">2</cn><cn type="integer" id="S7.p10.1.m1.1.1.2.3.cmml" xref="S7.p10.1.m1.1.1.2.3">240</cn></apply><ci id="S7.p10.1.m1.1.1.3.cmml" xref="S7.p10.1.m1.1.1.3">s</ci></apply></annotation-xml><annotation encoding="application/x-tex" id="S7.p10.1.m1.1c">\rm 2\times 240s</annotation><annotation encoding="application/x-llamapun" id="S7.p10.1.m1.1d">2 × 240 roman_s</annotation></semantics></math> exposures. Comparison with the formal
error of individual points given by the photometry code (based upon
the flux of the star, the background, and various parameters) showed
that the formal error underestimates our rms with stars brighter than
<math id="S7.p10.2.m2.1" class="ltx_Math" alttext="\rm I\sim 10^{m}" display="inline"><semantics id="S7.p10.2.m2.1a"><mrow id="S7.p10.2.m2.1.1" xref="S7.p10.2.m2.1.1.cmml"><mi mathvariant="normal" id="S7.p10.2.m2.1.1.2" xref="S7.p10.2.m2.1.1.2.cmml">I</mi><mo id="S7.p10.2.m2.1.1.1" xref="S7.p10.2.m2.1.1.1.cmml">∼</mo><msup id="S7.p10.2.m2.1.1.3" xref="S7.p10.2.m2.1.1.3.cmml"><mn id="S7.p10.2.m2.1.1.3.2" xref="S7.p10.2.m2.1.1.3.2.cmml">10</mn><mi mathvariant="normal" id="S7.p10.2.m2.1.1.3.3" xref="S7.p10.2.m2.1.1.3.3.cmml">m</mi></msup></mrow><annotation-xml encoding="MathML-Content" id="S7.p10.2.m2.1b"><apply id="S7.p10.2.m2.1.1.cmml" xref="S7.p10.2.m2.1.1"><csymbol cd="latexml" id="S7.p10.2.m2.1.1.1.cmml" xref="S7.p10.2.m2.1.1.1">similar-to</csymbol><ci id="S7.p10.2.m2.1.1.2.cmml" xref="S7.p10.2.m2.1.1.2">I</ci><apply id="S7.p10.2.m2.1.1.3.cmml" xref="S7.p10.2.m2.1.1.3"><csymbol cd="ambiguous" id="S7.p10.2.m2.1.1.3.1.cmml" xref="S7.p10.2.m2.1.1.3">superscript</csymbol><cn type="integer" id="S7.p10.2.m2.1.1.3.2.cmml" xref="S7.p10.2.m2.1.1.3.2">10</cn><ci id="S7.p10.2.m2.1.1.3.3.cmml" xref="S7.p10.2.m2.1.1.3.3">m</ci></apply></apply></annotation-xml><annotation encoding="application/x-tex" id="S7.p10.2.m2.1c">\rm I\sim 10^{m}</annotation><annotation encoding="application/x-llamapun" id="S7.p10.2.m2.1d">roman_I ∼ 10 start_POSTSUPERSCRIPT roman_m end_POSTSUPERSCRIPT</annotation></semantics></math>, while slightly overestimates at faint sources (see
Fig. <a href="#S7.F7" title="Figure 7 ‣ 7 Photometric precision of HAT ‣ System description and first light-curves of HAT, an autonomous observatory for variability search" class="ltx_ref"><span class="ltx_text ltx_ref_tag">7</span></a>). We also derived the <math id="S7.p10.3.m3.1" class="ltx_Math" alttext="J_{s}" display="inline"><semantics id="S7.p10.3.m3.1a"><msub id="S7.p10.3.m3.1.1" xref="S7.p10.3.m3.1.1.cmml"><mi id="S7.p10.3.m3.1.1.2" xref="S7.p10.3.m3.1.1.2.cmml">J</mi><mi id="S7.p10.3.m3.1.1.3" xref="S7.p10.3.m3.1.1.3.cmml">s</mi></msub><annotation-xml encoding="MathML-Content" id="S7.p10.3.m3.1b"><apply id="S7.p10.3.m3.1.1.cmml" xref="S7.p10.3.m3.1.1"><csymbol cd="ambiguous" id="S7.p10.3.m3.1.1.1.cmml" xref="S7.p10.3.m3.1.1">subscript</csymbol><ci id="S7.p10.3.m3.1.1.2.cmml" xref="S7.p10.3.m3.1.1.2">𝐽</ci><ci id="S7.p10.3.m3.1.1.3.cmml" xref="S7.p10.3.m3.1.1.3">𝑠</ci></apply></annotation-xml><annotation encoding="application/x-tex" id="S7.p10.3.m3.1c">J_{s}</annotation><annotation encoding="application/x-llamapun" id="S7.p10.3.m3.1d">italic_J start_POSTSUBSCRIPT italic_s end_POSTSUBSCRIPT</annotation></semantics></math> Stetson variability
index
for each star <cite class="ltx_cite ltx_citemacro_citep">(Stetson, <a href="#bib.bib47" title="" class="ltx_ref">1996</a>)</cite> following <cite class="ltx_cite ltx_citemacro_citet">Kaluzny et al. (<a href="#bib.bib19" title="" class="ltx_ref">1998</a>)</cite>, where the
formal errors are scaled by a linear relation to the true errors
(Fig. <a href="#S7.F7" title="Figure 7 ‣ 7 Photometric precision of HAT ‣ System description and first light-curves of HAT, an autonomous observatory for variability search" class="ltx_ref"><span class="ltx_text ltx_ref_tag">7</span></a>, lower panel).</p>
</div>
<figure id="S7.F7" class="ltx_figure"><img src="x7.png" id="S7.F7.g1" class="ltx_graphics" width="676" height="718" alt="Photometric precision of HAT with 8 minute exposures. The
upper panel shows the rms of light-curves, which contain at least 100
points. Open circles denote stars with ">
<figcaption class="ltx_caption"><span class="ltx_tag ltx_tag_figure">Figure 7: </span>Photometric precision of HAT with 8 minute exposures. The
upper panel shows the rms of light-curves, which contain at least 100
points. Open circles denote stars with <math id="S7.F7.3.m1.1" class="ltx_Math" alttext="\rm J_{s}" display="inline"><semantics id="S7.F7.3.m1.1b"><msub id="S7.F7.3.m1.1.1" xref="S7.F7.3.m1.1.1.cmml"><mi mathvariant="normal" id="S7.F7.3.m1.1.1.2" xref="S7.F7.3.m1.1.1.2.cmml">J</mi><mi mathvariant="normal" id="S7.F7.3.m1.1.1.3" xref="S7.F7.3.m1.1.1.3.cmml">s</mi></msub><annotation-xml encoding="MathML-Content" id="S7.F7.3.m1.1c"><apply id="S7.F7.3.m1.1.1.cmml" xref="S7.F7.3.m1.1.1"><csymbol cd="ambiguous" id="S7.F7.3.m1.1.1.1.cmml" xref="S7.F7.3.m1.1.1">subscript</csymbol><ci id="S7.F7.3.m1.1.1.2.cmml" xref="S7.F7.3.m1.1.1.2">J</ci><ci id="S7.F7.3.m1.1.1.3.cmml" xref="S7.F7.3.m1.1.1.3">s</ci></apply></annotation-xml><annotation encoding="application/x-tex" id="S7.F7.3.m1.1d">\rm J_{s}</annotation><annotation encoding="application/x-llamapun" id="S7.F7.3.m1.1e">roman_J start_POSTSUBSCRIPT roman_s end_POSTSUBSCRIPT</annotation></semantics></math> variability index
greater than <math id="S7.F7.4.m2.1" class="ltx_Math" alttext="0.75" display="inline"><semantics id="S7.F7.4.m2.1b"><mn id="S7.F7.4.m2.1.1" xref="S7.F7.4.m2.1.1.cmml">0.75</mn><annotation-xml encoding="MathML-Content" id="S7.F7.4.m2.1c"><cn type="float" id="S7.F7.4.m2.1.1.cmml" xref="S7.F7.4.m2.1.1">0.75</cn></annotation-xml><annotation encoding="application/x-tex" id="S7.F7.4.m2.1d">0.75</annotation><annotation encoding="application/x-llamapun" id="S7.F7.4.m2.1e">0.75</annotation></semantics></math>, i.e., suspected variables, while gray points mark
constant stars. The solid line shows the average <span id="S7.F7.7.1" class="ltx_text ltx_font_italic">estimated</span> error
by <span id="S7.F7.8.2" class="ltx_text ltx_font_smallcaps">daophot</span>. The lower panel shows the variability index for
the same stars.
</figcaption>
</figure>
<div id="S7.p11" class="ltx_para">
<p id="S7.p11.3" class="ltx_p">Stars with at least 100 data points and <math id="S7.p11.1.m1.1" class="ltx_Math" alttext="\rm J_{s}&gt;0.75" display="inline"><semantics id="S7.p11.1.m1.1a"><mrow id="S7.p11.1.m1.1.1" xref="S7.p11.1.m1.1.1.cmml"><msub id="S7.p11.1.m1.1.1.2" xref="S7.p11.1.m1.1.1.2.cmml"><mi mathvariant="normal" id="S7.p11.1.m1.1.1.2.2" xref="S7.p11.1.m1.1.1.2.2.cmml">J</mi><mi mathvariant="normal" id="S7.p11.1.m1.1.1.2.3" xref="S7.p11.1.m1.1.1.2.3.cmml">s</mi></msub><mo id="S7.p11.1.m1.1.1.1" xref="S7.p11.1.m1.1.1.1.cmml">&gt;</mo><mn id="S7.p11.1.m1.1.1.3" xref="S7.p11.1.m1.1.1.3.cmml">0.75</mn></mrow><annotation-xml encoding="MathML-Content" id="S7.p11.1.m1.1b"><apply id="S7.p11.1.m1.1.1.cmml" xref="S7.p11.1.m1.1.1"><gt id="S7.p11.1.m1.1.1.1.cmml" xref="S7.p11.1.m1.1.1.1"></gt><apply id="S7.p11.1.m1.1.1.2.cmml" xref="S7.p11.1.m1.1.1.2"><csymbol cd="ambiguous" id="S7.p11.1.m1.1.1.2.1.cmml" xref="S7.p11.1.m1.1.1.2">subscript</csymbol><ci id="S7.p11.1.m1.1.1.2.2.cmml" xref="S7.p11.1.m1.1.1.2.2">J</ci><ci id="S7.p11.1.m1.1.1.2.3.cmml" xref="S7.p11.1.m1.1.1.2.3">s</ci></apply><cn type="float" id="S7.p11.1.m1.1.1.3.cmml" xref="S7.p11.1.m1.1.1.3">0.75</cn></apply></annotation-xml><annotation encoding="application/x-tex" id="S7.p11.1.m1.1c">\rm J_{s}&gt;0.75</annotation><annotation encoding="application/x-llamapun" id="S7.p11.1.m1.1d">roman_J start_POSTSUBSCRIPT roman_s end_POSTSUBSCRIPT &gt; 0.75</annotation></semantics></math> were selected
as suspected variables (<math id="S7.p11.2.m2.1" class="ltx_Math" alttext="\sim 1700" display="inline"><semantics id="S7.p11.2.m2.1a"><mrow id="S7.p11.2.m2.1.1" xref="S7.p11.2.m2.1.1.cmml"><mi id="S7.p11.2.m2.1.1.2" xref="S7.p11.2.m2.1.1.2.cmml"></mi><mo id="S7.p11.2.m2.1.1.1" xref="S7.p11.2.m2.1.1.1.cmml">∼</mo><mn id="S7.p11.2.m2.1.1.3" xref="S7.p11.2.m2.1.1.3.cmml">1700</mn></mrow><annotation-xml encoding="MathML-Content" id="S7.p11.2.m2.1b"><apply id="S7.p11.2.m2.1.1.cmml" xref="S7.p11.2.m2.1.1"><csymbol cd="latexml" id="S7.p11.2.m2.1.1.1.cmml" xref="S7.p11.2.m2.1.1.1">similar-to</csymbol><csymbol cd="latexml" id="S7.p11.2.m2.1.1.2.cmml" xref="S7.p11.2.m2.1.1.2">absent</csymbol><cn type="integer" id="S7.p11.2.m2.1.1.3.cmml" xref="S7.p11.2.m2.1.1.3">1700</cn></apply></annotation-xml><annotation encoding="application/x-tex" id="S7.p11.2.m2.1c">\sim 1700</annotation><annotation encoding="application/x-llamapun" id="S7.p11.2.m2.1d">∼ 1700</annotation></semantics></math>). After standard Fourier analysis,
the selection was further narrowed by looking only at periods shorter
than <math id="S7.p11.3.m3.1" class="ltx_Math" alttext="\rm 25^{d}" display="inline"><semantics id="S7.p11.3.m3.1a"><msup id="S7.p11.3.m3.1.1" xref="S7.p11.3.m3.1.1.cmml"><mn id="S7.p11.3.m3.1.1.2" xref="S7.p11.3.m3.1.1.2.cmml">25</mn><mi mathvariant="normal" id="S7.p11.3.m3.1.1.3" xref="S7.p11.3.m3.1.1.3.cmml">d</mi></msup><annotation-xml encoding="MathML-Content" id="S7.p11.3.m3.1b"><apply id="S7.p11.3.m3.1.1.cmml" xref="S7.p11.3.m3.1.1"><csymbol cd="ambiguous" id="S7.p11.3.m3.1.1.1.cmml" xref="S7.p11.3.m3.1.1">superscript</csymbol><cn type="integer" id="S7.p11.3.m3.1.1.2.cmml" xref="S7.p11.3.m3.1.1.2">25</cn><ci id="S7.p11.3.m3.1.1.3.cmml" xref="S7.p11.3.m3.1.1.3">d</ci></apply></annotation-xml><annotation encoding="application/x-tex" id="S7.p11.3.m3.1c">\rm 25^{d}</annotation><annotation encoding="application/x-llamapun" id="S7.p11.3.m3.1d">25 start_POSTSUPERSCRIPT roman_d end_POSTSUPERSCRIPT</annotation></semantics></math> and Signal Detection Efficiency
<cite class="ltx_cite ltx_citemacro_citep">(Alcock et al., <a href="#bib.bib2" title="" class="ltx_ref">2000</a>; Kovács et al., <a href="#bib.bib22" title="" class="ltx_ref">2002</a>)</cite> of the main peak in the power spectrum
greater than 4.5. Finally, 60, mostly short period light-curves were
selected by hand (see Fig. <a href="#S9.F8" title="Figure 8 ‣ System description and first light-curves of HAT, an autonomous observatory for variability search" class="ltx_ref"><span class="ltx_text ltx_ref_tag">8</span></a>). We cross-correlated the
list with the General Catalogue of Variable Stars
<cite class="ltx_cite ltx_citemacro_citep">(Kholopov et al., <a href="#bib.bib20" title="" class="ltx_ref">1998</a>, GCVS)</cite>, including the New Suspected Variables (NSV), and found
only 12 known and 1 suspected variables. Stars were also looked up in
the Hipparcos <cite class="ltx_cite ltx_citemacro_citep">(Perryman et al., <a href="#bib.bib36" title="" class="ltx_ref">1997b</a>)</cite>, Tycho-2 <cite class="ltx_cite ltx_citemacro_citep">(Høg et al., <a href="#bib.bib17" title="" class="ltx_ref">2000</a>)</cite> catalogues, Tycho
variable stars <cite class="ltx_cite ltx_citemacro_citep">(Piquard et al., <a href="#bib.bib38" title="" class="ltx_ref">2001</a>)</cite>, and the Simbad database<span id="footnote15" class="ltx_note ltx_role_footnote"><sup class="ltx_note_mark">15</sup><span class="ltx_note_outer"><span class="ltx_note_content"><sup class="ltx_note_mark">15</sup><span class="ltx_tag ltx_tag_note">15</span>We
acknowledge the use of the SIMBAD database, operated at CDS,
Strasbourg, France: simbad.u-strasbg.fr</span></span></span>, but only one (out of those
not in the GCVS) was found with variability flag (GSC 0362701580 in
Fig.<a href="#S9.F8" title="Figure 8 ‣ System description and first light-curves of HAT, an autonomous observatory for variability search" class="ltx_ref"><span class="ltx_text ltx_ref_tag">8</span></a>). Eight stars out of the suspected new
discoveries were in the well-known BD, HD or SAO catalogues.</p>
</div>
<div id="S7.p12" class="ltx_para">
<p id="S7.p12.3" class="ltx_p">In addition, we retrieved all variables in our FOV from the GCVS
catalogue having brighter maximum than <math id="S7.p12.1.m1.1" class="ltx_Math" alttext="\rm 13^{m}" display="inline"><semantics id="S7.p12.1.m1.1a"><msup id="S7.p12.1.m1.1.1" xref="S7.p12.1.m1.1.1.cmml"><mn id="S7.p12.1.m1.1.1.2" xref="S7.p12.1.m1.1.1.2.cmml">13</mn><mi mathvariant="normal" id="S7.p12.1.m1.1.1.3" xref="S7.p12.1.m1.1.1.3.cmml">m</mi></msup><annotation-xml encoding="MathML-Content" id="S7.p12.1.m1.1b"><apply id="S7.p12.1.m1.1.1.cmml" xref="S7.p12.1.m1.1.1"><csymbol cd="ambiguous" id="S7.p12.1.m1.1.1.1.cmml" xref="S7.p12.1.m1.1.1">superscript</csymbol><cn type="integer" id="S7.p12.1.m1.1.1.2.cmml" xref="S7.p12.1.m1.1.1.2">13</cn><ci id="S7.p12.1.m1.1.1.3.cmml" xref="S7.p12.1.m1.1.1.3">m</ci></apply></annotation-xml><annotation encoding="application/x-tex" id="S7.p12.1.m1.1c">\rm 13^{m}</annotation><annotation encoding="application/x-llamapun" id="S7.p12.1.m1.1d">13 start_POSTSUPERSCRIPT roman_m end_POSTSUPERSCRIPT</annotation></semantics></math>, not being saturated
on our images, and having a period shorter than <math id="S7.p12.2.m2.1" class="ltx_Math" alttext="\rm 25^{d}" display="inline"><semantics id="S7.p12.2.m2.1a"><msup id="S7.p12.2.m2.1.1" xref="S7.p12.2.m2.1.1.cmml"><mn id="S7.p12.2.m2.1.1.2" xref="S7.p12.2.m2.1.1.2.cmml">25</mn><mi mathvariant="normal" id="S7.p12.2.m2.1.1.3" xref="S7.p12.2.m2.1.1.3.cmml">d</mi></msup><annotation-xml encoding="MathML-Content" id="S7.p12.2.m2.1b"><apply id="S7.p12.2.m2.1.1.cmml" xref="S7.p12.2.m2.1.1"><csymbol cd="ambiguous" id="S7.p12.2.m2.1.1.1.cmml" xref="S7.p12.2.m2.1.1">superscript</csymbol><cn type="integer" id="S7.p12.2.m2.1.1.2.cmml" xref="S7.p12.2.m2.1.1.2">25</cn><ci id="S7.p12.2.m2.1.1.3.cmml" xref="S7.p12.2.m2.1.1.3">d</ci></apply></annotation-xml><annotation encoding="application/x-tex" id="S7.p12.2.m2.1c">\rm 25^{d}</annotation><annotation encoding="application/x-llamapun" id="S7.p12.2.m2.1d">25 start_POSTSUPERSCRIPT roman_d end_POSTSUPERSCRIPT</annotation></semantics></math> (14
entries). Ten out of them were found by our survey, one was too
crowded, and omitted from the original coordinate lists (CZ And),
another variable was too elongated in the corner of our field, thus not
found by <span id="S7.p12.3.1" class="ltx_text ltx_font_smallcaps">daofind</span>. CU And, semi-detached binary was excluded
from our selection due to its small variability index (<math id="S7.p12.3.m3.1" class="ltx_Math" alttext="\rm J_{s}=0.5" display="inline"><semantics id="S7.p12.3.m3.1a"><mrow id="S7.p12.3.m3.1.1" xref="S7.p12.3.m3.1.1.cmml"><msub id="S7.p12.3.m3.1.1.2" xref="S7.p12.3.m3.1.1.2.cmml"><mi mathvariant="normal" id="S7.p12.3.m3.1.1.2.2" xref="S7.p12.3.m3.1.1.2.2.cmml">J</mi><mi mathvariant="normal" id="S7.p12.3.m3.1.1.2.3" xref="S7.p12.3.m3.1.1.2.3.cmml">s</mi></msub><mo id="S7.p12.3.m3.1.1.1" xref="S7.p12.3.m3.1.1.1.cmml">=</mo><mn id="S7.p12.3.m3.1.1.3" xref="S7.p12.3.m3.1.1.3.cmml">0.5</mn></mrow><annotation-xml encoding="MathML-Content" id="S7.p12.3.m3.1b"><apply id="S7.p12.3.m3.1.1.cmml" xref="S7.p12.3.m3.1.1"><eq id="S7.p12.3.m3.1.1.1.cmml" xref="S7.p12.3.m3.1.1.1"></eq><apply id="S7.p12.3.m3.1.1.2.cmml" xref="S7.p12.3.m3.1.1.2"><csymbol cd="ambiguous" id="S7.p12.3.m3.1.1.2.1.cmml" xref="S7.p12.3.m3.1.1.2">subscript</csymbol><ci id="S7.p12.3.m3.1.1.2.2.cmml" xref="S7.p12.3.m3.1.1.2.2">J</ci><ci id="S7.p12.3.m3.1.1.2.3.cmml" xref="S7.p12.3.m3.1.1.2.3">s</ci></apply><cn type="float" id="S7.p12.3.m3.1.1.3.cmml" xref="S7.p12.3.m3.1.1.3">0.5</cn></apply></annotation-xml><annotation encoding="application/x-tex" id="S7.p12.3.m3.1c">\rm J_{s}=0.5</annotation><annotation encoding="application/x-llamapun" id="S7.p12.3.m3.1d">roman_J start_POSTSUBSCRIPT roman_s end_POSTSUBSCRIPT = 0.5</annotation></semantics></math>), although it has a good quality light-curve. Finally, FL Lac was
faint and crowded, and our light-curve was too noisy for selection.</p>
</div>
<div id="S7.p13" class="ltx_para">
<p id="S7.p13.1" class="ltx_p">The 46 new variables are marked with their GSC numbers. Several
long-period variables were also found, but due to the limited time-span
of our observations, we do not include them in the present paper.</p>
</div>
</section>
<section id="S8" class="ltx_section">
<h2 class="ltx_title ltx_title_section">
<span class="ltx_tag ltx_tag_section">8 </span>Summary and future directions</h2>

<div id="S8.p1" class="ltx_para">
<p id="S8.p1.3" class="ltx_p">We described a small, autonomous observatory, which has been working
for one year at Steward Observatory, Kitt Peak. In spite of the small
telephoto lens used as the “telescope”, it can perform massive
photometry of bright sources. It completed on the order of <math id="S8.p1.1.m1.1" class="ltx_Math" alttext="20000" display="inline"><semantics id="S8.p1.1.m1.1a"><mn id="S8.p1.1.m1.1.1" xref="S8.p1.1.m1.1.1.cmml">20000</mn><annotation-xml encoding="MathML-Content" id="S8.p1.1.m1.1b"><cn type="integer" id="S8.p1.1.m1.1.1.cmml" xref="S8.p1.1.m1.1.1">20000</cn></annotation-xml><annotation encoding="application/x-tex" id="S8.p1.1.m1.1c">20000</annotation><annotation encoding="application/x-llamapun" id="S8.p1.1.m1.1d">20000</annotation></semantics></math>
pointings to different objects, yielding a data flow of <math id="S8.p1.2.m2.1" class="ltx_Math" alttext="\sim 10^{6}" display="inline"><semantics id="S8.p1.2.m2.1a"><mrow id="S8.p1.2.m2.1.1" xref="S8.p1.2.m2.1.1.cmml"><mi id="S8.p1.2.m2.1.1.2" xref="S8.p1.2.m2.1.1.2.cmml"></mi><mo id="S8.p1.2.m2.1.1.1" xref="S8.p1.2.m2.1.1.1.cmml">∼</mo><msup id="S8.p1.2.m2.1.1.3" xref="S8.p1.2.m2.1.1.3.cmml"><mn id="S8.p1.2.m2.1.1.3.2" xref="S8.p1.2.m2.1.1.3.2.cmml">10</mn><mn id="S8.p1.2.m2.1.1.3.3" xref="S8.p1.2.m2.1.1.3.3.cmml">6</mn></msup></mrow><annotation-xml encoding="MathML-Content" id="S8.p1.2.m2.1b"><apply id="S8.p1.2.m2.1.1.cmml" xref="S8.p1.2.m2.1.1"><csymbol cd="latexml" id="S8.p1.2.m2.1.1.1.cmml" xref="S8.p1.2.m2.1.1.1">similar-to</csymbol><csymbol cd="latexml" id="S8.p1.2.m2.1.1.2.cmml" xref="S8.p1.2.m2.1.1.2">absent</csymbol><apply id="S8.p1.2.m2.1.1.3.cmml" xref="S8.p1.2.m2.1.1.3"><csymbol cd="ambiguous" id="S8.p1.2.m2.1.1.3.1.cmml" xref="S8.p1.2.m2.1.1.3">superscript</csymbol><cn type="integer" id="S8.p1.2.m2.1.1.3.2.cmml" xref="S8.p1.2.m2.1.1.3.2">10</cn><cn type="integer" id="S8.p1.2.m2.1.1.3.3.cmml" xref="S8.p1.2.m2.1.1.3.3">6</cn></apply></apply></annotation-xml><annotation encoding="application/x-tex" id="S8.p1.2.m2.1c">\sim 10^{6}</annotation><annotation encoding="application/x-llamapun" id="S8.p1.2.m2.1d">∼ 10 start_POSTSUPERSCRIPT 6 end_POSTSUPERSCRIPT</annotation></semantics></math>
photometric measurements per night, and proved to work reliably. Using
5% of our data, and a single selected field, a few dozen bright
(<math id="S8.p1.3.m3.1" class="ltx_Math" alttext="I&lt;13" display="inline"><semantics id="S8.p1.3.m3.1a"><mrow id="S8.p1.3.m3.1.1" xref="S8.p1.3.m3.1.1.cmml"><mi id="S8.p1.3.m3.1.1.2" xref="S8.p1.3.m3.1.1.2.cmml">I</mi><mo id="S8.p1.3.m3.1.1.1" xref="S8.p1.3.m3.1.1.1.cmml">&lt;</mo><mn id="S8.p1.3.m3.1.1.3" xref="S8.p1.3.m3.1.1.3.cmml">13</mn></mrow><annotation-xml encoding="MathML-Content" id="S8.p1.3.m3.1b"><apply id="S8.p1.3.m3.1.1.cmml" xref="S8.p1.3.m3.1.1"><lt id="S8.p1.3.m3.1.1.1.cmml" xref="S8.p1.3.m3.1.1.1"></lt><ci id="S8.p1.3.m3.1.1.2.cmml" xref="S8.p1.3.m3.1.1.2">𝐼</ci><cn type="integer" id="S8.p1.3.m3.1.1.3.cmml" xref="S8.p1.3.m3.1.1.3">13</cn></apply></annotation-xml><annotation encoding="application/x-tex" id="S8.p1.3.m3.1c">I&lt;13</annotation><annotation encoding="application/x-llamapun" id="S8.p1.3.m3.1d">italic_I &lt; 13</annotation></semantics></math>) variables were found, further reinforcing the incompleteness
even on the bright end of previous variability searches. Some
parameters of the system are summarized in Table <a href="#S8.T3" title="Table 3 ‣ 8 Summary and future directions ‣ System description and first light-curves of HAT, an autonomous observatory for variability search" class="ltx_ref"><span class="ltx_text ltx_ref_tag">3</span></a>.</p>
</div>
<div id="S8.p2" class="ltx_para">
<p id="S8.p2.1" class="ltx_p">HAT-1 is capable of monitoring a <span id="S8.p2.1.1" class="ltx_text ltx_font_italic">small fraction</span> of the sky with
sufficient time resolution. Multiple filters would not only ease
classification of sources based upon their colors, but due to
differential refraction (vs. color), the photometric precision would be
also improved. Wider aperture would increase the incoming flux, and
either our time-resolution at constant photometric precision would
drop, or our limiting magnitude with the same exposure times would be
expanded. This would not be necesseraly <span id="S8.p2.1.2" class="ltx_text ltx_font_italic">improvement</span> of the
system, but in some sense the target of observations would be
different. HAT is designed to operate in a network. It is an
off-the-shelf system of <math id="S8.p2.1.m1.1" class="ltx_Math" alttext="\rm\sim 15K" display="inline"><semantics id="S8.p2.1.m1.1a"><mrow id="S8.p2.1.m1.1.1" xref="S8.p2.1.m1.1.1.cmml"><mi id="S8.p2.1.m1.1.1.2" xref="S8.p2.1.m1.1.1.2.cmml"></mi><mo id="S8.p2.1.m1.1.1.1" xref="S8.p2.1.m1.1.1.1.cmml">∼</mo><mrow id="S8.p2.1.m1.1.1.3" xref="S8.p2.1.m1.1.1.3.cmml"><mn id="S8.p2.1.m1.1.1.3.2" xref="S8.p2.1.m1.1.1.3.2.cmml">15</mn><mo id="S8.p2.1.m1.1.1.3.1" xref="S8.p2.1.m1.1.1.3.1.cmml">⁢</mo><mi mathvariant="normal" id="S8.p2.1.m1.1.1.3.3" xref="S8.p2.1.m1.1.1.3.3.cmml">K</mi></mrow></mrow><annotation-xml encoding="MathML-Content" id="S8.p2.1.m1.1b"><apply id="S8.p2.1.m1.1.1.cmml" xref="S8.p2.1.m1.1.1"><csymbol cd="latexml" id="S8.p2.1.m1.1.1.1.cmml" xref="S8.p2.1.m1.1.1.1">similar-to</csymbol><csymbol cd="latexml" id="S8.p2.1.m1.1.1.2.cmml" xref="S8.p2.1.m1.1.1.2">absent</csymbol><apply id="S8.p2.1.m1.1.1.3.cmml" xref="S8.p2.1.m1.1.1.3"><times id="S8.p2.1.m1.1.1.3.1.cmml" xref="S8.p2.1.m1.1.1.3.1"></times><cn type="integer" id="S8.p2.1.m1.1.1.3.2.cmml" xref="S8.p2.1.m1.1.1.3.2">15</cn><ci id="S8.p2.1.m1.1.1.3.3.cmml" xref="S8.p2.1.m1.1.1.3.3">K</ci></apply></apply></annotation-xml><annotation encoding="application/x-tex" id="S8.p2.1.m1.1c">\rm\sim 15K</annotation><annotation encoding="application/x-llamapun" id="S8.p2.1.m1.1d">∼ 15 roman_K</annotation></semantics></math> USD cost, plus the CCD. These
make multiple installations easy. It is also a flexible, multi-purpose
observatory, not only eligible for all-sky monitoring, but could be
used as a photometric monitor station at bigger observatories. Current
bottleneck is neither hardware development or operating HAT, but rather
efficient data reduction, archiving and web-availability of the data.
After efficient software can handle the data flow and overcome the
difficulties, we plan to upgrade HAT with more filters, bigger
telescopes, and to install multiple stations.</p>
</div>
<figure id="S8.T3" class="ltx_table">
<figcaption class="ltx_caption"><span class="ltx_tag ltx_tag_table">Table 3: </span>Specification of HAT-1 at Kitt Peak.
</figcaption>
<table id="S8.T3.26" class="ltx_tabular ltx_guessed_headers ltx_align_middle">
<thead class="ltx_thead">
<tr id="S8.T3.26.27.1" class="ltx_tr">
<th id="S8.T3.26.27.1.1" class="ltx_td ltx_align_center ltx_th ltx_th_row ltx_border_t" colspan="2">Equatorial mount</th>
</tr>
</thead>
<tbody class="ltx_tbody">
<tr id="S8.T3.1.1" class="ltx_tr">
<th id="S8.T3.1.1.2" class="ltx_td ltx_align_left ltx_th ltx_th_row ltx_border_tt">Max. RA slew</th>
<td id="S8.T3.1.1.1" class="ltx_td ltx_align_right ltx_border_tt">
<math id="S8.T3.1.1.1.m1.1" class="ltx_Math" alttext="\rm 2\arcdeg/" display="inline"><semantics id="S8.T3.1.1.1.m1.1a"><mrow id="S8.T3.1.1.1.m1.1b"><mn id="S8.T3.1.1.1.m1.1.1" xref="S8.T3.1.1.1.m1.1.1.cmml">2</mn><mi mathvariant="normal" id="S8.T3.1.1.1.m1.1.2" xref="S8.T3.1.1.1.m1.1.2.cmml">°</mi><mo id="S8.T3.1.1.1.m1.1.3" xref="S8.T3.1.1.1.m1.1.3.cmml">/</mo></mrow><annotation-xml encoding="MathML-Content" id="S8.T3.1.1.1.m1.1c"><cerror id="S8.T3.1.1.1.m1.1d"><csymbol cd="ambiguous" id="S8.T3.1.1.1.m1.1e">fragments</csymbol><cn type="integer" id="S8.T3.1.1.1.m1.1.1.cmml" xref="S8.T3.1.1.1.m1.1.1">2</cn><csymbol cd="unknown" id="S8.T3.1.1.1.m1.1.2.cmml" xref="S8.T3.1.1.1.m1.1.2">°</csymbol><divide id="S8.T3.1.1.1.m1.1.3.cmml" xref="S8.T3.1.1.1.m1.1.3"></divide></cerror></annotation-xml><annotation encoding="application/x-tex" id="S8.T3.1.1.1.m1.1f">\rm 2\arcdeg/</annotation><annotation encoding="application/x-llamapun" id="S8.T3.1.1.1.m1.1g">2 ° /</annotation></semantics></math>sec</td>
</tr>
<tr id="S8.T3.2.2" class="ltx_tr">
<th id="S8.T3.2.2.2" class="ltx_td ltx_align_left ltx_th ltx_th_row">Max. Dec slew</th>
<td id="S8.T3.2.2.1" class="ltx_td ltx_align_right">
<math id="S8.T3.2.2.1.m1.1" class="ltx_Math" alttext="\rm 5\arcdeg/" display="inline"><semantics id="S8.T3.2.2.1.m1.1a"><mrow id="S8.T3.2.2.1.m1.1b"><mn id="S8.T3.2.2.1.m1.1.1" xref="S8.T3.2.2.1.m1.1.1.cmml">5</mn><mi mathvariant="normal" id="S8.T3.2.2.1.m1.1.2" xref="S8.T3.2.2.1.m1.1.2.cmml">°</mi><mo id="S8.T3.2.2.1.m1.1.3" xref="S8.T3.2.2.1.m1.1.3.cmml">/</mo></mrow><annotation-xml encoding="MathML-Content" id="S8.T3.2.2.1.m1.1c"><cerror id="S8.T3.2.2.1.m1.1d"><csymbol cd="ambiguous" id="S8.T3.2.2.1.m1.1e">fragments</csymbol><cn type="integer" id="S8.T3.2.2.1.m1.1.1.cmml" xref="S8.T3.2.2.1.m1.1.1">5</cn><csymbol cd="unknown" id="S8.T3.2.2.1.m1.1.2.cmml" xref="S8.T3.2.2.1.m1.1.2">°</csymbol><divide id="S8.T3.2.2.1.m1.1.3.cmml" xref="S8.T3.2.2.1.m1.1.3"></divide></cerror></annotation-xml><annotation encoding="application/x-tex" id="S8.T3.2.2.1.m1.1f">\rm 5\arcdeg/</annotation><annotation encoding="application/x-llamapun" id="S8.T3.2.2.1.m1.1g">5 ° /</annotation></semantics></math>sec</td>
</tr>
<tr id="S8.T3.3.3" class="ltx_tr">
<th id="S8.T3.3.3.2" class="ltx_td ltx_align_left ltx_th ltx_th_row">RA resolution</th>
<td id="S8.T3.3.3.1" class="ltx_td ltx_align_right">
<math id="S8.T3.3.3.1.m1.1" class="ltx_Math" alttext="\rm 1\arcsec/" display="inline"><semantics id="S8.T3.3.3.1.m1.1a"><mrow id="S8.T3.3.3.1.m1.1b"><mn id="S8.T3.3.3.1.m1.1.1" xref="S8.T3.3.3.1.m1.1.1.cmml">1</mn><mi mathvariant="normal" id="S8.T3.3.3.1.m1.1.2" xref="S8.T3.3.3.1.m1.1.2.cmml">″</mi><mo id="S8.T3.3.3.1.m1.1.3" xref="S8.T3.3.3.1.m1.1.3.cmml">/</mo></mrow><annotation-xml encoding="MathML-Content" id="S8.T3.3.3.1.m1.1c"><cerror id="S8.T3.3.3.1.m1.1d"><csymbol cd="ambiguous" id="S8.T3.3.3.1.m1.1e">fragments</csymbol><cn type="integer" id="S8.T3.3.3.1.m1.1.1.cmml" xref="S8.T3.3.3.1.m1.1.1">1</cn><csymbol cd="unknown" id="S8.T3.3.3.1.m1.1.2.cmml" xref="S8.T3.3.3.1.m1.1.2">″</csymbol><divide id="S8.T3.3.3.1.m1.1.3.cmml" xref="S8.T3.3.3.1.m1.1.3"></divide></cerror></annotation-xml><annotation encoding="application/x-tex" id="S8.T3.3.3.1.m1.1f">\rm 1\arcsec/</annotation><annotation encoding="application/x-llamapun" id="S8.T3.3.3.1.m1.1g">1 ″ /</annotation></semantics></math>step</td>
</tr>
<tr id="S8.T3.4.4" class="ltx_tr">
<th id="S8.T3.4.4.2" class="ltx_td ltx_align_left ltx_th ltx_th_row">Dec resolution</th>
<td id="S8.T3.4.4.1" class="ltx_td ltx_align_right">
<math id="S8.T3.4.4.1.m1.1" class="ltx_Math" alttext="\rm 5\arcsec/" display="inline"><semantics id="S8.T3.4.4.1.m1.1a"><mrow id="S8.T3.4.4.1.m1.1b"><mn id="S8.T3.4.4.1.m1.1.1" xref="S8.T3.4.4.1.m1.1.1.cmml">5</mn><mi mathvariant="normal" id="S8.T3.4.4.1.m1.1.2" xref="S8.T3.4.4.1.m1.1.2.cmml">″</mi><mo id="S8.T3.4.4.1.m1.1.3" xref="S8.T3.4.4.1.m1.1.3.cmml">/</mo></mrow><annotation-xml encoding="MathML-Content" id="S8.T3.4.4.1.m1.1c"><cerror id="S8.T3.4.4.1.m1.1d"><csymbol cd="ambiguous" id="S8.T3.4.4.1.m1.1e">fragments</csymbol><cn type="integer" id="S8.T3.4.4.1.m1.1.1.cmml" xref="S8.T3.4.4.1.m1.1.1">5</cn><csymbol cd="unknown" id="S8.T3.4.4.1.m1.1.2.cmml" xref="S8.T3.4.4.1.m1.1.2">″</csymbol><divide id="S8.T3.4.4.1.m1.1.3.cmml" xref="S8.T3.4.4.1.m1.1.3"></divide></cerror></annotation-xml><annotation encoding="application/x-tex" id="S8.T3.4.4.1.m1.1f">\rm 5\arcsec/</annotation><annotation encoding="application/x-llamapun" id="S8.T3.4.4.1.m1.1g">5 ″ /</annotation></semantics></math>step</td>
</tr>
<tr id="S8.T3.26.28.1" class="ltx_tr">
<th id="S8.T3.26.28.1.1" class="ltx_td ltx_align_left ltx_th ltx_th_row">Max. tel. diam.</th>
<td id="S8.T3.26.28.1.2" class="ltx_td ltx_align_right">20cm</td>
</tr>
<tr id="S8.T3.26.29.2" class="ltx_tr">
<th id="S8.T3.26.29.2.1" class="ltx_td ltx_align_center ltx_th ltx_th_row ltx_border_t" colspan="2">CCD (Apogee AP10)</th>
</tr>
<tr id="S8.T3.6.6" class="ltx_tr">
<th id="S8.T3.6.6.3" class="ltx_td ltx_align_left ltx_th ltx_th_row ltx_border_tt">Dimensions</th>
<td id="S8.T3.6.6.2" class="ltx_td ltx_align_right ltx_border_tt">
<math id="S8.T3.5.5.1.m1.1" class="ltx_Math" alttext="\rm 2K\times 2K" display="inline"><semantics id="S8.T3.5.5.1.m1.1a"><mrow id="S8.T3.5.5.1.m1.1.1" xref="S8.T3.5.5.1.m1.1.1.cmml"><mrow id="S8.T3.5.5.1.m1.1.1.2" xref="S8.T3.5.5.1.m1.1.1.2.cmml"><mrow id="S8.T3.5.5.1.m1.1.1.2.2" xref="S8.T3.5.5.1.m1.1.1.2.2.cmml"><mn id="S8.T3.5.5.1.m1.1.1.2.2.2" xref="S8.T3.5.5.1.m1.1.1.2.2.2.cmml">2</mn><mo id="S8.T3.5.5.1.m1.1.1.2.2.1" xref="S8.T3.5.5.1.m1.1.1.2.2.1.cmml">⁢</mo><mi mathvariant="normal" id="S8.T3.5.5.1.m1.1.1.2.2.3" xref="S8.T3.5.5.1.m1.1.1.2.2.3.cmml">K</mi></mrow><mo id="S8.T3.5.5.1.m1.1.1.2.1" xref="S8.T3.5.5.1.m1.1.1.2.1.cmml">×</mo><mn id="S8.T3.5.5.1.m1.1.1.2.3" xref="S8.T3.5.5.1.m1.1.1.2.3.cmml">2</mn></mrow><mo id="S8.T3.5.5.1.m1.1.1.1" xref="S8.T3.5.5.1.m1.1.1.1.cmml">⁢</mo><mi mathvariant="normal" id="S8.T3.5.5.1.m1.1.1.3" xref="S8.T3.5.5.1.m1.1.1.3.cmml">K</mi></mrow><annotation-xml encoding="MathML-Content" id="S8.T3.5.5.1.m1.1b"><apply id="S8.T3.5.5.1.m1.1.1.cmml" xref="S8.T3.5.5.1.m1.1.1"><times id="S8.T3.5.5.1.m1.1.1.1.cmml" xref="S8.T3.5.5.1.m1.1.1.1"></times><apply id="S8.T3.5.5.1.m1.1.1.2.cmml" xref="S8.T3.5.5.1.m1.1.1.2"><times id="S8.T3.5.5.1.m1.1.1.2.1.cmml" xref="S8.T3.5.5.1.m1.1.1.2.1"></times><apply id="S8.T3.5.5.1.m1.1.1.2.2.cmml" xref="S8.T3.5.5.1.m1.1.1.2.2"><times id="S8.T3.5.5.1.m1.1.1.2.2.1.cmml" xref="S8.T3.5.5.1.m1.1.1.2.2.1"></times><cn type="integer" id="S8.T3.5.5.1.m1.1.1.2.2.2.cmml" xref="S8.T3.5.5.1.m1.1.1.2.2.2">2</cn><ci id="S8.T3.5.5.1.m1.1.1.2.2.3.cmml" xref="S8.T3.5.5.1.m1.1.1.2.2.3">K</ci></apply><cn type="integer" id="S8.T3.5.5.1.m1.1.1.2.3.cmml" xref="S8.T3.5.5.1.m1.1.1.2.3">2</cn></apply><ci id="S8.T3.5.5.1.m1.1.1.3.cmml" xref="S8.T3.5.5.1.m1.1.1.3">K</ci></apply></annotation-xml><annotation encoding="application/x-tex" id="S8.T3.5.5.1.m1.1c">\rm 2K\times 2K</annotation><annotation encoding="application/x-llamapun" id="S8.T3.5.5.1.m1.1d">2 roman_K × 2 roman_K</annotation></semantics></math>, <math id="S8.T3.6.6.2.m2.1" class="ltx_Math" alttext="\rm 14" display="inline"><semantics id="S8.T3.6.6.2.m2.1a"><mn id="S8.T3.6.6.2.m2.1.1" xref="S8.T3.6.6.2.m2.1.1.cmml">14</mn><annotation-xml encoding="MathML-Content" id="S8.T3.6.6.2.m2.1b"><cn type="integer" id="S8.T3.6.6.2.m2.1.1.cmml" xref="S8.T3.6.6.2.m2.1.1">14</cn></annotation-xml><annotation encoding="application/x-tex" id="S8.T3.6.6.2.m2.1c">\rm 14</annotation><annotation encoding="application/x-llamapun" id="S8.T3.6.6.2.m2.1d">14</annotation></semantics></math> micron</td>
</tr>
<tr id="S8.T3.26.30.3" class="ltx_tr">
<th id="S8.T3.26.30.3.1" class="ltx_td ltx_align_left ltx_th ltx_th_row">Gain</th>
<td id="S8.T3.26.30.3.2" class="ltx_td ltx_align_right">10 e-/ADU</td>
</tr>
<tr id="S8.T3.26.31.4" class="ltx_tr">
<th id="S8.T3.26.31.4.1" class="ltx_td ltx_align_left ltx_th ltx_th_row">Readout noise</th>
<td id="S8.T3.26.31.4.2" class="ltx_td ltx_align_right">2 ADU</td>
</tr>
<tr id="S8.T3.8.8" class="ltx_tr">
<th id="S8.T3.8.8.3" class="ltx_td ltx_align_left ltx_th ltx_th_row">Dark current</th>
<td id="S8.T3.8.8.2" class="ltx_td ltx_align_right">
<math id="S8.T3.7.7.1.m1.1" class="ltx_Math" alttext="\rm 0.05ADU/sec" display="inline"><semantics id="S8.T3.7.7.1.m1.1a"><mrow id="S8.T3.7.7.1.m1.1.1" xref="S8.T3.7.7.1.m1.1.1.cmml"><mrow id="S8.T3.7.7.1.m1.1.1.2" xref="S8.T3.7.7.1.m1.1.1.2.cmml"><mn id="S8.T3.7.7.1.m1.1.1.2.2" xref="S8.T3.7.7.1.m1.1.1.2.2.cmml">0.05</mn><mo id="S8.T3.7.7.1.m1.1.1.2.1" xref="S8.T3.7.7.1.m1.1.1.2.1.cmml">⁢</mo><mi id="S8.T3.7.7.1.m1.1.1.2.3" xref="S8.T3.7.7.1.m1.1.1.2.3.cmml">ADU</mi></mrow><mo id="S8.T3.7.7.1.m1.1.1.1" xref="S8.T3.7.7.1.m1.1.1.1.cmml">/</mo><mi id="S8.T3.7.7.1.m1.1.1.3" xref="S8.T3.7.7.1.m1.1.1.3.cmml">sec</mi></mrow><annotation-xml encoding="MathML-Content" id="S8.T3.7.7.1.m1.1b"><apply id="S8.T3.7.7.1.m1.1.1.cmml" xref="S8.T3.7.7.1.m1.1.1"><divide id="S8.T3.7.7.1.m1.1.1.1.cmml" xref="S8.T3.7.7.1.m1.1.1.1"></divide><apply id="S8.T3.7.7.1.m1.1.1.2.cmml" xref="S8.T3.7.7.1.m1.1.1.2"><times id="S8.T3.7.7.1.m1.1.1.2.1.cmml" xref="S8.T3.7.7.1.m1.1.1.2.1"></times><cn type="float" id="S8.T3.7.7.1.m1.1.1.2.2.cmml" xref="S8.T3.7.7.1.m1.1.1.2.2">0.05</cn><ci id="S8.T3.7.7.1.m1.1.1.2.3.cmml" xref="S8.T3.7.7.1.m1.1.1.2.3">ADU</ci></apply><ci id="S8.T3.7.7.1.m1.1.1.3.cmml" xref="S8.T3.7.7.1.m1.1.1.3">sec</ci></apply></annotation-xml><annotation encoding="application/x-tex" id="S8.T3.7.7.1.m1.1c">\rm 0.05ADU/sec</annotation><annotation encoding="application/x-llamapun" id="S8.T3.7.7.1.m1.1d">0.05 roman_ADU / roman_sec</annotation></semantics></math> at <math id="S8.T3.8.8.2.m2.1" class="ltx_Math" alttext="-15\arcdeg C" display="inline"><semantics id="S8.T3.8.8.2.m2.1a"><mrow id="S8.T3.8.8.2.m2.1.1" xref="S8.T3.8.8.2.m2.1.1.cmml"><mo id="S8.T3.8.8.2.m2.1.1.1" xref="S8.T3.8.8.2.m2.1.1.1.cmml">-</mo><mrow id="S8.T3.8.8.2.m2.1.1.2" xref="S8.T3.8.8.2.m2.1.1.2.cmml"><mn id="S8.T3.8.8.2.m2.1.1.2.2" xref="S8.T3.8.8.2.m2.1.1.2.2.cmml">15</mn><mo id="S8.T3.8.8.2.m2.1.1.2.1" xref="S8.T3.8.8.2.m2.1.1.2.1.cmml">⁢</mo><mi mathvariant="normal" id="S8.T3.8.8.2.m2.1.1.2.3" xref="S8.T3.8.8.2.m2.1.1.2.3.cmml">°</mi><mo id="S8.T3.8.8.2.m2.1.1.2.1a" xref="S8.T3.8.8.2.m2.1.1.2.1.cmml">⁢</mo><mi id="S8.T3.8.8.2.m2.1.1.2.4" xref="S8.T3.8.8.2.m2.1.1.2.4.cmml">C</mi></mrow></mrow><annotation-xml encoding="MathML-Content" id="S8.T3.8.8.2.m2.1b"><apply id="S8.T3.8.8.2.m2.1.1.cmml" xref="S8.T3.8.8.2.m2.1.1"><minus id="S8.T3.8.8.2.m2.1.1.1.cmml" xref="S8.T3.8.8.2.m2.1.1.1"></minus><apply id="S8.T3.8.8.2.m2.1.1.2.cmml" xref="S8.T3.8.8.2.m2.1.1.2"><times id="S8.T3.8.8.2.m2.1.1.2.1.cmml" xref="S8.T3.8.8.2.m2.1.1.2.1"></times><cn type="integer" id="S8.T3.8.8.2.m2.1.1.2.2.cmml" xref="S8.T3.8.8.2.m2.1.1.2.2">15</cn><ci id="S8.T3.8.8.2.m2.1.1.2.3.cmml" xref="S8.T3.8.8.2.m2.1.1.2.3">°</ci><ci id="S8.T3.8.8.2.m2.1.1.2.4.cmml" xref="S8.T3.8.8.2.m2.1.1.2.4">𝐶</ci></apply></apply></annotation-xml><annotation encoding="application/x-tex" id="S8.T3.8.8.2.m2.1c">-15\arcdeg C</annotation><annotation encoding="application/x-llamapun" id="S8.T3.8.8.2.m2.1d">- 15 ° italic_C</annotation></semantics></math>
</td>
</tr>
<tr id="S8.T3.9.9" class="ltx_tr">
<th id="S8.T3.9.9.2" class="ltx_td ltx_align_left ltx_th ltx_th_row">Readout time</th>
<td id="S8.T3.9.9.1" class="ltx_td ltx_align_right"><math id="S8.T3.9.9.1.m1.1" class="ltx_Math" alttext="\rm&lt;10s" display="inline"><semantics id="S8.T3.9.9.1.m1.1a"><mrow id="S8.T3.9.9.1.m1.1.1" xref="S8.T3.9.9.1.m1.1.1.cmml"><mi id="S8.T3.9.9.1.m1.1.1.2" xref="S8.T3.9.9.1.m1.1.1.2.cmml"></mi><mo id="S8.T3.9.9.1.m1.1.1.1" xref="S8.T3.9.9.1.m1.1.1.1.cmml">&lt;</mo><mrow id="S8.T3.9.9.1.m1.1.1.3" xref="S8.T3.9.9.1.m1.1.1.3.cmml"><mn id="S8.T3.9.9.1.m1.1.1.3.2" xref="S8.T3.9.9.1.m1.1.1.3.2.cmml">10</mn><mo id="S8.T3.9.9.1.m1.1.1.3.1" xref="S8.T3.9.9.1.m1.1.1.3.1.cmml">⁢</mo><mi mathvariant="normal" id="S8.T3.9.9.1.m1.1.1.3.3" xref="S8.T3.9.9.1.m1.1.1.3.3.cmml">s</mi></mrow></mrow><annotation-xml encoding="MathML-Content" id="S8.T3.9.9.1.m1.1b"><apply id="S8.T3.9.9.1.m1.1.1.cmml" xref="S8.T3.9.9.1.m1.1.1"><lt id="S8.T3.9.9.1.m1.1.1.1.cmml" xref="S8.T3.9.9.1.m1.1.1.1"></lt><csymbol cd="latexml" id="S8.T3.9.9.1.m1.1.1.2.cmml" xref="S8.T3.9.9.1.m1.1.1.2">absent</csymbol><apply id="S8.T3.9.9.1.m1.1.1.3.cmml" xref="S8.T3.9.9.1.m1.1.1.3"><times id="S8.T3.9.9.1.m1.1.1.3.1.cmml" xref="S8.T3.9.9.1.m1.1.1.3.1"></times><cn type="integer" id="S8.T3.9.9.1.m1.1.1.3.2.cmml" xref="S8.T3.9.9.1.m1.1.1.3.2">10</cn><ci id="S8.T3.9.9.1.m1.1.1.3.3.cmml" xref="S8.T3.9.9.1.m1.1.1.3.3">s</ci></apply></apply></annotation-xml><annotation encoding="application/x-tex" id="S8.T3.9.9.1.m1.1c">\rm&lt;10s</annotation><annotation encoding="application/x-llamapun" id="S8.T3.9.9.1.m1.1d">&lt; 10 roman_s</annotation></semantics></math></td>
</tr>
<tr id="S8.T3.10.10" class="ltx_tr">
<th id="S8.T3.10.10.2" class="ltx_td ltx_align_left ltx_th ltx_th_row">Cooling</th>
<td id="S8.T3.10.10.1" class="ltx_td ltx_align_right">Peltier (<math id="S8.T3.10.10.1.m1.1" class="ltx_Math" alttext="\rm\Delta T=32\arcdeg C" display="inline"><semantics id="S8.T3.10.10.1.m1.1a"><mrow id="S8.T3.10.10.1.m1.1.1" xref="S8.T3.10.10.1.m1.1.1.cmml"><mrow id="S8.T3.10.10.1.m1.1.1.2" xref="S8.T3.10.10.1.m1.1.1.2.cmml"><mi mathvariant="normal" id="S8.T3.10.10.1.m1.1.1.2.2" xref="S8.T3.10.10.1.m1.1.1.2.2.cmml">Δ</mi><mo id="S8.T3.10.10.1.m1.1.1.2.1" xref="S8.T3.10.10.1.m1.1.1.2.1.cmml">⁢</mo><mi mathvariant="normal" id="S8.T3.10.10.1.m1.1.1.2.3" xref="S8.T3.10.10.1.m1.1.1.2.3.cmml">T</mi></mrow><mo id="S8.T3.10.10.1.m1.1.1.1" xref="S8.T3.10.10.1.m1.1.1.1.cmml">=</mo><mrow id="S8.T3.10.10.1.m1.1.1.3" xref="S8.T3.10.10.1.m1.1.1.3.cmml"><mn id="S8.T3.10.10.1.m1.1.1.3.2" xref="S8.T3.10.10.1.m1.1.1.3.2.cmml">32</mn><mo id="S8.T3.10.10.1.m1.1.1.3.1" xref="S8.T3.10.10.1.m1.1.1.3.1.cmml">⁢</mo><mi mathvariant="normal" id="S8.T3.10.10.1.m1.1.1.3.3" xref="S8.T3.10.10.1.m1.1.1.3.3.cmml">°</mi><mo id="S8.T3.10.10.1.m1.1.1.3.1a" xref="S8.T3.10.10.1.m1.1.1.3.1.cmml">⁢</mo><mi mathvariant="normal" id="S8.T3.10.10.1.m1.1.1.3.4" xref="S8.T3.10.10.1.m1.1.1.3.4.cmml">C</mi></mrow></mrow><annotation-xml encoding="MathML-Content" id="S8.T3.10.10.1.m1.1b"><apply id="S8.T3.10.10.1.m1.1.1.cmml" xref="S8.T3.10.10.1.m1.1.1"><eq id="S8.T3.10.10.1.m1.1.1.1.cmml" xref="S8.T3.10.10.1.m1.1.1.1"></eq><apply id="S8.T3.10.10.1.m1.1.1.2.cmml" xref="S8.T3.10.10.1.m1.1.1.2"><times id="S8.T3.10.10.1.m1.1.1.2.1.cmml" xref="S8.T3.10.10.1.m1.1.1.2.1"></times><ci id="S8.T3.10.10.1.m1.1.1.2.2.cmml" xref="S8.T3.10.10.1.m1.1.1.2.2">Δ</ci><ci id="S8.T3.10.10.1.m1.1.1.2.3.cmml" xref="S8.T3.10.10.1.m1.1.1.2.3">T</ci></apply><apply id="S8.T3.10.10.1.m1.1.1.3.cmml" xref="S8.T3.10.10.1.m1.1.1.3"><times id="S8.T3.10.10.1.m1.1.1.3.1.cmml" xref="S8.T3.10.10.1.m1.1.1.3.1"></times><cn type="integer" id="S8.T3.10.10.1.m1.1.1.3.2.cmml" xref="S8.T3.10.10.1.m1.1.1.3.2">32</cn><ci id="S8.T3.10.10.1.m1.1.1.3.3.cmml" xref="S8.T3.10.10.1.m1.1.1.3.3">°</ci><ci id="S8.T3.10.10.1.m1.1.1.3.4.cmml" xref="S8.T3.10.10.1.m1.1.1.3.4">C</ci></apply></apply></annotation-xml><annotation encoding="application/x-tex" id="S8.T3.10.10.1.m1.1c">\rm\Delta T=32\arcdeg C</annotation><annotation encoding="application/x-llamapun" id="S8.T3.10.10.1.m1.1d">roman_Δ roman_T = 32 ° roman_C</annotation></semantics></math>)</td>
</tr>
<tr id="S8.T3.26.32.5" class="ltx_tr">
<th id="S8.T3.26.32.5.1" class="ltx_td ltx_align_center ltx_th ltx_th_row ltx_border_t" colspan="2">Lens (Nikon 180mm f/2.8)</th>
</tr>
<tr id="S8.T3.26.33.6" class="ltx_tr">
<th id="S8.T3.26.33.6.1" class="ltx_td ltx_align_left ltx_th ltx_th_row ltx_border_tt">Aperture</th>
<td id="S8.T3.26.33.6.2" class="ltx_td ltx_align_right ltx_border_tt">65mm</td>
</tr>
<tr id="S8.T3.11.11" class="ltx_tr">
<th id="S8.T3.11.11.2" class="ltx_td ltx_align_left ltx_th ltx_th_row">Plate scale</th>
<td id="S8.T3.11.11.1" class="ltx_td ltx_align_right"><math id="S8.T3.11.11.1.m1.1" class="ltx_Math" alttext="\rm 16\arcsec/pixel" display="inline"><semantics id="S8.T3.11.11.1.m1.1a"><mrow id="S8.T3.11.11.1.m1.1.1" xref="S8.T3.11.11.1.m1.1.1.cmml"><mrow id="S8.T3.11.11.1.m1.1.1.2" xref="S8.T3.11.11.1.m1.1.1.2.cmml"><mn id="S8.T3.11.11.1.m1.1.1.2.2" xref="S8.T3.11.11.1.m1.1.1.2.2.cmml">16</mn><mo id="S8.T3.11.11.1.m1.1.1.2.1" xref="S8.T3.11.11.1.m1.1.1.2.1.cmml">⁢</mo><mi mathvariant="normal" id="S8.T3.11.11.1.m1.1.1.2.3" xref="S8.T3.11.11.1.m1.1.1.2.3.cmml">″</mi></mrow><mo id="S8.T3.11.11.1.m1.1.1.1" xref="S8.T3.11.11.1.m1.1.1.1.cmml">/</mo><mi id="S8.T3.11.11.1.m1.1.1.3" xref="S8.T3.11.11.1.m1.1.1.3.cmml">pixel</mi></mrow><annotation-xml encoding="MathML-Content" id="S8.T3.11.11.1.m1.1b"><apply id="S8.T3.11.11.1.m1.1.1.cmml" xref="S8.T3.11.11.1.m1.1.1"><divide id="S8.T3.11.11.1.m1.1.1.1.cmml" xref="S8.T3.11.11.1.m1.1.1.1"></divide><apply id="S8.T3.11.11.1.m1.1.1.2.cmml" xref="S8.T3.11.11.1.m1.1.1.2"><times id="S8.T3.11.11.1.m1.1.1.2.1.cmml" xref="S8.T3.11.11.1.m1.1.1.2.1"></times><cn type="integer" id="S8.T3.11.11.1.m1.1.1.2.2.cmml" xref="S8.T3.11.11.1.m1.1.1.2.2">16</cn><ci id="S8.T3.11.11.1.m1.1.1.2.3.cmml" xref="S8.T3.11.11.1.m1.1.1.2.3">″</ci></apply><ci id="S8.T3.11.11.1.m1.1.1.3.cmml" xref="S8.T3.11.11.1.m1.1.1.3">pixel</ci></apply></annotation-xml><annotation encoding="application/x-tex" id="S8.T3.11.11.1.m1.1c">\rm 16\arcsec/pixel</annotation><annotation encoding="application/x-llamapun" id="S8.T3.11.11.1.m1.1d">16 ″ / roman_pixel</annotation></semantics></math></td>
</tr>
<tr id="S8.T3.12.12" class="ltx_tr">
<th id="S8.T3.12.12.2" class="ltx_td ltx_align_left ltx_th ltx_th_row">Field of view</th>
<td id="S8.T3.12.12.1" class="ltx_td ltx_align_right"><math id="S8.T3.12.12.1.m1.1" class="ltx_Math" alttext="9\arcdeg\times 9\arcdeg" display="inline"><semantics id="S8.T3.12.12.1.m1.1a"><mrow id="S8.T3.12.12.1.m1.1.1" xref="S8.T3.12.12.1.m1.1.1.cmml"><mrow id="S8.T3.12.12.1.m1.1.1.2" xref="S8.T3.12.12.1.m1.1.1.2.cmml"><mrow id="S8.T3.12.12.1.m1.1.1.2.2" xref="S8.T3.12.12.1.m1.1.1.2.2.cmml"><mn id="S8.T3.12.12.1.m1.1.1.2.2.2" xref="S8.T3.12.12.1.m1.1.1.2.2.2.cmml">9</mn><mo id="S8.T3.12.12.1.m1.1.1.2.2.1" xref="S8.T3.12.12.1.m1.1.1.2.2.1.cmml">⁢</mo><mi mathvariant="normal" id="S8.T3.12.12.1.m1.1.1.2.2.3" xref="S8.T3.12.12.1.m1.1.1.2.2.3.cmml">°</mi></mrow><mo id="S8.T3.12.12.1.m1.1.1.2.1" xref="S8.T3.12.12.1.m1.1.1.2.1.cmml">×</mo><mn id="S8.T3.12.12.1.m1.1.1.2.3" xref="S8.T3.12.12.1.m1.1.1.2.3.cmml">9</mn></mrow><mo id="S8.T3.12.12.1.m1.1.1.1" xref="S8.T3.12.12.1.m1.1.1.1.cmml">⁢</mo><mi mathvariant="normal" id="S8.T3.12.12.1.m1.1.1.3" xref="S8.T3.12.12.1.m1.1.1.3.cmml">°</mi></mrow><annotation-xml encoding="MathML-Content" id="S8.T3.12.12.1.m1.1b"><apply id="S8.T3.12.12.1.m1.1.1.cmml" xref="S8.T3.12.12.1.m1.1.1"><times id="S8.T3.12.12.1.m1.1.1.1.cmml" xref="S8.T3.12.12.1.m1.1.1.1"></times><apply id="S8.T3.12.12.1.m1.1.1.2.cmml" xref="S8.T3.12.12.1.m1.1.1.2"><times id="S8.T3.12.12.1.m1.1.1.2.1.cmml" xref="S8.T3.12.12.1.m1.1.1.2.1"></times><apply id="S8.T3.12.12.1.m1.1.1.2.2.cmml" xref="S8.T3.12.12.1.m1.1.1.2.2"><times id="S8.T3.12.12.1.m1.1.1.2.2.1.cmml" xref="S8.T3.12.12.1.m1.1.1.2.2.1"></times><cn type="integer" id="S8.T3.12.12.1.m1.1.1.2.2.2.cmml" xref="S8.T3.12.12.1.m1.1.1.2.2.2">9</cn><ci id="S8.T3.12.12.1.m1.1.1.2.2.3.cmml" xref="S8.T3.12.12.1.m1.1.1.2.2.3">°</ci></apply><cn type="integer" id="S8.T3.12.12.1.m1.1.1.2.3.cmml" xref="S8.T3.12.12.1.m1.1.1.2.3">9</cn></apply><ci id="S8.T3.12.12.1.m1.1.1.3.cmml" xref="S8.T3.12.12.1.m1.1.1.3">°</ci></apply></annotation-xml><annotation encoding="application/x-tex" id="S8.T3.12.12.1.m1.1c">9\arcdeg\times 9\arcdeg</annotation><annotation encoding="application/x-llamapun" id="S8.T3.12.12.1.m1.1d">9 ° × 9 °</annotation></semantics></math></td>
</tr>
<tr id="S8.T3.26.34.7" class="ltx_tr">
<th id="S8.T3.26.34.7.1" class="ltx_td ltx_align_center ltx_th ltx_th_row ltx_border_t" colspan="2">Pointing precision</th>
</tr>
<tr id="S8.T3.13.13" class="ltx_tr">
<th id="S8.T3.13.13.2" class="ltx_td ltx_align_left ltx_th ltx_th_row ltx_border_tt">Homing</th>
<td id="S8.T3.13.13.1" class="ltx_td ltx_align_right ltx_border_tt"><math id="S8.T3.13.13.1.m1.1" class="ltx_Math" alttext="&lt;0.5\arcmin" display="inline"><semantics id="S8.T3.13.13.1.m1.1a"><mrow id="S8.T3.13.13.1.m1.1.1" xref="S8.T3.13.13.1.m1.1.1.cmml"><mi id="S8.T3.13.13.1.m1.1.1.2" xref="S8.T3.13.13.1.m1.1.1.2.cmml"></mi><mo id="S8.T3.13.13.1.m1.1.1.1" xref="S8.T3.13.13.1.m1.1.1.1.cmml">&lt;</mo><mrow id="S8.T3.13.13.1.m1.1.1.3" xref="S8.T3.13.13.1.m1.1.1.3.cmml"><mn id="S8.T3.13.13.1.m1.1.1.3.2" xref="S8.T3.13.13.1.m1.1.1.3.2.cmml">0.5</mn><mo id="S8.T3.13.13.1.m1.1.1.3.1" xref="S8.T3.13.13.1.m1.1.1.3.1.cmml">⁢</mo><mi mathvariant="normal" id="S8.T3.13.13.1.m1.1.1.3.3" xref="S8.T3.13.13.1.m1.1.1.3.3.cmml">′</mi></mrow></mrow><annotation-xml encoding="MathML-Content" id="S8.T3.13.13.1.m1.1b"><apply id="S8.T3.13.13.1.m1.1.1.cmml" xref="S8.T3.13.13.1.m1.1.1"><lt id="S8.T3.13.13.1.m1.1.1.1.cmml" xref="S8.T3.13.13.1.m1.1.1.1"></lt><csymbol cd="latexml" id="S8.T3.13.13.1.m1.1.1.2.cmml" xref="S8.T3.13.13.1.m1.1.1.2">absent</csymbol><apply id="S8.T3.13.13.1.m1.1.1.3.cmml" xref="S8.T3.13.13.1.m1.1.1.3"><times id="S8.T3.13.13.1.m1.1.1.3.1.cmml" xref="S8.T3.13.13.1.m1.1.1.3.1"></times><cn type="float" id="S8.T3.13.13.1.m1.1.1.3.2.cmml" xref="S8.T3.13.13.1.m1.1.1.3.2">0.5</cn><ci id="S8.T3.13.13.1.m1.1.1.3.3.cmml" xref="S8.T3.13.13.1.m1.1.1.3.3">′</ci></apply></apply></annotation-xml><annotation encoding="application/x-tex" id="S8.T3.13.13.1.m1.1c">&lt;0.5\arcmin</annotation><annotation encoding="application/x-llamapun" id="S8.T3.13.13.1.m1.1d">&lt; 0.5 ′</annotation></semantics></math></td>
</tr>
<tr id="S8.T3.14.14" class="ltx_tr">
<th id="S8.T3.14.14.2" class="ltx_td ltx_align_left ltx_th ltx_th_row">Tracking</th>
<td id="S8.T3.14.14.1" class="ltx_td ltx_align_right"><math id="S8.T3.14.14.1.m1.1" class="ltx_Math" alttext="\rm&lt;0.5sec/2hr" display="inline"><semantics id="S8.T3.14.14.1.m1.1a"><mrow id="S8.T3.14.14.1.m1.1.1" xref="S8.T3.14.14.1.m1.1.1.cmml"><mi id="S8.T3.14.14.1.m1.1.1.2" xref="S8.T3.14.14.1.m1.1.1.2.cmml"></mi><mo id="S8.T3.14.14.1.m1.1.1.1" xref="S8.T3.14.14.1.m1.1.1.1.cmml">&lt;</mo><mrow id="S8.T3.14.14.1.m1.1.1.3" xref="S8.T3.14.14.1.m1.1.1.3.cmml"><mrow id="S8.T3.14.14.1.m1.1.1.3.2" xref="S8.T3.14.14.1.m1.1.1.3.2.cmml"><mrow id="S8.T3.14.14.1.m1.1.1.3.2.2" xref="S8.T3.14.14.1.m1.1.1.3.2.2.cmml"><mn id="S8.T3.14.14.1.m1.1.1.3.2.2.2" xref="S8.T3.14.14.1.m1.1.1.3.2.2.2.cmml">0.5</mn><mo id="S8.T3.14.14.1.m1.1.1.3.2.2.1" xref="S8.T3.14.14.1.m1.1.1.3.2.2.1.cmml">⁢</mo><mi id="S8.T3.14.14.1.m1.1.1.3.2.2.3" xref="S8.T3.14.14.1.m1.1.1.3.2.2.3.cmml">sec</mi></mrow><mo id="S8.T3.14.14.1.m1.1.1.3.2.1" xref="S8.T3.14.14.1.m1.1.1.3.2.1.cmml">/</mo><mn id="S8.T3.14.14.1.m1.1.1.3.2.3" xref="S8.T3.14.14.1.m1.1.1.3.2.3.cmml">2</mn></mrow><mo id="S8.T3.14.14.1.m1.1.1.3.1" xref="S8.T3.14.14.1.m1.1.1.3.1.cmml">⁢</mo><mi mathvariant="normal" id="S8.T3.14.14.1.m1.1.1.3.3" xref="S8.T3.14.14.1.m1.1.1.3.3.cmml">h</mi><mo id="S8.T3.14.14.1.m1.1.1.3.1a" xref="S8.T3.14.14.1.m1.1.1.3.1.cmml">⁢</mo><mi mathvariant="normal" id="S8.T3.14.14.1.m1.1.1.3.4" xref="S8.T3.14.14.1.m1.1.1.3.4.cmml">r</mi></mrow></mrow><annotation-xml encoding="MathML-Content" id="S8.T3.14.14.1.m1.1b"><apply id="S8.T3.14.14.1.m1.1.1.cmml" xref="S8.T3.14.14.1.m1.1.1"><lt id="S8.T3.14.14.1.m1.1.1.1.cmml" xref="S8.T3.14.14.1.m1.1.1.1"></lt><csymbol cd="latexml" id="S8.T3.14.14.1.m1.1.1.2.cmml" xref="S8.T3.14.14.1.m1.1.1.2">absent</csymbol><apply id="S8.T3.14.14.1.m1.1.1.3.cmml" xref="S8.T3.14.14.1.m1.1.1.3"><times id="S8.T3.14.14.1.m1.1.1.3.1.cmml" xref="S8.T3.14.14.1.m1.1.1.3.1"></times><apply id="S8.T3.14.14.1.m1.1.1.3.2.cmml" xref="S8.T3.14.14.1.m1.1.1.3.2"><divide id="S8.T3.14.14.1.m1.1.1.3.2.1.cmml" xref="S8.T3.14.14.1.m1.1.1.3.2.1"></divide><apply id="S8.T3.14.14.1.m1.1.1.3.2.2.cmml" xref="S8.T3.14.14.1.m1.1.1.3.2.2"><times id="S8.T3.14.14.1.m1.1.1.3.2.2.1.cmml" xref="S8.T3.14.14.1.m1.1.1.3.2.2.1"></times><cn type="float" id="S8.T3.14.14.1.m1.1.1.3.2.2.2.cmml" xref="S8.T3.14.14.1.m1.1.1.3.2.2.2">0.5</cn><ci id="S8.T3.14.14.1.m1.1.1.3.2.2.3.cmml" xref="S8.T3.14.14.1.m1.1.1.3.2.2.3">sec</ci></apply><cn type="integer" id="S8.T3.14.14.1.m1.1.1.3.2.3.cmml" xref="S8.T3.14.14.1.m1.1.1.3.2.3">2</cn></apply><ci id="S8.T3.14.14.1.m1.1.1.3.3.cmml" xref="S8.T3.14.14.1.m1.1.1.3.3">h</ci><ci id="S8.T3.14.14.1.m1.1.1.3.4.cmml" xref="S8.T3.14.14.1.m1.1.1.3.4">r</ci></apply></apply></annotation-xml><annotation encoding="application/x-tex" id="S8.T3.14.14.1.m1.1c">\rm&lt;0.5sec/2hr</annotation><annotation encoding="application/x-llamapun" id="S8.T3.14.14.1.m1.1d">&lt; 0.5 roman_sec / 2 roman_h roman_r</annotation></semantics></math></td>
</tr>
<tr id="S8.T3.16.16" class="ltx_tr">
<th id="S8.T3.16.16.3" class="ltx_td ltx_align_left ltx_th ltx_th_row">Absolute</th>
<td id="S8.T3.16.16.2" class="ltx_td ltx_align_right">RA:<math id="S8.T3.15.15.1.m1.1" class="ltx_Math" alttext="\rm 20sec" display="inline"><semantics id="S8.T3.15.15.1.m1.1a"><mrow id="S8.T3.15.15.1.m1.1.1" xref="S8.T3.15.15.1.m1.1.1.cmml"><mn id="S8.T3.15.15.1.m1.1.1.2" xref="S8.T3.15.15.1.m1.1.1.2.cmml">20</mn><mo id="S8.T3.15.15.1.m1.1.1.1" xref="S8.T3.15.15.1.m1.1.1.1.cmml">⁢</mo><mi mathvariant="normal" id="S8.T3.15.15.1.m1.1.1.3" xref="S8.T3.15.15.1.m1.1.1.3.cmml">s</mi><mo id="S8.T3.15.15.1.m1.1.1.1a" xref="S8.T3.15.15.1.m1.1.1.1.cmml">⁢</mo><mi mathvariant="normal" id="S8.T3.15.15.1.m1.1.1.4" xref="S8.T3.15.15.1.m1.1.1.4.cmml">e</mi><mo id="S8.T3.15.15.1.m1.1.1.1b" xref="S8.T3.15.15.1.m1.1.1.1.cmml">⁢</mo><mi mathvariant="normal" id="S8.T3.15.15.1.m1.1.1.5" xref="S8.T3.15.15.1.m1.1.1.5.cmml">c</mi></mrow><annotation-xml encoding="MathML-Content" id="S8.T3.15.15.1.m1.1b"><apply id="S8.T3.15.15.1.m1.1.1.cmml" xref="S8.T3.15.15.1.m1.1.1"><times id="S8.T3.15.15.1.m1.1.1.1.cmml" xref="S8.T3.15.15.1.m1.1.1.1"></times><cn type="integer" id="S8.T3.15.15.1.m1.1.1.2.cmml" xref="S8.T3.15.15.1.m1.1.1.2">20</cn><ci id="S8.T3.15.15.1.m1.1.1.3.cmml" xref="S8.T3.15.15.1.m1.1.1.3">s</ci><ci id="S8.T3.15.15.1.m1.1.1.4.cmml" xref="S8.T3.15.15.1.m1.1.1.4">e</ci><ci id="S8.T3.15.15.1.m1.1.1.5.cmml" xref="S8.T3.15.15.1.m1.1.1.5">c</ci></apply></annotation-xml><annotation encoding="application/x-tex" id="S8.T3.15.15.1.m1.1c">\rm 20sec</annotation><annotation encoding="application/x-llamapun" id="S8.T3.15.15.1.m1.1d">20 roman_s roman_e roman_c</annotation></semantics></math>, Dec: <math id="S8.T3.16.16.2.m2.1" class="ltx_Math" alttext="1\arcmin" display="inline"><semantics id="S8.T3.16.16.2.m2.1a"><mrow id="S8.T3.16.16.2.m2.1.1" xref="S8.T3.16.16.2.m2.1.1.cmml"><mn id="S8.T3.16.16.2.m2.1.1.2" xref="S8.T3.16.16.2.m2.1.1.2.cmml">1</mn><mo id="S8.T3.16.16.2.m2.1.1.1" xref="S8.T3.16.16.2.m2.1.1.1.cmml">⁢</mo><mi mathvariant="normal" id="S8.T3.16.16.2.m2.1.1.3" xref="S8.T3.16.16.2.m2.1.1.3.cmml">′</mi></mrow><annotation-xml encoding="MathML-Content" id="S8.T3.16.16.2.m2.1b"><apply id="S8.T3.16.16.2.m2.1.1.cmml" xref="S8.T3.16.16.2.m2.1.1"><times id="S8.T3.16.16.2.m2.1.1.1.cmml" xref="S8.T3.16.16.2.m2.1.1.1"></times><cn type="integer" id="S8.T3.16.16.2.m2.1.1.2.cmml" xref="S8.T3.16.16.2.m2.1.1.2">1</cn><ci id="S8.T3.16.16.2.m2.1.1.3.cmml" xref="S8.T3.16.16.2.m2.1.1.3">′</ci></apply></annotation-xml><annotation encoding="application/x-tex" id="S8.T3.16.16.2.m2.1c">1\arcmin</annotation><annotation encoding="application/x-llamapun" id="S8.T3.16.16.2.m2.1d">1 ′</annotation></semantics></math>
</td>
</tr>
<tr id="S8.T3.18.18" class="ltx_tr">
<th id="S8.T3.18.18.3" class="ltx_td ltx_align_left ltx_th ltx_th_row">Repeatability</th>
<td id="S8.T3.18.18.2" class="ltx_td ltx_align_right">RA: <math id="S8.T3.17.17.1.m1.1" class="ltx_Math" alttext="\rm 7sec" display="inline"><semantics id="S8.T3.17.17.1.m1.1a"><mrow id="S8.T3.17.17.1.m1.1.1" xref="S8.T3.17.17.1.m1.1.1.cmml"><mn id="S8.T3.17.17.1.m1.1.1.2" xref="S8.T3.17.17.1.m1.1.1.2.cmml">7</mn><mo id="S8.T3.17.17.1.m1.1.1.1" xref="S8.T3.17.17.1.m1.1.1.1.cmml">⁢</mo><mi mathvariant="normal" id="S8.T3.17.17.1.m1.1.1.3" xref="S8.T3.17.17.1.m1.1.1.3.cmml">s</mi><mo id="S8.T3.17.17.1.m1.1.1.1a" xref="S8.T3.17.17.1.m1.1.1.1.cmml">⁢</mo><mi mathvariant="normal" id="S8.T3.17.17.1.m1.1.1.4" xref="S8.T3.17.17.1.m1.1.1.4.cmml">e</mi><mo id="S8.T3.17.17.1.m1.1.1.1b" xref="S8.T3.17.17.1.m1.1.1.1.cmml">⁢</mo><mi mathvariant="normal" id="S8.T3.17.17.1.m1.1.1.5" xref="S8.T3.17.17.1.m1.1.1.5.cmml">c</mi></mrow><annotation-xml encoding="MathML-Content" id="S8.T3.17.17.1.m1.1b"><apply id="S8.T3.17.17.1.m1.1.1.cmml" xref="S8.T3.17.17.1.m1.1.1"><times id="S8.T3.17.17.1.m1.1.1.1.cmml" xref="S8.T3.17.17.1.m1.1.1.1"></times><cn type="integer" id="S8.T3.17.17.1.m1.1.1.2.cmml" xref="S8.T3.17.17.1.m1.1.1.2">7</cn><ci id="S8.T3.17.17.1.m1.1.1.3.cmml" xref="S8.T3.17.17.1.m1.1.1.3">s</ci><ci id="S8.T3.17.17.1.m1.1.1.4.cmml" xref="S8.T3.17.17.1.m1.1.1.4">e</ci><ci id="S8.T3.17.17.1.m1.1.1.5.cmml" xref="S8.T3.17.17.1.m1.1.1.5">c</ci></apply></annotation-xml><annotation encoding="application/x-tex" id="S8.T3.17.17.1.m1.1c">\rm 7sec</annotation><annotation encoding="application/x-llamapun" id="S8.T3.17.17.1.m1.1d">7 roman_s roman_e roman_c</annotation></semantics></math>, Dec: <math id="S8.T3.18.18.2.m2.1" class="ltx_Math" alttext="1\arcmin" display="inline"><semantics id="S8.T3.18.18.2.m2.1a"><mrow id="S8.T3.18.18.2.m2.1.1" xref="S8.T3.18.18.2.m2.1.1.cmml"><mn id="S8.T3.18.18.2.m2.1.1.2" xref="S8.T3.18.18.2.m2.1.1.2.cmml">1</mn><mo id="S8.T3.18.18.2.m2.1.1.1" xref="S8.T3.18.18.2.m2.1.1.1.cmml">⁢</mo><mi mathvariant="normal" id="S8.T3.18.18.2.m2.1.1.3" xref="S8.T3.18.18.2.m2.1.1.3.cmml">′</mi></mrow><annotation-xml encoding="MathML-Content" id="S8.T3.18.18.2.m2.1b"><apply id="S8.T3.18.18.2.m2.1.1.cmml" xref="S8.T3.18.18.2.m2.1.1"><times id="S8.T3.18.18.2.m2.1.1.1.cmml" xref="S8.T3.18.18.2.m2.1.1.1"></times><cn type="integer" id="S8.T3.18.18.2.m2.1.1.2.cmml" xref="S8.T3.18.18.2.m2.1.1.2">1</cn><ci id="S8.T3.18.18.2.m2.1.1.3.cmml" xref="S8.T3.18.18.2.m2.1.1.3">′</ci></apply></annotation-xml><annotation encoding="application/x-tex" id="S8.T3.18.18.2.m2.1c">1\arcmin</annotation><annotation encoding="application/x-llamapun" id="S8.T3.18.18.2.m2.1d">1 ′</annotation></semantics></math>
</td>
</tr>
<tr id="S8.T3.19.19" class="ltx_tr">
<th id="S8.T3.19.19.1" class="ltx_td ltx_align_center ltx_th ltx_th_row ltx_border_t" colspan="2">Photometry (<math id="S8.T3.19.19.1.m1.1" class="ltx_Math" alttext="\rm I_{c}" display="inline"><semantics id="S8.T3.19.19.1.m1.1a"><msub id="S8.T3.19.19.1.m1.1.1" xref="S8.T3.19.19.1.m1.1.1.cmml"><mi mathvariant="normal" id="S8.T3.19.19.1.m1.1.1.2" xref="S8.T3.19.19.1.m1.1.1.2.cmml">I</mi><mi mathvariant="normal" id="S8.T3.19.19.1.m1.1.1.3" xref="S8.T3.19.19.1.m1.1.1.3.cmml">c</mi></msub><annotation-xml encoding="MathML-Content" id="S8.T3.19.19.1.m1.1b"><apply id="S8.T3.19.19.1.m1.1.1.cmml" xref="S8.T3.19.19.1.m1.1.1"><csymbol cd="ambiguous" id="S8.T3.19.19.1.m1.1.1.1.cmml" xref="S8.T3.19.19.1.m1.1.1">subscript</csymbol><ci id="S8.T3.19.19.1.m1.1.1.2.cmml" xref="S8.T3.19.19.1.m1.1.1.2">I</ci><ci id="S8.T3.19.19.1.m1.1.1.3.cmml" xref="S8.T3.19.19.1.m1.1.1.3">c</ci></apply></annotation-xml><annotation encoding="application/x-tex" id="S8.T3.19.19.1.m1.1c">\rm I_{c}</annotation><annotation encoding="application/x-llamapun" id="S8.T3.19.19.1.m1.1d">roman_I start_POSTSUBSCRIPT roman_c end_POSTSUBSCRIPT</annotation></semantics></math> band, 8min time res.)</th>
</tr>
<tr id="S8.T3.20.20" class="ltx_tr">
<th id="S8.T3.20.20.2" class="ltx_td ltx_align_left ltx_th ltx_th_row ltx_border_tt">Accuracy (absolute)</th>
<td id="S8.T3.20.20.1" class="ltx_td ltx_align_right ltx_border_tt"><math id="S8.T3.20.20.1.m1.1" class="ltx_Math" alttext="\rm\approx 0.05^{m}" display="inline"><semantics id="S8.T3.20.20.1.m1.1a"><mrow id="S8.T3.20.20.1.m1.1.1" xref="S8.T3.20.20.1.m1.1.1.cmml"><mi id="S8.T3.20.20.1.m1.1.1.2" xref="S8.T3.20.20.1.m1.1.1.2.cmml"></mi><mo id="S8.T3.20.20.1.m1.1.1.1" xref="S8.T3.20.20.1.m1.1.1.1.cmml">≈</mo><msup id="S8.T3.20.20.1.m1.1.1.3" xref="S8.T3.20.20.1.m1.1.1.3.cmml"><mn id="S8.T3.20.20.1.m1.1.1.3.2" xref="S8.T3.20.20.1.m1.1.1.3.2.cmml">0.05</mn><mi mathvariant="normal" id="S8.T3.20.20.1.m1.1.1.3.3" xref="S8.T3.20.20.1.m1.1.1.3.3.cmml">m</mi></msup></mrow><annotation-xml encoding="MathML-Content" id="S8.T3.20.20.1.m1.1b"><apply id="S8.T3.20.20.1.m1.1.1.cmml" xref="S8.T3.20.20.1.m1.1.1"><approx id="S8.T3.20.20.1.m1.1.1.1.cmml" xref="S8.T3.20.20.1.m1.1.1.1"></approx><csymbol cd="latexml" id="S8.T3.20.20.1.m1.1.1.2.cmml" xref="S8.T3.20.20.1.m1.1.1.2">absent</csymbol><apply id="S8.T3.20.20.1.m1.1.1.3.cmml" xref="S8.T3.20.20.1.m1.1.1.3"><csymbol cd="ambiguous" id="S8.T3.20.20.1.m1.1.1.3.1.cmml" xref="S8.T3.20.20.1.m1.1.1.3">superscript</csymbol><cn type="float" id="S8.T3.20.20.1.m1.1.1.3.2.cmml" xref="S8.T3.20.20.1.m1.1.1.3.2">0.05</cn><ci id="S8.T3.20.20.1.m1.1.1.3.3.cmml" xref="S8.T3.20.20.1.m1.1.1.3.3">m</ci></apply></apply></annotation-xml><annotation encoding="application/x-tex" id="S8.T3.20.20.1.m1.1c">\rm\approx 0.05^{m}</annotation><annotation encoding="application/x-llamapun" id="S8.T3.20.20.1.m1.1d">≈ 0.05 start_POSTSUPERSCRIPT roman_m end_POSTSUPERSCRIPT</annotation></semantics></math></td>
</tr>
<tr id="S8.T3.22.22" class="ltx_tr">
<th id="S8.T3.21.21.1" class="ltx_td ltx_align_left ltx_th ltx_th_row">Precision <math id="S8.T3.21.21.1.m1.1" class="ltx_Math" alttext="\rm(I_{c}&lt;10^{m})" display="inline"><semantics id="S8.T3.21.21.1.m1.1a"><mrow id="S8.T3.21.21.1.m1.1.1.1" xref="S8.T3.21.21.1.m1.1.1.1.1.cmml"><mo stretchy="false" id="S8.T3.21.21.1.m1.1.1.1.2" xref="S8.T3.21.21.1.m1.1.1.1.1.cmml">(</mo><mrow id="S8.T3.21.21.1.m1.1.1.1.1" xref="S8.T3.21.21.1.m1.1.1.1.1.cmml"><msub id="S8.T3.21.21.1.m1.1.1.1.1.2" xref="S8.T3.21.21.1.m1.1.1.1.1.2.cmml"><mi mathvariant="normal" id="S8.T3.21.21.1.m1.1.1.1.1.2.2" xref="S8.T3.21.21.1.m1.1.1.1.1.2.2.cmml">I</mi><mi mathvariant="normal" id="S8.T3.21.21.1.m1.1.1.1.1.2.3" xref="S8.T3.21.21.1.m1.1.1.1.1.2.3.cmml">c</mi></msub><mo id="S8.T3.21.21.1.m1.1.1.1.1.1" xref="S8.T3.21.21.1.m1.1.1.1.1.1.cmml">&lt;</mo><msup id="S8.T3.21.21.1.m1.1.1.1.1.3" xref="S8.T3.21.21.1.m1.1.1.1.1.3.cmml"><mn id="S8.T3.21.21.1.m1.1.1.1.1.3.2" xref="S8.T3.21.21.1.m1.1.1.1.1.3.2.cmml">10</mn><mi mathvariant="normal" id="S8.T3.21.21.1.m1.1.1.1.1.3.3" xref="S8.T3.21.21.1.m1.1.1.1.1.3.3.cmml">m</mi></msup></mrow><mo stretchy="false" id="S8.T3.21.21.1.m1.1.1.1.3" xref="S8.T3.21.21.1.m1.1.1.1.1.cmml">)</mo></mrow><annotation-xml encoding="MathML-Content" id="S8.T3.21.21.1.m1.1b"><apply id="S8.T3.21.21.1.m1.1.1.1.1.cmml" xref="S8.T3.21.21.1.m1.1.1.1"><lt id="S8.T3.21.21.1.m1.1.1.1.1.1.cmml" xref="S8.T3.21.21.1.m1.1.1.1.1.1"></lt><apply id="S8.T3.21.21.1.m1.1.1.1.1.2.cmml" xref="S8.T3.21.21.1.m1.1.1.1.1.2"><csymbol cd="ambiguous" id="S8.T3.21.21.1.m1.1.1.1.1.2.1.cmml" xref="S8.T3.21.21.1.m1.1.1.1.1.2">subscript</csymbol><ci id="S8.T3.21.21.1.m1.1.1.1.1.2.2.cmml" xref="S8.T3.21.21.1.m1.1.1.1.1.2.2">I</ci><ci id="S8.T3.21.21.1.m1.1.1.1.1.2.3.cmml" xref="S8.T3.21.21.1.m1.1.1.1.1.2.3">c</ci></apply><apply id="S8.T3.21.21.1.m1.1.1.1.1.3.cmml" xref="S8.T3.21.21.1.m1.1.1.1.1.3"><csymbol cd="ambiguous" id="S8.T3.21.21.1.m1.1.1.1.1.3.1.cmml" xref="S8.T3.21.21.1.m1.1.1.1.1.3">superscript</csymbol><cn type="integer" id="S8.T3.21.21.1.m1.1.1.1.1.3.2.cmml" xref="S8.T3.21.21.1.m1.1.1.1.1.3.2">10</cn><ci id="S8.T3.21.21.1.m1.1.1.1.1.3.3.cmml" xref="S8.T3.21.21.1.m1.1.1.1.1.3.3">m</ci></apply></apply></annotation-xml><annotation encoding="application/x-tex" id="S8.T3.21.21.1.m1.1c">\rm(I_{c}&lt;10^{m})</annotation><annotation encoding="application/x-llamapun" id="S8.T3.21.21.1.m1.1d">( roman_I start_POSTSUBSCRIPT roman_c end_POSTSUBSCRIPT &lt; 10 start_POSTSUPERSCRIPT roman_m end_POSTSUPERSCRIPT )</annotation></semantics></math>
</th>
<td id="S8.T3.22.22.2" class="ltx_td ltx_align_right"><math id="S8.T3.22.22.2.m1.1" class="ltx_Math" alttext="\rm\lesssim 0.01^{m}" display="inline"><semantics id="S8.T3.22.22.2.m1.1a"><mrow id="S8.T3.22.22.2.m1.1.1" xref="S8.T3.22.22.2.m1.1.1.cmml"><mi id="S8.T3.22.22.2.m1.1.1.2" xref="S8.T3.22.22.2.m1.1.1.2.cmml"></mi><mo id="S8.T3.22.22.2.m1.1.1.1" xref="S8.T3.22.22.2.m1.1.1.1.cmml">≲</mo><msup id="S8.T3.22.22.2.m1.1.1.3" xref="S8.T3.22.22.2.m1.1.1.3.cmml"><mn id="S8.T3.22.22.2.m1.1.1.3.2" xref="S8.T3.22.22.2.m1.1.1.3.2.cmml">0.01</mn><mi mathvariant="normal" id="S8.T3.22.22.2.m1.1.1.3.3" xref="S8.T3.22.22.2.m1.1.1.3.3.cmml">m</mi></msup></mrow><annotation-xml encoding="MathML-Content" id="S8.T3.22.22.2.m1.1b"><apply id="S8.T3.22.22.2.m1.1.1.cmml" xref="S8.T3.22.22.2.m1.1.1"><csymbol cd="latexml" id="S8.T3.22.22.2.m1.1.1.1.cmml" xref="S8.T3.22.22.2.m1.1.1.1">less-than-or-similar-to</csymbol><csymbol cd="latexml" id="S8.T3.22.22.2.m1.1.1.2.cmml" xref="S8.T3.22.22.2.m1.1.1.2">absent</csymbol><apply id="S8.T3.22.22.2.m1.1.1.3.cmml" xref="S8.T3.22.22.2.m1.1.1.3"><csymbol cd="ambiguous" id="S8.T3.22.22.2.m1.1.1.3.1.cmml" xref="S8.T3.22.22.2.m1.1.1.3">superscript</csymbol><cn type="float" id="S8.T3.22.22.2.m1.1.1.3.2.cmml" xref="S8.T3.22.22.2.m1.1.1.3.2">0.01</cn><ci id="S8.T3.22.22.2.m1.1.1.3.3.cmml" xref="S8.T3.22.22.2.m1.1.1.3.3">m</ci></apply></apply></annotation-xml><annotation encoding="application/x-tex" id="S8.T3.22.22.2.m1.1c">\rm\lesssim 0.01^{m}</annotation><annotation encoding="application/x-llamapun" id="S8.T3.22.22.2.m1.1d">≲ 0.01 start_POSTSUPERSCRIPT roman_m end_POSTSUPERSCRIPT</annotation></semantics></math></td>
</tr>
<tr id="S8.T3.24.24" class="ltx_tr">
<th id="S8.T3.23.23.1" class="ltx_td ltx_align_left ltx_th ltx_th_row"><math id="S8.T3.23.23.1.m1.1" class="ltx_Math" alttext="\rm I_{c}=11^{m}" display="inline"><semantics id="S8.T3.23.23.1.m1.1a"><mrow id="S8.T3.23.23.1.m1.1.1" xref="S8.T3.23.23.1.m1.1.1.cmml"><msub id="S8.T3.23.23.1.m1.1.1.2" xref="S8.T3.23.23.1.m1.1.1.2.cmml"><mi mathvariant="normal" id="S8.T3.23.23.1.m1.1.1.2.2" xref="S8.T3.23.23.1.m1.1.1.2.2.cmml">I</mi><mi mathvariant="normal" id="S8.T3.23.23.1.m1.1.1.2.3" xref="S8.T3.23.23.1.m1.1.1.2.3.cmml">c</mi></msub><mo id="S8.T3.23.23.1.m1.1.1.1" xref="S8.T3.23.23.1.m1.1.1.1.cmml">=</mo><msup id="S8.T3.23.23.1.m1.1.1.3" xref="S8.T3.23.23.1.m1.1.1.3.cmml"><mn id="S8.T3.23.23.1.m1.1.1.3.2" xref="S8.T3.23.23.1.m1.1.1.3.2.cmml">11</mn><mi mathvariant="normal" id="S8.T3.23.23.1.m1.1.1.3.3" xref="S8.T3.23.23.1.m1.1.1.3.3.cmml">m</mi></msup></mrow><annotation-xml encoding="MathML-Content" id="S8.T3.23.23.1.m1.1b"><apply id="S8.T3.23.23.1.m1.1.1.cmml" xref="S8.T3.23.23.1.m1.1.1"><eq id="S8.T3.23.23.1.m1.1.1.1.cmml" xref="S8.T3.23.23.1.m1.1.1.1"></eq><apply id="S8.T3.23.23.1.m1.1.1.2.cmml" xref="S8.T3.23.23.1.m1.1.1.2"><csymbol cd="ambiguous" id="S8.T3.23.23.1.m1.1.1.2.1.cmml" xref="S8.T3.23.23.1.m1.1.1.2">subscript</csymbol><ci id="S8.T3.23.23.1.m1.1.1.2.2.cmml" xref="S8.T3.23.23.1.m1.1.1.2.2">I</ci><ci id="S8.T3.23.23.1.m1.1.1.2.3.cmml" xref="S8.T3.23.23.1.m1.1.1.2.3">c</ci></apply><apply id="S8.T3.23.23.1.m1.1.1.3.cmml" xref="S8.T3.23.23.1.m1.1.1.3"><csymbol cd="ambiguous" id="S8.T3.23.23.1.m1.1.1.3.1.cmml" xref="S8.T3.23.23.1.m1.1.1.3">superscript</csymbol><cn type="integer" id="S8.T3.23.23.1.m1.1.1.3.2.cmml" xref="S8.T3.23.23.1.m1.1.1.3.2">11</cn><ci id="S8.T3.23.23.1.m1.1.1.3.3.cmml" xref="S8.T3.23.23.1.m1.1.1.3.3">m</ci></apply></apply></annotation-xml><annotation encoding="application/x-tex" id="S8.T3.23.23.1.m1.1c">\rm I_{c}=11^{m}</annotation><annotation encoding="application/x-llamapun" id="S8.T3.23.23.1.m1.1d">roman_I start_POSTSUBSCRIPT roman_c end_POSTSUBSCRIPT = 11 start_POSTSUPERSCRIPT roman_m end_POSTSUPERSCRIPT</annotation></semantics></math></th>
<td id="S8.T3.24.24.2" class="ltx_td ltx_align_right"><math id="S8.T3.24.24.2.m1.1" class="ltx_Math" alttext="\rm 0.02^{m}" display="inline"><semantics id="S8.T3.24.24.2.m1.1a"><msup id="S8.T3.24.24.2.m1.1.1" xref="S8.T3.24.24.2.m1.1.1.cmml"><mn id="S8.T3.24.24.2.m1.1.1.2" xref="S8.T3.24.24.2.m1.1.1.2.cmml">0.02</mn><mi mathvariant="normal" id="S8.T3.24.24.2.m1.1.1.3" xref="S8.T3.24.24.2.m1.1.1.3.cmml">m</mi></msup><annotation-xml encoding="MathML-Content" id="S8.T3.24.24.2.m1.1b"><apply id="S8.T3.24.24.2.m1.1.1.cmml" xref="S8.T3.24.24.2.m1.1.1"><csymbol cd="ambiguous" id="S8.T3.24.24.2.m1.1.1.1.cmml" xref="S8.T3.24.24.2.m1.1.1">superscript</csymbol><cn type="float" id="S8.T3.24.24.2.m1.1.1.2.cmml" xref="S8.T3.24.24.2.m1.1.1.2">0.02</cn><ci id="S8.T3.24.24.2.m1.1.1.3.cmml" xref="S8.T3.24.24.2.m1.1.1.3">m</ci></apply></annotation-xml><annotation encoding="application/x-tex" id="S8.T3.24.24.2.m1.1c">\rm 0.02^{m}</annotation><annotation encoding="application/x-llamapun" id="S8.T3.24.24.2.m1.1d">0.02 start_POSTSUPERSCRIPT roman_m end_POSTSUPERSCRIPT</annotation></semantics></math></td>
</tr>
<tr id="S8.T3.26.26" class="ltx_tr">
<th id="S8.T3.25.25.1" class="ltx_td ltx_align_left ltx_th ltx_th_row ltx_border_bb"><math id="S8.T3.25.25.1.m1.1" class="ltx_Math" alttext="\rm I_{c}=12^{m}" display="inline"><semantics id="S8.T3.25.25.1.m1.1a"><mrow id="S8.T3.25.25.1.m1.1.1" xref="S8.T3.25.25.1.m1.1.1.cmml"><msub id="S8.T3.25.25.1.m1.1.1.2" xref="S8.T3.25.25.1.m1.1.1.2.cmml"><mi mathvariant="normal" id="S8.T3.25.25.1.m1.1.1.2.2" xref="S8.T3.25.25.1.m1.1.1.2.2.cmml">I</mi><mi mathvariant="normal" id="S8.T3.25.25.1.m1.1.1.2.3" xref="S8.T3.25.25.1.m1.1.1.2.3.cmml">c</mi></msub><mo id="S8.T3.25.25.1.m1.1.1.1" xref="S8.T3.25.25.1.m1.1.1.1.cmml">=</mo><msup id="S8.T3.25.25.1.m1.1.1.3" xref="S8.T3.25.25.1.m1.1.1.3.cmml"><mn id="S8.T3.25.25.1.m1.1.1.3.2" xref="S8.T3.25.25.1.m1.1.1.3.2.cmml">12</mn><mi mathvariant="normal" id="S8.T3.25.25.1.m1.1.1.3.3" xref="S8.T3.25.25.1.m1.1.1.3.3.cmml">m</mi></msup></mrow><annotation-xml encoding="MathML-Content" id="S8.T3.25.25.1.m1.1b"><apply id="S8.T3.25.25.1.m1.1.1.cmml" xref="S8.T3.25.25.1.m1.1.1"><eq id="S8.T3.25.25.1.m1.1.1.1.cmml" xref="S8.T3.25.25.1.m1.1.1.1"></eq><apply id="S8.T3.25.25.1.m1.1.1.2.cmml" xref="S8.T3.25.25.1.m1.1.1.2"><csymbol cd="ambiguous" id="S8.T3.25.25.1.m1.1.1.2.1.cmml" xref="S8.T3.25.25.1.m1.1.1.2">subscript</csymbol><ci id="S8.T3.25.25.1.m1.1.1.2.2.cmml" xref="S8.T3.25.25.1.m1.1.1.2.2">I</ci><ci id="S8.T3.25.25.1.m1.1.1.2.3.cmml" xref="S8.T3.25.25.1.m1.1.1.2.3">c</ci></apply><apply id="S8.T3.25.25.1.m1.1.1.3.cmml" xref="S8.T3.25.25.1.m1.1.1.3"><csymbol cd="ambiguous" id="S8.T3.25.25.1.m1.1.1.3.1.cmml" xref="S8.T3.25.25.1.m1.1.1.3">superscript</csymbol><cn type="integer" id="S8.T3.25.25.1.m1.1.1.3.2.cmml" xref="S8.T3.25.25.1.m1.1.1.3.2">12</cn><ci id="S8.T3.25.25.1.m1.1.1.3.3.cmml" xref="S8.T3.25.25.1.m1.1.1.3.3">m</ci></apply></apply></annotation-xml><annotation encoding="application/x-tex" id="S8.T3.25.25.1.m1.1c">\rm I_{c}=12^{m}</annotation><annotation encoding="application/x-llamapun" id="S8.T3.25.25.1.m1.1d">roman_I start_POSTSUBSCRIPT roman_c end_POSTSUBSCRIPT = 12 start_POSTSUPERSCRIPT roman_m end_POSTSUPERSCRIPT</annotation></semantics></math></th>
<td id="S8.T3.26.26.2" class="ltx_td ltx_align_right ltx_border_bb"><math id="S8.T3.26.26.2.m1.1" class="ltx_Math" alttext="\rm 0.05^{m}" display="inline"><semantics id="S8.T3.26.26.2.m1.1a"><msup id="S8.T3.26.26.2.m1.1.1" xref="S8.T3.26.26.2.m1.1.1.cmml"><mn id="S8.T3.26.26.2.m1.1.1.2" xref="S8.T3.26.26.2.m1.1.1.2.cmml">0.05</mn><mi mathvariant="normal" id="S8.T3.26.26.2.m1.1.1.3" xref="S8.T3.26.26.2.m1.1.1.3.cmml">m</mi></msup><annotation-xml encoding="MathML-Content" id="S8.T3.26.26.2.m1.1b"><apply id="S8.T3.26.26.2.m1.1.1.cmml" xref="S8.T3.26.26.2.m1.1.1"><csymbol cd="ambiguous" id="S8.T3.26.26.2.m1.1.1.1.cmml" xref="S8.T3.26.26.2.m1.1.1">superscript</csymbol><cn type="float" id="S8.T3.26.26.2.m1.1.1.2.cmml" xref="S8.T3.26.26.2.m1.1.1.2">0.05</cn><ci id="S8.T3.26.26.2.m1.1.1.3.cmml" xref="S8.T3.26.26.2.m1.1.1.3">m</ci></apply></annotation-xml><annotation encoding="application/x-tex" id="S8.T3.26.26.2.m1.1c">\rm 0.05^{m}</annotation><annotation encoding="application/x-llamapun" id="S8.T3.26.26.2.m1.1d">0.05 start_POSTSUPERSCRIPT roman_m end_POSTSUPERSCRIPT</annotation></semantics></math></td>
</tr>
</tbody>
</table>
</figure>
<div class="ltx_acknowledgements">
</div>
</section>
<section id="S9" class="ltx_section">
<h2 class="ltx_title ltx_title_section">
<span class="ltx_tag ltx_tag_section">9 </span>Acknowledgements</h2>

<div id="S9.p1" class="ltx_para">
<p id="S9.p1.1" class="ltx_p">This project was initiated by Prof. Bohdan Paczyński. We are grateful
for his tireless support, encouragement, advice and funds from the
generous gift of Mr. William Golden. We are indebted to G. Pojmański
for his collaboration and for sharing his plans and software with us.
It is a pleasure to thank P. Strittmatter the opportunity of installing
HAT to Kitt Peak, the partial funds in the installation costs and the
hospitality of Steward Observatory. We are thankful to the whole staff
of Steward Observatory (B. Peterson, G. Stafford, W. Wood, J. Rill) for
their consistent support in operating HAT. G. Bakos wishes to
acknowledge the kind hospitality of Konkoly Observatory during the test
period of HAT, while being an undergraduate and junior research fellow,
with special thanks to the director, L. G. Balázs. The project was
partially funded by the Hungarian OTKA T-038437 grant. Support is also
acknowledged to NASA grant NAG 5-10854. G. Bakos is greatly indebted
to the Smithsonian Astrophysical Observatory for support through the
SAO Predoctoral Fellowship program. We thank G. Kovács, R. W. Noyes,
D. D. Sasselov, K. Z Stanek and A. H. Szentgyorgy for valuable
discussions and their careful reading of this manuscript.</p>
</div>
</section>
<section id="bib" class="ltx_bibliography">
<h2 class="ltx_title ltx_title_bibliography">References</h2>

<ul class="ltx_biblist">
<li id="bib.bib1" class="ltx_bibitem">
<span class="ltx_tag ltx_role_refnum ltx_tag_bibitem">Alard (2000)</span>
<span class="ltx_bibblock"> Alard, C. 2000, A&amp;AS, 144, 363.

</span>
</li>
<li id="bib.bib2" class="ltx_bibitem">
<span class="ltx_tag ltx_role_refnum ltx_tag_bibitem">Alcock et al. (2000)</span>
<span class="ltx_bibblock"> Alcock, C. et al. 2000,
ApJ, 542, 257.

</span>
</li>
<li id="bib.bib3" class="ltx_bibitem">
<span class="ltx_tag ltx_role_refnum ltx_tag_bibitem">Allsman &amp; Axelrod (2001)</span>
<span class="ltx_bibblock"> Allsman, R. A. &amp; Axelrod,
T. S. 2001, astro-ph/0108444

</span>
</li>
<li id="bib.bib4" class="ltx_bibitem">
<span class="ltx_tag ltx_role_refnum ltx_tag_bibitem">Akerlof et al. (2000)</span>
<span class="ltx_bibblock"> Akerlof, C. et al. 
2000, AJ, 119, 1901.

</span>
</li>
<li id="bib.bib5" class="ltx_bibitem">
<span class="ltx_tag ltx_role_refnum ltx_tag_bibitem">Antipin (1997)</span>
<span class="ltx_bibblock"> Antipin, S. 1997, A&amp;A, 326, L1.

</span>
</li>
<li id="bib.bib6" class="ltx_bibitem">
<span class="ltx_tag ltx_role_refnum ltx_tag_bibitem">Bakos (2001)</span>
<span class="ltx_bibblock"> Bakos, G. Á. 2001, RTLinux driven HAT for
All Sky Monitoring, Smale Telescope Astronomy on Global scales,
Eds. B. Paczyński, Claudia Lemme, Wen Ping Chen, ASP Conf. Series, 246, 59
(IAU coll. 183)

</span>
</li>
<li id="bib.bib7" class="ltx_bibitem">
<span class="ltx_tag ltx_role_refnum ltx_tag_bibitem">Borucki et al. (2001)</span>
<span class="ltx_bibblock"> Borucki, W. J.,
Caldwell, D., Koch, D. G., Webster, L. D., Jenkins, J. M., Ninkov, Z., &amp;
Showen, R. 2001, PASP, 113, 439.

</span>
</li>
<li id="bib.bib8" class="ltx_bibitem">
<span class="ltx_tag ltx_role_refnum ltx_tag_bibitem">Boyd et al. (1984)</span>
<span class="ltx_bibblock"> Boyd, L. J., Genet, R. M.,
Sauer, D. J., Hawthorn, P. S., Slonaker, L. W., &amp; Chatto, J. 1984, BAAS,
16, 909.

</span>
</li>
<li id="bib.bib9" class="ltx_bibitem">
<span class="ltx_tag ltx_role_refnum ltx_tag_bibitem">Brown &amp; Charbonneau (1999)</span>
<span class="ltx_bibblock"> Brown, T. M. &amp;
Charbonneau, D. 1999, American Astronomical Society Meeting, 195, 110907.

</span>
</li>
<li id="bib.bib10" class="ltx_bibitem">
<span class="ltx_tag ltx_role_refnum ltx_tag_bibitem">Buchler, Serre &amp; Kolláth (1995)</span>
<span class="ltx_bibblock"> Buchler,
J. R., Serre, T., &amp; Kolláth, Z. 1995, Physical Review Letters, 73, .

</span>
</li>
<li id="bib.bib11" class="ltx_bibitem">
<span class="ltx_tag ltx_role_refnum ltx_tag_bibitem">Buffington, Booth, &amp; Hudson (1991)</span>
<span class="ltx_bibblock">
Buffington, A., Booth, C. H., &amp; Hudson, H. S. 1991, PASP, 103, 685.

</span>
</li>
<li id="bib.bib12" class="ltx_bibitem">
<span class="ltx_tag ltx_role_refnum ltx_tag_bibitem">Burki et al. (1986)</span>
<span class="ltx_bibblock"> Burki, G., Schmidt,
E. G., Arellano Ferro, A., Fernie, J. D., Sasselov, D., Simon, N. R.,
Percy, J. R., &amp; Szabados, L. 1986, A&amp;A, 168, 139

</span>
</li>
<li id="bib.bib13" class="ltx_bibitem">
<span class="ltx_tag ltx_role_refnum ltx_tag_bibitem">Chen (1999)</span>
<span class="ltx_bibblock"> Chen, W. P. 1999, Observational
Astrophysics in Asia and its Future, 181.

</span>
</li>
<li id="bib.bib14" class="ltx_bibitem">
<span class="ltx_tag ltx_role_refnum ltx_tag_bibitem">Chromey &amp; Hasselbacher (1996)</span>
<span class="ltx_bibblock"> Chromey,
F. R. &amp; Hasselbacher, D. A. 1996, PASP, 108, 944.

</span>
</li>
<li id="bib.bib15" class="ltx_bibitem">
<span class="ltx_tag ltx_role_refnum ltx_tag_bibitem">Duerbeck et al. (2000)</span>
<span class="ltx_bibblock"> Duerbeck, H. W. et
al. 2000, AJ, 119, 2360.

</span>
</li>
<li id="bib.bib16" class="ltx_bibitem">
<span class="ltx_tag ltx_role_refnum ltx_tag_bibitem">Helmer &amp; Morrison (1985)</span>
<span class="ltx_bibblock"> Helmer, L. &amp;
Morrison, L. V. 1985, Vistas in Astronomy, 28, 505.

</span>
</li>
<li id="bib.bib17" class="ltx_bibitem">
<span class="ltx_tag ltx_role_refnum ltx_tag_bibitem">Høg et al. (2000)</span>
<span class="ltx_bibblock"> Høg, E. et al. 2000,
A&amp;A, 357, 367.

</span>
</li>
<li id="bib.bib18" class="ltx_bibitem">
<span class="ltx_tag ltx_role_refnum ltx_tag_bibitem">Ishioka et al. (2001)</span>
<span class="ltx_bibblock"> Ishioka, R. et al. 
2001, IAU Circ., 7669, 1.

</span>
</li>
<li id="bib.bib19" class="ltx_bibitem">
<span class="ltx_tag ltx_role_refnum ltx_tag_bibitem">Kaluzny et al. (1998)</span>
<span class="ltx_bibblock"> Kaluzny, J., Stanek,
K. Z., Krockenberger, M., Sasselov, D. D., Tonry, J. L., &amp; Mateo, M. 
1998, AJ, 115, 1016.

</span>
</li>
<li id="bib.bib20" class="ltx_bibitem">
<span class="ltx_tag ltx_role_refnum ltx_tag_bibitem">Kholopov et al. (1998)</span>
<span class="ltx_bibblock"> Kholopov, P. N. et
al. 1998, Combined General Catalogue of Variable Stars (GCVS),
(Vizier Online Data at CDS, Strasbourg: II/214A)

</span>
</li>
<li id="bib.bib21" class="ltx_bibitem">
<span class="ltx_tag ltx_role_refnum ltx_tag_bibitem">Kiss et al. (2000)</span>
<span class="ltx_bibblock"> Kiss, L. L., Szatmáry, K., Szabó, G., &amp; Mattei, J. A. 2000, A&amp;AS, 145, 283.

</span>
</li>
<li id="bib.bib22" class="ltx_bibitem">
<span class="ltx_tag ltx_role_refnum ltx_tag_bibitem">Kovács et al. (2002)</span>
<span class="ltx_bibblock"> Kovács, G., Zucker, S. &amp;
Mazeh, T. 2002, A&amp;A, accepted

</span>
</li>
<li id="bib.bib23" class="ltx_bibitem">
<span class="ltx_tag ltx_role_refnum ltx_tag_bibitem">Landolt (1983)</span>
<span class="ltx_bibblock"> Landolt, A. U. 1983, AJ, 88,
853.

</span>
</li>
<li id="bib.bib24" class="ltx_bibitem">
<span class="ltx_tag ltx_role_refnum ltx_tag_bibitem">Landolt (1992)</span>
<span class="ltx_bibblock"> Landolt, A. U. 1992, AJ, 104, 340.

</span>
</li>
<li id="bib.bib25" class="ltx_bibitem">
<span class="ltx_tag ltx_role_refnum ltx_tag_bibitem">Lasker et al. (1990)</span>
<span class="ltx_bibblock"> Lasker, B. M., Sturch,
C. R., McLean, B. J., Russell, J. L., Jenkner, H., &amp; Shara, M. M. 1990,
AJ, 99, 2019.

</span>
</li>
<li id="bib.bib26" class="ltx_bibitem">
<span class="ltx_tag ltx_role_refnum ltx_tag_bibitem">van Leeuwen et al. (1997)</span>
<span class="ltx_bibblock"> van Leeuwen, F.,
Evans, D. W., Grenon, M., Grossmann, V., Mignard, F., &amp; Perryman,
M. A. C. 1997, A&amp;A, 323, L61.

</span>
</li>
<li id="bib.bib27" class="ltx_bibitem">
<span class="ltx_tag ltx_role_refnum ltx_tag_bibitem">Leung (1962)</span>
<span class="ltx_bibblock"> Leung, K. 1962, JRASC, 56, 242.

</span>
</li>
<li id="bib.bib28" class="ltx_bibitem">
<span class="ltx_tag ltx_role_refnum ltx_tag_bibitem">Macri, Sasselov, &amp; Stanek (2001)</span>
<span class="ltx_bibblock"> Macri,
L. M., Sasselov, D. D., &amp; Stanek, K. Z. 2001, ApJ, 550, L159.

</span>
</li>
<li id="bib.bib29" class="ltx_bibitem">
<span class="ltx_tag ltx_role_refnum ltx_tag_bibitem">Nakano &amp; Kushida (1996)</span>
<span class="ltx_bibblock"> Nakano, S. &amp;
Kushida, Y. 1996, IAU Circ., 6323, 1.

</span>
</li>
<li id="bib.bib30" class="ltx_bibitem">
<span class="ltx_tag ltx_role_refnum ltx_tag_bibitem">Newberry (1991)</span>
<span class="ltx_bibblock"> Newberry, M. V. 1991, PASP,
103, 122.

</span>
</li>
<li id="bib.bib31" class="ltx_bibitem">
<span class="ltx_tag ltx_role_refnum ltx_tag_bibitem">Paczyński (1997)</span>
<span class="ltx_bibblock"> Paczyński, B. 1997,
Variables Stars and the Astrophysical Returns of the Microlensing Surveys,
Edited by R. Ferlet, J. P. Maillard and B. Raban, 357.

</span>
</li>
<li id="bib.bib32" class="ltx_bibitem">
<span class="ltx_tag ltx_role_refnum ltx_tag_bibitem">Paczyński (2000)</span>
<span class="ltx_bibblock"> Paczyński, B. 
2000, PASP, 112, 1281.

</span>
</li>
<li id="bib.bib33" class="ltx_bibitem">
<span class="ltx_tag ltx_role_refnum ltx_tag_bibitem">Park et al. (1997)</span>
<span class="ltx_bibblock"> Park, H. S. et al. 1997,
ApJ, 490, L21.

</span>
</li>
<li id="bib.bib34" class="ltx_bibitem">
<span class="ltx_tag ltx_role_refnum ltx_tag_bibitem">Park et al. (2001)</span>
<span class="ltx_bibblock"> Park, H. S. et al. 2001,
American Astronomical Society Meeting, 198, 3805.

</span>
</li>
<li id="bib.bib35" class="ltx_bibitem">
<span class="ltx_tag ltx_role_refnum ltx_tag_bibitem">Pereira et al. (2000)</span>
<span class="ltx_bibblock"> Pereira, W. E.,
Nemiroff, R. J., Rafert, J. B., Ftaclas, C., &amp; Perez-Ramirez, D. 2000,
American Astronomical Society Meeting, 197, #115.10.

</span>
</li>
<li id="bib.bib36" class="ltx_bibitem">
<span class="ltx_tag ltx_role_refnum ltx_tag_bibitem">Perryman et al. (1997b)</span>
<span class="ltx_bibblock"> Perryman, M. A. C. et
al. 1997, The HIPPARCOS and TYCHO Catalogues, ESA SP-1200, Vol. 1-17

</span>
</li>
<li id="bib.bib37" class="ltx_bibitem">
<span class="ltx_tag ltx_role_refnum ltx_tag_bibitem">Perryman et al. (1997)</span>
<span class="ltx_bibblock"> Perryman, M. A. C. et
al. 1997, The HIPPARCOS and TYCHO Catalogues, ESA SP-1200, 1, 465,
http://astro.estec.esa.nl/Hipparcos/vis_stat.html

</span>
</li>
<li id="bib.bib38" class="ltx_bibitem">
<span class="ltx_tag ltx_role_refnum ltx_tag_bibitem">Piquard et al. (2001)</span>
<span class="ltx_bibblock"> Piquard, S., Halbwachs,
J.-L., Fabricius, C., Geckeler, R., Soubiran, C., &amp; Wicenec, A. 2001,
A&amp;A, 373, 576, also VizieR Online Data Catalog 337, 330576

</span>
</li>
<li id="bib.bib39" class="ltx_bibitem">
<span class="ltx_tag ltx_role_refnum ltx_tag_bibitem">Pojmański (1997)</span>
<span class="ltx_bibblock"> Pojmański, G. 1997, Acta
Astronomica, 47, 467.

</span>
</li>
<li id="bib.bib40" class="ltx_bibitem">
<span class="ltx_tag ltx_role_refnum ltx_tag_bibitem">Pojmański (1998)</span>
<span class="ltx_bibblock"> Pojmański, G. 1998, Acta
Astronomica, 48, 35.

</span>
</li>
<li id="bib.bib41" class="ltx_bibitem">
<span class="ltx_tag ltx_role_refnum ltx_tag_bibitem">Pojmański (2000)</span>
<span class="ltx_bibblock"> Pojmański, G. 2000, Acta
Astronomica, 50, 177.

</span>
</li>
<li id="bib.bib42" class="ltx_bibitem">
<span class="ltx_tag ltx_role_refnum ltx_tag_bibitem">Pojmański (2002)</span>
<span class="ltx_bibblock"> Pojmański, G. 2002, private communication

</span>
</li>
<li id="bib.bib43" class="ltx_bibitem">
<span class="ltx_tag ltx_role_refnum ltx_tag_bibitem">Smith et al. (2002)</span>
<span class="ltx_bibblock"> Smith, D. A. et al. 2002,
astro-ph/0204404

</span>
</li>
<li id="bib.bib44" class="ltx_bibitem">
<span class="ltx_tag ltx_role_refnum ltx_tag_bibitem">Richmond, Treffers, &amp; Filippenko (1993)</span>
<span class="ltx_bibblock">
Richmond, M., Treffers, R. R., &amp; Filippenko, A. V. 1993, PASP, 105, 1164.

</span>
</li>
<li id="bib.bib45" class="ltx_bibitem">
<span class="ltx_tag ltx_role_refnum ltx_tag_bibitem">Robinson et al. (1995)</span>
<span class="ltx_bibblock"> Robinson, L. B., Wei,
M. Z., Borucki, W. J., Dunham, E. W., Ford, C. H., &amp; Granados, A. F. 
1995, PASP, 107, 1094.

</span>
</li>
<li id="bib.bib46" class="ltx_bibitem">
<span class="ltx_tag ltx_role_refnum ltx_tag_bibitem">Stetson (1987)</span>
<span class="ltx_bibblock"> Stetson, P. B. 1987, PASP,
99, 191.

</span>
</li>
<li id="bib.bib47" class="ltx_bibitem">
<span class="ltx_tag ltx_role_refnum ltx_tag_bibitem">Stetson (1996)</span>
<span class="ltx_bibblock"> Stetson, P. B. 1996, PASP,
108, 851.

</span>
</li>
<li id="bib.bib48" class="ltx_bibitem">
<span class="ltx_tag ltx_role_refnum ltx_tag_bibitem">Tody (1993)</span>
<span class="ltx_bibblock"> Tody, D. 1993, ASP
Conf. Ser. 52: Astronomical Data Analysis Software and Systems II, 2, 173.

</span>
</li>
<li id="bib.bib49" class="ltx_bibitem">
<span class="ltx_tag ltx_role_refnum ltx_tag_bibitem">Walker (1987)</span>
<span class="ltx_bibblock"> Walker, A. 1987, Noao Newsletter No. 10, 16

</span>
</li>
<li id="bib.bib50" class="ltx_bibitem">
<span class="ltx_tag ltx_role_refnum ltx_tag_bibitem">Woźniak (2002)</span>
<span class="ltx_bibblock"> Woźniak, P. R. et al. 2002, astro-ph/0201377

</span>
</li>
</ul>
</section>
<div class="ltx_pagination ltx_role_newpage"></div>
<figure id="S9.F8" class="ltx_figure"><img src="x8.png" id="S9.F8.g1" class="ltx_graphics" width="559" height="802" alt="Selected variables from HAT observations of field “F077”. Known
variable stars are marked with their conventional names, new
discoveries with their GSC numbers. Numbers in the panels indicate the
period, average I-band brightness and the height of the box
(magnitudes). Colons (:) denote uncertainty in the
cross-identification.
">
<figcaption class="ltx_caption"><span class="ltx_tag ltx_tag_figure">Figure 8: </span>Selected variables from HAT observations of field “F077”. Known
variable stars are marked with their conventional names, new
discoveries with their GSC numbers. Numbers in the panels indicate the
period, average I-band brightness and the height of the box
(magnitudes). Colons (:) denote uncertainty in the
cross-identification.
</figcaption>
</figure>
</article>
</div>
<footer class="ltx_page_footer">
<div class="ltx_page_logo">Generated  on Sun Dec  6 04:28:21 2020 by <a href="http://dlmf.nist.gov/LaTeXML/">LaTeXML <img src="data:image/png;base64,iVBORw0KGgoAAAANSUhEUgAAAAsAAAAOCAYAAAD5YeaVAAAAAXNSR0IArs4c6QAAAAZiS0dEAP8A/wD/oL2nkwAAAAlwSFlzAAALEwAACxMBAJqcGAAAAAd0SU1FB9wKExQZLWTEaOUAAAAddEVYdENvbW1lbnQAQ3JlYXRlZCB3aXRoIFRoZSBHSU1Q72QlbgAAAdpJREFUKM9tkL+L2nAARz9fPZNCKFapUn8kyI0e4iRHSR1Kb8ng0lJw6FYHFwv2LwhOpcWxTjeUunYqOmqd6hEoRDhtDWdA8ApRYsSUCDHNt5ul13vz4w0vWCgUnnEc975arX6ORqN3VqtVZbfbTQC4uEHANM3jSqXymFI6yWazP2KxWAXAL9zCUa1Wy2tXVxheKA9YNoR8Pt+aTqe4FVVVvz05O6MBhqUIBGk8Hn8HAOVy+T+XLJfLS4ZhTiRJgqIoVBRFIoric47jPnmeB1mW/9rr9ZpSSn3Lsmir1fJZlqWlUonKsvwWwD8ymc/nXwVBeLjf7xEKhdBut9Hr9WgmkyGEkJwsy5eHG5vN5g0AKIoCAEgkEkin0wQAfN9/cXPdheu6P33fBwB4ngcAcByHJpPJl+fn54mD3Gg0NrquXxeLRQAAwzAYj8cwTZPwPH9/sVg8PXweDAauqqr2cDjEer1GJBLBZDJBs9mE4zjwfZ85lAGg2+06hmGgXq+j3+/DsixYlgVN03a9Xu8jgCNCyIegIAgx13Vfd7vdu+FweG8YRkjXdWy329+dTgeSJD3ieZ7RNO0VAXAPwDEAO5VKndi2fWrb9jWl9Esul6PZbDY9Go1OZ7PZ9z/lyuD3OozU2wAAAABJRU5ErkJggg==" alt="[LOGO]"></a>
</div></footer>
</div>
</body>
</html>
